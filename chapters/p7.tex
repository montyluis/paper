%\documentclass{article}
%\usepackage{amsmath,amssymb,amsfonts}
%\usepackage{fontspec}
%\usepackage{xunicode}
%\usepackage{xltxtra}
%\usepackage{polyglossia}
%\setdefaultlanguage{spanish}
%\usepackage{color}
%\usepackage{array}
%\usepackage{hhline}
%\usepackage{hyperref}
%\hypersetup{colorlinks=true, linkcolor=blue, citecolor=blue, filecolor=blue, urlcolor=blue}
%\newtheorem{theorem}{Theorem}
%\title{}
%\author{}
%\date{2014-09-04}
%\begin{document}

%%\clearpage\setcounter{page}{115}
\thispagestyle{empty}
\phantomsection{}
\addcontentsline{toc}{chapter}{Dimensión Didáctica de la Historia\newline y su Valor 
Formativo\newline $\diamond$
\normalfont\textit{Jaime Salazar Adame y Smirna Romero Garibay}}
{\centering {\scshape \large Dimensión Didáctica de la Historia y su Valor Formativo}\par}
\markboth{la formación del historiador}{dimensión didáctica}
\setcounter{footnote}{0}

\bigskip
\begin{center}
{\bfseries Jaime Salazar Adame\\
Smirna Romero Garibay}\\
{\itshape Universidad Autónoma de Guerrero\/}
\end{center}

\bigskip
\textbf{Resumen}

\enlargethispage{1\baselineskip}
El método didáctico tipo taller del grupo español 13--16, propuesto en esta ponencia, en su origen está proyectado para colaborar en la 
formación intelectual de los adolescentes aportando los valores que 
proporciona el conocimiento de la Historia. Cada unidad se propone como un trabajo detectivesco, donde lo importante no son los contenidos históricos sino el método de investigación de los historiadores. 
Consideramos que esta metodología resulta muy atractiva para que los 
estudiantes de bachillerato en el área de las ciencias humanístico 
sociales, así como los que inician los estudios profesionales propios 
del historiador, se interesen por el conocimiento del pasado a través 
de un novedoso procedimiento que cambia la visión memorialista y 
apegados a nombres y fechas, que priva en buena parte de las 
instituciones educativas de nuestro país.

\medskip
\textbf{Introducción}


Los profesores(as) de Historia probablemente pertenecemos a uno de los 
sectores docentes que mayor complejidad hemos sentido, en consideración 
a que los contenidos de los planes y programas de estudio dependen en 
gran medida de la función que se quiere que tenga la disciplina ante la 
renovación de contenidos que en cada revisión se introducen a los 
planes y programas de estudio de la licenciatura en Historia, 
principalmente porque sabemos mejor que nadie, que la enseñanza de las 
Ciencias Sociales y la Historia se encuentran\linebreak indisolublemente ligadas 
a la presencia de valores de tipo político e ideológico, al hacerlo, es 
portadora de un mensaje, dichos valores tendrán que ser explicitados 
por el profesor y éste deberá proporcionarlos al alumno a través de 
medios didácticos para reflexionar sobre ellos. 

 
Por su parte, los alumnos habrán de recibir la información social e 
histórica con un claro favorecimiento de la comprensión de otras 
visiones alternativas como la propuesta dedicada a que los 
historiadores en \mbox{ciernes} no adquieran los conocimientos históricos y 
sociales de manera pasiva sino constructiva, para ello, es necesario 
que podamos identificar la manera en que se llevan a cabo dichas 
inferencias, estos es, que los alumnos-as comprendan y utilicen el 
método del historiador. Así fue considerado y desarrollado por la línea 
de investigación británica en torno al <<proyecto curricular de ciencias 
sociales 13--16>>, que exitosamente se implementó en España en los años 70 y 80 del siglo precedente (Carretero~1995, p.~16). Tal 
posición se basaba en las ideas de Collingwood (1946), quien comparaba 
el conocimiento histórico con la actividad de un detective. Es decir, 
el historiador no debía aceptar los documentos históricos de manera 
acrítica, sino que tenía que verificar sus posiciones constantemente, 
indagando la autenticidad de sus fuentes y la plausibilidad de sus 
interpretaciones. Tal método es susceptible de adaptarse a la enseñanza 
del nivel superior de las licenciaturas de Historia.

\enlargethispage{1\baselineskip} 
El profesor Mario Carretero y su equipo de colaboradores de la 
Universidad Autónoma de Madrid hacen la crítica acerca de que en 
materia de comprensión y enseñanza de la historia los esfuerzos 
realizados en el ámbito internacional no han sido de la envergadura de 
los practicados en el área de las ciencias naturales y las matemáticas, 
medidos en razón de recursos, publicaciones, libros, congresos, 
etcétera.\linebreak También señala que excepciones como la línea de investigación 
británica en torno al <<proyecto 13--16>>, aplicado a la enseñanza media 
en España, se ha realizado a partir de un enfoque educativo y no 
psicológico. 

 
La presente ponencia ---haciendo referencia a dicho método--- se ocupa 
especialmente de su dimensión didáctica, así como de la importancia de 
su valor formativo en el proceso de enseñanza-aprendizaje centrado en 
el estudiante, y utilizando la tecnología educativa y audiovisual de 
última moda, que puede muy bien aplicarse en los jóvenes de los cursos 
de introducción, al trabajo del historiador que establece el currículo 
de la Licenciatura en Historia de la Unidad Académica de Filosofía y 
Letras de la Universidad Autónoma de Guerrero, y de otras instituciones 
afines. El Plan de estudios contiene seis cursos en el área de 
investigación: Introducción al trabajo del historiador I y II, más 
cuatro cursos de Seminario de tesis (UGRro~1995).

 
Los adolescentes que recién ingresan a la carrera de Historia, en su 
gran mayoría, conciben a esta ciencia como un conjunto de fechas, 
acontecimientos y relatos curiosos del pasado, animados con períodos 
ocasionales de realización de maquetas, dibujos e interpretaciones 
dramáticas. Tales actividades nos hacen mirar que pueden muy bien 
ocupar un tiempo en la jornada escolar, pero en sí mismas contribuyen 
muy poco a la construcción de los conceptos cruciales para un entendimiento 
de lo que supone ser un historiador y penetrar en el pasado.

\enlargethispage{1\baselineskip} 
La metodología propuesta en el Taller de Historia del proyecto 
curricular de ciencias sociales 13--16, concibe una didáctica que buscar 
no sólo enseñar, sino enseñar a aprender. Una didáctica que 
tiene como objetivo no sólo el camino para encontrar explicaciones 
coherentes, razonadas y sistemáticas a los problemas planteados, sino 
que\linebreak además posibilite el hacer partícipes a los alumnos de las 
experiencias a descubrir, o incluso con el riesgo de equivocarse.

 
Por otra parte, el proceso enseñanza aprendizaje centrado en el 
estudiante, como adopción de las recientes tendencias educativas en 
materia de competencias promovidas por la UNESCO, que le permita al 
educando adquirir las capacidades propias del historiador, ya no pueden 
continuar desarrollándose en el marco de la educación formal y 
puramente de las exigencias académicas, porque finalmente la educación 
es para la vida, y podemos considera que el material informativo es una 
importante ayuda para el profesor y para el alumno, porque como en el 
caso del libro de texto o del manual o de una síntesis, ofrece los 
conceptos que constituyen un determinado conocimiento ordenado y 
relacionados sistemáticamente entre sí (Delors~1997, pp.~94--95). 
El conocimiento de lo general también es necesario para elaborar 
hipótesis, porque estas resultan de poner en relación las leyes 
generales con la observación de los hechos concretos. Y las hipótesis 
son necesarias para dirigir el estudio, porque en su defecto éste puede 
ser errático.


\medskip
\textbf{Conocimiento: información, comprensión\\ y contenido}
%\enlargethispage{1\baselineskip}
 
El método propuesto para que el estudiante se compenetre rápida y 
objetivamente en qué consistirá su trabajo como futuro historiador 
comprende los conocimientos de los hechos ampliamente admitidos, por 
ejemplo, fechas, acontecimientos, lugares, el desarrollo vital de las 
personas, sobre las cuales puede situarse un marco histórico; la 
relación entre los hechos aceptados y los datos sobre los mismos y la 
materia de un determinado fragmento del estudio histórico.

 
Otra faceta consiste en la apreciación, por parte de los jóvenes 
alumnos, de los conceptos específicos que proporcionan a la historia su 
rango como disciplina académica, por ejemplo, cronología, secuencia, 
semejanza y cambio. La tercera es una faceta que complementa a los dos 
anteriores y consiste en las destrezas implícitas en todos los estudios 
académicos, la capacidad de indagar a partir de una gama de fuentes, de 
formular juicios sobre los descubrimientos realizados y de presentarlos 
de forma que puedan ser entendidos  por otras personas.

 
Todo ello nos indica que en la actualidad los profesores investigadores 
son capaces de brindar a sus estudiantes un enfoque mucho más complejo 
del aprendizaje de la historia que el existente hasta ahora. De manera 
sencilla y un tanto coloquial expondremos algunos conceptos sobre la 
materia.

 
Qué es la historia. La historia no es sólo un saber acerca del pasado 
destinado a que unos profesores de una generación se la expliquen a los 
más jóvenes. Posee herramientas de método a la última moda: 
tecnologías, informática, estadísticas, encuestas, dataciones, 
filologías y estilística, técnicas psicosociales, etcétera; todo es 
bienvenido para conocer mejor el objeto histórico. Es una ciencia 
humanista que domina caminos y técnicas para <<atacar>> las fuentes, por 
ello se ha especializado en la archivística, la conservación y 
restauración documental, etcétera.

 
Para qué sirve la historia.  No se puede utilizar lo que le sucedió a 
Miguel Hidalgo como testimonio de lo que se tiene que hacer en un 
determinado momento. Nos referimos al momento en que dispuesto a tomar 
la Cuidad de México decidió retirarse para evitar una masacre. Lo que 
si proporciona la historia es un mejor conocimiento de cómo actúan las 
personas. Hacemos historia para saber más acerca del ser humano, y 
haciéndola, descubrimos que quizá no sea maestra de la vida, como 
escribieron los historiadores clásicos, porque uno no puede sacar 
enseñanzas directas aplicables, pues las situaciones no se repiten. 

 
Qué hace el historiador. Investiga y enseña lo que investiga. El 
historiador es el encargado de interpretar y expresar con rigor 
científico la huella que el hombre deja tras su paso. Su trabajo no 
sólo consiste en estudiar y recopilar datos, hechos y acontecimientos, 
ha de analizarlos e interpretarlos, cruzando con rigor científico la 
sucesión cronológica de los hechos y estos con las áreas que configuran 
la vida del hombre histórico: política, ciencia, cultura, economía, 
costumbres, religión. Y todo ello con el fin de documentar e iluminar 
la memoria del hombre (Tusell~1993, pp.~5--8).

 
El historiador tiene además otra gran misión, que acaba siendo la de 
muchos, pero que no por ello deja de ser relevante: enseñar con rigor 
la historia en los distintos niveles educativos y ser un transmisor 
oral pero ecuánime de la existencia de hombres y acontecimientos 
remotos que determinaron la realidad política, económica y cultural de 
nuestros días.

 
Dedicarse a la historia es dedicarse a las humanidades, al saber 
humanístico por excelencia, desde una óptica  que es la del tiempo, con 
la idea de que se entiende mejor al ser humano a través de lo que ha 
sido. La dimensión temporal es absolutamente fundamental para un 
historiador, por eso, es que la geografía junto con la cronología son  
considerados como los \textit{ojos de la historia}.

 
La historia como conocimiento interpretable. Se trata de descubrir las 
dimensiones del ser humano en la historia. Y lo que se obtiene es un 
conocimiento interpretable que, por supuesto, está lleno de 
\mbox{dificultades} y dudas a la hora de precisar lo que aconteció en un 
tiempo concreto, pero a la vez, es el conocimiento que con más 
fiabilidad se acerca a cómo es el ser humano.

 
La historia proporciona saberes generales. La historia da, básicamente, 
la dimensión cultural y política del hombre, de cómo fue la vida humana 
en otros momentos.

 
La historia nos habla de cómo actuaron determinados protagonistas en
determinadas situaciones y da idea de lo contingente que son las cosas.

 
El estudio de la historia  forja una conciencia social y una conciencia
histórica.

 
A partir de la idea de que el hombre es un ser social que siempre ha 
vivido en comunidad, mostrando cómo eran aquellas primeras comunidades 
y como han ido evolucionando. Además, haciendo patente la noción de que 
todo presente tiene su origen en el pasado. 

 
También, teniendo la certeza de que las sociedades no son estáticas, 
sino mutables; asumiendo la noción de que cada quien, como integrante 
de la sociedad, forma parte del proceso de transformación; teniendo la 
percepción de que el presente es el pasado del futuro, y por lo tanto, 
se es responsable de la construcción de ese futuro, y finalmente, con 
la certeza de que soy parte del movimiento histórico y puedo, si 
quiero, participar de manera consciente en la transformación de la 
sociedad (García~1997, pp.~54--55).

 
La utilidad de la historia. Un historiador sabe por ejemplo que las 
crisis económicas son procesos que tienen tales o cuales repercusiones, 
que hay muchos tipos de crisis económicas y que sus soluciones pueden 
ser muy diferentes, pero sabe, también, que son superables. 

 
Igualmente, sabe que las fuerzas políticas situadas en un determinado 
contexto actúan de una determinada manera, o que la misión de un 
intelectual puede ser una u otra y que puede contribuir a que haya un 
orden democrático o a dificultarlo, y conoce el papel que los 
individuos juegan o han jugado en la historia, se llamen Benito 
Juárez, Juan Álvarez o Cuauhtémoc Cárdenas. La utilidad de la historia 
es la formación de la persona, pues juega un papel básico para 
comprender el entorno cultural en que se vive, y es un saber 
humanístico por excelencia.

 
La formación del historiador lo proporcionan tanto la curiosidad como 
la lectura. Ante todo afición por la lectura.  Sólo al leerse los 
libros y encontrar placer en la lectura es como el historiador  
aprovecha el caudal de influencias recibidas en su formación y cultura, 
porque sabe que es el eje de la Historia, porque es el protagonista de 
lo que investiga, descubre y escribe.

 
El historiador requiere gran dosis de curiosidad porque su campo de 
acción es una dimensión temporal donde la humanidad refleja un vivir 
inestable y móvil, al ser fundamentalmente cambio y mutabilidad. Toda 
investigación sólida y objetiva no puede prescindir del modelo de 
cambio como referencia fundamental, y los hechos que pueda conocer el 
historiador giran siempre alrededor de este \textit{logos} central de 
la mudanza, la crisis, la transformación o la subversión. Por tanto, el 
historiador es el biógrafo del movimiento.

 
La investigación cuestiona y explica por qué y cómo suceden las cosas. 
Trata de configurar el gran rompecabezas del pasado. Preguntar por el 
pasado es apasionarse por el presente. El placer de resolver un enigma, 
o la satisfacción de descubrir algo nuevo, algo que nadie más sabe, es 
contribuir al patrimonio del conocimiento humano.

 
Si no se es capaz de analizar lo que se lee en los periódicos o en lo 
que ves en la televisión, tampoco lo puedes entender con respecto al 
pasado, porque hay una sabiduría de la vida que es la misma que hace 
comprender la historia, y viceversa, hay una sabiduría de la historia 
que hace comprender la vida, y todo ese entramado forma parte de lo que 
es el trabajo del investigador. La sociedad necesita historiadores con 
mentes de capacidad crítica, que puedan emprender una investigación, 
hacer sus propias preguntas y encontrar sus propias respuestas.

 
Planteamos la difusión de la historia en la consideración de que la 
historia es una ciencia que se escribe para ser dada a conocer. Para 
matizar podríamos poner de ejemplo a otras ciencias, como la química, 
la física, y la medicina, que se desarrollan para utilizarlas en la 
práctica. En cambio, una vez investigada, no nos podemos quedar con la 
historia (Nava~2003, p.~24). Lo que debemos hacer con ella es 
darla a conocer, y esto no es otra cosa más que enseñarla. No sólo en 
las aulas, sino que cada vez se vuelve más necesario recurrir, para su 
difusión, a otros medios como la televisión, las publicaciones 
periódicas y las conferencias ({\itshape ibid.}).
\enlargethispage{1\baselineskip}
 
La docencia. Qué es ser un buen profesor. En la primaria y  secundaria 
muchos de los profesores de historia no son forzosamente historiadores 
profesionales. Aunque aprenden la historia y tratan de comunicarla a 
sus alumnos, no siempre lo hacen de la mejor manera. Generalmente la 
enseñan de forma tradicional, en donde la historia se convierte en una 
sucesión de nombres, fechas y héroes que los alumnos se tienen que 
aprender de memoria sin que les signifique nada, sin que tengan sentido 
para ellos.

 
La historia se convierte, así,  en una cronología que resulta 
enormemente pesada y aburrida. De tal manera que los alumnos conciben 
que eso es la historia y evidentemente, eso no es la historia.

 
Un buen profesor propicia que se  modifiquen, que se cambien los 
criterios para enseñar la historia, aprendiendo a mirar  la historia 
como un proceso que llega hasta  el presente.  La historia no es el 
mero relato de hechos,  debe encontrarse en ella una concatenación, una 
dinámica de todos los procesos que han dado lugar a lo que hoy estamos 
viviendo.

 
Es importante saber cómo es el alumno que resulta estimulante para un 
profesor haciéndole pensar que la tarea que está realizando no cae en 
saco roto, que verdaderamente tiene importancia, que merece la pena, 
más allá del sueldo mensual,  ir a clase cada día y participar en ella.

 
La vocación del historiador la encontramos en la disposición para 
dedicarse a la investigación, a escribir libros o a enseñar. Esta 
capacidad o interés la demuestra el alumno en la forma de analizar su 
propio entorno, en si se siente atraído por descubrir las causas que 
inciden en las modificaciones de lo inmediato. Si piensa que todo ello 
no responde a la casualidad, sino a unas causas y a una evolución que 
tal vez podría haber sucedido de otro modo; esta actitud es la que está 
relacionada con el tirón por la historia.

%\enlargethispage{1\baselineskip} 
Se afirma que se posee madera de historiador cuando se tiene interés 
por preguntarse cómo han llegado a suceder las cosas. Si se tienen 
intereses culturales amplios. Si se interesa en la literatura en lo que 
tiene de producto de unas determinadas circunstancias. O si se dispone 
de capacidad de imaginación, no capacidad fabulosa, pero si una 
imaginación capaz de reconstruir el pasado, y de darse cuenta de que 
cualquier momento del pasado tuvo infinitas  posibilidades abiertas 
aunque se produjera sólo una determinada y no las demás es que tienes 
madera de historiador (Sánchez~2002, pp.~86--87).

 
Una de las principales cualidades del historiador es la de actuar como 
detective porque un historiador tiene, inevitablemente, algo de 
detective, con el agravante de que el premio que consigue es escaso; 
pero cuando ha logrado explicar algo de forma convincente, interpretar 
la historia desde su propia perspectiva, la satisfacción intelectual es 
incomparable. También una curiosidad capaz de reconstruir los hechos, 
al menos ha de tener la de comprender cómo ocurren las cosas cuando se 
las explica otro, la de por ejemplo, por qué el río Huacapa se 
extinguió a mediados del siglo XX y cómo evolucionó después.

 
El perfil del historiador, en cuanto a imagen negativa es la que lo 
retrata como rata de archivo, enterrado en un tiempo pasado que no 
interesa a nadie, porque su trabajo resulta tan aburrido como ajeno al 
tiempo en que vive. En cuanto a la imagen real, esta lo concibe como 
persona con una inmensa curiosidad, con imaginación y ciertas dotes 
detectivescas que se interrogan continuamente por el pasado a través 
del presente que están viviendo, e intentan analizar, comprender y 
explicar su tiempo.

 
La calidad de vida o el nivel de vida de un historiador que investiga y 
enseña, no es gran cosa, pero la calidad de vida que tiene como 
profesor universitario es maravillosa, porque además de dar asesorías y 
tutorías a los estudiantes, puede escribir en periódicos, hacer 
programas de radio y televisión, publicar libros, entre tantas otras 
actividades intelectuales.

 
Las salidas profesionales del historiador o sea, las oportunidades que 
tiene frente al mercado laboral parten de que el historiador que parece 
destinado a ejercer exclusivamente la enseñanza de la historia o la 
investigación, tiene en cambio otros campos para desarrollarse 
profesionalmente. Por ejemplo, en áreas como la museografía, la 
archivística sobre todo con la Ley de acceso a la información, la 
biblioteconomía, o el sector editorial. 

 
También son numerosos los historiadores que ingresan en las 
administraciones públicas. O como promotor cultural, promotor 
turístico, ayudante de documentación, bibliotecario, comentarista en 
medios de comunicación, crítico literario, director de museos, 
documentalista, editor, editor de textos, profesor de bachillerato, 
profesor universitario, redactor publicitario. Algunas de estas salidas 
no son exclusivas del historiador, pero si puede perfectamente 
desempeñarlas. 

 
Finalmente, podemos decir, junto con la historiadora Andrea Sánchez 
Quintanar, que la investigación del pasado, además de ser interesante 
por sí misma, es lo que nos permite conocer la vida actual, pues lo que 
ha sucedido integra todo lo que estamos viviendo hoy. Pierre Vilar 
escribió que el estudio de la historia sirve para leer bien el 
periódico y entender, con ello, lo que está sucediendo 
(Zermeño~1994, pp.~\mbox{20--21}).


\medskip
\textbf{El Taller de historia. Proyecto curricular\\ de ciencias 
sociales}

 
El taller de historia y su proyecto curricular de ciencias sociales 
incorpora un método activo, estrechamente vinculado con la reforma 
educativa española, respecto de la cual en México no estamos nada 
lejanos, en la que los promotores del \textit{Grupo~13--16} se hallan 
individualmente implicados. El proyecto se conforma con dos libros. Uno 
es la guía didáctica, y otro es un estuche que contiene 14 cuadernillos 
para las actividades de  los alumnos, de los cuales cada uno 
corresponde a una unidad de estudio y ninguno pasa de las veinte 
páginas de contenidos.

 
La guía didáctica es para la aplicación del método por el profesor, 
ésta aborda la utilización de cada una de las unidades didácticas del 
alumno, describiéndolas, determinando su objetivo, contenido, 
estrategia a seguir y criterios de evaluación, así como una propuesta 
de bibliografía específica.

 
De sus propósitos nos dice que 

\begin{quotation}
Sus objetivos fundamentales son 
colaborar en la formación intelectual de los adolescentes, aportando 
los valores que proporciona el conocimiento de la Historia, primando 
aquellos objetivos específicos que desarrollan capacidades para 
identificar los valores que rigen el funcionamiento de las sociedades, 
manifiestan actitudes de tolerancia y respeto y abordan la resolución 
de los problemas desde una perspectiva crítica (Camino~1990\textsuperscript{a}, p.~114).
\end{quotation} 
 
Respecto de cómo se integran las unidades de estudio, exponen que 

\begin{quotation}
\begin{sloppypar}
Cada unidad didáctica integra contenidos con\-cep\-tua\-les, 
pro\-ce\-di\-men\-ta\-les y actitudinales, entendiéndolos como una 
totalidad cuyo aprendizaje se realiza a través de las diferentes 
actividades. Son instrumentos de análisis las aportaciones de la 
arqueología, la antropología, la demografía y la 
sociología ({\itshape ibid.}).
\end{sloppypar}
\end{quotation}

\enlargethispage{1\baselineskip} 
En cuanto al rol del docente en el nuevo plan de estudios, éste se ha 
ampliado a más actividades, tales como son la Investigación (líneas de 
generación y aplicación de conocimientos), la docencia, el trabajo en redes 
locales, nacionales e internacionales a través de cuerpos académicos, 
la asesoría académica, la asesoría de tesis, la tutoría, la gestión académica, 
la vinculación con los sectores productivos y la sociedad en general, 
difusión y la participación en seminarios o congresos 
especializados (UAGro~1995, pp.~5--10). En ese contexto, el citado 
proyecto curricular, adaptado a nuestra realidad histórica y social, 
viene a constituir un valioso auxiliar para los cursos de introducción 
al trabajo del historiador con situaciones dadas por el propio 
acontecer histórico regional, nacional e internacional, e incluso para 
adaptarse a los cursos de bachillerato, presentando los contenidos 
históricos de manera problematizada. 

 
Los autores del proyecto consideraron las cargas de trabajo de los 
profesores(as) en ocasiones excesivas, puesto que 

\begin{quotation}
Somos conscientes de 
que el profesorado debe preparar y saber muchas cosas a la vez. Por 
ello, cada unidad ha sido planteada con cierto detalle, incorporando 
aquella información básica que el profesor necesitará tener a mano para 
trabajar con sus alumnos. La bibliografía que se adjunta es sólo para 
aquellos que crean que necesitan más información sobre el 
tema (Camino 1990$^b$, p. 11).
\end{quotation}
 
Pongamos por ejemplo el caso más destacado del taller, se trata de la 
primera unidad, denominada {\itshape La extraña muerte de Marta}. En ella 
se plantea la muerte de una joven y el papel que un detective ---que es como 
debe proceder el historiador--- debe desempeñar para aclarar las causas del 
deceso, así como la identificación del o los responsables. Los tópicos 
en la guía son: Descripción de la unidad; Objetivos; Contenidos; 
Estrategias y Evaluación.
 
El estuche del Taller contiene un cuadernillo en el que los contenidos 
son: 1) la presentación del caso; 2) Guía de investigación y 3) 
relación de los documentos que llevaba Marta en su bolso.

%\enlargethispage{1\baselineskip} 
En cuanto a la descripción de la unidad, en esta parte se presentan las 
indicaciones para el profesor para esta primera unidad del método. Se trata 
de una introducción al curso de Historia. Consta de un {\itshape dossier} en 
el que se incluye la guía de investigación y materiales sobre un caso ficticio, 
supuestamente ocurrido en Madrid. 
%\enlargethispage{-1\baselineskip}
 
Los materiales de trabajo del alumnado se hallan en el interior de un 
sobre. El documento que plantea el caso es un informe policial 
sobre una chica llamada Marta. Junto a este detallado informe se 
adjunta una serie de documentos, que se supone estaban en el bolso de 
la muchacha. 

Los documentos son los siguientes: 
\begin{Obs}
\item[1.] Una credencial de lectora del Instituto
Alemán. 
\item[2.] Una fotografía de un grupo de jóvenes. 
\item[3.] Un cartel del grupo <<Aire
Libre>>. 
\item[4.] Una pegatina <<Yo esquío>>.  
\item[5.] Una tarjeta de circulación de RENFE
(bono de tren). 
\item[6.] Un recorte de periódico referente a la Filmoteca Nacional.
\item[7.] Recorte de periódico con un anuncio de <<\textit{Remey Magic}>>. 
\item[8.] Una carta de Juan. 
\item[9.] Una ficha de la Hermandad de Donantes de Sangre. 
\item[10.] Un recorte de
un periódico sobre el Día Internacional de la Mujer. 
\item[11.] Documento Nacional de
Identidad de Marta Sanz Martín. 
\item[12.] Carné (credencial) de la Universidad
Autónoma de Madrid. 
\item[13.] Carné (credencial) de deporte universitario. 
\item[14.] Hoja de un calendario. 
\item[15.] Tarjeta de autobús. 
\item[16.] Entradas para un concierto. 
\item[17.] Una carta de Marta. 
\item[18.] La foto de un chico no identificado. 
\item[19.] Recorte de un
periódico sobre el Festival de Teatro de Madrid. 
\item[20.] Recorte de un periódico
alemán. 
\item[21.] Foto de \textit{Hans}. 
\item[22.] Hoja de un diario íntimo. 
\item[23.] Trozo de papel con un número de teléfono.
\end{Obs}
 
Tres son los objetivos de la unidad introductoria: 
\begin{Obs}
\item[a)] Que los 
alumnos(as) lleguen a conocer el concepto de prueba-fuente, como 
elemento que proporciona información para la investigación. 
\item[b)] Que 
adquieran experiencia en el análisis e interpretación de las pruebas 
para, a partir de ellas, llegar a identificar un personaje o narrar una 
posible evolución de los hechos. 
\item[c)] Que aprendan a utilizar la 
capacidad de inferencia para formular hipótesis basadas en pruebas y 
que vean las lagunas que aparecen y que impiden, en ocasiones, llegar a 
conclusiones definitivas.
\end{Obs}
 
En esta unidad no se formulan objetivos conceptuales o de contenidos 
históricos, porque no los hay. El caso es meramente detectivesco y 
sirve como introducción a las unidades siguientes. Sin embargo, si 
existen contenidos de tipo procedimental y actitudinal, ya que se 
pretende, desde el comienzo del curso, que el alumno(a) sepa ordenar 
correctamente una información e intente formular hipótesis sobre 
cualquier caso que se le presente, así como que acepte opiniones 
contrarias a las suyas.

 
Igualmente, debemos tener en cuenta que esta unidad es importante para 
iniciar a los alumnos(as) en la técnica del debate.

 
Las estrategias consisten en que el profesor(a) expondrá brevemente en 
que consiste el trabajo de los historiadores. Se destacará la tarea de 
búsqueda y recogida de fuentes, y se acompañará esta actividad con las 
que realiza un detective para resolver un caso cualquiera.

 
Después, se propondrá a los alumnos(as) investigar un caso ficticio, 
como si ellos fueran los detectives encargados de su solución. Para 
ello, deberá procederse según las pautas marcadas en la guía de 
investigación: primero harán la lectura del informe policial en el que 
se detallan los hechos. 

 
Posteriormente a la realización de la lectura, irán surgiendo las 
preguntas: ¿Quién era esta chica?, ¿Con qué tipo de gente se 
relacionaba?, ¿Qué hizo durante las horas que antecedieron a su muerte? 
¿Cuál ha sido la causa de esa muerte?, etcétera. Si se quiere proceder 
ordenadamente, hay que seguir el cuestionario de la Guía de 
Investigación. Para contestar el mismo, será necesario utilizar todos 
los documentos que se adjuntan en el sobre, y ello dará ocasión a 
formular hipótesis muy diversas.

Luego, la presentación y argumentación de las hipótesis es el 
paso siguiente. Aquí es muy importante que cada alumno(a) o grupo 
explique cómo enlaza todos los documentos, qué razones tiene para 
formular su hipótesis, etc. El papel del profesor es el de moderador, 
que hace respetar los turnos de intervención, regula el tiempo de 
exposición, evita que se hable a la vez y, sobre todo, pone de 
manifiesto la insuficiencia de pruebas, las contradicciones, los 
documentos que no encajan, etc.

Hay que tener presente que el caso no tiene previsto un desenlace. No 
se trata de hallar la solución. Se trata de aprender cómo se debe 
trabajar en la clase de historia.

 
En lo referente a la evaluación, en esta unidad lo que se puede evaluar 
es: 1) La capacidad de explicar por escrito la hipótesis que el 
alumno-a considera más lógica. Se trata de evaluar la expresión 
escrita, necesariamente importante para alcanzar resultados aceptables 
durante el curso; 2) La capacidad de observación del alumno(a) ya que la 
unidad está llena de pequeños detalles (tachaduras o arrepentimientos 
en una carta de Marta, fecha de los diversos documentos, etc.);\linebreak (3) La 
capacidad de discusión: sostener con argumentos una hipótesis y 
aceptar la del contrario cuando hay pruebas 
evidentes (Camino~1990$^{b}$, pp.~\mbox{13--15}).

 
Hemos examinado una unidad en la que se destaca la labor tendiente a la 
formación intelectual de los adolescentes, aportando los valores que 
proporciona el conocimiento de la Historia, enfatizando el trabajo 
detectivesco de los alumnos, donde lo importante no son los contenidos 
históricos, sino el método de investigación de los historiadores. Ahora 
reseñaremos una unidad en la que priman ambos procedimientos, el 
detectivesco y los conocimientos del acontecer histórico.
\enlargethispage{1\baselineskip}

La unidad dedicada a las causas y motivos de los descubrimientos 
geográficos, y tiene dos partes claramente diferenciadas. En la primera 
se presentan una serie de fuentes (periódicos, fragmentos de 
memorias, fotografías, informes, etc.) relacionadas con los viajes 
interplanetarios del siglo XX, mientras que en la segunda hay fuentes 
relativas a los viajes de los descubridores españoles y portugueses de 
la época moderna, correspondiente a los siglos XV y XVI\@. Ambas partes 
van precedidas por una introducción y preguntas destinadas a los 
alumnos.

 
Los objetivos que se persiguen con los educandos son que logren 
comprender la variedad de motivos que las personas podemos tener 
para realizar nuestras acciones, y, al mismo tiempo, ayudarles a 
comprender la variedad de causas, algunas deseadas o provocadas 
conscientemente, otras dictadas por las circunstancias o por 
accidentes, que actúan en cualquier situación histórica.

 
Otro objetivo pretende que el estudiante tenga experiencias de 
formulación de preguntas, y de esta forma estimular la investigación 
sobre causación y motivación de la Historia.

\begin{sloppypar} 
Un último objetivo se propone comparar situaciones análogas de épocas 
muy diferentes y deducir sus semejanzas  y  diferencias \mbox{(Camino~1990\textsuperscript{a}, p.~69)}.
\end{sloppypar}

 
En lo que respecta a los contenidos, esta unidad tiene unos contenidos 
conceptuales muy concretos: la comprensión de un hecho histórico, en el 
caso que nos ocupa son los descubrimientos de la era espacial y los de 
los navegantes del siglo XV, esto implica saber diferenciar entre las 
causas que lo han hecho posible y los motivos que tenemos las personas 
para involucrarnos en él.

 
A través de fuentes primarias, se trata de investigar las causas y los 
motivos del descubrimiento de América, efectuados por los españoles y 
los portugueses, así como de las dos grandes potencias del siglo XX, la 
URSS y los Estados Unidos, en relación con la conquista del espacio. En 
definitiva, se trata de empezar a mirar cómo y por qué la civilización 
europea se expandió en el pasado, y las razones que subyacen en los 
descubrimientos espaciales del siglo XX\@.

 
Los contenidos procedimentales son más difíciles de precisar, ya que 
lo que se pretende es que el estudiante, ante un hecho, sepa\linebreak interrogar 
correctamente. Evidentemente, la metodología no puede concentrarse en un 
rígida normativa de preguntas y respuestas, ya que ni las preguntas 
pueden se preestablecidas ni tampoco las respuestas son siempre 
posibles.

 
Los contenidos actitudinales se hallan en relación con el interés por 
descubrir las conexiones causales y la valoración de la importancia de 
los motivos personales en los acontecimientos del pasado y del 
presente.

 
Finalmente, diremos que la estrategia prevé que sea el profesor el que 
inicialmente aclare cómo se van a formular las preguntas que suele 
hacer un historiador(a) ante un acontecimiento del pasado. Ejemplo: 
¿Por qué el ser humano se puede interesar por los viajes espaciales? 
¿Por qué no se han desarrollado hasta la segunda mitad del siglo XX y 
no antes? ¿Es que no les interesaba el espacio a finales del siglo XIX\@?
\enlargethispage{1\baselineskip}
 
Considerando que ese hecho histórico es muy complejo, deberá 
poner en antecedentes a los alumnos sobre la Historia de esta carrera 
espacial, es recomendable que lo haga con diapositivas, así el 
conocimiento será mejor. También, deberá estimular la búsqueda de 
información en hemerotecas. Para facilitar el trabajo de los alumnos, puede 
proporcionar una cronología de los acontecimientos más significativos 
como punto de partida de los estudiantes a la hora de comenzar la
investigación (\textit{ibid.}, pp.~69--76).
%\newpage

\bigskip
\textbf{Conclusión}

Se trata de explicar que aprender historia no constituye un ejercicio de la
imaginación, memorista o de lectura de una historia que se ofrece acabada,
porque se propone hacer, en este caso concreto, de una forma elemental, lo que
realizan los historiadores. Ellos se proponen, durante el curso académico,
acercarse lo más posible al método con el que trabajan los investigadores de la
historia, a fin de llegar a comprender que el conocimiento del pasado se halla aún
en construcción.

\bigskip 
\textbf{Referencias}
\enlargethispage{1\baselineskip}

\medskip
Camino García, María {\itshape et al\@.} (1990\textsuperscript{a}), 
\textit{Proyecto didáctico Quirón. Taller de 
historia. Proyecto curricular de ciencias sociales. Guía didáctica para 
el profesor}. \textit{Grupo 13--16}, Madrid, Ediciones de la Torre. 
 
\_\_\_\_\_\_ (1990\textsuperscript{b}),
\textit{Taller de historia. Proyecto curricular de ciencias sociales. 
Guía didáctica para el alumno. Grupo 13--16}, Madrid, Ediciones de la 
Torre.

Carretero, Mario (1995), \textit{Construir y 
enseñar las ciencias sociales y la historia}, Buenos Aires, Aique. 

Corral, José Luis {\itshape et al\@.} (2006), \textit{Taller de historia. El oficio que amamos}, España, Ensayo Edhasa.

Delors, Jacques {\itshape et al\@.} (1997), \textit{La educación encierra un tesoro}, México, UNESCO\@.

\begin{sloppypar}
Díaz Barrado, Mario P. (ed.) (1996), 
\textit{Imagen e historia}, Madrid,\linebreak Ayer\slash{}Marcial Pons.
\end{sloppypar}
 
Nava, José, <<No todos saben la importancia de la historia>>, \textit{El Financiero}, México, 18 de junio de 2003, p. 24.

Friera Suárez, Florencio (1995), \textit{Didáctica de las Ciencias 
Sociales. Geografía e Historia}, Madrid, Ediciones de la Torre.
 
UAGro. (2005), <<Reforma a los planes de estudio>>, \textit{Gaceta 
Universitaria. Órgano informativo el H. Consejo Universitario de la 
UAGro}, año 6, No. 12, febrero.

González, Isaac (2001), \textit{Una didáctica de la Historia}, Madrid, Ediciones de la Torre.

González Simancas, Jaspe Luis y Carbajo 
López, Fernando (2005), \textit{Tres principios de la acción educativa}, España, EUNSA\@.

García González, Magdalena {\itshape et al\@.} (1997), 
\textit{Investigación histórica. Manual para la enseñanza de la 
Historia de la Filosofía. (Para acompañar a luces y sombras)}, Madrid, 
Ediciones de la Torre.

<<Procedimientos en historia>>. Secuenciación y enseñanza (1994), Los procedimientos en 
Historia número 1\@, Madrid, Número 1, año 1, julio.
 
Ortiz de Ortuño, José María (ed.) (1998), \textit{Historia y sistema 
educativo}, Madrid,  Ayer\slash\ Marcial Pons.

Pluckrose, Henri (1993), \textit{Enseñanza y aprendizaje de la 
historia}, Madrid, Ministerio de Educación y Ciencia/ Ediciones Morata, 
S.L.

Sánchez Quintanar, Andrea (2002), \textit{Reencuentro con la historia. 
Teoría y praxis de su enseñanza en México}, México, UNAM\slash{}Facultad de 
Filosofía y Letras. 

Tusell, Javier (1993), \textit{Profesiones la historia}, Madrid, Acento 
editorial.

Universidad Autónoma de Guerrero\slash{}Facultad de 
Filosofía y Letras (1995), \textit{Reforma al plan de estudios de la 
Licenciatura en Historia. Aprobado por el Consejo Técnico el 17 de mayo de 1995 
y por el Consejo Universitario el día 15 de noviembre de 1996}, Chilpancingo, mimeo, 
s\slash{}f.

Zermeño Padilla, Guillermo (comp.) (1994), 
\textit{Pensar la historia. Introducción a la 
teoría y metodología de la historia en el siglo XX}, México,\linebreak 
UIA\slash{}Departamento de Historia.

%\documentclass{article}
%\usepackage{amsmath,amssymb,amsfonts}
%\usepackage{fontspec}
%\usepackage{xunicode}
%\usepackage{xltxtra}
%\usepackage{polyglossia}
%\setdefaultlanguage{spanish}
%\usepackage{color}
%\usepackage{array}
%\usepackage{supertabular}
%\usepackage{hhline}
%\usepackage{hyperref}
%\hypersetup{colorlinks=true, linkcolor=blue, citecolor=blue, filecolor=blue, urlcolor=blue}
% Text styles:
%\newcommand\textstyleappleconvertedspace[1]{#1}
%\makeatletter
%\newcommand\arraybslash{\let\\\@arraycr}
%\makeatother
%\setlength\tabcolsep{1mm}
%\renewcommand\arraystretch{1.3}
%\newtheorem{theorem}{Theorem}
%\title{}
%\author{Historia-soporte}
%\date{2014-06-24}

%\begin{document}

%%\clearpage\setcounter{page}{137}
\thispagestyle{empty}
\phantomsection{}
\addcontentsline{toc}{chapter}{La Enseñanza de la Historia: Una Modalidad\newline No Convencional\newline $\diamond$
\normalfont\textit{Wilfrido Llanes Espinoza y
Eduardo Frías Sarmiento}}
{\centering {\scshape \large La Enseñanza de la Historia: Una Modalidad\\ No Convencional}\par}
\markboth{la formación del historiador}{enseñanza de la historia}
\setcounter{footnote}{0}


\bigskip
\begin{center}
{\bfseries Wilfrido Llanes Espinoza \\
Eduardo Frías Sarmiento}\\
{\itshape\ Facultad de Historia, UAS\/}
\end{center}


\bigskip
{\bfseries Resumen}

La intención que anima este trabajo es la de explorar  y reconocer las 
áreas de oportunidad que significa poner en marcha modalidades 
alternativas o no convencionales en la enseñanza de la historia, como 
la que se propone ejecutar en la Licenciatura en Historia (UAS), así 
como las implicaciones de hacerlo. 

\medskip
{\bfseries Introducción}

En alusión al título de la disertación, vale señalar que cuando hacemos 
referencia al binomio enseñanza de la historia-modalidad no 
convencional, nos referimos concretamente al caso de la Licenciatura en 
Historia de la Universidad Autónoma de Sinaloa [en adelante UAS], por 
ser una experiencia vivida recientemente en nuestra Facultad.

Lo que se expone es pues la experiencia de implementar la nueva 
modalidad como parte del ensanchamiento de la oferta educativa a través 
de una estrategia de diversificación académica. Discutiremos 
especialmente cuatro aspectos fundamentales: 
\begin{Obs}
\item[1)] la justificación de la 
implementación de la modalidad no convencional o semiescolarizada; 
\item[2)] el esquema del modelo educativo; 
\item[3)] el diplomado «formación docente en 
modalidades no convencionales»; 
\item[4)] el uso de la tecnología\slash{}Internet 
en  el marco de la nueva modalidad. 
\end{Obs}

\medskip
{\bfseries Justificación del programa}
%\enlargethispage{1\baselineskip}

Ante la nueva realidad que se presenta en tiempos de cambio y reformas 
de carácter educativo que se viven al presente, la Facultad de Historia 
de la UAS resolvió formar parte de las mejoras a la calidad, 
pertinencia y equidad de los programas educativos y servicios de la 
universidad, esto en el marco del proyecto de consolidación educativa 2017 
de la UAS (Guerra Liera 2013).\footnote{Al respecto, véase el eje estratégico 1: Docencia. Calidad e innovación educativa, más especialmente el Objetivo estratégico 3, cuyo 
propósito es contribuir a la ampliación de la cobertura a partir de la 
consolidación de modalidades educativas no escolarizadas, se contempla 
y reconoce la importancia de vigorizar la oferta educativa a través de 
una estrategia de diversificación académica y ampliando la cobertura de 
la oferta educativa semiescolarizada.}

Las necesidades educativas de los diferentes grupos poblacionales de 
nuestro país ha ido en aumento, la demanda de instrucción universitaria 
cada vez es mayor. Ante este escenario, particularmente hablando del 
área de las ciencias sociales, el caso de la historia resulta 
ilustrativo.

Sobre este punto la educación abierta y a distancia se ha planteado en 
uno de sus enfoques como una alternativa en la búsqueda de la 
democratización y socialización del conocimiento. Concretamente, la 
educación superior no está al margen de ello, en la UAS se ha empezado 
a alternar con su oferta educativa el diseño e implementación de 
modalidades no convencionales de educación, como parte de una respuesta 
concreta a la equidad formativa de grupos determinados de la población.

Son pioneros al interior de la UAS, las áreas de la 
Educación y Humanidades y las Ciencias Sociales y 
Administrativas. La Facultad de Historia se integra a este grupo 
partiendo del interés existente en algunos sectores 
de la población, especialmente en profesores del área de las ciencias 
sociales y humanidades en general, por cursar la Licenciatura en esta 
modalidad, opción que resuelve a los interesados la problemática que 
representa acudir de forma continua a las aulas.

De este modo la modalidad semiescolarizada, amplía su 
viabilidad por ser incluyente con un mercado académico que ha quedado 
sin la posibilidad de integrarse al estudio de la historia, por las 
distintas actividades que desarrollan, puesto que son los interesados, 
sobre todo, educadores en algún otro nivel o personas formadas en 
disciplinas diferentes. 


\bigskip
{\bfseries El esquema del modelo educativo}
%\enlargethispage{-1\baselineskip}

Según el modelo pedagógico y de aprendizaje basado en el modelo 
curricular por competencias profesionales e implementado en el nuevo 
Plan de Estudios de la Licenciatura en Historia de la Universidad 
Autónoma de Sinaloa ---nos referimos al escolarizado---, es necesario 
concretar un  modelo pedagógico y de aprendizaje distinto al 
tradicional; el modelo centra su interés en el aprendizaje, con la 
intención de promover la capacidad de las y los alumnos para gestionar 
sus propios aprendizajes, acrecentar sus niveles de autonomía en su 
carrera académica, así como para disponer de herramientas intelectuales 
y sociales que les permitan aprender y desaprender continuamente a lo 
largo de su vida (Delors 1996; Pozo 1999; Savater 1997), ya que el 
cambio continuo de los contextos y necesidades de la profesión, 
requiere de profesionales capaces de aprender nuevas competencias\linebreak y de 
«desaprender» las que eventualmente se vuelvan obsoletas, esto 
significa que las y los estudiantes deben aprender a identificar y 
manejar la emergencia de nuevas competencias y mantener apertura para 
actualizarse.  

La implementación de la Licenciatura en Historia en la modalidad 
semiescolarizada implica que los estudiantes acudirán a sesiones 
presenciales únicamente el día sábado, continuando su formación de 
forma complementaria, haciendo uso de las diversas plataformas de 
en\-se\-ñan\-za-apren\-di\-za\-je virtual. 

La modalidad semipresencial o semiescolarizada funciona bien cuando no 
es posible cumplir a cabalidad con la modalidad presencial. Lo 
presencial en esta modalidad se ha planeado para que el docente utilice 
los medios y estrategias que permitirán al participante desarrollar con 
éxito la fase complementaria a la presencial.  

Bajo este modelo los medios didácticos se tornan en facilitadores y 
mediadores del conocimiento para el aprendizaje. Las acciones 
presenciales estarán dadas por los niveles de participación e 
interacción que surgen cuando el docente y participante interaccionen 
en las sesiones presenciales, donde el docente se valdrá de los medios 
para motivar al estudiante a una mayor apropiación o reconstrucción del 
conocimiento.

Sabedores de que los problemas educativos no se resuelven incorporando 
más tecnologías en el aula, no estamos negados a entender que estas 
proporcionan un sin número de oportunidades para desarrollar nuevos 
métodos de enseñanza, basados en lo que ya se ha investigado sobre cómo 
se produce la comprensión y el aprendizaje.
\newpage

%\medskip
{\bfseries El diplomado}

Para ello se implementó un diplomado con la intención de introducir y 
actualizar a la planta docente de la Facultad de Historia en los 
requerimientos propios de la nueva modalidad.

Sobre el Diplomado «Formación Docente en Modalidades no 
Convencionales», vale destacar que a través de su estructura se puede 
observar el soporte de la modalidad no convencional de la Licenciatura 
en Historia. 
%\enlargethispage{-1\baselineskip}
%\newpage 

\textsl{Los objetivos específicos del Programa Académico fueron formar 
cuadros de facilitadores-asesores capaces de:}
%\enlargethispage{1\baselineskip}

%\smallskip 
\begin{Obs} 
\item[$\rhd$] Ofrecer a los participantes los 
elementos teórico-metodológicos que le permitan resolver la 
problemática del aula y proponer alternativas de solución. 
\item[$\rhd$] Impulsar la superación profesional de los docentes involucrados en la 
tarea de la educación, el intercambio de experiencias relativas a la 
práctica docente lo que le permitirá cuestionar su cotidianidad en el aula 
con la finalidad de que el usuario del diplomado proponga estrategias 
didácticas en su espacio de trabajo más acordes a la sociedad del 
conocimiento y al momento actual en que vivimos. 
\item[$\rhd$] Ofrecer a los 
usuarios del diplomado conocimientos teóricos sobre el desarrollo del 
aprendizaje del alumno, para que este tenga una mejor comprensión del 
mismo a través de su conocimiento social, mejorando los procesos de 
construcción del conocimiento autónomo. 
\item[$\rhd$] Proporcionar a los 
participantes estrategias de aprendizaje educativas, pedagógicas y 
tecnológicas, así como, las técnicas didácticas vivenciales de estudio, 
que sean un medio para acceder de una manera más dinámica a los 
aprendizajes a fin de que reconceptualicen su práctica docente como un 
espacio de reflexión y creatividad permanente. 
\item[$\rhd$] Utilizar y 
propiciar el uso de ambientes de aprendizaje, para estudiantes y 
docentes, y que  reconozcan  los roles de los diferentes actores del 
proceso educativo de un curso. 
\end{Obs}
%\newpage

Todo esto con la intención de que el docente esté actualizado 
en el conocimiento del modelo socio-formativo y enfoque por 
competencias, en el uso de las técnicas didácticas y estrategias de 
estudio, y de que cuente con los conocimientos necesarios para el 
diseño de paquete didáctico para modalidades no convencionales 
(Véase el Anexo~I).

El docente que egresó del Diplomado confirmó la importancia del alto 
sentido ético; la implementación variada de estrategias de intervención 
en el aula; de crear una planeación educativa y de realizar un paquete 
didáctico adecuado para el aprendizaje propicio en la construcción de 
conocimientos escolares, mediados por herramientas tecnológicas, a 
nivel individual y colectivo.

En parte, el propósito del diplomado fue hacer de nuestro conocimiento 
la importancia de tener en cuenta tres componentes claves: que el 
modelo socio formativo y enfoque se basa en el modelo por competencias; 
la importancia que revisten las técnicas didácticas y estrategias de 
estudio para esta modalidad y, finalmente, la importancia de innovar en 
el diseño de paquetes didácticos para modalidades no convencionales.
\enlargethispage{1\baselineskip}

El Diplomado tuvo como objetivo sustancial buscar que los profesores 
reconocieran las capacidades adecuadas para la implementación de la 
nueva modalidad. Las y los profesores deberán ser capaces de:
\begin{Obs}
\item[$\bullet$] Utilizar adecuadamente la tecnología de vanguardia y aplicarla en el
ámbito educativo.
\item[$\bullet$] Diseñar e implementar planes de clase, programas de curso y materiales de
instrucción de diferentes niveles educacionales y  modalidades.
\item[$\bullet$] Utilizar efectivamente las habilidades docentes, las estrategias de
enseñanza y de aprendizaje ya sea para el trabajo en el aula o mediadas por la
tecnología.
\enlargethispage{1\baselineskip}
\item[$\bullet$] Favorables para el uso intensivo y adecuado de la tecnología educativa y
la renovación constante.
\item[$\bullet$] Humanista, de equidad, democrática y participativa. 
\item[$\bullet$] Gestionar, organizar y coordinar equipos de trabajo.
\item[$\bullet$] Diseñar, implementar y administrar redes de conocimiento y sistemas de
información aglutinando actores diversos.
\item[$\bullet$] Facilitar el conocimiento.  
\end{Obs}

\textsl{Como se puede observar, las
capacidades no son pocas, como tampoco los requerimientos que componen el
perfil del docente, quien deberá ser un académico capaz de:}

\begin{Obs}
\item[$\star$] Acreditar por lo menos el grado de especialidad.
\item[$\star$] Demostrar formación y\slash{}o experiencia en el área del conocimiento a tratar y
dominar en forma amplia la asignatura que imparte y otras relacionadas. 
\item[$\star$]  Aptitud y disposición para la enseñanza personalizada a través de
asesorías semipresenciales y\slash\ o virtuales.
\item[$\star$]  Ser accesible y flexible con sus alumnos. 
\item[$\star$]  Estar capacitado pedagógicamente para ayudar a los alumnos a diagnosticar,
resolver y evaluar sus tareas académicas y apoyarlos en la planeación de las
medidas correctivas apropiadas.
\item[$\star$]  Tener una actitud positiva, flexible y abierta al cambio. 
\item[$\star$]  Ser objetivo en su trabajo de orientación, dirección y enseñanza. 
\item[$\star$]  Procurar ser sistemático, organizado y metódico, a fin de afrontar
cualquier situación académica.
\item[$\star$]  Practicar la investigación educativa permanente.
\item[$\star$]  Conocer los servicios que la UAS y otras IES ponen  a disposición de los
alumnos usuarios, a fin de poder remitirlos con precisión cuando así lo
requieran.
\item[$\star$]  Actualizar permanentemente sus conocimientos.
\end{Obs}

\bigskip
{\bfseries Internet y el nuevo modelo}
\enlargethispage{1\baselineskip}

Sin lugar a dudas, la aplicación de la tecnología ha ensanchado los 
horizontes en diversos sentidos y ha repercutido en las diversas esferas 
sociales; la educación no es la excepción, pues es un campo fértil para 
implementar el uso de la Tecnología de la Información y Comunicaciones (TIC).\footnote{Véase:  Abril Herrera Chávez y Karla Valverde Viesca, «El uso de las tecnologías de la información y comunicaciones (TIC’S) como un nuevo elemento de análisis social. La Internet como rizoma social», en Cristina Puga Espinoza (coord.), {\itshape Formación en Ciencias Sociales en México. Una mirada desde las universidades del país}, México, Asociación para la Acreditación y la Certificación en Ciencias Sociales, A.C., 2008, pp. 131--115.}

\begin{quotation}
El sistema educativo, una de las instituciones sociales por 
excelencia, se encuentra inmerso en un proceso de cambios, enmarcados 
en el conjunto de transformaciones  sociales propiciadas por la 
innovación tecnológica y, sobre todo, por el desarrollo de las 
tecnologías de la información y de la comunicación, por los cambios en 
las relaciones sociales y por una nueva concepción de las relaciones 
tecnología-sociedad que determinan las relaciones tecnología-educación. 
Cada época ha tenido sus propias instituciones educativas, adaptando 
los procesos educativos a las circunstancias. En la actualidad esta 
adaptación supone cambios en los modelos educativos, cambios en los 
usuarios de la formación y cambios en los escenarios donde ocurre el 
aprendizaje (Salinas, 1997).
\end{quotation}

Las nuevas tecnologías, en unión con las condiciones sociales y 
laborales del Siglo XXI, hacen necesario realizar grandes cambios en la 
educación y la docencia, donde ya usamos términos como Educación 2.0 y Docente 
3.0 para referirnos a una educación centrada en el estudiante, con el 
profesor como guía, enfatizando el trabajo colaborativo y el desarrollo de 
proyectos, aprovechando redes sociales y tecnología móvil. Se presenta 
como imprescindible que los estudiantes dominen «nuevas» habilidades y 
competencias para poder llegar a ser ciudadanos útiles en esta nueva 
sociedad.

La estrategia que busca aterrizar este objetivo es: «consolidar y 
diversificar la oferta educativa del programa UAS virtual y ampliar la 
cobertura de la oferta educativa semiescolarizada». Particularmente, en 
el caso de la Licenciatura en Historia en la modalidad no convencional 
o no escolarizada, ha resultado de gran interés promover la adaptación 
de nuevos ambientes de aprendizaje o entornos distintos de educación a 
la enseñanza de la historia.

Si bien se reconocen beneficios, en la implementación de esta nueva 
modalidad también se presentan importantes retos, mismos que como parte 
del mismo proceso de enseñanza-aprendizaje tendremos que afrontar en esta 
nueva experiencia.
\newpage

{\bfseries Referencias}

Delors, Jacques (comp.) (1998), «Informe a la UNESCO de la Comisión Internacional sobre la Educación para el siglo XXI», \textit{Compendio. La educación encierra un tesoro}, España, Ediciones UNESCO\@. 

Guerra Liera, Juan Eulogio (2013), \textit{Plan de Desarrollo 
Institucional Consolidación 2017}, Culiacán, Universidad Autónoma de 
Sinaloa.  

Herrera Chávez Abril y Karla Valverde Viesca (2008) «El uso de las 
tecnologías de la información y comunicaciones (TIC’S) como un nuevo 
elemento de análisis social. La Internet como rizoma social», en 
Cristina Puga Espinoza (coord.), \textit{Formación en Ciencias Sociales 
en México. Una mirada desde las universidades del país}, México, 
Asociación para la Acreditación y la Certificación en Ciencias 
Sociales, A.C\@.

Pozo,Mauricio Ignacio (1999), \textit{Aprendices y Maestros}, Madrid, Alianza Editorial. 

Savater, Fernando (1997), \textit{El valor de educar}, Barcelona,
Ariel Ed\@. 

Salinas, J. (1997), «Nuevos ambientes de aprendizaje para una sociedad de la información», \textit{Revista Pensamiento Educativo}, 20, Santiago, Pontificia Universidad Católica de Chile, disponible en
\url{http://www.uib.es/depart/gte/ambientes.html}. 
\newpage

\begin{center}
\begin{scriptsize}
{\bfseries Anexo I. Programa del Diplomado en Formación Docente\\ en Modalidades No Convencionales.}
\end{scriptsize}
\end{center}

\begin{tiny}
\begin{center}
\tablefirsthead{}
\tablehead{}
\tabletail{}
\tablelasttail{}
\setlength{\extrarowheight}{4pt}
\begin{supertabular}{|llllllllll|}
\hline
\rowcolor{lsLightBlue}\multicolumn{4}{|l|}{\bfseries NOMBRE DEL DIPLOMADO} &
\multicolumn{6}{l|}{\bfseries Formación Docente en Modalidades No
Convencionales.}\\\hline
\rowcolor{lsLightGray}\multicolumn{4}{|l|}{\bfseries NOMBRE DEL MÓDULO:} &
\multicolumn{6}{l|}{\bfseries Modelo educativo y Enfoque por
Competencias.}\\\hline
\multicolumn{1}{|l|}{\bfseries MÓDULO:}&
\multicolumn{1}{l|}{PRIMERO}&
\multicolumn{1}{l|}{\bfseries HT:}&
\multicolumn{1}{l|}{16}&
\multicolumn{1}{l|}{\bfseries HI:}&
\multicolumn{1}{l|}{32}&
\multicolumn{1}{l|}{\bfseries HP:}&
\multicolumn{1}{l|}{~}&
\multicolumn{1}{l|}{\bfseries CRÉDITOS:}&
3\\\hline
\multicolumn{5}{|l|}{\bfseries TOTAL DE HORAS DEL MÓDULO:} &
\multicolumn{5}{l|}{48}\\\hline
\end{supertabular}
\end{center}
\end{tiny}

%\bigskip
\begin{tiny}
\begin{center}
\tablefirsthead{}
\tablehead{}
\tabletail{}
\tablelasttail{}
\setlength{\extrarowheight}{4pt}
\begin{supertabular}{|m{0.25\textwidth}m{0.58\textwidth}|}
\hline
\rowcolor{lsLightGray}\multicolumn{2}{|m{0.85\textwidth}|}{\centering{\bfseries PERFIL PROFESIONAL DEL DOCENTE (CONOCIMIENTOS, EXPERIENCIA Y FORMACIÓN)}}\\\hline

{\bfseries PERFIL DEL DOCENTE} &
Es un profesional que posee conocimientos en:

Educación, Pedagogía, Didáctica, Teorías del Aprendizaje, Andragogía,\par 
Psicología Educativa.\\\hline

{\bfseries EXPERIENCIA DOCENTE Y NIVEL DE FORMACIÓN PROFESIONAL}&

Es un profesional con formación en psicología educativa o pedagogía.\par Deberá 
haber laborado en contextos educativos a nivel medio superior\par y superior, con
una experiencia como docente mínima de dos años.

El nivel profesional que debe poseer el docente está en el grado\par de Licenciatura
o Postgrado.\\\hline

\rowcolor{lsLightGray}\multicolumn{2}{|p{0.85\textwidth}|}{\centering{\bfseries PROPÓSITO Y COMPETENCIA GENERAL DEL MÓDULO}}\\\hline

\multicolumn{2}{|p{0.85\textwidth}|}{{\bfseries Propósito:}

Conozca y valore el modelo de educación basado en competencias y sus
implicaciones educativas  al adoptarse en un esquema curricular de nivel
superior congruente con el contexto institucional actual y las políticas
educativas contemporáneas.}\\\hline  

\multicolumn{2}{|p{0.83\textwidth}|}{{\bfseries Competencia:}

Conoce el nuevo modelo de educación basado en competencias y lo aplica 
según necesidades de materiales educativos según su contexto educativo.}\\\hline
\end{supertabular}
\end{center}
\end{tiny}
%\newpage

%\bigskip
\begin{tiny}
\begin{center}
%\tablefirsthead{}
%\tablehead{}
%\tabletail{}
%\tablelasttail{}
\setlength{\extrarowheight}{4pt}
\begin{supertabular}{|m{13.5mm}m{32mm}m{18mm}m{3mm}m{3mm}m{3mm}m{3mm}m{3mm}m{3mm}m{3mm}|}
\hline
\rowcolor{lsLightBlue}\multicolumn{1}{|m{13.5mm}|}{{\centering\bfseries UNIDAD\par}
\centering{\bfseries I}} &
\multicolumn{1}{m{32mm}|}{Las competencias son la actuación\par eficiente en un
contexto determinado.} &
\multicolumn{1}{m{18mm}|}{\centering{\bfseries ASIGNACIÓN EN HORAS}} &
\multicolumn{1}{m{3mm}|}{HT} &
\multicolumn{1}{m{3mm}|}{\centering 5} &
\multicolumn{1}{m{3mm}|}{\centering HI} &
\multicolumn{2}{m{3mm}|}{\centering 10} &
\multicolumn{1}{m{3mm}|}{\centering HP} &
\multicolumn{1}{m{3mm}|}{\centering ~ }\\\hline
 &
 &
 &
\multicolumn{4}{|m{21mm}|}{\centering{\bfseries TOTAL DE HORAS:}} &
\multicolumn{3}{m{2mm}|}{\centering 15}\\\hline
\rowcolor{lsLightGray}\multicolumn{10}{|m{.85\textwidth}|}{\centering{\bfseries CONTENIDOS
TEMÁTICOS}}\\\hline
\multicolumn{10}{|p{83mm}|}{{\bfseries Programa Teórico:}

1.1.- En busca de una definición de competencia.

\quad 1.1.1.- Distintas definiciones del término «competencias».

\quad 1.1.2.- ¿Qué entendemos por competencia?

\quad 1.1.3.- El concepto de competencias: Un abordaje complejo.

1.2.- Proceso desarrollado en una actuación competente.

1.3.- Ser competente no es una cuestión de todo o nada.

1.4.- En la práctica.}\\\hline
\end{supertabular}
\end{center}

%\bigskip
\enlargethispage{2\baselineskip} 
\begin{center}
%\tablefirsthead{}
%\tablehead{}
%\tabletail{}
%\tablelasttail{}
\setlength{\extrarowheight}{2pt}
\begin{supertabular}{|m{13.5mm}m{32mm}m{18mm}m{3mm}m{3mm}m{3mm}m{3mm}m{3mm}m{3mm}m{3mm}|}
\hline
\rowcolor{lsLightBlue}\multicolumn{1}{|m{13.5mm}|}{{\centering\bfseries UNIDAD\par}
\centering{\bfseries II}} &
\multicolumn{1}{m{32mm}|}{El modelo de competencias;\par un enfoque
socioformativo.} &
\multicolumn{1}{m{18mm}|}{\centering{\bfseries ASIGNACIÓN EN HORAS}} &
\multicolumn{1}{m{3mm}|}{HT} &
\multicolumn{1}{m{3mm}|}{\centering 6} &
\multicolumn{1}{m{3mm}|}{HI} &
\multicolumn{2}{m{3mm}|}{\centering 12} &
\multicolumn{1}{m{3mm}|}{HP} &
~\\\hline
 &
 &
 &
\multicolumn{4}{m{21mm}|}{{\bfseries TOTAL DE HORAS:}} &
\multicolumn{3}{m{2mm}|}{\centering 18}\\\hline
\rowcolor{lsLightGray}\multicolumn{10}{|m{.85\textwidth}|}{\centering{\bfseries CONTENIDOS
TEMÁTICOS}}\\\hline
\multicolumn{10}{|p{83mm}|}{{\bfseries Programa Teórico:}

2.1.- El nuevo paradigma de las competencias.

2.2.- Principios comunes al modelo de competencias como nuevo paradigma
educativo.

2.3.- Enfoques actuales de las competencias. Análisis comparado.

2.4.- El enfoque socioformativo; El pensamiento sistémico-complejo en la
práctica educativa.

2.5.- Descripción y formación de una competencia desde el enfoque
socioformativo.

2.6.- Bases teóricas y filosóficas de la formación de las competencias.

\quad 2.6.1.- El humanismo como base de la socioformación y las competencias.

\qquad 2.6.1.1.- Concepción del hombre.

\qquad 2.6.1.2.- Concepción de sociedad.

\qquad 2.6.1.3.- Concepción de institución educativa.

\quad 2.6.2.- Bases constructivistas en la formación y evaluación de las
competencias.

\quad 2.6.3.- El aprendizaje cooperativo en las secuencias didácticas.

\quad 2.6.4.- Las secuencias didácticas y el papel de la enseñanza problémica.

\quad 2.6.5.- Bases en el aprendizaje significativo.

\quad 2.6.6.- El pensamiento complejo; Las competencias desde el proyecto ético de
vida.}\\\hline
%\end{supertabular}
%\end{flushleft}
%\bigskip
%\begin{flushleft}
%\tablefirsthead{}
%\tablehead{}
%\tabletail{}
%\tablelasttail{}
%\begin{supertabular}{|m{13.5mm}m{32mm}m{18mm}m{5mm}m{5mm}m{5mm}m{5mm}m{5mm}m{3mm}m{3mm}|}
%\hline
\rowcolor{lsLightBlue}\multicolumn{1}{|m{13.5mm}|}{{\centering\bfseries UNIDAD\par}
\centering{\bfseries III}} &
\multicolumn{1}{m{32mm}|}{Educación basada en competencias \textbf{(EBC)}.} &
\multicolumn{1}{m{18mm}|}{\centering{\bfseries ASIGNACIÓN EN HORAS}} &
\multicolumn{1}{m{3mm}|}{HT} &
\multicolumn{1}{m{3mm}|}{\centering 5} &
\multicolumn{1}{m{3mm}|}{\centering HI} &
\multicolumn{2}{m{3mm}|}{\centering 10} &
\multicolumn{1}{m{3mm}|}{\centering HP} &
~
\\\hline
 &
 &
 &
\multicolumn{4}{m{21mm}|}{\centering{\bfseries TOTAL DE HORAS:}} &
\multicolumn{3}{m{2mm}|}{\centering 15}\\\hline
\rowcolor{lsLightGray}\multicolumn{10}{|m{.85\textwidth}|}{\centering{\bfseries CONTENIDOS
TEMÁTICOS}}\\\hline
\multicolumn{10}{|p{83mm}|}{{\bfseries Programa Teórico:}\par
3.1.- Las competencias genéricas en el Proyecto DeSeCo de la OCDE.\par
3.2- Las competencias genéricas en el Proyecto Tuning.\par
3.3.- Las competencias genéricas en el Espacio Europeo de Educación Superior.\par
3.4.- Las competencias específicas.\par
3.5.- Las competencias profesionales.\par
3.6.- Las competencias en el mundo laboral y en el mundo educativo.\par
3.7.- La vinculación universidad-empresas.\par 
3.8.-  Fundamentos epistemológicos de la EBC.}\\\hline
\rowcolor{lsLightGray}\multicolumn{10}{|m{.85\textwidth}|}{\centering \bfseries ESTRATEGIA DIDÁCTICA SUGERIDA}\\\hline
\multicolumn{10}{|p{83mm}|}{Modalidad didáctica:
\begin{itemize}
\item Modelado.
\item Trabajo en equipos.
\item Cuestionamiento dirigido.
\item Consulta documental.
\item Trabajo cooperativo mediante acciones dirigidas.
\item Exposición.
\item Lectura comentada.
\item Evaluar conocimientos previos sobre la base de la técnica <<lluvia de
ideas>> o cuestionario dirigido para conocer los elementos de comunicación que
maneja el alumno.
\item Exponer las características y propiedades externas de los textos.
\item Organizar equipos de trabajo cooperativo para que se utilicen las
estrategias de comprensión lectora de un texto diferente en cada equipo.
\item Coordinar la evaluación formativa de productos y\slash\ o desempeño, con apoyo de
listas de cotejo y/o guías de observación en situaciones de auto y\slash\ o
co-evaluación.
\end{itemize}}\\\hline
\end{supertabular}
\end{center}
%\enlargethispage{4\baselineskip}
%\newpage

\begin{center}
%\tablefirsthead{}
%\tablehead{}
%\tabletail{}
%\tablelasttail{}
\setlength{\extrarowheight}{2.5pt}
\begin{supertabular}{|p{0.85\textwidth}|}
\hline
\rowcolor{lsLightGray}\centering\arraybslash{\bfseries ESTRATEGIA DE EVALUACIÓN SUGERIDA}\\\hline
\centering\arraybslash{\bfseries Sistema de Evaluación}\\\hline
\centering\arraybslash{\bfseries Criterios y Procedimientos de Evaluación y
Acreditación}\\\hline
{\bfseries Teoría y práctica:}

Responsabilidad \quad 15\%

Participación \quad 15\%

Trabajos parciales \quad 20\%

Trabajo final \quad 50\%

(Ensayo).

Nota: Para acreditar el curso se requiere un mínimo del 80\,\% de 
asistencias.\\ En consecuencia, la nota final obtenida por cada 
alumno es el resultado de la valoración del conjunto de actividades, 
evaluaciones, trabajos prácticos, etc.\ desarrollados a lo largo del 
Curso. \\\hline \rowcolor{lsLightGray}{\bfseries EVALUACIÓN DE LOS APRENDIZAJES DE LOS 
ALUMNOS:}\\\hline Evaluación Diagnóstica:

\begin{itemize}
\item Lluvia de ideas. 
\item Equipos de trabajo. 
\item Listas de cotejo.
\item Guías de observación en ejercicios de autoevaluación y/o coevaluación.
\end{itemize}
\\\hline
Evaluación Formativa:

\begin{itemize}
\item Debates.
\item Exposiciones. 
\item Temas relevantes.
\item Control de lecturas.
\item Mapas conceptuales.
\item Rúbrica.
\item Escalas valorativas.
\end{itemize}
\\\hline
Evaluación Sumaria:

\begin{itemize}
\item Productos.
\item Desempeños.
\item Informe de actividades o lecturas realizadas.
\item Ensayo.
\item Participación en discusión.
\item Portafolio de evidencias (Redacción de competencia)
\end{itemize}
\\\hline
\end{supertabular}
\end{center}


\begin{center}
%\tablefirsthead{}
%\tablehead{}
%\tabletail{}
%\tablelasttail{}
\setlength{\extrarowheight}{2.5pt}
\begin{supertabular}{|p{0.85\textwidth}|}
\hline
\rowcolor{lsLightGray}\centering\arraybslash{\bfseries MATERIALES Y RECURSOS}\\\hline

\begin{itemize}
\item Selección de ejemplos o modelos por competencias de textos funcionales
utilizados dentro del propio ámbito escolar: currículum vitae, cuadro
sinóptico, mapa conceptual, carta petición.
\item Cuestionarios para evaluación diagnóstica.
\item Guías de observación para evaluar desempeño y participaciones en equipo
que distingan la correspondiente de cada integrante.
\item Bibliografía básica y complementaria para consulta documental.
\item Organizador previo (Guía de lectura).
\item Mapas conceptuales.
\item Listas de cotejo, guías de observación para evaluar productos y desempeño.
\end{itemize}
\\\hline
\end{supertabular}
\end{center}

\bigskip
\begin{center}
%\tablefirsthead{}
%\tablehead{}
%\tabletail{}
%\tablelasttail{}
\setlength{\extrarowheight}{2.5pt}
\begin{supertabular}{|p{0.85\textwidth}|}
\hline
\rowcolor{lsLightGray}\centering\arraybslash{\bfseries BIBLIOGRAFÍA BÁSICA}\\\hline

1.- Bellocchio, Mabel, Educación Basada en Competencias y Constructivismo: Un
enfoque y un modelo para la formación pedagógica del siglo XXI\@. Editado por
ANUIES, Universidad de Colima y Universidad Autónoma de Ciudad Juárez, México, 2010.

2.- Tobón, Sergio et al\@. Secuencias didácticas: Aprendizaje y evaluación de competencias. Ed\@. PEARSON, México, 2010. (Capítulo 1 y 2). 

3.- Tobón, Sergio. Aspectos básicos de la formación basada en competencias. Talca: Proyecto Mesesup, 2006.  

4.- Zabala, Antoni y Arnau, Laia. 11 ideas clave; Cómo aprender y enseñar competencias. Ed\@. GRAÓ, Barcelona. 2008. (pp. 31--52). \\\hline

\rowcolor{lsLightGray}\centering\arraybslash{\bfseries BIBLIOGRAFÍA COMPLEMENTARIA}\\\hline

1.- Delors, Jacques. La educación; encierra un tesoro. Ediciones UNESCO, México, 2001.

2.- Gadotti, Moacir (colabs). Perspectivas actuales de la educación. Ed\@. Siglo XXI editores, Argentina, 2003.

3.- Ontoria A. et  al. \ Mapas conceptuales una técnica para aprender Ediciones Narsea, S. A., España, 1996.

4.- Perrenoud, Philippe. Diez nuevas competencias para enseñar. Ed\@. GRAÓ, Barcelona, 2007. 

5.- Trilla, J. et. \ al. \ El legado pedagógico del siglo XX para la escuela del siglo XXI\@. Ed\@. GRAÓ, España, 2005.\\\hline
\end{supertabular}
\end{center}
\end{tiny}


