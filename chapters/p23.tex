%\documentclass{article}
%\usepackage{amsmath,amssymb,amsfonts}
%\usepackage{fontspec}
%\usepackage{xunicode}
%\usepackage{xltxtra}
%\usepackage{polyglossia}
%\setdefaultlanguage{spanish}
%\usepackage{color}
%\usepackage{array}
%\usepackage{supertabular}
%\usepackage{hhline}
%\usepackage{hyperref}
%\hypersetup{colorlinks=true, linkcolor=blue, citecolor=blue, filecolor=blue, urlcolor=blue}
%% Text styles
%\newcommand\textstyleFootnoteCharacters[1]{\textsuperscript{#1}}
%\makeatletter
%\newcommand\arraybslash{\let\\\@arraycr}
%\makeatother
%\setlength\tabcolsep{1mm}
%\renewcommand\arraystretch{1.3}
%\newtheorem{theorem}{Theorem}
%\title{}
%\author{}
%\date{2014-08-19}
%\begin{document}
%%\clearpage\setcounter{page}{507}

\thispagestyle{empty}
\phantomsection{}
\addcontentsline{toc}{chapter}{Entre PROMEPsas y PROEDsas\newline $\diamond$
\normalfont\textit{Gloria Pedrero Nieto y Graciela Isabel Badía Muñoz}}
{\centering {\scshape \large Entre PROMEPsas y PROEDsas}\par}
\markboth{la formación del historiador}{promepsas y proedsas}
\setcounter{footnote}{0}

\bigskip
\begin{center}
{\bfseries Gloria Pedrero Nieto\\
Graciela Isabel Badía Muñoz}\\
{\itshape Universidad Autónoma del Estado de México}
\end{center}

\bigskip
\textbf{Resumen}


\bigskip
El programa de mejoramiento profesional y los programas al desempeño docente
 han tasado desigualmente las funciones sustantivas de las universidades
públicas y privadas bajo criterios ideales  muy distantes a las realidades
institucionales, se somete al profesor de educación media y superior a una
mal entendida cultura de la evaluación  en la cual debe mostrar habilidades
cada vez más complejas en todos los ámbitos de su función profesional, lo
que ha  generado expertos en el cumplimiento de los requerimientos
administrativos de los programas, en detrimento de la actividad docente. La
presente ponencia tiene por propósito analizar cómo la docencia es la
actividad de mayor impacto y es la menos valorada y remunerada en estos
programas afectando la calidad de la enseñanza de la historia.\\
{\bfseries Palabras claves}: Salario, estímulos económicos, profesores
universitarios.


\bigskip
\textbf{Abstract}

The program of professional development and programs for teaching
performance have been unequally assessed the substantive functions of
public and private universities under very distant institutional realities
ideal criteria, submits teacher and higher education to a misunderstood
culture of evaluation in which to display increasingly complex skills in
all areas of their professional role, which has led experts in meeting the
administrative requirements of the programs, in sharp detriment of
teaching. 

The purpose of this proposal is to analyze how teaching activity that is
more impact and is the least valued and remunerated in these programs
affecting the quality of the teaching of history.\\
{\bfseries Keywords:} Wage, economic stimulus, academics.

\bigskip
\textbf{A manera de Introducción}

Para poder acreditar un plan de estudios a nivel federal es requerido
cumplir con estándares acordes al tipo de profesión, entre ellos,  cada
año, cada dos o tres, uno de sus aspectos es que los profesores de las
universidades públicas seamos sometidos a evaluaciones académicas, llámense
PROED, PROMEP (ahora  PRODEP), SNI, a través de las cuales la institución 
adquiere  recursos y bienes, sin contar, el detrimento de la planta
docente, pues no se  considera  el estado de estrés, el cual nos dura
varios meses, entre la búsqueda de la información, la organización de la
misma, escanearla, vaciarla a los formatos de currículum (ahora en línea,
vía Internet), claro después de interpretar lo que se nos pide y no
equivocarnos en su clasificación, viene la larga espera del resultado de la
misma.


Pero antes de analizarlos hagamos algo de historia, el <<Programa de
estímulos al desempeño del personal docente de la UAEM>> surge a instancias
de la Secretaría de Educación Pública, cuando ésta convoca a las
universidades públicas del país para poner en marcha el  <<Programa de
Carrera Docente del Personal Académico el cual  tenía como base \textit{El
Plan Nacional para la Modernización Educativa}.
\newpage

En su primera convocatoria a la comunidad universitaria, junio de 1992, el
entonces rector M.\ en C.\ Efrén Rojas Dávila manifiesta que <<(\ldots) este
ejercicio nos dará  la oportunidad de iniciar una nueva cultura de
EVALUACIÓN ACADÉMICA,  pues será la primera ocasión en que se llevará a
cabo en la totalidad del personal académico>>. En los objetivos específicos
se vuelve  a mencionar el establecimiento de <<un sistema de evaluación
institucional con base en variables de permanencia, dedicación,
productividad, rendimiento, escolaridad, participación, profesionalización,
calidad y experiencia>>, tendientes a otorgar recursos económicos
adicionales al salario (UAEM 1992). 


En los documentos emanados en enero de 1993 y enero de 1994, en los
objetivos se cambia el término de evaluación por el de valorización, con lo
que se trata de dar una apreciación más cualitativa que cuantitativa al
proceso. Esto se mantiene hasta 1998, en el  año 1999, se vuelve a la idea
de evaluación, lo que responde más a la intención del programa que es el de
dar valor numérico a las actividades académicas, que dentro de la política
neoliberal permite competir y comparar.  


Otro cambio que llama nuestra atención fue la insistencia en la docencia, en
los  últimos programas, prueba de ello fue el cambio de nombre del programa
de carrera académica a carrera docente y ahora al de estímulos al personal
docente, la insistencia es tal que donde antes se hablaba de personal
académico, ahora se dice docente, pero si bien en el discurso se da el
cambio, no sucede lo mismo con el sistema de puntaje y por lo tanto en los
estímulos.


El programa está dirigido a todo el personal académico,  a partir de
entonces  llamado docente, pero el estímulo económico desde esa época ha
sido  diferencial, ya que a los profesores de asignatura, personal de
carrera de medio tiempo y técnicos académicos sólo podían aspirar, con la
máxima calificación a un 30\,\% sobre su sueldo base. El personal de carrera
de tiempo completo, con la máxima calificación llegaba a alcanzar más del
100\% de su sueldo base (UAEM 1998, pp. 91--93). 


En el programa actual el estímulo se calcula en salarios mínimos, los
profesores y técnicos de tiempo completo pueden aspirar a tener una
compensación  de uno a 14 salarios mínimos, con una puntuación que va de
195 a 650 puntos. Los profesores y técnicos académicos de medio tiempo sólo
puede aspirar a cinco salarios mínimos, con las mismas puntuaciones de los
tiempos completos, los de asignatura únicamente alcanzan cuatro salarios
mínimos, es decir para obtener el máximo tienen que tener los 650 puntos. 


 Lo anterior lo han  justificado en el supuesto de que los académicos de las
primeras categorías mencionadas dedican todo el tiempo a la universidad y
si este no era el caso,  se intentaba lograr su exclusividad. Lo que no se
ha tomado en cuenta, es que una gran parte de los profesores  de asignatura
y de medio tiempo, sobre todo en las carreras humanísticas, tienen como
única actividad y por lo tanto fuente de ingresos a la universidad.


La justificación para esto es la siguiente:


La financiación del Programa se efectuará con base en la disponibilidad
presupuestaria que otorguen, para este efecto, los Gobiernos Federal y
Estatal, la propia Institución y los recursos derivados de reducciones del
capítulo 1000, ---Servicios Personales---, conforme lo determine la SHCP, a
través de la Unidad de Política y Control Presupuestario. 


El personal de medio tiempo y asignatura podrá participar en el Programa
cuando la Universidad cuente con los recursos adicionales obtenidos por
reducciones al capítulo 1000, ---Servicios Personales---, por aportaciones del
Gobierno Estatal y por ingresos propios (H.\ Consejo Universitario 2001,
p.7). 


Para lograr los objetivos del programa se establece un sistema de puntaje
actualmente contiene tres criterios a evaluar:

\begin{Obs}
\item[1.] Puntaje de Calidad en el Desempeño a la Docencia
\item[1.] Puntaje de dedicación a la Docencia
\item[3.] Puntaje de Permanencia en las Actividades de la Docencia.
\end{Obs}

En realidad lo que se evalúa es el punto uno, los otros dos no cuentan en la
calificación pero  sí son necesarios para poder participar, los profesores
de tiempo completo para obtener los 250 puntos del rubro dos, tienen que
cubrir 12 horas semanales de cátedra frente a grupo . El rubro tres se
refiere a la antigüedad, los 100 puntos se obtienen por más de 10 años en
la universidad (10 puntos por año). 


Los rubros a evaluar de ese primer punto son:

\begin{Obs}
\item[1.1] Formación académica y experiencia profesional
\item[1.2] Docencia
\item[1.3] Generación y aplicación del conocimiento
\item[1.4] Tutoría académica
\item[1.5] Gestión académica y cuerpos colegiados
\end{Obs}


En el punto 1.1 únicamente se toma en cuenta la formación académica, que
cabe aclarar que es a partir de la maestría; a esto se añade los cursos de
actualización y los diplomados, todos con vigencia anual. El puntaje máximo
es de 120 puntos (18.46\,\%), o sea que hay que tener doctorado y  acudir a
un diplomado  más dos cursos de 25 horas o seis cursos; es decir, 150 horas.
Nota no se toman en cuenta  los cursos profesionalizantes, no vinculados
directamente al campo de trabajo  y los diplomados cada vez son más caros.


El más valorado (hasta 190 puntos, igual a 29.23\,\%), ya que estamos en una
universidad cuya función primordial es la docencia es el 1.2, curiosamente
es el más difícil, cuando menos en nuestra experiencia de alcanzar el
máximo, pues en él se valora la apreciación estudiantil, no siempre
objetiva, también se califica la elaboración de material didáctico, el cual
los profesores por lo general lo hacemos pero la acreditación del mismo es
muy difícil, pues corresponde a una comisión nombrada <<Comisión de Calidad
de Medios Educativos>> que cuando aprueba (Material didáctico, programación
pedagógica, antología, apuntes cuaderno de ejercicios y\slash{}o problemarios,
prácticas de laboratorio, monografías,  prontuarios, revisión
bibliográfica, Material audiovisual y material informático), únicamente
sirve para la evaluación anual. También en este apartado se califica la
traducción de libros, conferencias y teleconferencias por invitación
expresa de otra institución, cursos de actualización disciplinaria para
profesores, elaboración de exámenes departamentales, que en nuestra
facultad no existen, pues muchas de las unidades de aprendizaje cuando
mucho se dan en tres grupos. Elaboración o actualización de programas, lo
cual si lo hacemos y son evaluados por las áreas, academias y consejo
académico. También aquí se evalúa el conocimiento de una lengua el puntaje
es alto pero se tiene que contar con los certificados internacionales y su
vigencia salvo algunos casos tiene que ser renovada, de la misma manera se
califica la formación docente en lenguas. El último punto son los
reconocimientos académicos,  los cuales, en su mayoría tienen una vigencia
anual. 


El rubro 1.3 es el que nos valora como investigadores o creadores, en él se
califica  la participación en eventos académicos, las publicaciones, el
desarrollo de la investigación, la presentación de obras artísticas y de
producción dramatúrgica. También la investigación y el desarrollo tecnológico y la
participación en cuerpos académicos y en redes, así como la  función de
dirección de trabajos de titulación. En este renglón el tope máximo es de 130 puntos
(el 20\,\%).


El punto 1.4 corresponde a la tutoría académica, en este rubro se toma en
cuenta la permanencia como tutor de los alumnos de licenciatura, la
asesoría académica disciplinaria y en actividades externas. También se toma
en cuenta las tesis dirigidas, revisadas y los protocolos, el tope es de
140 puntos y corresponden al  21.53\,\%.


El 1.5 corresponde a la <<Gestión académica y cuerpos colegiados>>. Aquí se valoran las
actividades académico administrativas (coordinaciones de academia, órganos
colegiados), la organización de eventos,  la participación en el desarrollo
curricular  y las actividades de difusión. El puntaje máximo es de 70
puntos, o sea el 10.79\,\%.


Por cierto que los puntos que rebasan el límite de cada rubro son puntos
perdidos para el programa, aunque no lo sean para la experiencia del
profesor. 


El Perfil PROMEP, ahora PRODEP, está íntimamente relacionado con los
estímulos académicos, pues es valorado en el punto 1.2.5.3
<<Reconocimientos académicos>> con un puntuaje alto, equivalente a un capítulo
de libro. De la misma manera, para obtener el perfil es necesario participar en el
programa institucional. Lo que califica el PRODEP es fundamentalmente la
publicación de artículos o libros, de no contar con tres, no se puede
participar, además se toma en cuenta la\linebreak docencia, la dirección de tesis,
proyectos de investigación, las tutorías, la gestión académica y los
premios y distinciones. Es importante destacar que para participar debe
tener cuando menos el grado de Maestro.


El análisis de los programas de estímulos y el PRODEP nos lleva a la
conclusión que para obtener buenas calificaciones, que tienen implicaciones
 de percepciones económicas, se tiene que ser un \textit{buen todólogo}
(estudios avanzados, publicaciones, investigación, difusión, elaboración de
material didáctico, ser muy puntual y cumplido, dar clases, etc.), por lo
que quedan fuera, los que se dedican únicamente para lo que supuestamente
fue creado el programa la docencia.


Por otra parte, sentimos que el programa no ha propiciado la evaluación
académica de los profesores, sino que lo ha convertido en un perseguidor de
constancias que le permitan elevar su salario, aún cuando esto no implique
que se esté superando académicamente y, menos aún, que se esté apoyando a la
docencia.


Consideramos que la UAEM está desprotegiendo la columna vertebral del
quehacer institucional. Los profesores de asignatura son el grueso de los
docentes que integran la universidad. 


Resulta curioso que la universidad no contemple entre sus prioridades el
mejoramiento económico de los profesores; si bien es cierto que el programa
de estímulos en la UAEM está destinado para la totalidad del personal
docente (cuando el recurso alcance), también es cierto que los profesores
de asignatura son los menos favorecidos. Como ya lo mencionamos, la
diferencia de estímulos entre un profesor de asignatura y uno de carrera de
tiempo completo es de diez salarios mínimos. Es interesante
subrayar que un profesor que únicamente se dedica a la docencia, nunca
podrá alcanzar el porcentaje máximo. En los últimos años la UAEM
 se ha preocupado por implementar programas encaminados a la actualización
de los profesores; sin embargo, la asistencia a estos eventos, en su
mayoría, es de profesores de tiempo completo, quienes  muchas veces
atienden uno o dos grupos. No se puede asegurar que de parte de los
profesores de asignatura no exista interés por investigar o por
actualizarse, nos consta que si hay deseo y responsabilidad por mejorar,
pero ante esto volvemos al punto de partida de esta intervención, y es que
no existe un estímulo real, para que un profesor de asignatura pueda
dedicar parte de su tiempo en su actualización, situación que se refleja en
la preparación de los alumnos. Es verdad que la investigación es el
fundamento de la docencia, pero también es cierto que para realizar esta
labor, es necesario que el docente cuente no sólo con recursos materiales y
económicos, sino también con el tiempo y el apoyo institucional. 


Sí la aspiración de la UAEM es lograr la excelencia académica, tendrá que
implementar un programa tendiente a apoyar la economía del profesor, sino
lo hace el sueño de la excelencia quedará en proyecto.


Con lo hasta ahora expuesto se podría pensar que los profesores de tiempo
completo, <<los consentidos de la universidad>>, pueden contar con un buen
sueldo, pues cada año se les aumenta el salario, por  ello haremos un
ejercicio, para conocer que tanto ha sido un premio extra el programa de
estímulos o bien a correspondido al aumento de salario que deberíamos
tener. En el Cuadro 1 (véase Anexo) mostramos los salarios de los
profesores de tiempo completo, hemos tomado la categoría menor  A (TCA), 
la mayor F(TCF)  y una intermedia C (TCC). El salario total incluye las
prestaciones, pero no es el salario neto, pues de ahí hay que descontar el
impuesto, la seguridad social y el fondo de pensiones.


Estos salarios los hemos sometido al índice INPC, que es el índice de
precios al consumidor; los datos fueron tomados del Banco de México
(2010).\footnote{Para realizar el Cuadro con
el índice INPC contamos con la valiosa colaboración de la Dra.\ en
demografía Mercedes Pedrero Nieto, a quien le estamos nuevamente  muy
agradecidas. El Cuadro completo se incluye en el Anexo.} Cabe aclarar que
se tomó el 2010 como base, por ser el año que toma la fuente, es decir, no
está actualizada al presente, al año de 2014. 


El análisis de los  Cuadros 1 y 2 (ver Anexo) nos demuestra que no hemos tenido
un aumento real en nuestros salarios,  incluso en algunos años hubo una
disminución, el más bajo fue el  2000 cuando el salario disminuyó para los
profesores TCA un 3.6\,\%, para los TCC 4.7\,\% y para los TCF 4.2\,\%, por lo
anterior, podemos afirmar que únicamente cada año, el supuesto aumento de
salarios ha tenido que ver con nivelarnos con el índice de inflación.


Por lo anterior concluimos que la realidad es que sólo pocos profesores a
nivel de la UAEM y de la mayoría de las universidades del país han recibido
un aumento real en sus percepciones económicas a base de un esfuerzo extra
o de una estrategia de saber juntar puntos. De ahí que hayamos titulado
este trabajo <<Entre PROMEPsas y PROEDsas>>.


Finalmente,  sólo nos resta preguntamos ¿Cuántos puntos nos darán por este
trabajo?
\newpage

\textbf{Referencias}


H.\ Consejo Universitario (2001), \textit{Reglamento del programa de estímulos 
al desempeño del personal docente}, Toluca,
Universidad Autónoma del Estado de México.


UAEM (1992), \textit{Programa de carrera académica}, Toluca,
Universidad Autónoma del Estado de México.


UAEM (1998), \textit{Programa de Carrera Docente 1998}, Toluca,
Universidad Autónoma del Estado de México.


FAAPAUAEM (1993--2013), \textit{Tabulador del personal académico}.


\bigskip
{\bfseries Página electrónica}

\begin{sloppypar}
\url{http://www.banxico.org.mx/politica-monetaria-e-inflacion/material-de-referencia/intermedio/inflacion/elaboracion-inpc-udis.html}.
Consultada el 30 mayo de 2014.
\end{sloppypar}
\newpage

{\centering \textbf{Anexo}}

\bigskip
\begin{scriptsize}
\begin{flushleft}
\setlength{\extrarowheight}{2pt}
\tablefirsthead{}
\tablehead{}
\tabletail{}
\tablelasttail{}
\begin{supertabular}{|m{10mm}|m{16mm}|m{14mm}|m{16mm}|m{14mm}|m{16mm}|m{14mm}|}
\hline
\rowcolor{lsLightBlue}\multicolumn{7}{|m{.98\textwidth}|}{\centering{
%\multicolumn{7}{|m{1\textwidth}|}{{
\textbf{CUADRO 1.\ SALARIOS PROFESORES TIEMPO COMPLETO UAEM 1993--2013}}
\centering{\textbf{ (En pesos)}}}\\\hline
\centering{\textbf{Año}} &
\centering{\textbf{TCA}\par\textbf{Tabulador}}
&
{\centering \textbf{TCA}\par}

\centering{\textbf{Total}} &
\centering{\textbf{TCC}\par\textbf{Tabulador}}
&
{\centering \textbf{TCC}\par}

\centering{\textbf{Total}} &
\centering{\textbf{TCF}\par\textbf{Tabulador}}
&
{\centering \textbf{TCF}\par}

\centering\arraybslash{
\textbf{Total}}\\\hline
\centering{1993} &
\centering{1813.65} &
\centering{2169.33} &
\centering{2297.29} &
\centering{2748.07} &
\centering{3627.30} &
\centering\arraybslash{4331.29}\\\hline
\centering{1995} &
\centering{2356.84} &
\centering{2807.06} &
\centering{2968.82} &
\centering{3536.48} &
\centering{4687.66} &
\centering\arraybslash{5576.23}\\\hline
\centering{1998} &
\centering{4455.89} &
\centering{5291.49} &
\centering{5613.19} &
\centering{6666.24} &
\centering{8862.60} &
\centering\arraybslash{10515.28}\\\hline
\centering{1999} &
\centering{5253.51} &
\centering{6343.11} &
\centering{6617.95} &
\centering{7999.51} &
\centering{10449.02} &
\centering\arraybslash{12616.14}\\\hline
\centering{2000} &
\centering{5883.93} &
\centering{7218.90} &
\centering{7412.10} &
\centering{9087.55} &
\centering{11702.90} &
\centering\arraybslash{14407.39}\\\hline
\centering{2001} &
\centering{6643.64} &
\centering{8149.93} &
\centering{8361.37} &
\centering{10250.22} &
\centering{13274.88} &
\centering\arraybslash{16338.67}\\\hline
\centering{2002 enero} &
\centering{7243.51} &
\centering{9006.41} &
\centering{9108.34} &
\centering{11317.75} &
\centering{14537.28} &
\centering\arraybslash{18133.81}\\\hline
\centering{2002 mayo} &
\centering{7385.36} &
\centering{9035.83} &
\centering{9279.37} &
\centering{11353.23} &
\centering{14880.46} &
\centering\arraybslash{18204.98}\\\hline
\centering{2003} &
\centering{7702.93} &
\centering{9470.62} &
\centering{9678.38} &
\centering{11899.52} &
\centering{15520.32} &
\centering\arraybslash{19081.02}\\\hline
\centering{2004} &
\centering{7995.64} &
\centering{9862.90} &
\centering{10046.16} &
\centering{12392.29} &
\centering{16110.09} &
\centering\arraybslash{19872.35}\\\hline
\centering{2005} &
\centering{8514.28} &
\centering{10444.75} &
\centering{10697.80} &
\centering{13123.35} &
\centering{17155.07} &
\centering\arraybslash{21044.70}\\\hline
\centering{2006} &
\centering{8850.59} &
\centering{10946.72} &
\centering{11120.36} &
\centering{13754.05} &
\centering{17832.70} &
\centering\arraybslash{22056.09}\\\hline
\centering{2007} &
\centering{9293.12} &
\centering{11619.68} &
\centering{11676.38} &
\centering{14599.60} &
\centering{18724.34} &
\centering\arraybslash{23412.04}\\\hline
\centering{2008} &
\centering{9850.71} &
\centering{12311.50} &
\centering{12376.96} &
\centering{15468.84} &
\centering{19847.80} &
\centering\arraybslash{24805.96}\\\hline
\centering{2009} &
\centering{10343.24} &
\centering{12922.61} &
\centering{12995.81} &
\centering{16236.66} &
\centering{20840.19} &
\centering\arraybslash{26037.25}\\\hline
\centering{2010} &
\centering{10860.41} &
\centering{13564.27} &
\centering{13645.60} &
\centering{17042.88} &
\centering{21882.20} &
\centering\arraybslash{27330.10}\\\hline
\centering{2011} &
\centering{11392.57} &
\centering{14409.48} &
\centering{14314.23} &
\centering{18104.84} &
\centering{22954.42} &
\centering\arraybslash{29033.07}\\\hline
\centering{2012} &
\centering{11939.41} &
\centering{15365.62} &
\centering{15001.31} &
\centering{19306.19} &
\centering{24056.23} &
\centering\arraybslash{30959.57}\\\hline
\centering{2013} &
\centering{12524.44} &
\centering{16281.35} &
\centering{15736.37} &
\centering{20456.76} &
\centering{25234.98} &
\centering\arraybslash{32804.63}\\\hline
\rowcolor{lsLightGray}\multicolumn{7}{|m{.98\textwidth}|}{{{\bfseries Fuente:} FAAPAUAEM, 1993--2013.}}\\\hline
\end{supertabular}
\end{flushleft}

\newpage
\begin{flushleft}
\tablefirsthead{}\setlength{\extrarowheight}{2pt}

\tablehead{}
\tabletail{}
\tablelasttail{}
\begin{supertabular}{|m{10mm}|m{16mm}|m{14mm}|m{16mm}|m{14mm}|m{16mm}|m{14mm}|}
%\begin{supertabular}{|m{10mm}|m{17mm}|m{17mm}|m{17mm}|m{17mm}|m{17mm}|m{17mm}|}
\hline
\rowcolor{lsLightBlue}\multicolumn{7}{|m{.98\textwidth}|}{\centering{
\textbf{CUADRO 2. SALARIOS REALES DE LOS PROFESORES TIEMPO COMPLETO UAEM, 
1993--2013} \textbf{(En pesos)}}}\\\hline
\centering{\textbf{Año}} &
\centering{\textbf{\textcolor{black}{TCA
}}\textbf{\textcolor{black}{Tabulador}}} &
{\centering \textbf{\textcolor{black}{TCA
}}\par}

\centering{
\textbf{\textcolor{black}{Total}}} &
\centering{\textbf{\textcolor{black}{TCC
}}\textbf{\textcolor{black}{Tabulador}}} &
{\centering \textbf{\textcolor{black}{TCC
}}\par}

\centering{
\textbf{\textcolor{black}{Total}}} &
\centering{\textbf{\textcolor{black}{TCF
}}\textbf{\textcolor{black}{Tabulador}}} &
{\centering \textbf{TFC}\par}

\centering\arraybslash{
\textbf{Total}}\\\hline
\centering{1993} &
\raggedleft{\textcolor{black}{11,145.11}} &
{ \textcolor{black}{13,330.81}} &
\raggedleft{\textcolor{black}{14,117.14}} &
{ \textcolor{black}{16,887.24}} &
\raggedleft{\textcolor{black}{22,290.22}} &
\centering\arraybslash{
\textcolor{black}{26,616.33}}\\\hline
\centering{1995} &
\raggedleft{\textcolor{black}{12,336.89}} &
{ \textcolor{black}{14,693.57}} &
\raggedleft{\textcolor{black}{15,540.30}} &
{ \textcolor{black}{18,511.72}} &
\raggedleft{\textcolor{black}{24,537.58}} &
\centering\arraybslash{
\textcolor{black}{29,188.80}}\\\hline
\centering{1998} &
\raggedleft{\textcolor{black}{10,658.83}} &
{ \textcolor{black}{12,657.65}} &
\raggedleft{\textcolor{black}{13,427.19}} &
{ \textcolor{black}{15,946.16}} &
\raggedleft{\textcolor{black}{21,200.03}} &
\centering\arraybslash{
\textcolor{black}{25,153.36}}\\\hline
\centering{1999} &
\raggedleft{\textcolor{black}{10,840.14}} &
{ \textcolor{black}{13,088.43}} &
\raggedleft{\textcolor{black}{13,655.54}} &
{ \textcolor{black}{16,506.26}} &
\raggedleft{\textcolor{black}{21,560.61}} &
\centering\arraybslash{
\textcolor{black}{26,032.26}}\\\hline
\centering{2000} &
\raggedleft{\textcolor{black}{10,413.77}} &
{ \textcolor{black}{12,776.49}} &
\raggedleft{\textcolor{black}{13,118.43}} &
{ \textcolor{black}{16,083.75}} &
\raggedleft{\textcolor{black}{20,712.57}} &
\centering\arraybslash{
\textcolor{black}{25,499.15}}\\\hline
\centering{2001} &
\raggedleft{\textcolor{black}{10,739.05}} &
{ \textcolor{black}{13,173.88}} &
\raggedleft{\textcolor{black}{13,515.66}} &
{ \textcolor{black}{16,568.87}} &
\raggedleft{\textcolor{black}{21,458.06}} &
\centering\arraybslash{
\textcolor{black}{26,410.49}}\\\hline
\centering{2002 enero} &
\raggedleft{\textcolor{black}{11,202.50}} &
{ \textcolor{black}{13,928.92}} &
\raggedleft{\textcolor{black}{14,086.56}} &
{ \textcolor{black}{17,503.54}} &
\raggedleft{\textcolor{black}{22,482.72}} &
\centering\arraybslash{
\textcolor{black}{28,044.96}}\\\hline
\centering{2002 mayo} &
\raggedleft{\textcolor{black}{11,274.61}} &
{ \textcolor{black}{13,794.24}} &
\raggedleft{\textcolor{black}{14,166.03}} &
{ \textcolor{black}{17,332.02}} &
\raggedleft{\textcolor{black}{22,716.74}} &
\centering\arraybslash{
\textcolor{black}{27,792.01}}\\\hline
\centering{2003} &
\raggedleft{\textcolor{black}{11,145.24}} &
{ \textcolor{black}{13,702.88}} &
\raggedleft{\textcolor{black}{14,003.49}} &
{ \textcolor{black}{17,217.22}} &
\raggedleft{\textcolor{black}{22,456.09}} &
\centering\arraybslash{
\textcolor{black}{27,608.01}}\\\hline
\centering{2004} &
\raggedleft{\textcolor{black}{10,570.05}} &
{ \textcolor{black}{13,038.52}} &
\raggedleft{\textcolor{black}{13,280.79}} &
{ \textcolor{black}{16,382.32}} &
\raggedleft{\textcolor{black}{21,297.16}} &
\centering\arraybslash{
\textcolor{black}{26,270.78}}\\\hline
\centering{2005} &
\raggedleft{\textcolor{black}{10,824.01}} &
{ \textcolor{black}{13,278.17}} &
\raggedleft{\textcolor{black}{13,599.87}} &
{ \textcolor{black}{16,683.42}} &
\raggedleft{\textcolor{black}{21,808.85}} &
\centering\arraybslash{
\textcolor{black}{26,753.65}}\\\hline
\centering{2006} &
\raggedleft{\textcolor{black}{10,857.48}} &
{ \textcolor{black}{13,428.92}} &
\raggedleft{\textcolor{black}{13,641.93}} &
{ \textcolor{black}{16,872.82}} &
\raggedleft{\textcolor{black}{21,876.31}} &
\centering\arraybslash{
\textcolor{black}{27,057.37}}\\\hline
\centering{2007} &
\raggedleft{\textcolor{black}{10,965.38}} &
{ \textcolor{black}{13,710.59}} &
\raggedleft{\textcolor{black}{13,777.50}} &
{ \textcolor{black}{17,226.74}} &
\raggedleft{\textcolor{black}{22,093.71}} &
\centering\arraybslash{
\textcolor{black}{27,624.94}}\\\hline
\centering{2008} &
\raggedleft{\textcolor{black}{11,056.65}} &
{ \textcolor{black}{13,818.70}} &
\raggedleft{\textcolor{black}{13,892.17}} &
{ \textcolor{black}{17,362.57}} &
\raggedleft{\textcolor{black}{22,277.61}} &
\centering\arraybslash{
\textcolor{black}{27,842.76}}\\\hline
\centering{2009} &
\raggedleft{\textcolor{black}{11,025.42}} &
{ \textcolor{black}{13,774.92}} &
\raggedleft{\textcolor{black}{13,852.94}} &
{ \textcolor{black}{17,307.54}} &
\raggedleft{\textcolor{black}{22,214.70}} &
\centering\arraybslash{
\textcolor{black}{27,754.53}}\\\hline
\centering{2010} &
\raggedleft{\textcolor{black}{10,860.41}} &
{ \textcolor{black}{13,564.27}} &
\raggedleft{\textcolor{black}{13,645.60}} &
{ \textcolor{black}{17,042.88}} &
\raggedleft{\textcolor{black}{21,882.20}} &
\centering\arraybslash{
\textcolor{black}{27,330.10}}\\\hline
\centering{2011} &
\raggedleft{\textcolor{black}{11,275.13}} &
{ \textcolor{black}{14,260.94}} &
\raggedleft{\textcolor{black}{14,166.67}} &
{ \textcolor{black}{17,918.21}} &
\raggedleft{\textcolor{black}{22,717.80}} &
\centering\arraybslash{
\textcolor{black}{28,733.78}}\\\hline
\centering{2012} &
\raggedleft{\textcolor{black}{11,349.69}} &
{ \textcolor{black}{14,606.67}} &
\centering{\textcolor{black}{14,260.35}} &
{ \textcolor{black}{18,352.60}} &
\raggedleft{\textcolor{black}{22,868.03}} &
\centering\arraybslash{
\textcolor{black}{29,430.39}}\\\hline
\centering{2013} &
\raggedleft{\textcolor{black}{11,469.26}} &
{ \textcolor{black}{14,909.65}} &
\raggedleft{\textcolor{black}{14,410.58}} &
{ \textcolor{black}{18,733.28}} &
\raggedleft{\textcolor{black}{23,108.94}} &
\centering\arraybslash{
\textcolor{black}{30,040.85}}\\\hline
\rowcolor{lsLightGray}\multicolumn{7}{|m{.98\textwidth}|}{{
{\bfseries Fuente:} \url{http://www.banxico.org.mx/politica-monetaria-e-inflacion/material-de-referencia/intermedio/inflacion/elaboracion-inpc-udis.html}. 30 mayo de 2014.}
%~
}\\\hline
\end{supertabular}
\end{flushleft}
\end{scriptsize}
\newpage
\thispagestyle{empty}
\phantom{abc}