%%\clearpage\setcounter{page}{323}

\thispagestyle{empty}	
\phantomsection
\addcontentsline{toc}{chapter}{Clima social en estudiantes universitarios:\newline Análisis obligado para el mejoramiento\newline\ de la eficiencia terminal\newline $\diamond$
\normalfont\textit{Ivett Reyes-Guillén y Carlos Arcos Vázquez}}
{\centering {\scshape \large Clima social en estudiantes universitarios: Análisis obligado
para el mejoramiento de la eficiencia terminal\par}}
\markboth{formación del historiador}{clima social}

\bigskip
\begin{center}
{\bfseries Ivett Reyes{-}Guillén\\
Carlos Arcos Vázquez}\\
{\itshape Universidad Autónoma de Chiapas}
\end{center}

\bigskip
\textbf{Resumen}

El análisis del clima social en estudiantes universitarios, nos permite
determinar las líneas de acción para la elaboración de un plan de mejora de
una institución educativa. Con ello, se busca atender diversos aspectos que
los alumnos perciben como necesidades, que de ser cubiertas, impactarán de
modo positivo sobre su desempeño, permanencia y eficiencia terminal. Los
resultados encontrados permiten visualizar las percepciones de los
estudiantes en tres direcciones, hacia sí mismos, hacia sus docentes y
hacia la infraestructura con que cuentan para su formación. Se muestran
varios elementos que reflejan el estado situacional del clima social de los
estudiantes y su impacto en la calidad de los programas a los que están
insertos. Específicamente, se determinaron necesidades de atención en áreas
como infraestructura física, planta docente, grado de especialización,
disponibilidad de los profesores, autoestima, relaciones académicas
docente-alumnos.\\
\textbf{Palabras Clave:} Clima social, deserción, eficiencia terminal.
\enlargethispage{1\baselineskip}

\medskip 
\textbf{Abstract}

The analysis of the social climate in University, allows us to determine the
lines of action for the development of a plan of improvement of an
educational institution. Therefore it seeks to address aspects that
students perceived needs, that of being covered, will impact positively on
their performance, permanence and terminal efficiency. The found results
allow to visualize the perceptions of students in three directions, towards
themselves, towards their teachers and to the infrastructure available for
training. Shows several elements that reflect the situational status of
students social climate and its impact on the quality of the programs.
Specifically, identified needs of care in areas such as infrastructure,
degree of specialization, availability of teachers, self-esteem, academic
relations betwen teachers-students.\\ 
\textbf{Keywords:} Social climate,  school dropout,
terminal efficiency.

\medskip 
\textbf{Introducción}
%\enlargethispage{1\baselineskip}
 
La educación formal se ha transformado en uno de los ámbitos de mayor
preocupación de los gobiernos a nivel mundial, debido principalmente a la
constante búsqueda de la calidad para el desarrollo. Es decir, el impacto
que los procesos educativos tienen en la promoción del desempeño social y
económico de la población, permite insertarse en el nuevo orden de competencia 
internacional (Pascual 1995).

 
Para México, el Plan de Desarrollo Nacional 2013-2018 (PND) estima la
necesidad de trabajar por un México incluyente, con educación de calidad,
prosperidad y responsabilidad global. El PDN considera varias estrategias
claras, con líneas de acción concretas, tal es el caso de la estrategia
3.1.3, que implica garantizar que los planes y programas de estudio sean
pertinentes y contribuyan a que los estudiantes puedan avanzar exitosamente
en su trayectoria educativa, al tiempo que desarrollen aprendizajes
significativos y competencias que les sirva a lo largo de su vida (Gobierno
Federal 2013).

 
Así también, el PDN plantea la estrategia 3.1.5, la cual tiene el objetivo de 
disminuir el abandono escolar, mejorar la eficiencia terminal en cada nivel educativo
y aumentar las tasas de transición entre un nivel y otro (\textit{ibid.}). 
Lo anterior exige la participación social y los aportes que la
ciencia pueda desarrollar para el logro de las metas trazadas dentro
del PDN 2013--2018.

 
Las estrategias anteriores son algunas de las que se han considerado para
atender de modo específico la problemática educativa respecto a la
deserción escolar y la eficiencia terminal. Ambos indicadores en las
Instituciones de Educación Superior, representan un problema complejo y
comúnmente estudiado desde una perspectiva cuantitativa por la naturaleza
misma de estos indicadores.

 
Al respecto se han realizado numerosos estudios, no obstante, sigue siendo necesario y
conveniente tener un mayor acercamiento a la realidad del estudiante con la
finalidad de identificar lo que está ocurriéndole, las dificultades que va
encontrando a lo largo de su trayectoria, de modo que se esté en
posibilidades de diseñar estrategias e\linebreak implementar acciones encaminadas hacia
la conclusión de sus estudios\linebreak \mbox{(Domínguez~\textit{et~al.}~2013)}.

 
Precisamente dentro de los estudios que se han realizado varios en el ámbito
de la calidad y la comprensión de los procesos educativos, los análisis de
percepciones sobre el ambiente en el que se desarrollan las actividades
humanas habituales, las relaciones interpersonales que establecen y el
marco en que se dan tales interacciones. Es decir, el conocimiento del
clima institucional, social, laboral, entre otros. En este sentido, en el
ámbito escolar se estaría hablando del Clima Social Escolar (Arón y Millic
2000).

 
El presente estudio, muestra interés principal en el análisis de los
factores biopsicosociales relacionados con la percepción del estudiante
universitario sobre sí mismo y sobre el clima social del cual participa.
Todo ello, relacionado con la necesidad de identificación y comprensión de
los factores que influyen, de modo directo o indirecto, con el
aprovechamiento, permanencia y eficiencia terminal de los estudiantes.

 
\textbf{Objetivo General}

 
Analizar mediante valoración auto-perceptiva, el clima social de estudiantes
de nivel superior.

 
\textbf{Objetivo Específico}

 
Incidir en la planeación de programas de mejora para reducir el índice de
deserción y mejorar la eficiencia terminal.

 
\textbf{Marco teórico y conceptual}

 
Partiendo de que en la actualidad, los procesos educativos deben estar
ocupados en mejorar la calidad de los mismos, la educación superior
enfrenta fuertes retos relacionados no solo con la calidad de sus programas
educativos, sino también con la permanencia de los mismos. Dos indicadores
de interés para este caso son la Deserción Escolar y la Eficiencia Terminal,
teniendo éstos relación directa entre ambos.

 
Es preciso definir ambos conceptos, mismos en los que fijamos interés al
analizar el Clima Social de jóvenes universitarios. Deserción escolar es un
término que hace referencia a aquellos alumnos que dejan de asistir a
clases y quedan fuera del sistema educativo. Mientras tanto, la eficiencia
terminal puede definirse como la relación porcentual entre una generación
de egresados de un nivel educativo y el número de estudiantes que dieron
origen a esta misma generación años antes (SEP~2001).
\newpage
 
Los esfuerzos que se han realizado para atender esta problemática  y
explicar la naturaleza de ambos procesos, han sido múltiples. Dentro de
ellos  se ha considerado que la deserción obedece a la falta de integración
de los jóvenes al sistema universitario, así también es atribuido al bajo
rendimiento escolar, limitada capacidad de estudiar y la existencia de
problemas socioeconómicos familiares (Bean~1982; SREB~2000; Herrera~1999; 
Langbeing y Zinder~1999). 

 
Estudios en Latinoamérica argumentan además como factores de deserción a la
desintegración familiar, violencia intrafamiliar y social, problemas de
salud, problemas jurídicos. Otros estudios sugieren, por la otra parte, que
el sistema de aprendizaje y relación docente-alumno, son elementos muy
importantes al momento de analizar la deserción escolar (Páramo y Correa~1999, 
Osorio y Jaramillo~2000, Pérez Franco~2001).

 
Para el caso de México, son pocos los estudios que se han hecho al respecto,
algunos han podido encontrar como factores determinantes de la deserción a
las  presiones económicas familiares, desintegración familiar, falta o
inadecuada orientación profesional, reprobación, problemas de salud y el
traslape de horarios de escuela-trabajo de los estudiantes (Chaín Revueltas
\textit{et al.} 2001).

 
Si bien, los elementos anteriormente mencionados son factores causales de la
deserción, no debemos dejar de lado las percepciones que los jóvenes tienen
con relación a ellos mismos, con sus docentes y con la universidad en la
que están insertos, es decir, el clima social del estudiante universitario
y su influencia en los indicadores deserción escolar y eficiencia terminal
puede estar marcando factores sensibles para el abordaje de esta
problemática educativa.

 
La forma en que los individuos se evalúan a sí mismos y a su entorno, tiene
repercusión en todas las áreas de su desarrollo social, emocional,
intelectual, conductual y escolar. Es decir, la percepción de un individuo
determina sus actitudes y generalmente sus conductas\linebreak (González-Arratia \textit{et
al.} 2000). Estas percepciones se tornan importante para la educación
formal, toda vez que sabemos que la universidad es la institución a quien
compete preparar seres humanos integrales, desarrollando no solo sus
conocimientos sino sus aptitudes y actitudes para que sean profesionistas
capaces de resolver con éxito los problemas sociales.

 
Dentro del ambiente escolar, hablar de clima social es referirnos a la
estructura relacional configurada por la interacción de varios factores que
incidirán en el proceso de enseñanza-aprendizaje. La escuela, la dinámica
propia de las clases, las características físicas y de infraestructura, los
factores organizativos, las características tanto del docente como del
estudiante, son factores determinantes del clima social (Bassedas 1991),
en consecuencia, medirlos es una necesidad tanto para la comprensión de las
dinámicas escolares como de los indicadores de la estabilidad, es decir, la
deserción escolar y la eficiencia terminal.

 
\textbf{Métodos}

 
El estudio se realizó en el período comprendido por los meses de febrero y
marzo del 2014, en la Facultad de Ciencias Sociales de la UNACH, misma que
se encuentra localizada en la ciudad de San Cristóbal de las Casas,
Chiapas. Se caracteriza por una población pluriétnica, donde la lengua
materna del 35\,\% de los estudiantes es indígena (tsotsil 18\,\%,  tzeltal
10\,\% y chol 7\,\%). El resto corresponde a estudiantes de lengua castellana
 (Reyes-Guillén y Arcos 2014).

 El estudio es de tipo cuali-cuantitativo, perfilando las siguientes etapas:
\enlargethispage{1\baselineskip}
\begin{Obs}
\item[1.]  Diseño de instrumento de medición de clima social en estudiantes nivel
universitario. El instrumento consta de distintas secciones, las cuales
determinan el grado de comodidad del estudiante en el ambiente social en
que se desenvuelve.
\item[2.]  Prueba piloto del instrumento (con planeación estadística). Se probó el
instrumento, haciendo un análisis de su funcionalidad para el levantamiento
formal de los datos.
\item[3.]  Ajustes de diseño del instrumento de acuerdo a los resultados de la prueba
piloto. Se realizaron ajustes de acuerdo a las preguntas que no tenían
claridad para el entrevistado.
\item[4.]  Aplicación del instrumento para la obtención del clima social en estudiantes
nivel universitario de la FCS. Se aplicó el instrumento a la totalidad de
estudiantes encontrados en la Universidad durante el período comprendido
por los meses de febrero y marzo de 2014. Del total de estudiantes
inscritos $(N = 1,058)$ (Economía 540, Sociología 300, Historia 120,
Antropología 98 Servicios Escolares FCS, 2014) se entrevistaron a $n = 496$,
representando un 47\,\% de los estudiantes de la facultad.
\end{Obs}


\medskip
\textbf{Resultados y Discusión}
 
La edad promedio de los estudiantes $(n = 496)$ a quienes se les aplicó la
escala de medición de clima social en la Facultad de Ciencias Sociales,
UNACH, 2014, es de 21 años. En cuanto al género, fue menor el porcentaje de
mujeres (43\,\%) que de hombres (57\,\%).

 
Varios estudios han demostrado que se ha incrementado el número de mujeres
estudiantes  (Acker 1994; Andrade  y  León 2001; Eli 2000), en este
estudio puede verse un porcentaje alto de mujeres (43\,\%) inscritas en los
distintos programas de la facultad de Ciencias Sociales.

 
Aunado a lo anterior, el 35\,\% de los estudiantes son de origen indígena,
teniendo como lengua materna el tzeltal, tsotsil y chol principalmente,
estos estudiantes son bilingües (lengua materna y español).

 
A pesar de que los resultados muestran varios puntos de importancia para la
discusión, comenzaremos con el análisis general del clima social de jóvenes
universitarios. Posteriormente, se analizarán elementos que la Universidad
debe considerar para un mejor desarrollo de las actividades académicas y la
formación de profesionistas.

 
La información que se presenta, muestra la percepción de los estudiantes en
tres ejes, hacia ellos mismos, hacia los docentes y hacia la
infraestructura con que cuenta la universidad a la que pertenecen.

 
Percepción hacia ellos mismos. Un importante porcentaje de estudiantes,
demuestran elementos de baja autoestima, como es el considerarse con poca
inteligencia (42\,\%).

 
Respecto a los elementos de relaciones sociales, el 70\,\% se consideran
humildes, entendiéndose como humildad a la capacidad de reconocer sus
propias limitantes y debilidades y actuar de acuerdo a este conocimiento.
Aunque su concepto de humildad no está relacionado con el concepto de
servicio ya que 52\,\% de estos estudiantes, no se considera servicial, es
decir, no tienen disposición a prestar ayuda hacia los demás. Un elemento
de importancia en este apartado, es el de considerarse jóvenes sensibles
(con capacidad de percibir sensaciones), ya que el 45\,\% se consideran así.

\enlargethispage{1\baselineskip} 
Por lo anterior, las relaciones interpersonales de amistad entre los
estudiantes se ven reflejadas con solo el 59\,\% de jóvenes que pertenece a
algún grupo de amigos en la facultad y fuera de ella.

 
Apoyados en González-Arratia \textit{et al.} (2000), podemos afirmar que la
autoestima influye directamente en el comportamiento de los individuos e
incide en la promoción del bienestar psicológico. La forma en que estos
estudiantes se evalúan a sí mismos, tiene repercusión en todas las áreas de
su desarrollo social, emocional, intelectual, conductual y escolar. Lo
anterior es de suma importancia para este estudio, ya que sabemos que es a
la universidad a quien compete preparar seres humanos integrales,
desarrollando no solo sus conocimientos sino sus aptitudes y actitudes para
que sean profesionistas capaces de resolver con éxito los problemas
sociales.
\enlargethispage{1\baselineskip}
 
Percepción sobre sus actividades académicas y físicas. El 48\,\% de los
estudiantes no realizan programación de metas o pocas veces lo hacen, lo
que les conduce a no realizar actividades planeadas y solo el 47\,\% de los
estudiantes consideran guardar equilibrio entre lo que piensan y hacen; la
mayoría (53\,\%) nunca o pocas veces guardan este equilibrio.

 
De acuerdo al modelo educativo, modelo de competencias, en el que la
Universidad basa su quehacer académico (UNACH 2010), toma al estudiante
como un agente participativo, responsable, reflexivo, crítico, en su propio
aprendizaje. La planeación permite al estudiante lograr estas competencias
y asegurar el proceso de mejor manera. Si el estudiante no centra su
interés en ello, lógicamente quedará con menores posibilidades de formarse
profesionalmente y de modo integral.

 
Respecto a las actividades deportivas, tampoco son un elemento que acompañe
a sus intereses, ya que el 60\,\% de los estudiantes no realizan algún
deporte o pocas veces lo hace. Estos resultados son muy comunes en ámbitos
universitarios, donde es alto el porcentaje de jóvenes que no practica
algún deporte. Por ello se han desarrollado diversos estudios que expresan
la necesidad de fomentarse la actividad física para mejorar la calidad de
vida, así como prevenir padecimientos\linebreak crónicos y degenerativos. Así
también, dar a conocer la importancia de la actividad física para la salud
mental (López~\textit{et al.}~2006).

 
Por lo anteriormente discutido, cabe mencionar que, el potencial humano no
puede ser prescrito, sólo se pueden crear las condiciones propicias para
que el alumno encuentre sus propias capacidades\linebreak (Munguía~\textit{et al.}~1999).
Algunos elementos que favorecen este desarrollo del potencial son la
comunicación interpersonal, creatividad, planeación, trabajo en equipo y
liderazgo. De aquí la importancia de haber evaluado la percepción de los
estudiantes sobre todos estos elementos.

 
Percepción hacia sus materias y docentes. El 85\,\% de los estudiantes
consideran que sus materias son complicadas, sin duda alguna este
importante porcentaje tiene relación con los elementos antes discutidos. Es
decir, los problemas de comprensión del idioma, falta de planeación, falta
de actividades deportivas, dificulta el desarrollo intelectual y formación
integral de los estudiantes.  La falta de confianza en sí mismos respecto a
sus capacidades intelectuales es quizá la principal condicionante. Esta
información fue corroborada con la percepción de accesibilidad a sus
materias, donde el 55\,\% las consideran poco accesibles.

 
Ahora bien, al momento de analizar la percepción respecto al cumplimiento de
expectativas, el 64\,\% de los estudiantes consideran no estar cubriendo sus
expectativas con el programa al que están inscritos, o bien, no las
perciben cubiertas de modo completo. Esto, aunado al mismo porcentaje
(64\,\%) que no percibe elementos de especialización dentro del programa al
que están inscritos.

 
Percepción de la relación con sus docentes. No debe pasar desapercibido que
el 40\,\% de los estudiantes afirman tener problemas, siempre o
frecuentemente, con el idioma en que sus docentes imparten las clases:
español.

 
Aunado a lo anterior, el 65\,\% de los estudiantes consideran que sus docentes
pocas veces cubren el perfil adecuado al programa académico al que
pertenecen. Reflejado también, la falta de compromiso que tienen con su
actividad docente (70\,\%) así como escasa actitud de servicio (59\,\%). Un
elemento muy sensible de este apartado, explicado por los porcentajes
anteriormente expuestos, es que el 50\,\% de los estudiantes percibe falta de
respeto por parte de sus docentes hacia ellos como estudiantes.
 
Lo anterior explica a la vez que los estudiantes perciben poca o nula
actitud cordial por parte de sus docentes (63\,\%) y en algunas ocasiones
falta de actitud humilde por parte de ellos (66\,\%), siendo el 77\,\% de los
estudiantes que perciben actitudes prepotentes por parte de sus docentes.

\enlargethispage{1\baselineskip} 
Específicamente sobre el adecuado manejo de la información, los estudiantes
perciben falta de estas capacidades por parte de sus docentes (67\,\%).
 
Todas las variables analizadas anteriormente, nos permiten visualizar un
panorama delicado de las posibles relaciones docente-alumno que a nivel
universitario se están desarrollando. Por ello, vale la pena rescatar lo
que Batanaz (1997) ha reflexionado al respecto: A través de varios
estudios, se ha tenido registro de factores estrechamente relacionados con
lo que los estudiantes consideran calidad educativa a nivel universitario.
Dentro de estos elementos a los que un docente debe prestar atención, se
encuentran los relacionados a la motivación de sus estudiantes, 1)
acercamiento, accesibilidad y orientación; 2) adaptación al nivel de
conocimientos; 3) objetividad y tolerancia; 3) relación de\linebreak contenidos con
problemas significativos, 4) entusiasmo en la exposición de los temas. Es
decir, el docente debe ser sensible a la motivación dentro del proceso de
enseñanza aprendizaje, no solo en los primeros niveles de educación formal,
sino también en ambientes universitarios donde lo que se está formando es
precisamente a profesionistas integrales.

 
Si los docentes universitarios prestamos atención y practicamos estas
variables, nos dará como resultado la formación de profesionistas de alto
rendimiento, sensibles a la realidad del entorno y no solo al manejo de
información documental.

 
Elementos generales de la percepción de los estudiantes respecto a la
Universidad. Un porcentaje bastante elevado de los estudiantes, consideran
que en pocas o nulas ocasiones la Universidad les proporciona los servicios
 que faciliten el aprendizaje y formación dentro del programa al que están
inscritos (76\,\%). Ellos explican que esto puede deberse a que la
infraestructura con que se cuenta no es la suficiente para lograrlo
(78\,\%).

 
En cuanto a la infraestructura física y tecnológica, existen estudios
nacionales que atienden estas necesidades. Los resultados de estas
investigaciones consideran la posibilidad de reconocer los cambios que ha
sufrido la construcción de infraestructura educativa en México. En el país,
a lo largo del tiempo se han generado distintos programas de creación de
escuelas, por el principal objetivo de llevar educación al mayor número de
alumnos, sin detenerse  a analizar que estas construcciones requieren de
elementos para una formación holística (Santa Ana Lozada 2007). Esto
sucede en todos los ámbitos educativos, aunque si bien es cierto que dentro
del nivel universitario se atienden los requerimientos tecnológicos
actuales dentro de la infraestructura, también es cierto que aún resta
mucho que hacer para adaptar estos espacios hacia ambientes que permitan el
desarrollo y formación holista e integral de profesionistas.

 
Por último, es relevante mencionar que habrá que atender estas necesidades,
con relación a los programas educativos universitarios. En este caso, el
mayor porcentaje de la matrícula de la facultad, está representada por
estudiantes inscritos al programa de Economía (51\,\%), seguido del programa
de sociología aunque con un muy bajo porcentaje al compararse (28\,\%). Es de
importancia y objeto de otro estudio, el poder analizar los factores que
influyen para que se esté reflejando muy baja matricula de los programas de
Historia y Antropología (11\,\%~y~9\,\%, respectivamente).
\enlargethispage{1\baselineskip}

 
Relación de Clima Social con la deserción escolar y la eficiencia terminal.
El Clima Social de jóvenes universitarios logra mostrar las relaciones de
los alumnos entre sí y su dinámica. Estas interacciones y su naturaleza,
permiten a los estudiantes las condiciones de generar mecanismos para
perseverar y desarrollar un sentido de pertenencia con los compañeros del
programa y con la comunidad académica, dando lugar al establecimiento de
vínculos más fuertes entre ellos y a una vida académica integrada. Esto
implica un proceso de formación que se refleje en el aprovechamiento
académico y la eficiencia terminal, como resultado.

 
En este sentido, el presente estudio, muestra varios elementos que deben ser
atendidos para superar las condiciones académicas de la Facultad, toda vez
que consideremos que mejorar el clima social de los jóvenes facilitará su
integración a la universidad. Como se ha revisado en varios estudios, los
factores familiares, económicos y sociales están influyendo en la deserción
y la eficiencia terminal. En este estudio, se ha logrado identificar
factores internos que explican el clima\linebreak social percibido por los jóvenes
universitarios y son también elementos que requieren atención al momento de
buscar los mecanismos para mejorar estos indicadores. En el caso de
estudio, al abordar este análisis y planear proyectos de mejora, se
incidirá de modo directo en el mejoramiento de la eficiencia terminal,
siendo ésta actualmente de 42.89\,\% (UNACH FCS 2013) y evitar la
deserción, que abarca 57.11\,\% (UNACH FCS 2013).


\bigskip
\textbf{Conclusiones}
%\enlargethispage{1\baselineskip}
 
Del estudio de Clima Social en Jóvenes Universitarios, los resultados
encontrados en los tres ejes de análisis de percepciones, nos permiten
concluir lo siguiente:

\begin{Obs}
\item[1.]  Dentro de la población estudiantil de la Facultad, existe un alto porcentaje
de mujeres inscritas a los programas.
\item[2.] Un alto porcentaje de estudiantes son de origen indígena de las zonas Altos
tzeltal-tsotsil y Selva de Chiapas.
\item[3.] El estudio refleja problemas respecto a la autoestima de los estudiantes,
principalmente al momento de considerar sus capacidades intelectuales. Esto
debe ser considerado un importante elemento de análisis e intervención para
el mejoramiento en el rendimiento académico de los estudiantes.
\item[4.] La Facultad necesita planear intervenciones de fomento al deporte dentro de
la población estudiantil, así como mejorar las condiciones de
infraestructura para ello. No olvidando que la actividad física permite
mejorar la salud física, emocional y mental de quien la practica.
Específicamente dentro de las actividades académicas, el deporte favorecerá
el trabajo en equipo y liderazgo.
\item[5.] Los problemas de comprensión del idioma, falta de planeación, falta de
actividades deportivas, dificultan el desarrollo intelectual y formación
integral de los estudiantes, aunado a esto, la falta de compromiso y
motivación por parte de la actividad docente, nos deja con pocas
posibilidades de resultados satisfactorios y de calidad en la formación. Es
por ello, que la facultad debe prestar especial atención a la capacitación
didáctica y pedagógica de sus docentes, facilitando de esta manera los
procesos de enseñanza-aprendizaje en este nivel educativo, así como la
satisfacción de los usuarios, es decir, la satisfacción de los estudiantes
en cuanto a sus expectativas de formación profesional.
\item[6.]   
Los estudiantes de la facultad, reconocen debilidades en cuanto a la
infraestructura universitaria con que cuentan. Es de interés para la
Facultad, retomar estos indicadores para considerar dentro de su planeación
estratégica la necesidad de adecuar la infraestructura no solamente a las
necesidades tecnológicas, sino también, a las requeridas para una formación
holista basada en competencias, donde los estudiantes deberán facilitar sus
capacidades en el ámbito intelectual, físico, artístico y la sensibilidad
para atender problemas sociales concretos.
\item[7.] Los estudios de evaluación del clima social de estudiantes universitarios,
son útiles para el diagnóstico situacional de la institución y sus
resultados permiten identificar las líneas de acción para la planeación de
programas de mejora con el objetivo específico de disminuir la deserción e
incrementar el índice eficiencia terminal creando un ambiente social
saludable para los estudiantes.
\end{Obs}
\newpage

\textbf{Referencias}

\medskip 
Acker, S. (1994), \textit{Género y educación.
Reflexiones sociológicas sobre las mujeres, la enseñanza y el feminismo},
Madrid, Narcea.

 
Andrade, SM y ZM León (2001), \textit{La organización del trabajo doméstico,
en Maestros Universitarios}, Puebla: COESPO.

 
Arón A. y N. Milicic (2000), \textit{Desgaste profesional de los profesores y
clima social escolar}, Santiago de Chile, Ediciones Pontificia Universidad Católica de Chile.

Bartra, E. (2000), \textit{Estudios de la
mujer. ¿Un paso adelante, dos pasos atrás?},
México, Universidad Autónoma Metropolitana-Xochimilco.

 
Bassedas, BE (1991), \textit{Intervención Educativa y Diagnóstico
Psicopedagógico}, Barcelona,  Ed.\ Paidós.

 
Batanaz, P. (1997), <<Las variables de la relación profesor alumno en el
contexto universitario>>. \textit{Revista Electrónica Interuniversitaria del
Profesorado}, 1(0).

 
Bean JP (1980), \textit{Dropouts and Turnover: The synthesis and Test of a Causal
Model of Student Attrition}. Research in Higher Education.

\begin{sloppypar} 
Chain Revueltas, R (comp.) (2001), \textit{Deserción, rezago e ineficiencia
terminal en la IES}, Propuesta metodológica para su estudio, México,
ANUIES.
\end{sloppypar} 
 
Domínguez PD, Sandoval CM, Cruz CF, y T Pulido (2013), <<Problemas
relacionados con la eficiencia terminal desde la perspectiva de estudiantes
universitarios>>, \textit{Revista Iberoamericana sobre Calidad, Eficacia y
Cambio en Educación}, 12(1): pp. 25--34.
\newpage
 
González-Arratia NI, Gil LM y JL Valdez (2000), <<Autoconcepto en mujeres
mexicanas y españolas. Un análisis transcultural>>, \textit{Revista de
Psicología y Salud}, Vol. 10. Núm. 1, enero-junio, Universidad
Veracruzana.

 
Herrera, ME. (1999), <<Fracaso escolar, códigos y disciplina: una
aproximación etnográfica>>. En \textit{Última Década} No 10, Viña del Mar,
Ediciones CIDPA.

 
Langbein, L. I. y K. Zinder (1999), \textit{The impact of teaching on
retention: some quantitative evidence}. Social Science Quarterly.

 
López, B. González de Cossio O y G Rodríguez (2006), \textit{Actividad
física en estudiantes universitarios: prevalencia, características y
tendencia}, Med.\ Int.\ Mex.\ 22: 189--96.

 
Mungía, AG, Patiño SC y AD Perales (1999), \textit{Desarrollo del potencial en
educación superior}, Memorias del XXVI  Congreso del Consejo
Nacional para la Enseñanza e Investigación en Psicología,  México, CENEIP.

 
Osorio, AR. y C Jaramillo (2000), <<Deserción universitaria en los programas
de pregrado de la universidad EAFIT>>. \textit{Revista Universidad},
\textit{EAFIT}.

 
Páramo, GJ y CA. Correa (1999), <<Deserción estudiantil universitaria.
Conceptualización>>. \textit{Revista Universidad EAFIT}.

 
Pascual, E. (1995), <<Incidencia de las condiciones laborales e
institucionales en el desempeño profesional de los educadores de enseñanza
media>>. En \textit{Revista Pensamiento Educativo}, Vol. 16. 1995, pp.
245--264.

 
Pérez Franco, L. (2001), <<Los factores socioeconómicos que inciden en el
rezago y la deserción escolar>>. En Chain Revueltas, Ragueb (comp.),
\textit{Deserción, rezago y eficiencia terminal en las IES}, México,
ANUIES.

 
Gobierno Federal (2013), \textit{Plan Nacional de Desarrollo 2013--2018},
México.


\begin{sloppypar}  
Reyes-Guillén I. y Arcos Vásquez C. (2014), <<Estudio de percepciones sobre
clima organizacional de la Facultad de Ciencias Sociales, UNACH>>.
Reporte técnico, San Cristóbal de las Casas, Chiapas.
\end{sloppypar} 
 
Santa Ana Lozada, L. (2007). <<Arquitectura escolar en
México>>. Bitácora arquitectónica de la UNAM, núm.\ 17, pp.\ 70--75.

 
SEP (2001), \textit{Programa Nacional de Educación 2001--2006}, México.

 
Southern Regional Education Board-SREB (2000), \textit{Reducing dropout rates}.
SREB Educational Benchmarks 2000 Series.


\begin{sloppypar}  
UNACH (2010), \textit{Modelo educativo de la UNACH}, Tuxtla Gutiérrez, Chiapas, México, Universidad Autónoma de Chiapas. 
\end{sloppypar} 

\begin{sloppypar} 
UNACH, FCS (2013), \textit{Segundo informe de actividades}, Facultad de
Ciencias Sociales. Gestión 2011-2015, Universidad Autónoma de Chiapas.
\end{sloppypar}
%\newpage
%\thispagestyle{empty}
%\phantom{abc}

%\documentclass{article}
%\usepackage{amsmath,amssymb,amsfonts}
%\usepackage{fontspec}
%\usepackage{xunicode}
%\usepackage{xltxtra}
%\usepackage{polyglossia}
%\setdefaultlanguage{spanish}
%\usepackage{color}
%\usepackage{array}
%\usepackage{supertabular}
%\usepackage{hhline}
%\usepackage{hyperref}
%\hypersetup{colorlinks=true, linkcolor=blue, citecolor=blue, filecolor=blue, urlcolor=blue}
%\usepackage{graphicx}
%% Text styles
%\newcommand\textstylefootnotereference[1]{\textsuperscript{#1}}
%\makeatletter
%\newcommand\arraybslash{\let\\\@arraycr}
%\makeatother
%\setlength\tabcolsep{1mm}
%\renewcommand\arraystretch{1.3}
%\newtheorem{theorem}{Theorem}
%\title{}
%\author{Humanidades}
%\date{2014-06-24}
%\begin{document}
%\clearpage\setcounter{page}{341}
%\bigskip
\thispagestyle{empty}	
\phantomsection
\addcontentsline{toc}{chapter}{La estancia profesional en el plan de estudios\newline de la Licenciatura en Historia. Facultad de\newline humanidades Universidad Autónoma\newline del Estado de México\newline $\diamond$
\normalfont\textit{Georgina Flores García y  Marcela J. Arellano González}}
{\centering {\scshape \large La estancia profesional en el plan de estudios de la licenciatura en Historia. Facultad de humanidades Universidad Autónoma del Estado de México\par}}
\markboth{formación del historiador}{estancia profesional}
\setcounter{footnote}{0}

\renewcommand*{\thefootnote}{\fnsymbol{footnote}}

\bigskip
\begin{center}
{\bfseries Georgina Flores García}\footnote{Historiadora. Docente de Tiempo
Completo de la Facultad de Humanidades, Miembro del Cuerpo Académico
Historia. Universidad Autónoma del Estado de México.}\\
{\bfseries Marcela J. Arellano González}\footnote{Licenciada en Historia,
Auxiliar en Proyectos de investigación de Historia. Facultad de
Humanidades, Universidad Autónoma del Estado de México.}\\
{\itshape Universidad Autónoma del Estado de México}
\end{center}

\bigskip
\textbf{Resumen}
\renewcommand*{\thefootnote}{\arabic{footnote}}
\setcounter{footnote}{0}
\enlargethispage{1\baselineskip}

Estancia profesional en Plan de Estudios de Historia es una Unidad de
Aprendizaje que tiene el mayor número de créditos de toda la currícula de
la licenciatura, con un correspondiente mayor número de horas 12, las
cuales de acuerdo al mapa curricular son prácticas al 100\,\%.


El objetivo general es <<Vincular a los alumnos con el mercado laboral,
contribuyendo a su formación integral. Los objetivos específicos son:
Formar alumnos competentes para servir a la sociedad desde su profesión.
Analizar los perfiles profesionales humanos que requiere y demanda la
sociedad. Responder a las exigencias sociales de nuestro tiempo. Canalizar
programas de prácticas profesionales a zonas prioritarias>>.

La forma en que se efectúa la Estancia Profesional en la Facultad de
Humani\-dades tiene más que ver con una práctica de la Unidad de Aprendiza\-je de
una de las Áreas de Acentuación que con los objetivos trazados en el Plan
de Estudios, porque a través de un ejercicio, de uno o dos días en el que
el estudiante no demuestra todas las competencias adquiridas a lo largo de
la licenciatura.

En este trabajo se desarrolla la forma en que se lleva a la práctica la Estancia
Profesional, y se formula una propuesta para el cambio de Plan de Estudios.\\
\textbf{Palabras clave:} Historia, Estancia Profesional, Plan de Estudios.

\medskip
\textbf{La estancia profesional en el Plan de Estudios de la Licenciatura en
Historia. Facultad de Humanidades Universidad Autónoma del Estado de
México}


Estancia profesional en Plan de Estudios de Historia es la Unidad de
Aprendizaje que tiene el mayor número de créditos de toda la currícula de
la licenciatura, con un correspondiente mayor número de horas aula, sin
embargo por estancia profesional pensamos que se debería entender lo que de
acuerdo al Diccionario de la Real Academia es una estancia.
\textit{Permanencia durante cierto tiempo en un lugar determinado},
complementando con lo profesional, comprenderíamos que el estudiante de
Historia tendría que mantenerse durante un tiempo, que a nuestro parecer
debiera ser un mínimo de tres meses, sin embargo la mayor parte del
semestre se ocupa en horas aula elaborando el proyecto de lo que se
realizará, con un Marco Referencial, uno Conceptual, Objetivos, Desarrollo
y Fuentes.
\enlargethispage{1\baselineskip}

La estancia profesional debe estar ligada con el área de acentuación que el
estudiante ha seleccionado desde el quinto semestre. Las áreas de
acentuación son: Docencia, Bibliotecas y Archivos, Procesos Editoriales,
Servicios Históricos culturales y asesorías y Medios de Comunicación
masiva. En la trayectoria académica ideal Estancia Profesional está ubicada
en la octava fase, es decir que de cinco Unidades de Aprendizaje que el
estudiante debe cursar en su área de acentuación, solamente ha cursado
tres, sin embargo la flexibilidad administrativa permite que realicen la
estancia en fases pares o nones, dejándola en muchos casos en el noveno
semestre, cuando solamente les falta una Unidad del área de acentuación, lo
malo es el caso de quienes la presentan en el séptimo, cuando solamente han
abordado lo elemental y no tienen los suficientes elementos para poder
hacer un proyecto y mucho menos desarrollarlo en institución alguna. 


De acuerdo al Currículum 2004 de la Licenciatura en Historia de la
Universidad Autónoma del Estado de México, el objetivo general de Estancia
Profesional es: <<Vincular a los alumnos con el mercado laboral,
contribuyendo a su formación integral>>, los objetivos específicos son:
\begin{Obs}
\item[$\bullet$] Formar alumnos competentes para servir a la sociedad desde su profesión.
Analizar los perfiles profesionales humanos que requiere y demanda la
sociedad.
\item[$\bullet$] Responder a las exigencias sociales de nuestro tiempo. 
\item[$\bullet$] Canalizar programas de prácticas profesionales a zonas prioritarias.
\end{Obs}

Estos objetivos se lograrán con los siguientes contenidos: <<Desarrollar
competentemente su función de acuerdo al lugar en donde realicen su
práctica profesional.

Requiere la formación de un taller, la elaboración de un proyecto previo y
su implementación en la práctica profesional en: docencia, bibliotecas y
archivos, procesos editoriales, servicios históricos culturales y asesorías
y medios masivos de comunicación.

La práctica profesional es un espacio importante en la vida del estudiante
universitario, porque se enfrentará a problemáticas reales de acuerdo con su
perfil profesional.
\enlargethispage{1\baselineskip}

El alumno realizará su práctica profesional con una asesoría guiada y
supervisada.

El aprendizaje, aplicado a su práctica es responsabilidad del alumno.

Su práctica profesional puede ser disciplinaria, inter y
multidisciplinaria.

La práctica profesional se desarrolla fuera del espacio de la Facultad de
Humanidades.

La práctica profesional vincula al alumno con el contexto de aprendizaje
ubicado en situaciones reales, familiarizándose con el quehacer de su
profesión>> (Currículum 2004: 140--141).

El Reglamento de Estudios Profesionales en su capítulo III Art. 52° fracción
VI entiende por <<Práctica o estancia profesional. <<Actividad académica
obligatoria que el alumno deberá realizar en ámbitos reales de desempeño
profesional, para integrar y aplicar los conocimientos adquiridos.
\enlargethispage{1\baselineskip}

Art.54° Las prácticas o estancias profesionales previstas en la fracción VI
del art. 52° del presente reglamento se ajustarán a los criterios
siguientes:
\begin{Obs}
\item[I.-] Ser congruentes con los objetivos del programa educativo.
\item[II.-] Respaldarse en convenios institucionales de colaboración y en los
acuerdos operativos.
\item[III.-] Realizarse en los últimos periodos escolares del plan de estudios, en
organizaciones de los sectores público, privado o social.
\item[IV.-] Efectuarse en un plazo no menor de seis meses ni mayor de un año.
\item[V.-] Tener una duración mínima de 280 horas para estudios técnico profesional y
de 480 horas en los estudios de licenciatura.
\item[VI.-] Tener valor en créditos como parte del plan de estudios.
\end{Obs}

Si atendemos a lo que señala la Legislación universitaria, no se cumple
nada, porque curricularmente se tienen programadas 192 horas al semestre,
de las cuales por lo menos ciento ochenta se dedican a la realización del
proyecto y en el mejor de los casos diez se llevan a cabo en la institución
en la que supuestamente realizarían la Estancia. Esto no se paga, mientras
que las prácticas profesionales que son solicitadas directamente por
escuelas o empresas, sí son remuneradas.


¿Qué hacen los estudiantes en la Unidad de Aprendizaje Estancia Profesional?
De entrada son atendidos por un solo titular de la Unidad de Aprendizaje
que debe justificar doce horas, por lo que divide en tres las doce horas
miércoles y sábado, el primer día cuatro horas matutinas y cuatro
vespertinas y el segundo día cuatro horas matutinas, en esas horas en forma
individualizada y en grupos diferentes, atiende a cada estudiante viendo la
actividad que desarrollará, y pidiéndole busque un asesor del proyecto, con
quien trabajará hasta el final del semestre.


El titular de la Unidad de Aprendizaje revisa forma, redacción del proyecto
que el estudiante presenta. Es el único docente que tiene al cien por
ciento de los estudiantes de la licenciatura en su y es el único que la ha
impartido desde que entró en vigencia el actual Plan de Estudios.

\enlargethispage{1\baselineskip}
Hablaremos desde la experiencia como asesoras y asesoradas de proyectos de
estancia, en primer término los estudiantes asisten no más de cuatro horas
a la semana al avance de su proyecto de estancia, en el que se corrige
redacción, ortografía y se asesora principalmente\linebreak acerca de la forma de su
proyecto, con cuestionamientos que detonan en algunos casos reflexión sobre
lo que están proyectando para desarrollar.

Los estudiantes del área de acentuación de docencia por lo general eligen
dar una o dos sesiones del tema que les indiquen en una escuela
preparatoria o en una secundaria; tomando en cuenta que en ambas se dan 50 minutos
de clase, hablamos de que la verdadera estancia en el lugar del ejercicio
práctico es de cien minutos, en el mejor de los casos de ciento cincuenta
por tres sesiones, es decir que hablamos de menos de tres horas para 480
que pide la legislación o para 192 que señala en Plan de Estudios.


¿Es una Estancia Profesional? La inversión de tiempo que se hace en las
sesiones \textit{teóricas} es mayor al que se efectúa en el lugar de la
estancia, por ejemplo para las clases se prepara un plan para cada una de
ellas, este debe llevar encabezado con los datos institucionales, datos de
identificación grado, grupo, responsable, fecha y hora. Datos del tema
título, propósitos u objetivos, dependiendo del Plan de Estudios: Objetivos
específicos o competencias. Datos metodológicos: Método, procedimiento,
técnica, estrategias. Recursos materiales: material didáctico. Síntesis
temática, Forma de evaluar y Bibliografía. (Ver Anexo~1)


Los objetivos de su proyecto los redacta uno de los estudiantes el siguiente
es un ejemplo:
%\enlargethispage{1\baselineskip}

\medskip
\textbf{OBJETIVOS}

{\bfseries Generales:} Aplicar los conocimientos y competencias adquiridas durante las
unidades de aprendizaje de docencia. Obtener mayor dominio del modelo por
competencias. Formular el plan de clase bajo la perspectiva de competencias
que permite el estudio de la Historia. Aplicar en la práctica docente los
conocimientos y habilidades adquiridos en las unidades de aprendizaje del
área de acentuación de\linebreak  docencia, específicamente; teoría y método de
enseñanza-aprendizaje y habilidades para la docencia.  

La distribución del tiempo que el estudiante le destina a la Estancia, es
mínimo. Señalamos con cursivas y negritas lo que en realidad sería la
Estancia, este cronograma corresponde a otra estudiante:

\medskip
\begin{small}
\begin{flushleft}
\tablefirsthead{}
\tablehead{}
\tabletail{}
\tablelasttail{}
\begin{supertabular}{|m{38.5mm}|m{16.5mm}|m{14mm}|m{11mm}|m{11mm}|m{11mm}|}
%\begin{supertabular}{|l|c|c|c|c|c|}
\hline
\rowcolor{lsLightBlue}{\bfseries Actividad\slash{}Mes} &
{\bfseries Febrero} &
{\bfseries Marzo} &
{\bfseries Abril} &
{\bfseries Mayo} &
{\bfseries Junio}\\\hline
{\bfseries \textmd{Identificación del curso}} & \cellcolor{lsLightGray}{} & ~ & ~ & ~ & ~\\\hline
{\bfseries \textmd{Elección del tema y formulación de objetivos}} &
\cellcolor{lsLightGray}{}
 &
~
 &
~
 &
~
 &
~
\\\hline
{\bfseries \textmd{Justificación social, académica y teórica.}} & \cellcolor{lsLightGray}{} &
\cellcolor{lsLightGray}{}
 &
~
 &
~
 &
~
\\\hline
{\bfseries \textmd{Formulación de planes de clases}} & ~ & 
\cellcolor{lsLightGray}{} &
\cellcolor{lsLightGray}{}
 &
~
 &
~
\\\hline
{\bfseries \textmd{Revisión y corrección}} &
~
 &
~ 
 &
\cellcolor{lsLightGray}{}
 &
\cellcolor{lsLightGray}{}
 &
~
\\\hline
{\bfseries \textmd{Realización de la Estancia}} &
~
 &
~
 &
~
 &
\cellcolor{lsLightGray}{}
 &
~
\\\hline
{\bfseries \textmd{Presentación de evidencias}} &
~
 &
~
 &
~
 &
~
 &
\cellcolor{lsLightGray}{}
\\\hline
\end{supertabular}
\end{flushleft}
\end{small}

\bigskip
No solamente presentan la Estancia Profesional con sesiones docentes, ha
habido otras formas, por ejemplo una línea del tiempo musical, este es una
caso muy interesante, porque los jóvenes realizaron la línea grabando un
disco para la Unidad de aprendizaje Historia e Historiografía de México
primera mitad de siglo XX, con la colaboración de los estudiantes de Artes
Teatrales, quienes tocaron y cantaron, mientras que los estudiantes de
Historia daban las explicaciones pertinentes. Esta actividad les satisfizo
a tal grado que la desarrollaron como su Estancia Profesional,
presentándose en diferentes foros, inclusive en otros estados del país.


De igual forma otro equipo de trabajo organizó el Coloquio Homenaje a
Enrique Semo, y aunque no fue idea de ellos, organizaron, diseñaron el
cartel, los programas, difundieron, y el evento fue un éxito, debido en
gran parte al proyecto de Estancia de los jóvenes.


Por último queremos señalar a otro equipo que por primera vez se conformó
por dos estudiantes de dos diferentes áreas de acentuación: Docencia y
Patrimonio Cultural y artístico, ellos organizaron el Primer Coloquio de
Historia Bélica, planearon, desarrollaron y evaluaron el evento en todos
los aspectos, fue una experiencia que permitió la práctica de la unión de
dos áreas distintas que hicieron comprobar a los estudiantes lo integral
del núcleo profesionalizante.


\bigskip
\textbf{Referencias}

Plan de Estudios de la Licenciatura en Historia, \textit{Currículum 2004}.
Universidad Autónoma del Estado de México. Facultad de Humanidades. 179
páginas.


Reglamento de Estudios Profesionales de la Universidad  Autónoma del Estado
de México. \textit{Gaceta Universitaria} Núm. 15, Enero 2008, Época XII, Año
XXIV.
\newpage

\begin{footnotesize}
\textbf{ANEXO N° 1}

{Plan de clase }

{\textbf{Escuela Secundaria Federalizada }}

{\textbf{Andrés Molina Enríquez }}

{Grado: 2° }

{Grupo: E }

{Docente: Zaira Garcés Gómez }

{Unidad de Aprendizaje: Historia Universal }

{Miércoles 6 de Noviembre del 2013 }

{HORA: 8:20 am }

{TIEMPO DE DURACIÓN: 50 minutos }

{OBJETIVOS: El discente identificará las principales características en el
que surge Renacimiento, mediante un mapa conceptual. }

{El estudiante comparará la perspectiva de dos autores que aborden el
renacimiento. }

{SÍNTESIS: }

{El período del renacimiento, se le conoció como el movimiento cultural que
produjo cambios en Europa. Los cambios que presento esta época fueron en
todos os ámbitos es decir, arte, ciencia, tecnología y literatura. }

{Lo que se buscó en el Renacimiento fue retomar los clásicos principalmente
de las dos civilizaciones antiguas más influyentes como Grecia y Roma, es
por eso que {no era de esperarse que
surgiera en Italia. Durante el Renacimiento surgió una corriente cultural,
que fue }{el Humanismo, dándole gran
peso a la figura del hombre como el centro de todo y dedicarse su
pensamiento, imagen y anatomía. Durante este movimiento cultural, los
cambios permitieron ampliar la visión del hombre con los nuevos
descubrimientos, que se pudieron dar gracias al avance tecnológico. }}

{{MÉTODO: Análisis síntesis }}

{{TÉCNICA: Expositiva }}

{{EVALUACIÓN: El estudiante identificará
mediante una sopa de letras las principales características del
Renacimiento. }}

{{MATERIAL DIDÁCTICO: Mapa mental y sopa
de letras. }}

{{BIBLIOGRAFÍA: }}

{{Velázquez, Jorge
(1998}\textit{{). ¿Qué es el
renacimiento? La idea del renacimiento en la conciencia histórica de la
modernidad, }}{Universidad Autónoma
Metropolitana Iztapalapa, México, 210 pp. }}


Romano, Ruggiero y Alberto (1999). \textit{Los fundamentos del mundo
moderno, edad media. tardía, reforma, renacimiento}, Volumen 12, Siglo
veintiuno, México, 327 pp.


\bigskip
{\centering  \includegraphics{p16-img001.jpg} \par}

\bigskip
{\centering \textbf{PRIMER COLOQUIO DE HISTORIA BÉLICA A CIEN AÑOS 
DEL INICIO DE LA PRIMERA GUERRA MUNDIAL}
\par}
El presente coloquio de historia tiene la finalidad de discutir, comprender
y reflexionar los procesos históricos nacionales e internacionales en campo
de la historia militar, pues este ramo de la historia no ha sido
completamente desarrollado en la historiografía mexicana, principalmente a
que la comunidad intelectual ha desdeñado los estudios histórico-militares
de México y una interpretación de historiadores mexicanos sobre sucesos
bélicos a nivel de historia universal.


Es así, que se pretende con este espacio dar a conocer los principales
puntos de vista de la comunidad estudiantil sobre los procesos militares
más importantes a lo largo de la Historia de México y la Historia
Universal, no tan solo de estudiantes en Historia sino también estudiantes
de todas las carreras universitarias los cuales estén interesados en
discutir tales procesos.

\medskip
\textbf{Justificación social}
\enlargethispage{1\baselineskip}

Este proyecto está fundamentado en el \textit{Plan de Desarrollo 2013-2018
del Gobierno de la República}, en el apartado III \textit{México con
Educación de Calidad} en lo referente a cultura:

\begin{quotation}
Una sociedad culturalmente desarrollada tendrá una mayor capacidad para
entender su entorno y estará mejor capacitada para identificar
oportunidades de desarrollo. […] Para que la cultura llegue a más mexicanos
es necesario implementar programas culturales con un alcance más amplio.
Sin embargo, un hecho que posiblemente impida este avance es que las
actividades culturales aún no han logrado madurar suficientemente para que
sean autosustentables. […] Por otro lado, dado que la difusión cultural
hace un uso limitado de las tecnologías de la información y la
comunicación, la gran variedad de actividades culturales que se realizan en
el país, lo mismo expresiones artísticas contemporáneas que manifestaciones
de las culturas indígenas y urbanas, es apreciada por un número reducido de
ciudadanos.\footnote{Gobierno de la República, \textit{Plan de Desarrollo
2013-2018},  Gobierno de la República, México, 2013, p. 63}
\end{quotation}

%\medskip
\textbf{Justificación académica}


El presente proyecto se fundamenta en el \textit{Plan de estudios de la
licenciatura en historia: currículum 2004}, cuya matriz de competencias en
docencia, también correspondiente a la competencia de docencia histórica
del área de servicios histórico-culturales y asesorías, las cuales
establecen en su apartado dedicado a divulgación educativa y despacho de
asesorías, respectivamente, que el egresado en historia debe:

\begin{quotation}
Manejar educación continua: Diseñar, organizar e impartir: conferencias,
ciclos de conferencias, cursos, diplomados, diplomados superiores y
especialidades con reconocimiento de grado académico. Además de las
universidades se imparte en Centros culturales, museos, archivos,
dependencias de gobierno e instituciones
privadas.\footnote{Universidad Autónoma del
Estado de México, Facultad de Humanidades. \textit{Plan de estudios de la
Licenciatura en Historia}: currículum 2004, UAEM, Toluca, 2004,\linebreak  pp. 96--97.}
\end{quotation}

\vspace{3cm}
\begin{center}
\includegraphics{p16-img002.jpg} 
\end{center}
\newpage

\bigskip
{\centering  \includegraphics{p16-img003.jpg} \par}

\bigskip
{\centering\includegraphics{p16-img004.jpg} \par}
\end{footnotesize}
\newpage
\thispagestyle{empty}
\phantom{abc}