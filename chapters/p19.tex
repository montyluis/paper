% This file was converted to LaTeX by Writer2LaTeX ver. 1.3.1
% see http://writer2latex.sourceforge.net for more info
%\documentclass{article}
%\usepackage{amsmath,amssymb,amsfonts}
%\usepackage{fontspec}
%\usepackage{xunicode}
%\usepackage{xltxtra}
%\usepackage{polyglossia}
%\setdefaultlanguage{english}
%\usepackage{color}
%\usepackage{array}
%\usepackage{hhline}
%\usepackage{hyperref}
%\hypersetup{colorlinks=true, linkcolor=blue, citecolor=blue, filecolor=blue, urlcolor=blue}
%\newtheorem{theorem}{Theorem}
%\title{}
%\author{Socorro Aguayo}
%\date{2014-06-15}
%\begin{document}
%%\clearpage\setcounter{page}{409}
\thispagestyle{empty}
\phantomsection{}
\addcontentsline{toc}{chapter}{El Programa de Tutorías en la Licenciatura\newline en Historia de la Universidad Autónoma\newline de Ciudad Juárez\newline $\diamond$
\normalfont\textit{María Socorro Aguayo Ceballos y Ana Karent Muñoz Chávez}}
{\centering {\scshape \large El programa de tutorías en la licenciatura en historia\\ de la
Universidad Autónoma de Ciudad Juárez}\par}
\markboth{la formación del historiador}{programa de tutorías}
\setcounter{footnote}{0}

\bigskip
\begin{center}
{\bfseries María Socorro Aguayo Ceballos\\
Ana Karent Muñoz Chávez}\\
{\itshape Universidad Autónoma de Ciudad Juárez}
\end{center}

\bigskip
\textbf{Resumen}

La tutoría, un programa institucional que aún no logra los objetivos
planteados, al menos no en la Licenciatura en Historia que se imparte en la
 UACJ donde los docentes la ven como una imposición o como un requisito a
cumplir cuando se pretende obtener el perfil PROMEP; otro \ «aliciente» \ para
llevarla a cabo es lograr un buen nivel en el Pro\-gra\-ma de Estímulos al
Desempeño Docente, cuyo reglamento establece determinado número de horas de
tutorías para  acceder al mismo.


Los tutores afirman que no pueden obligar a los alumnos a asistir a las
sesiones de tutoría pero éstos opinan diferente; en una encuesta realizada
tanto a los docentes como a los alumnos se encontraron resultados
reveladores y es lo que se presentará en esta ponencia.\\
\textbf{Palabras clave:} tutoría, requisito, alumnos, encuesta, resultados.


\bigskip
\textbf{Abstract}

Tutoring, an institutional program that has not yet achieved the stated
objectives, at least not in the history program that is being taught at
UACJ where teachers see tutoring as an imposition or as a requisite to
fulfill for getting the PROMEP profile. Another «incentive « for doing
tutoring is to achieve a good level on the Teachers Development Stimuli
Program (Programa de Estímulos al Desempeño Docente) where its stated that
teachers have to complete a certain number of tutoring hours to get the
stimuli.


Tutors affirm that they cannot force their students to attend tutoring
sessions, but students think differently. On a questionnaire applied to
both (professors and students) important results came out, and these
results will be presented at this conference.\\
\textbf{Keywords:} tutoring, requisite, students, questionnaire, results.


\bigskip
\textbf{Antecedentes}

El programa institucional de tutorías fue propuesto por la a Asociación
Nacional  de Universidades e Instituciones de Educación Superior (ANUIES) 
como una alternativa para contribuir a abatir los problemas de deserción y
rezago en la educación superior y como una estrategia para mejorar la
calidad de la misma (ANUIES 2000). En el texto publicado por la misma
asociación  se definen  claramente los conceptos, enuncia los compromisos
de cada uno de los participantes y  propone la organización y operación de
las acciones  para la implantación  de dicho programa.


Explicar de forma detallada los siete capítulos que integran el texto antes
mencionado no es precisamente el propósito de esta presentación así que
solamente se retomarán algunos aspectos breves del mismo cuando sea
necesario ahondar en conceptos clave para analizar el programa de tutorías
de la Licenciatura en Historia en la Universidad Autónoma de Ciudad Juárez
(UACJ). Además de recurrir a otros autores que han escrito temas
relacionados a la tutoría.


Catorce años después de iniciar este proyecto a nivel nacional y a 10 de que
se implementó  en la UACJ surge la inquietud acerca de si se han logrado
los objetivos propuestos en dicho programa, es por lo anterior que se
realizó una  investigación con los alumnos para conocer su opinión respecto
a la tutoría;  se eligió  a los estudiantes de la Licenciatura  en Historia
por las siguientes razones:

\begin{Obs}
\item[1.] La población total  de alumnos inscritos es de 60 y por lo tanto fue más
sencillo localizarlos y  solicitar que respondieran la encuesta.
\item[2.] El programa cuenta con una planta docente de 22 profesores,\linebreak 12 de ellos
tienen la categoría de tiempo  y son los que tienen asignada la actividad
de las tutorías.
\item[3.] Es un programa que solamente se ofrece cada año  por eso la cantidad de
alumnos es baja y ello permite ubicarlos y conocerlos más.
\end{Obs}

\medskip
\textbf{Metodología}

La ANUIES diseñó un instrumento para evaluar el desempeño en la tutoría y lo
incluyó en el  anexo 3 del texto mencionado anteriormente,  para llevar a
cabo nuestra investigación retomamos ese cuestionario y solamente le
agregamos  seis preguntas más para intentar recabar información muy
específica  relacionada con la Licenciatura en Historia y para relacionarla
con  el sistema de captura  y validación de tutorías que se utiliza en la
UACJ.


Se respetó la escala Likert utilizada en el cuestionario original y
posteriormente se convirtió a escala ordinal para procesar los resultados
con el SPSS. Además se tomó nota de los comentarios expresados por los
alumnos al momento de llenar el cuestionario pues se consideró que eran
relevantes para analizar sus respuestas, es por ello que se incluyen en
algunas de ellas.


De la población total de estudiantes únicamente 30 aceptaron
responder el cuestionario, 12 rechazaron hacerlo y además argumentaron que
no tenía caso hacerlo porque de nada iba a servir lo que estábamos haciendo
pues su opinión nunca era tomada en cuenta por la coordinación del programa
(estas opiniones llevarán a otro tema que no corresponde al presente
trabajo), los restantes no respondieron a nuestra solicitud.


\medskip
\textbf{Resultados}

A continuación se presentan los resultados obtenidos en la encuesta y se
analizan los mismos, posteriormente se elaborará la sección de conclusiones
y comentarios.


En la primera parte se incluyeron 3 preguntas de tipo \textit{signalíctico} para
caracterizar a las personas encuestadas, estos datos son: edad, sexo y
semestre, esta información es necesaria para conocer cómo está distribuida
la población total que cursa esta licenciatura.

%\medskip
De las personas encuestadas el 60\,\% corresponde al sexo masculino y  el 40\,\%
al femenino, respecto  a la edad, se distinguen 3 grupos donde 2 personas
son menores de 20 años, 9 se ubican en el rango de 21 a 30 y  2 tienen
entre  31 y 40. En este aspecto es necesario mencionar que ha habido un
cambio notorio en la edad de quienes deciden estudiar esta carrera pues
anteriormente predominaban los estudiantes de más de 30 años ya que eran
personas que tenían una profesión y ejercían la misma pero decidían cursar
otra licenciatura como una satisfacción o gusto personal, además que esta
carrera no tenía el estatus de profesionalizante, es decir, además del
perfil de egreso no incluía un perfil profesional que estableciera los
conocimientos y habilidades necesarios para ejercer la profesión de
historiador; es a partir del rediseño curricular  realizado en 2011 que
adquiere esa característica  y que se modifica el tipo de estudiantes de
nuevo ingreso.


La carrera está diseñada para que se concluya en 9 semestres y encontramos
que existen  6 alumnos que se encuentran cursando desde el décimo hasta el
décimo cuarto; los restantes 26 se distribuyen de la siguiente manera:
segundo, 8; tercero 1; quinto, 1, sexto, 6; séptimo, 3; octavo, 1 y noveno,
4. Al revisar los mapas curriculares de la población estudiantil detectamos
que hay trayectorias erráticas, existen alumnos que aún no concluyen el
nivel principiante pero que inscriben materias de nivel intermedio e
inclusive de avanzado.


Las preguntas se transcriben tal y como están redactadas en el cuestionario
que se entregó a los alumnos.


\textit{Pregunta 1}: muestra el tutor disposición para atender a los alumnos,
las respuestas se dividieron solamente en tres: 16 (53.3\,\%) dijeron estar
totalmente de acuerdo, 11 (36.7\,\%) de acuerdo y 3 (10\,\%) más o menos de
acuerdo. El que 27 alumnos hayan elegido las primeras dos respuestas
permite saber que esta característica del tutor es vista como algo positivo
por quienes participan en el programa.


\textit{Pregunta 2}: La cordialidad y capacidad del tutor para lograr un
clima de confianza para que el alumno pueda exponer sus problemas obtuvo
los siguientes resultados, totalmente de acuerdo 13 (43.3\,\%), de acuerdo 10
(33.3\,\%) y más o menos de acuerdo 7 (23.3\,\%). En estos aspectos también la
mayoría considera que se cumplen pero llama la atención que se incrementa
el número de estudiantes que eligieron la opción \textit{más o menos de
acuerdo}.


\textit{Pregunta 3}: trata el tutor con respeto y atención a los alumnos,
las respuestas tuvieron una modificación pues los datos fueron los
siguientes: 21 (70\,\%) dice estar totalmente de acuerdo, 5 (16.7\,\%) de
acuerdo, 2~(6.7\,\%) más o menos de acuerdo y otros 2~(6.7\,\%) eligieron la opción en
desacuerdo. La percepción acerca del respeto  presentó variantes pues
aparece la respuesta \textit{en desacuerdo} y aunque el número de
estudiantes que eligió esta opción puede considerarse bajo si es de llamar
la atención.


\textit{Pregunta 4}: muestra el tutor interés por los problemas
académicos y personales que afectan el rendimiento de los alumnos arrojó la
siguiente información: 13~(43\,\%) respondió totalmente de acuerdo;\linebreak 11~(36.7\,\%) 
de acuerdo y 6~(20\,\%) más o menos de acuerdo.


Las respuestas siguen siendo consideradas dentro de los rangos aceptables
pero el número de los que eligieron \textit{más o menos de acuerdo} se
incrementó y se considera necesario continuar indagando el porqué de esta
situación.


\textit{Pregunta 5}: muestra el tutor capacidad para escuchar los
problemas de los alumnos, esta fue valorada así: 10~(33\,\%) totalmente de
acuerdo, 14~(46.7\,\%) de acuerdo, 5~(16.7\,\%) más o menos de acuerdo y 1~(3.3\,\%) en desacuerdo. Otras 24 personas consideran que los escuchan pero una
considera que no, se revisaron los cuestionarios para ver si era la misma
que muestra descuerdo en otras respuestas pero no corresponde, es decir,
son diferentes alumnos los que muestran desacuerdo en algunas de las
preguntas.


\textit{Pregunta 6}: muestra el tutor disposición para mantener una
comunicación permanente con el alumno fue valorada de la siguiente manera:
11 (36.7\,\%) totalmente de acuerdo, 11~(36.7\,\%) de acuerdo, 7~(23.3\,\%) más o
menos de acuerdo y 1 (3.3\,\%) en desacuerdo. Aquí se puede observar que
existe semejanza con las respuestas de la pregunta \mbox{anterior}.
\newpage

\textit{Pregunta 7}: tiene el tutor capacidad para resolver dudas
académicas de los alumnos obtuvo lo siguiente: 17~(56.7\,\%) totalmente de
acuerdo, 7~(23.3\,\%) de acuerdo y 6~(20\,\%) más o menos de acuerdo. En el
aspecto académico todos respondieron en los parámetros aceptables pues el
80\,\% reconoce este aspecto en su tutor.


\textit{Pregunta 8}: tiene el tutor capacidad para orientar al alumno en
metodología y técnicas de estudio fue respondida con los siguientes datos:
15 (50\,\%) totalmente de acuerdo, 7 (23.3\,\%) de acuerdo, 7 (23.3\,\%) más o
menos de acuerdo y 1 (3.3\,\%) en desacuerdo. Aquí vuelve a aparecer la
respuesta \textit{en desacuerdo} aunque el 57\,\% considera que si existe
esta capacidad en los tutores.


\textit{Pregunta 9}: Tiene el tutor la capacidad para diagnosticar las
dificultades y realizar las acciones pertinentes para resolverlas dio estos
datos: 9~(30\,\%) totalmente de acuerdo, 13~(43.3\,\%) de acuerdo, 6~(20\,\%) más
o menos de acuerdo y 2~(6.7\,\%) en desacuerdo.


Al parecer algunos estudiantes no consideran que los tutores se percaten de
sus dificultades pues 8 respondieron entre \textit{más o menos de acuerdo y
en desacuerdo}.


\textit{Pregunta 10}: tiene el tutor capacidad para estimular el estudio
independiente fue evaluada de la siguiente forma: 8 (26.7\,\%) totalmente de
acuerdo, 12 (40\,\%) de acuerdo, 8 (26.7\,\%) más o menos de acuerdo y 2
(6.7\,\%) en desacuerdo. Se considera que esta respuesta está relacionada con
la de la pregunta anterior pues de igual manera 8 respondieron entre
\textit{más o menos de acuerdo y en desacuerdo}.


\textit{Pregunta 11}: posee el tutor formación profesional en su
especialidad, estos son los datos: 17 (56.7\,\%) totalmente de acuerdo, 10
(33.3\,\%) de acuerdo y 3 (10\,\%) más o menos de acuerdo. Las respuestas a
esta pregunta reflejan la formación que tienen los docentes que imparten
clases en la licenciatura pues de los doce profesores de tiempo completo
solamente dos cursaron licenciatura en historia, cuatro en sociología, dos
en antropología social y uno en cada una de las siguientes áreas:
filosofía, derecho, educación y etnohistoria.


\textit{Pregunta 12}: posee el tutor dominio de métodos pedagógicos para la
atención individualizada o grupal, las respuestas son: 6 (20\,\%) totalmente
de acuerdo, 14 (46.7\,\%) de acuerdo, 9 (30\,\%) más o menos de acuerdo y 1
(3.3\,\%) en descuerdo. Los alumnos en general comentan este aspecto más con
relación a sus clases que a la tutoría.


\textit{Pregunta 13}: es fácil localizar al tutor que tiene asignado, obtuvo
lo siguiente: 9 (30\,\%) totalmente de acuerdo, 13 (43.3\,\%) de acuerdo, 
7~(23.3\,\%) más o menos de acuerdo y 1 (3.3\,\%) en desacuerdo. Estas respuestas
reflejan la asistencia y permanencia de los docentes en la universidad pues
al ser tiempos completos cuentan con un cubículo donde es fácil
localizarlos durante los horarios de labores.

\begin{sloppypar}
\textit{Pregunta 14}: el tutor conoce suficientemente bien la normatividad
institucional para aconsejarle las opciones adecuadas a sus problemas
escolares, fue respondida así: 9 (30\,\%) totalmente de acuerdo, 13~(43.3\,\%)
de acuerdo, 7~(23.3\,\%) más o menos de acuerdo y 1~(3.3\,\%) en desa\-cuer\-do.
Los alumnos que respondieron el cuestionario comentaron que en realidad son
pocas las ocasiones en que requieren de este tipo de información, pues
cuando es necesario acuden con las secretarias y son ellas quienes les
indican qué hacer.
\end{sloppypar}

\enlargethispage{1\baselineskip}
\textit{Pregunta 15}: la orientación recibida por parte del tutor ha
permitido realizar una selección adecuada de cursos y créditos. Los
resultados son: 7 (23.3\,\%) totalmente de acuerdo, 13 (43.3\,\%) de acuerdo,\linebreak 
9~(30\,\%) más o menos de acuerdo y 1 (3.3\,\%) en desacuerdo. Veinte alumnos
dudaron para responder esta pregunta pues dijeron que aun y cuando el tutor
les indique cuáles cursos deben seleccionar esto se pierde cuando consultan
los horarios de las clases y en ocasiones terminan eligiendo algo
totalmente distinto a lo sugerido; los otros diez comentaron que no siempre
acuden con su tutor para consultar esto.


\textit{Pregunta 16}: el tutor lo canaliza a las instancias adecuadas
cuando tiene algún problema que rebase su área de acción. Solamente
responda esta pregunta si lo ha requerido. Aquí los resultados variaron por
la condición enunciada en la misma pregunta: 15 (50\,\%) la dejaron sin
respuesta, 3 (10\,\%) totalmente de acuerdo, 8 (26.7\%) de acuerdo y 4~(13.3\,\%) 
más o menos de acuerdo. Los alumnos que sí respondieron dijeron
que básicamente les canalizaron a las áreas administrativas y finalmente
fue la secretaria quien los atendió, ninguno mencionó otras áreas, tales
como Centro de Orientación y Bienestar Estudiantil (COBE), Unidad de
Atención Médica Inicial (UAMI), etc\@.


\textit{Pregunta 17}: su participación en el programa de tutoría ha
mejorado su desempeño académico, aquí se obtuvieron estos resultados: 8~(26.7\,\%) 
totalmente de acuerdo, 12 (40\,\%) de acuerdo, 7 (23.3\,\%) más o
menos de acuerdo y 3 (10\,\%) en desacuerdo. En esta pregunta llama la
atención que el 33\,\% ubicó su respuesta entre \textit{más o menos de
acuerdo} y \textit{en desacuerdo} pues éste es uno de los objetivos básicos
del programa de tutoría y se esperaría que el mayor porcentaje se ubicara
en \textit{totalmente de acuerdo}. 

\enlargethispage{1\baselineskip}
\textit{Pregunta 18}: su integración a la universidad ha mejorado con
el programa de tutoría. Las respuestas fueron: 7 (23.3\,\%) totalmente de
acuerdo, 13 (43.3\,\%) de acuerdo, 7 (23.3\,\%) más o menos de acuerdo y\linebreak 3
(10\,\%) en desacuerdo. Aunque 20 alumnos respondieron de manera\linebreak positiva
vuelve a aparecer el porcentaje que ubica su opinión entre \textit{más o
menos de acuerdo y en desacuerdo}.


\textit{Pregunta 19}: es satisfactorio el programa de tutoría, esto
fue evaluado con estas cantidades: 11 (36.7\,\%) totalmente de acuerdo, 10
(33.3\,\%) de acuerdo, 7 (23.3\,\%) más o menos de acuerdo y 2 (6.7\,\%) en
desa\-cuer\-do. El 70\,\% de los encuestados considera que sí es satisfactorio
pero nuevamente se presenta el 30\,\% que no lo considera así.


\textit{Pregunta 20}: el tutor que le fue asignado es el adecuado a lo que
respondieron de esta forma: 8 (26.7\,\%) totalmente de acuerdo, 8 (26.7\,\%) de
acuerdo, 13 (43.3\,\%) más o menos de acuerdo y 1 (3.3\,\%) en des\-acuer\-do. Al
parecer el hecho de que su tutor les sea asignado por la coordinación de la
licenciatura no es del agrado de los estudiantes pues 14 de ellos ubicaron
su respuesta en \textit{más o menos de acuerdo y en desacuerdo}.


\medskip
Las siguientes son las preguntas que se agregaron al cuestionario original.


\textit{Pregunta 21}: le gustaría elegir a su tutor, es decir, que no
se le asigne en la coordinación del programa. Respondieron esto: 26
(86.7\,\%) totalmente de acuerdo, 1 (3.3\,\%) de acuerdo y 3 (10\,\%) más o menos
de acuerdo. Esta opción resulto ser más atractiva para los estudiantes pues
26 respondieron \textit{totalmente de acuerdo}.

\enlargethispage{1\baselineskip}
\textit{Pregunta 22}: a cuántas sesiones de tutoría acude durante el
semestre (aquí se establecieron rangos para facilitar el procesamiento de
la información). Los resultados fueron: 3 (10\,\%) sin respuesta, 7 (23.3\,\%)
acudieron solamente de una a dos veces, 8 (26.7\%) asistieron de tres a
cuatro ocasiones, 4 (13.3\,\%)  de cinco a seis, 1 (3.3\,\%)  fue a siete
reuniones y 7 (23.3\%) lo hicieron más de ocho veces durante el\linebreak semestre.
Las respuestas a esta pregunta tienen  relación con la siguiente pues
quienes asistieron mayor cantidad de veces a la tutoría realmente lo
hicieron para asesoría de tesis.

\textit{Pregunta 23}: subraye los motivos por los que acude a tutoría,
el listado que se incluyó corresponde con el de la plataforma o sistema
digital en el que se captura esta actividad en la UACJ, también se hizo una
agrupación para manejo de los datos. Los resultados fueron:\linebreak 25 (83.3\,\%)
acudieron por asuntos relacionados con asesoría académica, tesis,
orientación educativa, plan de estudios; 3 (10\,\%) por información general,
trámites, servicio social y 2 (6.7\,\%) por ser beneficiarios del Programa
Nacional de Becas (PRONABES) y éste establece como requisito asistir a 5
sesiones de tutoría durante el semestre, la opción denominada
\textit{relaciones personales} no tuvo respuestas.


\textit{Pregunta 24}: su tutor es docente de la Licenciatura en
Historia.\linebreak El 29~(96.7\,\%) respondió de manera afirmativa y solamente 1~(3.3\,\%)
tiene un tutor de otro programa pero no indicó a cuál pertenece. En este
aspecto los alumnos comentaron que les gustaría elegir tutor de otros
programas pero como se los asignan en la coordinación pues no tienen esa
opción.


\textit{Pregunta 25}: usted es quién solicita la tutoría. Estos son
los datos: 14~(46.7\,\%) totalmente de acuerdo, 7~(23.3\,\%) de acuerdo, 6~(20\,\%) 
de acuerdo, 1~(3.3\,\%) en desacuerdo, 2~(6.7\,\%) totalmente en
desacuerdo. Más del 50\,\% respondió de manera afirmativa pero comentaron que
no siempre son atendidos cuando acuden y por eso dejan de ir. Los que
respondieron \textit{en desacuerdo} y \textit{totalmente en desacuerdo}
dijeron no tener interés en la tutoría y prefieren esperar a que los citen
para ir a firmar.
%\newpage

\textit{Pregunta 26}: es su tutor quién lo busca para que asista a
tutoría.\linebreak El 7~(23.3\,\%) afirma estar totalmente de acuerdo, otros 7~(23.3\,\%)
dijeron que de acuerdo, 4~(13.3\,\%) más o menos de acuerdo, 3~(10\,\%) en
desacuerdo y 9~(30\,\%) en total desacuerdo. Esta pregunta se\linebreak incluyó con el
objetivo de tratar de valorar el interés del tutor porque los alumnos
participen de este programa y la mayoría de los alumnos respondió que sí
les buscan para que asistan pero también comentaron que solamente lo hacen
para pedirle que firmen y no para preguntarles cómo van con sus materias o
si tienen necesidad de otro tipo de orientación.

\smallskip
Después de realizar el procesamiento de lo los datos se hizo una revisión
del contenido del cuestionario y se logró identificar lo siguiente: en seis
de las veinte preguntas se  enfocan a las capacidad del tutor, en dos
mencionan la disposición, tres se refieren a las acciones que debe realizar
como tutor, tres mencionan específicamente los beneficios o resultados que
se esperan de la tutoría, una preguntan por un valor  como lo es el
respeto, las cinco restantes se enfocan en su  formación y conocimientos.


Las respuestas que dieron los alumnos nos dieron un panorama general acerca
de cómo perciben el programa de tutorías, específicamente las preguntas 17,
18 y 19 son las que cuestionan la efectividad del mismo y aunque la mayoría
opina estar entre \textit{totalmente de acuerdo, de acuerdo y más o menos
de acuerdo} se considera que no son los resultados esperados y que se debe
tomar medidas para modificar la forma en que se lleva a cabo el programa.


El primer punto a considerar es el relacionado con el perfil del tutor, la
ANUIES lo establece de la siguiente manera «las principales características
que debe tener son: poseer experiencia docente y de\linebreak investigación, conocer
los procesos de enseñanza aprendizaje; estar contratado de manera
definitiva, contar con habilidades como la comunicación fluida, la
creatividad, la capacidad de planeación y actitudes empáticas en su
relación con el alumno» (ANUIES 2000, p. 132).


Se consultó  y se tomó como referencia documental el trabajo de
investigación realizado en el 2011 por Gabriela de la Cruz Flores, Edith
Chehaybar y Luis Felipe Abreu, ya que ellos realizaron una revisión de la
literatura relacionada con la tutoría y en su escrito se encontraron 
otras definiciones de tutor que se enfocan en términos de: atributos,
propósitos, funciones y actividades, como  ejemplos de las que se enfocan
en los atributos están las que señalan: «El tutor es una persona hábil, que
cuenta con información, es dinámico y está comprometido en mejorar las
habilidades de otro individuo. Los tutores entrenan, enseñan y modelan a
los tutorados» (Young y Wringht 2001).


También se incluye esta otra: «Los tutores son individuos con experiencia,
conocimiento y compromiso para proveer soporte y movilidad a las carreras
de sus tutorados» (Ragins 1997).


Al relacionar estas definiciones con la propuesta por la ANUIES se puede
distinguir que el compromiso no está contemplado en la misma y es un rasgo
que sí se encuentra en las otras dos, se hace énfasis en este elemento
porque se considera que es clave para involucrarse en el programa de
tutoría y general en todas las actividades relacionadas con la docencia.

\enlargethispage{1\baselineskip}
Para comentar las funciones y acciones que debe cumplir un tutor recurrimos
nuevamente al glosario de términos que se incluyen en el texto de la ANUIES,
 donde define la acción tutorial como: 

\begin{quotation}
Ayuda u orientación que ofrecen los profesores-tutores a los alumnos en un
centro educativo, organizados en una red o equipo de\linebreak tutorías. Se concreta
en una planificación general de actividades, una formulación de objetivos y
en una programación concreta y realista. La asignación a cada tutor de
funciones específicas es básica para realizar adecuadamente la tutoría
(ANUIES~2000, p.~123).
\end{quotation}


A continuación incluiremos otras tres acepciones acerca de lo que debe hacer
un tutor: «Los tutores son modelos, confidentes y maestros. Son una fuente
de consejo, apoyo, patrocinio, entrenamiento, guía, enseñanza, retos,
protección, confidencialidad y amistad» (Bedy 1999). 

Como podemos observar son conceptos que no tienen relación entre sí pues en la de ANUIES  no se
mencionan de manera específica las acciones tal y como se vuelve a
encontrar en esta: «Los tutores son guías que logran la excelencia
académica, clarifican las metas y la planificación de los estudios. Enseñan
y depuran los conocimientos propios de su área de conocimiento, así como
los procesos o estándares de la conducta profesional. Estos estándares
incluyen las actitudes, los valores profesionales, la ética y la excelencia
académica» (Peyton 2001).


Después de analizar los conceptos anteriores nos preguntamos: ¿Todos los
docentes tenemos el perfil para ser tutores? ¿El hecho de tener una
contratación definitiva conlleva las características y cualidades que se
supone debe tener un tutor? ¿Solamente porque el Programa de Mejoramiento
del Profesorado (PROMEP) así lo establece somos tutores? ¿Será esto lo que
influye en los resultados que se obtienen con el programa de tutorías?

\enlargethispage{1\baselineskip}
Los docentes de educación superior tenemos diversas formaciones académicas
pero no fuimos formados para ser docentes, he ahí porque los alumnos
respondieron que existe falta de dominio de métodos pedagógicos, los
atributos formativos del tutor (Maloney 2001; Young y Wright 2001) se
refieren a su preparación académica, a su\linebreak experiencia y dominio de
conocimientos sobre su campo de estudio. Este atributo es mencionado por la
mayoría de los autores como indispensable. Por lo tanto los tutores deben
estar formados en su área, poseer conocimiento y comprensión de la
disciplina, dominar teorías y metodologías. Aunado a lo anterior debe
contar con los atributos didácticos que le faciliten en proceso de
enseñanza aprendizaje, tales como: 

\begin{quotation}
Conocimiento de la didáctica y de estrategias para facilitar el
aprendizaje, ofrecer múltiples ejemplos y enseñar en contextos donde se
aplique el conocimiento, brindar ayuda y consejos más allá de asuntos
técnicos, como la enseñanza de hábitos de trabajo, habilidades de
organización y establecimiento de prioridades, orientar en la escritura y
revisión de manuscritos (Collis 1998; Dolmas 1994; Maloney 1999;
Richardson y King 1998; Viator 2001).
\end{quotation}

Los atributos interpersonales de los tutores también se han comentado por
diversos autores y se refieren a la facilidad del tutor para relacionarse,
comunicarse, comprender y empatizar con los otros; Berger (1990);
Fagenson–Eland, Marks y Amendola (1997); Hartung (1995) y Maloney (1999)
enlistan los siguientes:


\begin{Obs}
\item[$\bullet$] Disponibilidad. Los tutores establecen un compromiso con el tutorado por
un periodo de tiempo. El tiempo implica dedicación y accesibilidad. Dentro
de las sesiones de tutoría establecen tiempo protegido, aminorando las
interrupciones por llamadas telefónicas o visitantes.
\item[$\bullet$] Habilidades de comunicación. Los tutores ofrecen confianza, saben escuchar
y permiten la expresión libre de las dudas de los tutorados. Son capaces de
analizar las necesidades de sus estudiantes y orientarlos en la toma de
decisiones, mostrando pros y contras de un actuar determinado. Mantienen
comunicación constante para verificar los resultados de las acciones que
los estudiantes han tomado como producto de su consejo. Para facilitar el
aprendizaje de los estudiantes, los tutores deben usar terminología
adaptada al nivel de competencia de los alumnos, así como brindar
explicaciones sobre los cornos y los porqués.
\item[$\bullet$] Habilidades afectivas. Un tutor eficaz es capaz de aceptar a sus tutorados
y empatizar con sus metas e intereses. Favorece la satisfacción de los
estudiantes durante los procesos de tutoría.
\item[$\bullet$] Habilidades de socialización. Los tutores usan el poder de su posición y
experiencia para participar en el desarrollo de la carrera de los
tutorados, relacionándolos con otros expertos o pares de la profesión.
Además les ayudan a incorporarse al rol de la profesión facilitando la
adquisición de valores, normas, tradiciones, conocimientos y prácticas
propias.
\end{Obs}

\medskip
Como podemos darnos cuenta, son muchos los requisitos para ser un buen
tutor, por lo tanto no podemos esperar que los resultados sean óptimos y
mucho menos evaluar este proceso con un instrumento que no incluye todos
los elementos mencionados como características, atributos y funciones;
además el desconocimiento que tienen los tutorados acerca del programa es
otro factor a considerar para el funcionamiento de las tutorías.

\enlargethispage{-1\baselineskip}
¿Qué es un tutorado? No se proporciona una definición en el texto de la
ANUIES, sin embargo de manera general se dice que es un novato, un
aprendiz, alguien inexperto y por lo tanto requiere de la guía de un tutor,
D.\ Campbell y T.\ Campbell (2000) identificaron algunas de las necesidades
que tienen los tutorados para establecer vínculos con los tutores, entre
ellas mencionan:  

\begin{quotation}
Recibir ayuda en la toma de decisiones para planificar
sus estudios, obtener guía académica durante todos sus estudios, tener
consejos para enfrentar las demandas académicas, contar con orientaciones
sobre requisitos del grado y recibir apoyo en problemas y crisis
personales.
\end{quotation}

También es necesario que los tutorados conozcan qué se espera de ellos pues
este proceso no puede delegarse solamente en el tutor, Adams (1993) señala
como atributos de los tutorados: responsabilidad, iniciativa, ingeniosidad,
habilidad para desarrollar un plan a fin de alcanzar sus metas y escuchar
los consejos del tutor, además de no asumir el rol de niño necesitado a
expensas de lo que disponga el tutor.


El desconocimiento del programa de tutoría conlleva a que los alumnos no
participen de manera adecuada en el mismo por ejemplo cuando se les
preguntó acerca de los motivos por los cuales asisten a tutoría  ubican en
primer lugar la asesoría académica y la de tesis, aun y cuando ANUIES
establece claramente que la primera de éstas es una actividad cotidiana en
las IES y que es distinta a la tutoría; específicamente menciona: «(\ldots) da
apoyo a las unidades de enseñanza aprendizaje que imparte el personal
académico. Consultas que  brinda un profesor fuera de lo que se considera
su tiempo docente, para resolver dudas o preguntas de un alumno o grupo de
alumnos sobre temas específicos que domina» (Latapí 1988). Mientras que la
de tesis consiste en la orientación y el apoyo metodológico que el asesor o
director proporciona al alumno para su trabajo de investigación. Es común
que los alumnos usen indistintamente el término de tutor.


La influencia del programa de tutoría en relación con la permanencia y la
eficiencia terminal de los estudiantes de la Licenciatura en\linebreak Historia es
demasiado relativa, aún no se logran los resultados esperados y los alumnos
continúan con trayectorias académicas erráticas, las cohortes
generacionales  tienen altos índices de rezago y por lo tanto la cantidad
de egresados y titulados no corresponde a los porcentajes establecidos, los
datos concretos no fueron proporcionados por la coordinación del programa
pero fue uno de los aspectos que ameritó una recomendación por parte del
Consejo para la Acreditación de Programas Educativos en Humanidades
(COAPEHUM) en el reciente proceso de evaluación a que fue sometida la
licenciatura.


Es necesario analizar varios aspectos para modificar la forma en que se está
llevando a cabo la tutoría, no solamente en la Licenciatura en Historia
sino en toda la UACJ pues la situación que se comenta en el presente
trabajo no es privativa de este programa ya que se conocen situaciones
similares en otros.


El primer punto a considerar es si realmente los docentes tenemos claro en
qué consiste la función del tutor, a pesar de que se han ofrecido cursos de
capacitación los resultados no son alentadores y hay quienes lo siguen
viendo como una obligación o como un requisito que se debe cumplir como
parte de las actividades establecidas por el PROMEP y peor aún como un
documento que se debe conseguir para participar en el programa de estímulos
al desempeño docente. 


Es necesario que la figura del tutor deje de ser solamente un indicador en
puntos, estímulos o gratificaciones, recuperar la vocación ---sí es que
alguna vez existió--- de ser docente y no solamente investigador, de tener
interés por los alumnos y su desempeño académico, de no verlos como una
pérdida de tiempo cuando en realidad son la razón de ser de la universidad.
Y tampoco creer que el tutor será quién resuelva de tajo todos los
problemas del alumno, éste es un concepto erróneo de lo que es un tutor,
desafortunadamente algunos docentes creen que esa la función que se
requiere que desempeñen y muestran un rechazo por la tutoría.
\newpage

También es necesario cuestionar si realmente la tutoría ayudará a mejorar la
calidad de la educación superior, porque como afirma 
el\linebreak Dr.~Aristeo~Santos (2003):

\begin{quotation}
Poco se sabe de las fórmulas que se seguirán inventando para descubrir cómo
la tutoría funciona o de los costos que implican el tener una tutoría. Lo
que si se tiene claro es que la tutoría debe enseñar a ser académicos en
las tareas educativas, a ser investigadores y sobre todo seres humanos. Esa
fórmula sólo se conseguirá cuando en los cuadros educativos se tengan a
verdaderos estudiantes y verdaderos tutores. Pero, para ello, se tiene que
quitar la tarea burocrática a la tutoría y se debe pensar en el sentido de
la educación para el futuro.
\end{quotation}

Aún queda mucho por hacer, este es solamente el principio de un largo camino
que debemos iniciar cuanto antes para lograr que nuestros estudiantes se
involucren realmente en un verdadero programa de tutorías, que logren
valorar el propósito del mismo pero sobre todo que los docentes asumamos
realmente el compromiso que adquirimos cuando optamos por  ejercer la
docencia, cuando decidimos impactar sus vidas porque aunque suene utópico
debemos siempre recordar que los alumnos son reflejo de sus maestros.
\newpage

\textbf{Referencias}


\bigskip
ANUIES (2000), \textit{Programas Institucionales de Tutoría. Una propuesta de la
ANUIES para su organización y funcionamiento en las instituciones de
educación superior}, México, Colección Biblioteca de la Educación Superior,
Serie Investigaciones, ANUIES.  

\begin{sloppypar}
Cruz Flores, Gabriela de la; Chehaybar y Kury, Edith y Abreu, Luis Felipe,
\textit{Tutoría en educación superior: una revisión analítica de la literatura}.
Rev.\ Educ.\ Sup.\ [online].\ 2011, vol.\ 40, n.\ 157 [citado 2014--06--11], pp.\
189--209. Disponible en
\url{http://www.scielo.org.mx/scielo.php?script=sci_arttext&pid=S0185-27602011000100009}
\end{sloppypar}


Santos López, A. y colabs. (2003),  «El tutor una innovación en educación para
una universidad de clase mundial: reflexiones del cotidiano frente a
desafíos de pobreza y comportamiento», Memorias del primer foro interno de tutoría académica de la mano con la formación integral del estudiante de la UAEM, del 26 al 28 de
noviembre, Toluca, México.

%\documentclass{article}
%\usepackage{amsmath,amssymb,amsfonts}
%\usepackage{fontspec}
%\usepackage{xunicode}
%\usepackage{xltxtra}
%\usepackage{polyglossia}
%\setdefaultlanguage{spanish}
%\usepackage{color}
%\usepackage{array}
%\usepackage{hhline}
%\usepackage{hyperref}
%\hypersetup{colorlinks=true, linkcolor=blue, citecolor=blue, filecolor=blue, urlcolor=blue}
%\newtheorem{theorem}{Theorem}
%\title{}
%\author{Humanidades}
%\date{2014-07-01}
%\begin{document}
%%\clearpage\setcounter{page}{429}
\thispagestyle{empty}
\phantomsection{}
\addcontentsline{toc}{chapter}{La tutoría académica: una experiencia de\newline vida.
Facultad de Humanidades Universidad\newline Autónoma del Estado de México\newline $\diamond$
\normalfont\textit{Georgina Flores García y Belén Benhumea Bahena}}
{\centering {\scshape \large La tutoría académica: una experiencia de vida. Facultad de Humanidades Universidad Autónoma del Estado\\ de México}\par}
\markboth{la formación del historiador}{tutoría académica}
\setcounter{footnote}{0}
\renewcommand*{\thefootnote}{\fnsymbol{footnote}}

\bigskip
\begin{center}
{\bfseries Georgina Flores García}\footnote{Doctora en Educación. Docente de
Tiempo Completo en la Facultad de Humanidades. Universidad Autónoma del
Estado de México. Perfil PROMEP. Miembro del Cuerpo Académico Historia.
Ginaflores5601@yahoo.com.mx}\\
{\bfseries Belén Benhumea Bahena}\footnote{Maestra en Historia. Docente de
asignatura  en la Facultad de Humanidades. Universidad Autónoma del Estado
de México. Estudiante de Doctorado en Historia.}\\
{\itshape Universidad Autónoma del Estado de México}
\end{center}
\renewcommand*{\thefootnote}{\arabic{footnote}}
\setcounter{footnote}{0}

\bigskip
La tutoría académica entendida como el acompañamiento de un experto hacia el
estudiante en formación profesional, durante la trayectoria académica,
significa una responsabilidad nodal para quien guía, considerada como una
acción molar en la que interviene el factor académico como columna
vertebral y aparejados los aspectos biológico, psicológico y pedagógico del
estudiante, obliga al tutor a estar preparado lo suficiente en esos tres
terrenos para poder llevar a buen término su labor.


En este trabajo se describe la experiencia de un grupo de tutorados de la
licenciatura en Historia de la Facultad de Humanidades de la Universidad
Autónoma del Estado de México durante el periodo 2005--2010 quienes formaban
un grupo heterogéneo en términos de edades y fases curriculares, y los
cuales fueron tutorados en sesiones grupales e individuales, logrando
resultados excelentes.


Cabe aclarar que no fue la tutoría por sí sola la que llevó al éxito, fue el
interés y la capacidad de los diferentes integrantes del grupo, quienes al
día de hoy, pueden contarse entre los profesionales exitosos en su
desempeño laboral.


La presente disertación versa sobre la experiencia que representó conjuntar 
los objetivos del Programa Institucional de Tutoría Académica con los del Plan de 
Estudios de Historia, tanto el rígido como el flexible.


\bigskip
{\bfseries La tutoría académica como experiencia de vida}

\medskip
Pensando en voz alta a cerca del proceso sistematizado a partir del año 2001
en la Universidad Autónoma del Estado de México y al mismo tiempo en otras
universidades públicas del país, debemos de decir que la experiencia ya se
tenía sin un método, sin programas ni informes, mucho menos con una serie
de pasos burocráticos que hacen en ocasiones que se pierda el objetivo.


¿Por qué afirmamos que ya se realizaba la actividad? Por varios motivos,
entre los que cabe señalar que cuando un estudiante le tenía confianza a un
docente, se acercaba con él para tratar problemas que impedían su avance
académico, si el docente estaba dispuesto y tenía los elementos que
pudieran dirigir al estudiante para lograr que continuara y\slash{}o 
terminara sus estudios, lo hacía.


En el mes de julio del 2001, durante la administración del Dr.~Rafael\linebreak
López Castañares, la Universidad Autónoma del Estado de México inició el
Programa Institucional de Tutoría Académica tomado directamente de la
Asociación Nacional de Universidades e Instituciones de Educación Superior
(ANUIES). Tal programa estuvo en un inicio bajo la responsabilidad del
Centro de Innovación, Desarrollo e Investigación Educativa (CIDIE),
dependiente de la Secretaría de Investigación; posteriormente, y en mayor
concordancia con sus principios,\linebreak el programa se turnó a la Dirección de
Desarrollo de Personal Académico (DIDEPA), dependiente de la Secretaría de
Docencia.


En la siguiente administración, encabezada por el Dr.~José Martínez\linebreak Vilchis,
el ProInsTA, tuvo un cambio fuerte al generar en forma electrónica el
sistema inteligente de tutoría, que padeció durante casi cuatro años muchos
problemas, tanto en el sistema como en el uso, generando molestia por parte
de tutores y tutorados, debido a la pérdida de tiempo que significaba. Poco a
poco se ha convertido en un instrumento para medir frecuencias de consulta,
afinándose, hasta lograr en la actual administración el vínculo con la
Dirección de Control Escolar, lo que ha permitido el ahorro de papel, tinta
y molestias a las trabajadoras del Departamento de Control escolar de la
Facultad, a la hora de consultar las historias académicas de los estudiantes.


\enlargethispage{1\baselineskip}
En el año 2001 inició la aplicación del Programa Institucional de Tutoría
Académica en la Universidad Autónoma del Estado de México, exclusivamente
en el nivel superior. El objetivo principal: abatir la deserción y
disminuir el índice de reprobación. Cada uno de los Organismos Académicos
elaboró su propio programa; dentro de los objetivos generales propuestos se
estimaban en un principio:

\begin{Obs}
\item[1.] «Detectar la problemática específica de los alumnos de primer ingreso a la
Facultad, a través de un diagnóstico que permitiera diseñar actividades
estratégicas preventivas y remediales a ser implementadas en el Programa de
Tutoría.»
\item[2.] «Contribuir a elevar la calidad del proceso formativo del estudiante, en
cuanto valores, actitudes y hábitos positivos, así como la promoción del
desarrollo de habilidades intelectuales en los estudiantes mediante la
utilización de estrategias de atención personalizada que complementaran las
actividades docentes regulares.»
\item[3.] «Favorecer el mejoramiento de las circunstancias o condiciones del
aprendizaje de los estudiantes, a través de la reflexión colegiada sobre la
información generada en el proceso tutorial.» (Universidad Autónoma del Estado de
México 2001).
\end{Obs}

Cabe aclarar que esta ponencia está escrita por una tutora y una tutorada
que para ese momento era estudiante de licenciatura y hoy es estudiante de
doctorado. La tutora había servido durante veinte años en la Facultad de
Ciencias Políticas y Administración Pública, espacio académico con
estudiantes de diferente nivel económico y social al de la Facultad de
Humanidades, lo que hizo difícil acoplarse al nuevo ambiente, sin embargo,
los jóvenes que me fueron asignados en la Facultad de Humanidades, entre los
que se encontraba la otra autora de esta ponencia, eran personas
interesadas en su carrera, estudiosas, responsables, que quizá no hubieran
necesitado de este programa tutorial, y, sin embargo, caminamos durante cinco
años en perfecta armonía y para crecimiento académico de ellos y el personal
de la tutora.


Para el logro de los objetivos había que iniciar por el conocimiento
personal de los estudiantes, de tal forma que en un espacio académico
diferente al que había servido, en  diferentes tiempos, se empezó a
activar el programa, el tipo de trabajo fue en seminario, con la aplicación
de una metodología cualitativa y participativa mediante el método 
investigación–acción, con unos excelentes resultados que se mencionarán 
a lo largo de esta ponencia.


El Programa Institucional de Tutoría Académica lleva doce años y medio de
aplicación. Durante este tiempo el procedimiento ha cambiado, pero en
esencia los pasos se han seguido sistemáticamente en forma semejante a la
inicial, porque el objetivo principal sigue siendo el \mbox{mismo}.


El grupo motivo de estudio se conformaba por personas de las generaciones
2003--2008, 2004--2009, 2005--2010 y 2006--2011\footnote{Todos los
tutorados formaron el grupo heterogéneo,  asignado a mi persona en el año
2006.} de la licenciatura en Historia de la Facultad de Humanidades. El
trabajo con este grupo fue un poco más complejo por la diferencia de
intereses personales y por el estudio de currículos de diferente índole:
una generación estudiante de Plan Rígido y las otras de Plan Flexible. Sin
embargo, esta técnica metodológica permitió encontrar algunas soluciones a
los problemas que inciden directamente en el aprovechamiento, como son:
problemas familiares, adicciones y problemas emocionales, entre otros.


\bigskip
\textbf{Antecedentes}

La tutoría la he ejercido sistemáticamente desde el año 2000, primero con
un grupo de 18 jóvenes de la Facultad de Ciencias Políticas y\linebreak
Administración Pública, y posteriormente con 18 jóvenes de la Facultad de
Humanidades. Los primeros fueron formados como licenciados en Ciencias
Políticas y Administración Pública, Sociología y Comunicación; y los
segundos estaban en proceso de formación como licenciados en Historia.


Los estudiantes de la Facultad de Humanidades tienen un perfil diferente a
los de Ciencias Políticas, con un punto a su favor, todos pertenecen a la
misma licenciatura; sin embargo, los problemas académicos y personales que,
por los semestres que cursaban al momento de tomar el grupo, se tuvieron
fueron complejos; había que atender cuatro grupos, los intereses, la
evolución académica era distinta, las sesiones grupales que podían
realizarse solamente eran cinco al año: la primera al iniciar cada
semestre, la segunda al término de cada ciclo escolar, y una más en
diciembre. Cada inicio de semestre se daba la bienvenida a todos, se
realizaba una sesión en donde se incluía un ejercicio sobre las
expectativas de cada uno sobre el nuevo semestre, cada uno de los escritos
se analizaba por parte de la tutora, se comentaban en la siguiente sesión
aquellos que tuvieran autorización por parte del tutorado, y aquellos que
no lo desean en sesión personalizada. Los escritos se comparaban con otros
que redactaban los tutorados al finalizar el semestre, y ellos evaluaban 
si habían logrado o rebasado sus expectativas.


El proceso tutorial, sus mecanismos y su ejercicio han sido diferentes, aún
cuando los objetivos y los recursos metodológicos han sido los mismos.


Los perfiles de los estudiantes en ambos casos son diferentes, de igual
forma su estatus social y económico hacía que tuvieran formas e
instrumentos de estudio sin puntos de comparación; sin embargo, ambos eran
jóvenes universitarios que estudiaban para ejercer una profesión al
egresar.


\bigskip
\textbf{Compartiendo el conocimiento}

\enlargethispage{1\baselineskip}
En la primera sesión di a conocer el Programa Institucional de Tutoría. En la
segunda sesión se les aplicó un cuestionario para tener conocimiento
de sus datos generales: domicilio, lugar de residencia, preparatoria de
procedencia, personas con las que vivían, estado civil, teléfono, dirección
electrónica, persona a la que contactar por cualquier eventualidad. En el
mismo instrumento había un sociograma, mediante el cual plasmaron su forma o
formas de estudiar, lo que llevaría a la tutora a ver cuáles eran los
hábitos de estudio, porque de ahí derivaría el éxito o fracaso en su
aprovechamiento escolar.


Al inicio de cada ciclo tutorial se aplicó un instrumento para detectar
necesidades académicas,\footnote{El grupo motivo de esta ponencia era
completamente heterogéneo, en su origen estuvo formado por una estudiante
de intercambio académico proveniente de Francia, dos estudiantes
correspondientes a la generación 2003--2008 (quienes egresaron  en julio
de 2008, ella actualmente es Maestra en Historia por el Colegio Mexiquense y
él no se ha titulado de la licenciatura), dos estudiantes de la generación
2004--2009 (a los cuales posteriormente se sumaron otros dos), seis estudiantes
de la generación 2005--2010 (de los cuales uno desertó, y una se sumó,
quedando el mismo número) y ocho estudiantes de la generación 2006--2011 (de
los cuales una desertó en los primeros meses de carrera y otra solicitó
cambio de tutor).} el instrumento contemplaba un apartado en el que se hace
un sociograma, analizando el ambiente en el que estudia el joven, un
ejercicio sobre administración del tiempo y preguntas sobre sus técnicas  y
métodos, para ver los diferentes hábitos de estudio.


Detectadas las necesidades sociales, durante el primer semestre del
ejercicio tutorial, y analizando paulatinamente el aprovechamiento
académico a través de los resultados parciales en y de su sentir en el
grupo y en la Facultad,\footnote{Esto a través de las tutorías grupales, en
donde comentaban cómo se sentían en el grupo, en la Facultad, qué les
parecía cada una de las sesiones de docencia, cómo eran sus docentes, cómo
era cada uno de ellos en el interior del aula, con relación al conocimiento
que adquirían en cada una de las Unidades de Aprendizaje.} se inició la toma
de decisiones con el fin de mejorar su aprovechamiento escolar,
interesándome más por el conocimiento adquirido que por la calificación,
aunque son dos factores que van íntimamente relacionados en la mayoría de
los casos.


Cuando había detectado problemas en algunos, era el momento de aplicar otro
instrumento, mediante el cual solicitaba que los jóvenes me dijeran cuáles
habían sido sus últimos resultados e hicieran una proyección para el final
del semestre;  posteriormente preguntaba ¿Qué alternativas de estudio
pensaban que existían para mejorar su aprovechamiento? Y con base en su
respuesta formulaba la siguiente pregunta, en términos de apoyar a alguno
de sus compañeros o de ser apoyado por alguno de los mismos. Cuando la
respuesta era afirmativa, directamente pedía dijeran a quién y por qué a
ella o a él. Estaban los estudiantes que dicen y evidencian no necesitar
apoyo de nadie  porque van muy bien, pero desean apoyar a algún compañero,
durante todo el tiempo de tutorías con ese grupo; no encontré a nadie que
no quisiera apoyar ni ser apoyado. Pedía la mecánica de trabajo que les
gustaría tener con su par y agradecía su tiempo.


Posteriormente al análisis de cada una de las respuestas, hablaba con los
pares elegidos; en el grupo aparecieron dos líderes de un mismo semestre y
uno de los otros dos. Hablé personalmente con cada uno, establecí el
compromiso y las reglas que incluían en primer término la
humildad,\footnote{No por ser la persona guía es superior al compañero que
requiere la ayuda, todos tenemos distintas capacidades, por lo que a uno se
le facilitan las matemáticas y a otro la redacción; el mundo será mejor si
compartimos el conocimiento y aceptamos el apoyo de los otros, brindando el
apoyo en lo que somos mejores.} asimismo el compromiso de trabajar con
interés por ambas partes.


Los jóvenes se sentían bien escuchando a sus compañeros, preguntando aquello
que no comprendieron en las sesiones; y su par se sentía bien compartiendo
lo que él comprendió, para tener por ambos lados mejores resultados.


Por supuesto que se les brindó la posibilidad de ser asesorados por un
experto, lo cual en la mayor parte de las ocasiones fue rechazado; cuando
se aceptó, se canalizó, siempre con la finalidad de que el estudiante
comprendiera el conocimiento nodal en la formación como cientista social.
\newpage

%\bigskip
\textbf{Trabajo en equipo: hacia el desarrollo de una\\ competencia
profesional}


Durante los años que duró la tutoría con este grupo las sesiones se fueron
espaciando paulatinamente, tal y como lo señalaba el modelo curricular
innovador que fue con el que inició el ProInsTA (Programa Institucional de
Tutoría Académica). ¿Por qué se fueron espaciando las reuniones grupales
conforme avanzaba la carrera? ¿No tenía que ser al contrario, que a mayor
complejidad y avance de los estudios profesionales, mayor apoyo tutorial?
No, el modelo educativo, que inició a la par del Programa Institucional de
Tutoría Académica, tenía como principio la información inicial y la
formación final, proceso que requería la autopoiesis y la independencia.


Conforme avanzaba el acercamiento se inició el conocimiento del grupo y del
Programa, se presentaron los objetivos, los propósitos y los beneficios que
obtendrían cada uno de ellos. En una de las sesiones del primer semestre se
les comentó que para poder trabajar en red académica, tendríamos que
conocernos profundamente, y para ello aplicaríamos la Historia de vida,
cada uno escribiría su propia Historia y, cuando lo deseara, podría
compartirla, siempre teniendo libertad de hacerlo o no hacerlo; lo que sí,
era necesario escribirla, porque esa sería la única forma de encontrar las
fortalezas y debilidades que como estudiantes tenían; ello haría que se
dieran cuenta y pudieran reforzar o corregir lo necesario para tener
resultados satisfactorios en los estudios profesionales.


En cada sesión se hacían comentarios sobre su desempeño
escolar, y acerca de los problemas que tenían para adaptarse al nuevo nivel escolar,
de las relaciones estudiantiles que iniciaban, y sobre la inserción en grupos de
trabajo presentes en la Facultad; fue de mucha utilidad en cada inicio de
semestre un ejercicio de administración del tiempo, que al final del
semestre se comparaba con otro análogo o semejante que permitía al
estudiante darse cuenta del tiempo invertido en el estudio, aun cuando no
tuviera clases, para ello era indispensable que conocieran el Plan de
Estudios y la carga de créditos, porque de esa forma lograban ver la
diferencia entre un curso, un taller y un seminario, lo que hacía
obligatorio invertir más tiempo en el primero y en el último.


En dos sesiones más, elaboraban y se analizaban sociogramas de los tutorados
para compararlos de alguna forma con lo que habían escrito en sus datos
generales, en el ejercicio de administración del tiempo y en lo que
llevaban de su Historia de vida; se analizaban solamente aquellos que
querían que se externara el análisis ante el grupo; quienes no lo deseaban
tuvieron en sesión individual su propio análisis. De éste se extrajeron
debilidades y fortalezas; las primeras paulatinamente se fueron atendiendo
para combatirlas, y las segundas se reforzaron para el aprovechamiento
académico de los estudiantes.  


Qué mayor orgullo que saberse parte importante de la conducción de la
carrera profesional desde el inicio hasta el final. En el año 2006, con la
experiencia de formación tutorial iniciaba un nuevo reto, como queda dicho
anteriormente, se efectuaban tres sesiones grupales al semestre; si era el
caso y se necesitaban otras, se llevaban a cabo en cada una no solamente se
compartía la experiencia de vida; siempre hubo algo que generó satisfacción
al sentido del gusto, y siempre se trató de que las sesiones se llevaran a
cabo en un lugar confortable, no en un aula; el ambiente influye mucho para
que los tutorados se involucraran, se sintieran parte de un grupo, a pesar
de ser de generaciones diferentes.

\enlargethispage{1\baselineskip}
En las sesiones intermedias de cada semestre, con cada uno de los grupos
(por fase), se analizaban temáticas que correspondían al avance del Plan de
Estudios, según correspondiera a la generación. En los distintos grupos se
hacía un análisis minucioso de la Legislación con la finalidad de que
tuvieran una idea clara de sus derechos y obligaciones; en cada semestre en
forma independiente, por grado, se abordaba el Plan de Estudios, con cada
una de las Unidades de Aprendizaje, haciendo hincapié en los Seminarios de
Titulación y en las áreas de acentuación, porque es crucial esta elección,
dado que forma parte de su futuro ejercicio profesional.


El grupo tutorial tomó muy en serio uno de los temas que mencionamos en el
párrafo anterior: el que corresponde a las formas y momentos de los
trabajos de titulación, dado que es nodal para la terminación de sus
estudios; el desconocimiento de estos tópicos es motivo por el cual la
mayor parte de los egresados no se titula; sin embargo, de este grupo
solamente dos estudiantes no se han titulado, el resto trabaja en el área
de acentuación de su elección y por lo menos tres han seguido estudios de
posgrado, estando ya una de ellas en el doctorado.


Debemos dejar clara la importancia del área de acentuación profesional. La
mayor parte de los estudiantes de este grupo tutorial eligieron el área de
acentuación de docencia por varias razones; posiblemente la tutora influyó
en sus decisiones al inclinar la balanza con comentarios hacia esa área,
porque comprobado está, a través de los dos únicos seguimientos de egresados,
que más del 80\,\% de los egresados de la licenciatura en Historia se dedican
a la docencia. No negamos que hay algunos, los menos, que trabajan para el
gobierno, sea municipal, estatal o federal, para medios de comunicación o
en archivos y museos. Con estos últimos espacios hay problemas porque la
misma Facultad oferta la Licenciatura en Ciencias de la Información
Documental, área específica que lanza a sus egresados a laborar en los
archivos.


De los dieciocho tutorados solamente dos eligieron otra área distinta a la
de docencia; sin embargo, ambos están ejerciendo la docencia. Del resto, con
mucho orgullo y responsabilidad debo decir que su elección por historia de
vida, por convicción y por la tutoría, fueron excelentes, porque son muy
buenos docentes de Historia; eso se constata cuando se habla con sus
estudiantes. Debido a que han pasado varios años desde su egreso, el
seguimiento por parte de la tutora no ha sido constante; sin embargo, no se
ha olvidado ni alejado de su vida profesional.


Por último quisiéramos comentar que la experiencia relacionada con la 
aplicación de la Historia de vida fue difícil de aplicar en la Facultad de Humanidades;
independientemente de que se piense que el estudiante de historia escribe,
escribir su propia historia es dolorosa, amén de que los estudiantes están
acostumbrados a escribir lo que les piden para sus Unidades de Aprendizaje,
y dependiendo del valor que sobre la calificación vaya a tener lo escrito.
Por lo anterior, la resistencia a escribir su propia historia es fuerte;
cuando se les conmina a compartirla, es aún mayor; sin embargo, lo hicieron
algunos.


Comentaremos un caso que detectado con tiempo hizo que el estudiante
egresara, aunque otros factores han impedido su titulación: el primero de
Octubre de 2008 se dio atención personalizada a quien llamaremos con el
seudónimo de Juan Prez; durante la lectura de sus Historia de vida se
detectaron problemas serios con su padre, desde la infancia, siendo un
joven estudioso no se explicaban las calificaciones regulares, los
sentimientos iban más allá con sus hermanas, las labores del campo y el
trabajo en su casa eran requisito; el estudiante hacía jornadas de doce
horas en la Facultad porque en caso contrario el padre lo enviaba al campo
y no podía leer por las tardes; cuando se le veía leyendo o escribiendo,
para sus padres significaba perder el tiempo. Inició con problemas de
salud, ello deterioró su rendimiento académico y, posteriormente, pensó en
una decisión terminal al creer que había contraído el virus de 
inmunodeficiencia humana. La opción fue canalizarlo con un psicólogo y un médico.
Ahora tenemos a Juan Pérez como uno más de nuestros egresados; con problemas fuertes, 
pero vivo y aún sin titularse.


Por otro lado tenemos el ejemplo de otra de nuestras estudiantes de
extracción mazahua. Ella llevó su historia, con partes adaptadas, a un
concurso nacional, y presentó un escrito denominado \textit{Problemas
socioeconómicos de los estudiantes Indígenas en la Facultad de Humanidades
de la UAEMéx}, hizo de todos los estudiantes indígenas de la Facultad la
historia de ella, una historia de discriminación, de marginación y de
migración. En ese concurso conquistó un premio.


No existió una calendarización para la presentación de las Historias de
vida; los jóvenes solos determinaban si compartían o no su historia. Ello
ayudó mucho; la mayor parte de los que la escribieron y compartieron han
logrado graduarse. Esto no quiere decir que si no escriben su historia no
se graduarán, simplemente que al escribir, al recordar, al compartir, se
analizan categorías que ayudan a mejorar hábitos de estudio, a tener mejor
noción del contexto histórico que se vive, y, como estudioso de la historia, a
cobrar conciencia de la parte que nos toca jugar dentro de la misma. Se
analizaron modelos educativos, modelos de enseñanza, de aprendizaje;
lograron descubrir sus propios canales de aprendizaje, y reconocerlos en los
otros compañeros; y ello ayudó para poner mayor atención en esas formas de
aprender, lo cual,  de alguna u otra manera, permitió que los resultados fueran
exitosos.


Quizá parezca invento o fantasía; lo mejor de todo es que a mi lado tengo a
una tutorada que ha caminado conmigo, cerca o lejos, pero conmigo en su
trayecto de licenciatura y de maestría; y ahora que inicia su doctorado, ella
y el resto de ese primer grupo de tutorados que tuve y sigo teniendo en la
Facultad de Humanidades de la Licenciatura en Historia, puede afirmar o
negar si nuestra labor fue fértil o estéril.


\bigskip
\textbf{Conclusiones}

\begin{Obs}
\item[$\star$] 
La Tutoría Académica es una actividad que permite elevar el nivel de
satisfacción académica para el sujeto estudiante universitario. No resta
más que comprender el desarrollo de los cuatro pilares que propone la
UNESCO, y las esferas en las que cualquier ser humano se desarrolla, de tal
forma que con la ausencia de estabilidad física, emocional, psicológica y
social, cualquier intento por alcanzar un excelente desarrollo académico,
será en vano. De ahí la importancia de saber, comprender y canalizar --en su
caso-- las problemáticas sociales y familiares que afectan el buen
desarrollo académico del estudiante.
\item[$\star$]  El Programa Institucional de Tutoría académica en la Universidad Autónoma
del Estado de México ha sido una estrategia adecuada para disminuir el
índice de deserción. El programa permite prevenir el índice de reprobación,
siempre y cuando sea llevado en forma metódica, sistemática y con un
compromiso real por parte del tutor y de los tutorados.
\item[$\star$] La aplicación de un instrumento para detección y diagnóstico de necesidades
permite conocer métodos y técnicas de estudio para remediar, fortalecer o
cambiar hábitos en los estudiantes.
\item[$\star$] El hábito de administrar el tiempo permite al estudiante aprovechar al
máximos sus tiempos libres, sin convertirlo en un extraño amargado,
simplemente en un ser que sabe lo que quiere.
\item[$\star$] La escritura y el análisis de la Historia de vida del estudiante, durante el
proceso de formación profesional, genera en el tutor y en el tutorado procesos
de ubicación y conciencia de los problemas que han entorpecido o facilitado
el aprendizaje.
\item[$\star$] Pensar al estudiante como un ente aislado del contexto familiar y social,
inserto en exclusiva en una organización académica denominada Universidad,
equivale a descontextualizarlo y, por ende, a quitarle sentido de existencia al ser
humano en formación. La Tutoría Académica, al pretender disminuir los
índices de reprobación y aumentar el nivel de permanencia en la Universidad
(a la vez que incrementar los índices de eficiencia terminal), permite que
través de programas como éste se tenga un acercamiento con el estudiante, a
fin de poder detectar posibles problemas en cuanto a los diversos aspectos
que favorecen o perjudican el avance académico del universitario.
\item[$\star$] El tutor, al detectar problemas familiares o sociales del estudiante
universitario, no se transforma en el psicólogo familiar y\slash{}o social, sino
que detecta y canaliza para la posible solución a la problemática, con la
consecuente mejora en el aprovechamiento del estudiante.
\item[$\star$] La experiencia de doce años y medio es una prueba de que este tipo de
programas puede tener éxito, no es sólo el afianzar al estudiante, sino el
permitirle adquirir elementos para continuar, para crecer académicamente, o
para partir, siempre en beneficio de él mismo, el de su familia y el de
su entorno social.
\end{Obs}
\newpage

\textbf{Referencias}


\textit{Programa Institucional de Tutoría Académica}, Universidad Autónoma
del Estado de México.
%\newpage
%\thispagestyle{empty}
%\phantom{abc}

