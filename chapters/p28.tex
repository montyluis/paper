%\documentclass{article}
%\usepackage{amsmath,amssymb,amsfonts}
%\usepackage{fontspec}
%\usepackage{xunicode}
%\usepackage{xltxtra}
%\usepackage{polyglossia}
%\setdefaultlanguage{spanish}
%\usepackage{color}
%\usepackage{array}
%\usepackage{hhline}
%\usepackage{hyperref}
%\hypersetup{colorlinks=true, linkcolor=blue, citecolor=blue, filecolor=blue, urlcolor=blue}
%% Text styles
%\newcommand\textstyleInternetlink[1]{\textcolor[rgb]{0.0,0.0,0.5019608}{#1}}
%\newcommand\textstyleStrong[1]{\textbf{#1}}
%\newcommand\textstyleInternetLink[1]{\textcolor[rgb]{0.019607844,0.3882353,0.75686276}{#1}}
%\newtheorem{theorem}{Theorem}
%\title{}
%\author{}
%\date{2014-08-25}
%\begin{document}
%\clearpage\setcounter{page}{1}{ 

\textbf{Autores}}

{ 
\textit{Elva Rivera Gómez}}

{ 
Es Doctora en Historia por la Universidad Veracruzana. Maestra en Ciencias
Históricas por la Universidad Amistad de los Pueblos, Moscú, Rusia. Es
Profesora Investigadora del Colegio de Historia de la Facultad de Filosofía
y Letras de la Benemérita Universidad Autónoma de Puebla. Es integrante del
Sistema Nacional de Investigadores. Pertenece al Cuerpo Académico
Consolidado de Estudios Históricos. Cuenta con el perfil PROMEP. Sus líneas
de investigación son Género e Historia, Historia Social y Enseñanza de la
Historia. Es autora y coautora de libros, capítulos de libros y artículos
de dedicados a la Historia de las Mujeres, estudios de las masculinidades,
género y educación y enseñanza de la historia. En fundadora y presidenta de
la Red Nacional de Licenciaturas en Historia y sus Cuerpos Académicos,
fundadora e integrante del Comité Internacional de la \textit{Revista La
Manzana} de la Red Internacional de Estudios de las Masculinidades, de la
Red Iberoamericana de Ciencia, Tecnología y Género.}


\bigskip

{ 
\textit{María Elena Bribiesca Sumano}}

{ 
Es originaria de la Ciudad de México. Profesora Emérita de la Universidad
Autónoma del Estado de México. Maestra en Historia. Docente de Tiempo
Completo en la Facultad de Humanidades de la UAEMéx. Maestra en Historia
por la Universidad Pontificia de México, Maestra en Pedagogía por la Normal
Superior de México, laboró en El Archivo General de la Nación durante 35
años, prestó sus servicios formando profesionales en la Escuela Nacional de
Biblioteconomía y Archivonomía, durante 11 años en donde impartió la
asignatura de Paleografía y Diplomática. En la Facultad de Humanidades de
la Universidad Autónoma del Estado de México es maestra emérita, ha
impartido la misma asignatura durante los últimos 46 años, contribuyendo a
la formación de todos aquellos profesionales de la Historia que se dedican
al estudio del periodo Novohispano y siglo XIX y que les es indispensable
el uso de la Paleografía. Ha sido docente  en diversas instituciones de
diferentes entidades federativas, entre las que se cuentan: Yucatán, Quinta
Roo, Guerrero, San Luis Potosí, Querétaro, Guanajuato, Nuevo León, Chiapas,
Campeche, Michoacán, por mencionar algunas.}

{ 
Tiene publicados varios manuales acerca de Paleografía y Diplomática,
artículos y capítulos de libro con temas de investigación histórica y ha
participado en coloquios internacionales y nacionales, contándose dentro de
los últimos su presencia en la Universidad de Viena y en Pernambuco,
Brasil.}

{ 
Actualmente se desempeña como docente de Tiempo Completo en la Facultad de
Humanidades de la Universidad Autónoma del Estado de México. Correo
electrónico:\textit{
}\href{mailto:bribiescas3603@yahoo.com.mx}{\textstyleInternetlink{\textit{bribiescas3603@yahoo.com.m}}}\href{mailto:bribiescas3603@yahoo.com.mx}{\textstyleInternetlink{\textit{x}}}}


\bigskip

{ 
\textit{María Guadalupe Zárate Barrios}}

{ 
Es Licenciada en Historia. Docente de asignatura en la Licenciatura en
Historia de la Facultad de Humanidades de la Universidad Autónoma, además
lectora de los trabajos de titulación del Seminario de Catalogación de
Documentos de Archivo de la licenciatura. Correo electrónico:\textit{ }}


\bigskip

{ 
\textit{Catalina Sáenz Gallegos}}

{ 
Maestra en Historia Regional continental, e Historiografía.  Profesora de la
Facultad de Historia de la Universidad Michoacana de San Nicolás Hidalgo.
Se ha dedicado a la investigación en el área económica, agraria y
documental  con una experiencia de más de 9 años trabajando en diversos
archivos, dedicados a su rescate, creación y organización de los mismos,
así como la difusión de su importancia a través de conferencias y
organización de seminarios y encuentros de archivos Estatales y
municipales. Correo electrónico:
\href{mailto:sagacat33@hotmail.com}{\textstyleInternetlink{sagacat33@hotmail.com}},
y \href{mailto:catsa37@gmailcom}{\textstyleInternetlink{catsa37@gmailcom}}}


\bigskip


\bigskip

{ 
\textit{María Guadalupe Carapia Medina }}

{ 
Maestra en Historia Regional continental, e Historiografía. Profesora de la
Facultad de Historia de la Universidad Michoacana de San Nicolás Hidalgo.
Se ha dedicado a la investigación en el área económica, agraria y
documental  con una experiencia de más de 9 años trabajando en diversos
archivos, dedicados a su rescate, creación y organización de los mismos,
así como la difusión de su importancia a través de conferencias y
organización de seminarios y encuentros de archivos Estatales y
municipales. Correo electrónico: 
\href{mailto:guadap73@hotmail.com}{\textstyleInternetlink{guadap73@hotmail.com}}}


\bigskip

{ 
\textit{Rubén Darío Núñez Altamirano }}

{ 
Doctor en Ciencias del Desarrollo Regional. Profesor de la Facultad de
Historia de la Universidad Michoacana de San Nicolás Hidalgo, su actividad
profesional aparte de la docencia la han dedicado a la investigación en el
área económica, agraria y documental  con una experiencia de más de 9 años
trabajando en diversos archivos, dedicados a su rescate, creación y
organización de los mismos, así como la difusión de su importancia a través
de conferencias y organización de seminarios y encuentros de archivos
Estatales y municipales. Correo electrónico:
\href{mailto:oso_no@hotmailcom}{\textstyleInternetlink{oso\_no@hotmailcom}}}


\bigskip

{ 
\textstyleStrong{\textmd{\textit{Paulina Latapí Escalante }}}}

{ 
\textstyleStrong{\textmd{Es }}Licenciada en Historia por la UNAM y Maestra
en Educación con Especialidad en Cognición de los Procesos de Enseñanza
Aprendizaje. Actualmente es profesora de tiempo completo adscrita a la
Facultad de Filosofía, Colegio de Historia, de la Universidad Autónoma de
Querétaro en donde coordina la Línea Terminal en Enseñanza de la Historia
de la Licenciatura en Historia y  forma parte del Núcleo Académico Básico
de la Maestría en Estudios Amerindios y Educación Bilingüe con
reconocimiento como programa de calidad de CONACYT. Cuenta con
reconocimiento de perfil Promep. Autora de 36 libros, de investigación
histórica y de educación; entre éstos se encuentran cinco  libros de texto
gratuito vigentes para el eje curricular  de ciencias sociales para nivel
secundaria.  Correo electrónico:\textit{
}\href{mailto:platapik@prodigy.net.mx}{\textstyleInternetlink{\textit{platapik@prodigy.net.mx}}}}


\bigskip

{ 
\textit{Ma. Gabriela Guerrero Hernández}}

{ 
Es pasante del Doctorado en Educación en la Universidad Marista de
Guadalajara. Maestra en Metodología de la Ciencia y Licenciada en Historia
por la UANL, actualmente es Maestra de tiempo completo  de la Facultad de
Filosofía y Letras de la Universidad Autónoma de Nuevo León (UANL) y tiene
Perfil PROMEP (2012-2015). En el área de investigación se desarrolla en la
línea de Las competencias Docentes en la Educación Superior. Entre las
publicaciones más recientes destacan: La formación docente y sus
implicaciones en el proceso de enseñanza aprendizaje en la educación
superior (2012), En la Memoria del Tercer Encuentro Nacional de Docencia,
Difusión y Enseñanza de la Historia, La invisibilidad de la mujer en el
siglo XIX, En: Tejiendo Género. Desde perspectivas teóricas y testimonios,
(2013). Correo electrónico:
\href{mailto:gaguh_70@yahoo.com.mx}{\textstyleInternetlink{gaguh\_70@yahoo.com.mx}}}


\bigskip

{ 
\textit{María del Rocío Rodríguez Román}}

{ 
Es Candidata a Doctora en Educación por la Universidad Marista de
Guadalajara. Maestra en Metodología de la Ciencia y Licenciada en Historia
egresada de la Facultad de Filosofía y  Letras de la UANL. Cuenta con
Perfil Promep (2012-2015) y su línea de investigación es la enseñanza de la
historia. Se desempeña como Profesora de Tiempo Completo en la Facultad de
Filosofía y Letras,  donde además es la Coordinadora del Colegio de
Historia y Estudios de Humanidades,  y en la Normal Superior “Profr. Moisés
Sáenz Garza”. Entre sus publicaciones se cuentan: Principales problemas y
retos que enfrenta la enseñanza de la historia: una revisión bibliográfica.
(2012). En la memoria del XIII Encuentro Internacional de Historia en la
Educación. La compilación del libro El valor de las abuelas (2012). Correo
electrónico:\textit{
}\href{mailto:rocio_rodriguez_ens@yahoo.com.mx}{\textstyleInternetlink{\textit{\textcolor[rgb]{0.0,0.0,0.8}{rocio\_rodriguez\_ens@yahoo.com.mx}}}}\textstyleInternetlink{\textcolor[rgb]{0.0,0.0,0.8}{
}}\textstyleInternetlink{\textcolor[rgb]{0.0,0.0,0.039215688}{y
}}\href{mailto:colegiohistoria@yahoo.com.mx}{\textstyleInternetlink{\textit{\textcolor[rgb]{0.0,0.0,0.8}{colegiohistoria@yahoo.com.mx}}}}}


\bigskip

{ 
\textit{\textcolor{black}{Hugo Torres Salazar}}}

{ 
Es Doctor en Historia por la Universidad “Paul Valéry” Montpellier, Francia.
Psicoanalista. APG. APM. FEPAL. IPA. Profesor- Investigador en el
Departamento de Historia. Presidente de Academia: Docencia de la Historia.
Cuenta con el Perfil PROMEP. Responsable de Cuerpo Académico: Historia,
memoria y cultura. Coordinador de la Maestría en Historia de México. Centro
Universitario de Ciencias Sociales y Humanidades. Universidad de
Guadalajara. Coordinador Regional. Centro Occidente. COMECSO. Presidente de
la Red de Especialistas en Docencia, Difusión e Investigación en Enseñanza
de la Historia. (REDDIEH). Correo electrónico:\textit{
}\href{mailto:htorres@cencar.udg.mx}{\textstyleInternetlink{\textit{htorres@cencar.udg.mx}}}}


\bigskip

{ 
\textit{Gil Arturo Ferrer Vicario}}

{ 
\textcolor{black}{Es Profesor de Tiempo Completo Titular C, adscrito a la
Unidad Académica de Filosofía y Letras de la Universidad Autónoma de
Guerrero, Perfil PROMEP 2009-2012 y actualmente en proceso de evaluación en
dicho Programa; Cuerpo Académico: Humanismo y Sustentabilidad,
Reconocimiento por participar en la Mesa Redonda: "Educación y
Sustentabilidad"; Reconocimientos como ponente de las Conferencias: "La
importancia de la Historia"; "Sor Juana Inés de la Cruz vista desde el
presente". Reconocimiento por la participación en la XXV Semana
Altamiranista. Publicaciones recientes: "Guerrero en el contexto de las
Revoluciones en México"; "Guerrero: la disputa por la tierra 1856-1933";
"Educación para la sustentabilidad". Correo electrónico:
}\href{mailto:gil_uagro@hotmail.com}{\textstyleInternetlink{\textcolor[rgb]{0.0,0.0,0.8}{gil\_uagro@hotmail.com}}}}


\bigskip

{ 
\textit{Jaime Salazar Adame}}

{ 
Es Historiador  y Doctor/a en Ciencias Políticas por la Universidad de
Madrid, España, graduándose con mención honorífica Sobresaliente Cum Laude.
Profesor de Tiempo Completo Titular “C”, adscrito a la licenciatura en
Historia de la Unidad Académica de Filosofía y Letras de la Universidad
Autónoma de Guerrero. Se desempeña como Coordinador de la DES de Ciencias
Sociales y Humanidades de la propia Universidad. Funge como Subdirector
Académico de la UA de Filosofía y Letras. Es Coordinador del Comité
Directivo del Consejo de la Crónica del Estado de Guerrero. Cuenta con el
reconocimiento de PERFIL PROMEP.  Pertenece al Cuerpo Académico en
Consolidación “Problemas Sociales y Humanos”. Sus últimas publicaciones en
coautoría son: Los Sentimientos de la Nación. Interpretaciones recientes.
México: UNAM/Gobierno del Estado de Guerrero/ LX Legislatura, 2014.
Presidentes y guerreros. Instituto de Estudios Parlamentarios/ LX
Legislatura del Edo. de Guerrero/El Colegio de Guerrero/ Editorial Lama,
2014. Coordinador y coautor. El ecocidio del siglo XXI. Cosmovisiones,
premisas, impactos y alternativas. México: EÓN/UAGro/Intercambio Social/
A.C./ Centeotl, A.C., 2014. Correo electrónico:
\href{mailto:jaime48sa@hotmail.com}{\textstyleInternetlink{jaime48sa@hotmail.com}}
 }


\bigskip

{ 
\textit{Smirna Romero Garibay}}

{ 
Es  Abogada y Maestra en Ciencias por la Universidad Autónoma de Guerrero,
donde se graduó con mención honorífica.  Profesora de Tiempo Completo
adscrita a la Unidad Académica de Derecho de la UAGro.  Cuenta con el
reconocimiento de PERFIL PROMEP. Pertenece al Cuerpo Académico en Formación
Sistemas de Justicia en México. Se desempeña como Coordinadora de la DES de
Derecho  y funge como Subdirectora de Planeación de la U.A. de Derecho. 
Sus últimas publicaciones en coautoría son: El Congreso de Chilpancingo en
su bicentenario. México: Consejo de la Crónica del Estado de Guerrero/
UAGro, 2014.  Vida y Obra de Ignacio Manuel Altamirano. México:
CONACULTA/Secretaría de Cultura/ UAGro/ Asociación de cronistas del Estado
de Guerrero, 2013. Bienes, derechos reales y sucesiones. Cuestionamientos
básicos. México, Universidad Autónoma de Guerrero, 2011. Correo
electrónico:
\href{mailto:smro07@yahoo.com.mx}{\textstyleInternetlink{smro07@yahoo.com.mx}}}


\bigskip

{ 
\textit{Wilfrido Llanes Espinoza}}

{ 
Es Doctor en Ciencias Sociales (Universidad de Guadalajara); coordinador de
la Licenciatura en Historia (Universidad Autónoma de Sinaloa, 2011-2014);
pertenece al Sistema Nacional de Investigadores,  Nivel C;  integrante del
Cuerpo Académico Consolidado Historia Socio-Cultural. Correo electrónico:
\href{mailto:wllanes@gmail.com}{\textstyleInternetlink{wllanes@gmail.com}}}


\bigskip

{ 
\textit{Eduardo Frías Sarmiento}}

{ 
\textcolor{black}{Es Doctor en Historia (Benemérita Universidad Autónoma de
Puebla); Director de la Facultad de Historia (Universidad Autónoma de
Sinaloa, 2011-2014); pertenece al Sistema Nacional de Investigadores, 
Nivel I;  Profesor con Perfil Deseable, PROMEP; integrante del Cuerpo
Académico Consolidado Historia Socio-Económico. Correo
electrónico:}\textit{\textcolor{black}{
}}\href{mailto:eduardofrias@uas.edu.mx}{\textstyleInternetlink{eduardofrias@uas.edu.mx}}}


\bigskip


\bigskip

\textit{Vanessa Magaly Moreno Coello} 

Es Doctorante en Ciencias Sociales por el CESMECA, Centro de Estudios
Superiores de México y Centroamérica. Maestra en Ciencias Sociales y
Humanísticas por el CESMECA y Licenciada en historia por la UNACH, Campus
III. Facultad de Ciencias Sociales. En el 2013 publicó: La escasez de
recursos en las cárceles de San Cristóbal de Las Casas durante el
porfiriato, en María Eugenia Claps Arenas (Coord.). Formación y gestión del
Estado en Chiapas “Algunas aproximaciones históricas” UNICACH, Pps 113-130.
En la actualidad se desempeña como profesora de asignatura en la
Universidad Autónoma de Chiapas, donde imparte diversas materias
relacionadas con el área de historia. Correo electrónico:
\href{mailto:vagmo18@hotmail.com}{\textstyleInternetlink{vagmo18@hotmail.com}}


\bigskip

\textit{Patricia Gutiérrez Casillas }

Es Doctorante en Ciencias Sociales y Humanísticas por el CESMECA- UNICACH.
Desde 2003 se desempeña como profesora de la Licenciatura en Historia por
la UNACH- Campus III. Fue titular del proyecto: "Cupanda y la historia del
cooperativismo en Tacámbaro, Michoacán”. CIESAS-OCC./SIMORELOS/ CONACYT,
Catalogadora y Paleógrafa del Archivo Gral. de Guadalajara. Sus últimas
publicaciones son: Gutiérrez Casillas, Patricia. 2013. "Por la ruta del
comercio: Justificando el género", en libro digital del III Encuentro
Internacional de Investigación de Genero. Estudios de Género en el S. XXI:
Experiencias de Transversalidad. pp. 1537-1551 UAQ. b) Gutiérrez Casillas,
Patricia y José Rubén ORANTES G. "Organizaciones emergentes y relaciones
sociopolíticas informales en la industria del aguacate michoacano”. Anuario
2010 CESMECA-UNICACH, n. 21. Pp. 187-202. Correo electrónico:
\href{mailto:paty_gutierrez05@hotmail.com}{\textstyleInternetlink{paty\_gutierrez05@hotmail.com}}


\bigskip

\textit{\textcolor{black}{Mario Heriberto Arce Moguel}}

\textcolor{black}{Es Licenciado en Historia por la Universidad Autónoma de
Chiapas, donde obtuvo Mención Honorífica con la tesis
}\textit{\textcolor{black}{Los Juegos del poder: el proceso histórico
político del sistema presidencialista en el ámbito del estado de
Chiapas}}\textcolor{black}{. Realizó estudios de Historia Contemporánea en
la Universidad Autónoma de Madrid. En 2013 publicó: "La sucesión de los
gobernadores en el estado de Chiapas durante la Revolución Mexicana,
1913-1920" artículo que forma parte del libro:
}\textit{\textcolor{black}{Formación y gestión}}\textcolor{black}{
}\textit{\textcolor{black}{del Estado en Chiapas. Algunas aproximaciones
históricas}}\textcolor{black}{, editado por CESMECA-UNICACH. En la
actualidad se desempeña como profesor de asignatura en la Universidad
Autónoma de Chiapas, donde imparte diversas materias relacionadas con el
área de historia. Correo electrónico:
}\href{mailto:marioheribertoarce@hotmail.com}{\textstyleInternetlink{marioheribertoarce@hotmail.com}}


\bigskip

{ 
\textit{Arturo Carrillo Rojas }}

{ 
Es Doctor en Ciencias Sociales y Maestro en Historia Regional por la
Universidad Autónoma de Sinaloa. Profesor investigador de tiempo completo
de la Facultad de Historia de la UAS. Director de la Facultad (2001-2004) y
presidente fundador de la Academia de Historia de Sinaloa A.C. (2005-2008).
Responsable del Cuerpo Académico Consolidado “Historia económica social”
(2001-2011). Miembro del S.N.I. desde 2001 (nivel I). Publicaciones
relevantes: “La irrigación en Sinaloa: cambios en la infraestructura
hidráulica y sistemas de regadío entre los siglos XIX y XX” (2011),
\textit{Agua, agricultura y agroindustria, Sinaloa en el siglo XX }(2013);
“Hacia la formación de competencias docentes en la Licenciatura en Historia
de la Universidad Autónoma de Sinaloa”, en \textit{Curriculum, formación y
prácticas de la enseñanza de la historia}, 2013. Correo:
\href{mailto:acarrillo_35@hotmail.com}{\textstyleInternetlink{acarrillo\_35@hotmail.com}}}


\bigskip

{ 
\textit{Luis Demetrio Meza López}}

{ 
Es candidato a Doctor en Educación por la Facultad de Ciencias de la 
Educación de la Universidad Autónoma de Sinaloa y profesor en el área de
investigación de la misma. Es encargado del Centro de Divulgación Enseñanza
y Comunicación en la Facultad de Historia de la UAS. Su publicación más
reciente\textcolor{black}{ es “}\textcolor{black}{La relevancia del modelo
instruccional para el diseño e implementación de entornos virtuales
educativos. Una experiencia en la enseñanza de la historia a nivel
superior”, capí}\textcolor{black}{tulo del  libro
}\textit{\textcolor{black}{Experiencias metodológicas en la investigación
cualitativa}}\textcolor{black}{ (en prensa). Además ha sido ponente en
diversos eventos presenciales y virtuales relacionados con la educación.
}Correo:
\href{mailto:luismezalopez@hotmail.com}{\textstyleInternetlink{luismezalopez@hotmail.com}}}


\bigskip

{ 
\textit{\textcolor{black}{Antonio F. de Jesús González Barroso}}}

{ 
\textcolor{black}{Es Doctor en Historia. Profesor tiempo completo de base,
titular C. Adscrito a las Unidades Académicas de Docencia Superior e
Historia en la Universidad Autónoma de Zacatecas. Cuenta con el
reconocimiento de PERFIL PROMEP 2013-2016. Pertenece al Cuerpo Académico en
consolidación “Enseñanza y difusión de la historia”. Sus últimas
publicaciones son: }\textit{\textcolor{black}{Teoría, consideraciones y
experiencias. El proceso de enseñanza-aprendizaje de la historia en
Zacatecas }}\textcolor{black}{(UAZ, 2013), }\textit{\textcolor{black}{La
pobreza de origen rural en la ciudad de Puebla entre 1878 y
1889}}\textcolor{black}{ (EAE, 2012) y }\textit{\textcolor{black}{Teoría y
metodología en la enseñanza-aprendizaje de la historia. Problemas de la
educación básica en Zacatecas }}\textcolor{black}{(UAZ, 2011). Correo
electrónico:}\textit{\textcolor{black}{
}}\href{mailto:jacobino@prodigy.net.mx}{\textstyleInternetlink{\textit{\textcolor{black}{jacobino@prodigy.net.mx}}}}}


\bigskip

{ 
\textit{María R. Magallanes Delgado}}

{ 
Es doctora en Historia por la Universidad Autónoma de Zacatecas y docente
investigadora en la Unidad Académica de Docencia Superior en la misma
institución; pertenece al Sistema Nacional de Investigadores Nivel I,  al
Registro de Evaluadores Nacionales de CONACYT y es perfil PROMEP. Impulsa
las líneas de historia de la educación en el siglo XIX y XX, los problemas
del aprendizaje y enseñanza de la historia en Educación Básica, la
actualización y profesionalización en la enseñanza de la Historia.
Coordinadora y coautora de \textit{Miradas y voces en la historia de la
educación en Zacatecas. Protagonistas, instituciones y enseñanza (XIX-XXI)
}(2013), \textit{Historia comparada de las mujeres en las Américas }(2012),
\textit{Grupos marginados de la educación (siglos XIX y XX}), (2011),
\textit{Teoría y metodología en la enseñanza de la historia. Problemas de
la educación básica en Zacatecas} (2011), \textit{Historia de la Educación
en Zacatecas I. Problemas, tendencias e instituciones en el siglo XIX}
(2010). \textcolor{black}{Correo electrónico:
}\href{mailto:rmdhistoria@yahoo.com.mx}{\textstyleInternetlink{rmdhistoria@yahoo.com.mx}}}


\bigskip

{ 
\textit{\textcolor{black}{Ángel Román Gutiérrez}}}

{ 
\textcolor{black}{Correo electrónico:}\textit{\textcolor{black}{
}}\href{mailto:angelemiliano0724@hotmail.com}{\textstyleInternetlink{\textcolor[rgb]{0.0,0.0,0.8}{angelemiliano0724@hotmail.com}}}}


\bigskip

{ 
\textit{\textcolor{black}{Marco Antonio Peralta Peralta}}}

{ 
Es Licenciado en Historia por la Facultad de Humanidades de la UAEMéx.
Profesor adjunto del 2009 al 2012 del seminario de investigación
\textit{Etnohistoria de las estructuras sociales y culturales del centro
}\textit{de Nueva España} en la misma institución. Participante en varios
proyectos de Investigación de la UAEMéx y de El Colegio Mexiquense A.C.
Participante, ponente y organizador en varios encuentros de la RENALIHCA y
la REDDIEH. Actualmente es alumno del programa de Posgrado en Estudios
Históricos de la Facultad de Filosofía de la Universidad Autónoma de
Querétaro. Las líneas de interés son: La cultura barroca en el centro de la
Nueva España, religiosidad e identidades colectivas, en donde tiene algunas
publicaciones y, los procesos de enseñanza y aprendizaje de la Historia.
Correo electrónico:
\href{mailto:marco_p1017@hotmail.com}{\textstyleInternetlink{marco\_p1017@hotmail.com}}}


\bigskip

{ 
\textit{\textcolor{black}{Marcela Janette Arellano González}}}

{ 
Es Licenciada en Historia por la Facultad de Humanidades.\textit{ }Profesora
adjunta de la Unidad de Aprendizaje \textit{Paleografía y Diplomática} de
la Facultad de Humanidades (2013). Participante en varios proyectos de
investigación en la UAEMéx. Ponente, organizadora y participante en varios
encuentros de la RENALIHCA y la REDDIEH. Actualmente se desempeña como
colaboradora de investigación de la Dra. Georgina Flores García y la Mtra.
María Elena Bribiesca Súmano. Sus líneas de interés versan sobre los
estudios de afro descendencia en Nueva España, Catálogos notariales de la
época novohispana, donde es autora de varios artículos y de un libro
académico \textit{Azúcar, enfermedad y esclavitud en la Hacienda de
Xalmolonga, siglo XVII} y, los paradigmas educativos de la enseñanza de la
Historia. Correo electrónico: 
\href{mailto:marboreanaz@hotmail.com}{\textstyleInternetlink{\textit{marboreanaz@hotmail.com}}}}


\bigskip


\bigskip

{ 
\textit{Ofelia Janeth Chávez Ojeda}}

{ 
Es Maestra en Historia por la Universidad Autónoma de Sinaloa,  Profesor
Investigador de Tiempo Completo con adscripción a la Facultad de Historia
de la misma Universidad. Secretaria Administrativa de octubre de 2007 a
mayo de 2011, Secretaria Académica de enero de 2012 a la fecha.
Organizadora de diversos eventos académicos nacionales e internacionales.
Investigador Asistente del Sistema Sinaloense de Investigadores y
Tecnólogos por el Instituto de Apoyo a la Investigación e Innovación.
Diplomados cursados: Diplomado en Desarrollo en Competencias Docentes en el
uso de las TIC impartido por la Universidad Davinci y Diplomado en
Formación Docente en Modalidades Alternativas impartido por el Sistema de
Universidad Abierta de Educación Continua de la Universidad Autónoma de
Sinaloa. Integrante del Núcleo Académico Básico de la Licenciatura en
Historia, programa acreditado por ACCECESISO desde 2008 a la fecha. Correo
electrónico:  
\href{mailto:mega_1845@hotmail.com}{\textstyleInternetlink{mega\_1845@hotmail.com}}}


\bigskip

{ 
\textit{Mayra Lizzete Vidales Quintero}}

{ 
Es Profesora e Investigadora en la Facultad de Historia de la Universidad
Autónoma de Sinaloa. Licenciada en Historia y Maestra en Historia Regional
por la Facultad de Historia de la UAS. Doctora en Ciencias Sociales por la
UAS-UNISON. Coordinadora de Investigación en la Dirección General de
Investigación y Posgrado de la Universidad Autónoma de Sinaloa. Integrante
del Cuerpo Académico Consolidado de Historia Sociocultural de la Facultad
de Historia. Investigador Nacional Nivel I por el Consejo Nacional para la
Ciencia y la Tecnología (CONAYT) desde 2006 a la fecha y Profesor con
Perfil Deseable por el Programa de Mejoramiento al Profesorado (PROMEP)
desde 2006 a la fecha. Coordinadora del tomo V\textbf{ }correspondiente al
tema de \textbf{Vida social y cotidiana,} del Proyecto \textit{Historia
temática de Sinaloa}\textbf{\textit{,  }}financiado por  Gobierno del
Estado a través del Instituto Sinaloense de Cultura. Actualmente
Colaboradora\textbf{ }del Proyecto \textit{Historia General de
Sinaloa}\textbf{\textit{, }}como en el apartado \textbf{Vida cotidiana y
festividades, S. XIX-XXI} del Tomo III, financiado por el Gobierno del
Estado de Sinaloa a través del Colegio de Sinaloa. Integrante del Núcleo
Académico Básico en los siguientes Programas de Posgrado: Doctorado y
Maestría en Historia, de la Facultad de Historia, reconocidos en el
Programa Nacional de Posgrado en Calidad (PNPC/SEP/CONACYT). Doctorado y
Maestría en Trabajo Social, de la Facultad de Trabajo Social reconocidos en
el Programa Nacional de Posgrado en Calidad (PNPC/SEP/CONACYT).
Licenciatura en Historia, programa acreditado por ACCECESISO desde 2008 a
la fecha. Correo electrónico: 
\href{mailto:maylivi@uas.edu.mx}{\textstyleInternetlink{maylivi@uas.edu.mx}}}


\bigskip

{ 
\textit{Edna Elizabeth Alvarado Mascareño}}

{ 
Es Maestra de Asignatura Base en la Facultad de Historia y Maestra de
Asignatura en la Preparatoria Emiliano Zapata de la Universidad Autónoma de
Sinaloa. Licenciada en Informática en la Universidad Autónoma de Sinaloa.
Doctorado en Pedagogía en el Centro de Investigación e Innovación 
educativa del Noroeste. Certificaciones en Teaching Knowledge Test de la
University of Cambridge ESOL examinations en “\textit{Language and
background to language learning and teaching}”, “\textit{Lesson planning
and use of resources for language teaching}”, “\textit{Managing the
teaching and learning process}”. Certificación  Nacional del Nivel de
Idioma Inglés, por la Secretaria de Educación Pública. Diplomados cursados:
Diplomado en la enseñanza del idioma inglés enfocado a niños y
adolescentes, impartido por el Centro de Idiomas Culiacán de la Universidad
Autónoma de Sinaloa, Diplomado en Competencias Docentes del Nivel Medio
Superior, impartido en la Universidad Autónoma de Sinaloa y Diplomado en
Desarrollo en Competencias Docentes en el uso de las TIC impartido por la
Universidad Davinci. Correo electrónico:
\href{mailto:edna83_am@hotmail.com}{\textstyleInternetlink{\textit{\textcolor{blue}{edna83\_am@hotmail.com}}}}}


\bigskip

{ 
\textit{Patricia Montoya Rivero}}

{ 
Es Maestra en Historiografía de México por la UAM Azcapotzalco y Licenciada
en Historia por la FES Acatlán. Profesora titular de Historiografía de
México y de Métodos y Técnicas de investigación histórica. Sus líneas de
investigación son la historiografía general, la mexicana y en particular la
de publicaciones periódicas en México. Autora de libros de texto para
educación media básica y media superior. Ha coordinado libros colectivos y
escrito diversos artículos de su especialidad. \textcolor{black}{Correo
electrónico:
}\href{mailto:pa_mon_ri@yahoo.com.mx}{\textstyleInternetlink{\textcolor{blue}{pa\_mon\_ri@yahoo.com.mx}}}}


\bigskip


\bigskip

{ 
\textit{María Cristina Montoya Rivero}}

{ 
FES Acatlán, UNAM}

{ 
Es Maestra en Historia del Arte y Licenciada en Historia por la Facultad de
Filosofía y Letras, UNAM. Profesora Titular de Historia y teoría del Arte y
Tutora en la MADEMS, en la FES Acatlán. Sus líneas de investigación son la
historia del arte y la historiografía de publicaciones periódicas. Ha
escrito diversos artículos y libros de su especialidad y es autora de
libros de texto para secundaria desde hace más de 20 años.
\textcolor{black}{Correo electrónico:
}\href{mailto:montriv_2000@yahoo.com}{\textstyleInternetlink{\textcolor{blue}{montriv\_2000@yahoo.com}}}}


\bigskip


\bigskip

{ 
\textit{Ivett Reyes-Guillén}}

{ 
Es Doctora en Ciencias, Ecología y Desarrollo Sustentable, Maestra en
Recursos Naturales y Desarrollo Rural, Bióloga. Profesora de Tiempo
Completo. Línea de Investigación Análisis de Percepciones en Procesos
Sociales. Adscrita a la Facultad de Ciencias Sociales, Universidad Autónoma
de Chiapas. Apoyo PROMEP como nuevo PTC con financiamiento de proyecto de
investigación. Coordinadora de Acreditación de la Facultad de Ciencias
Sociales, UNACH. \textcolor{black}{Correo electrónico:
}\href{mailto:ivettrg2@hotmail.com}{\textstyleInternetLink{ivettrg2@hotmail.com}}\textstyleInternetLink{
}}


\bigskip

{ 
\textit{Carlos Arcos Vázquez}}

{ 
Es Maestro en Educación Indígena. Historiador. Profesor de Tiempo Completo.
Línea de investigación, Historia Regional, Educación y Religión. Adscrito a
la Facultad de Ciencias Sociales, Universidad Autónoma de Chiapas.
Secretario Administrativo de la Facultad de Ciencias Sociales, UNACH.
\textcolor{black}{Correo electrónico:}
\href{mailto:karkos2005@hotmail.com}{\textstyleInternetLink{karkos2005@hotmail.com}}\textstyleInternetLink{
}\textstyleInternetLink{\textcolor[rgb]{0.0,0.0,0.039215688}{y
}}\href{mailto:carcos@unach.mx}{\textstyleInternetLink{carcos@unach.mx}}\textstyleInternetLink{
}}


\bigskip


\bigskip

{ 
\textit{Georgina Flores García}}

{ 
Es oriunda de la Ciudad de Toluca en donde ha realizado la mayor parte de
sus estudios y vida laboral académica. Doctora en Educación por la
Universidad La Salle. Licenciada en Historia,  por la Facultad de
Humanidades, Especialista en Innovaciones Educativas y Maestra en Educación
Superior por la Facultad de Ciencias de la Conducta, ambas de la
Universidad Autónoma del Estado de México y Profesora de Educación Primaria
por la Benemérita y Centenaria Normal para Profesores.  Ha prestado sus
servicios docentes de manera ininterrumpida durante los últimos treinta y
cinco años a diferentes Facultades de su Alma Máter, entre las que se
cuentan: Humanidades, Turismo, Antropología, Medicina, Economía, Ciencias
Políticas y Administración Pública. Ha ocupado diferentes cargos académicos
a lo largo de estos siete lustros. Correo electrónico: 
\href{mailto:ginaflores5601@yahoo.com.mx}{\textstyleInternetlink{\textcolor{blue}{ginaflores5601@yahoo.com.mx}}}}


\bigskip

{ 
\textit{\textcolor{black}{Miriam Edith León Méndez }}\textcolor{black}{ }}

{ 
\textcolor{black}{Es Maestra en Historiografía de México, por la Universidad
Autónoma Metropolitana-Unidad Azcapotzalco,  se desempeña como Profesor e
Investigador de Tiempo Completo adscrito a la Facultad de Humanidades de la
Universidad Autónoma de Campeche. Cuenta con el Reconocimiento de PERFIL
PROMEP desde el año de 2003. Pertenece al Cuerpo Académico }Problemas de
Teorías del Lenguaje, Historiografía  y Exégesis del Discurso Literario,
bajo la\textbf{ }LGAC: Historiografía, Lingüística y exégesis del Discurso
Literario: La construcción sociopolítica. \textcolor{black}{Sus últimas
publicaciones son}\textcolor{black}{ }\textit{Origen y desarrollo de las
Haciendas en Campeche}\textit{\textcolor{black}{ }}\textcolor{black}{y
}\textit{\textcolor{black}{Visiones e interpretaciones históricas de
Campeche }}\textcolor{black}{(Coord.), y }“Tutorías académicas: influencia
del autoconcepto en el rendimiento académico”.  Correo electrónico:
\href{mailto:leonmiry@yahoo.com.mx}{\textstyleInternetlink{leonmiry@yahoo.com.mx}}}


\bigskip

{ 
\textit{Laura Elena Dávila Díaz de León}}

{ 
Es Doctorante en Educación en la Universidad Cuauthémoc. Es Maestra en
Sociología de la Cultura por la Universidad Autónoma de Aguascalientes.
Maestra en Historia por El Colegio de Michoacán, A.C. Profesora adscrita al
Departamento de Historia del Centro de Ciencias Sociales y Humanidades en
donde imparte materias en el área de Talleres y Seminarios e
Historiografía.  Impartición de aproximadamente 95 cursos en presenciales;
12 blended y 10 línea. Perfil Promep. Cuerpo Académico en Formación
Historia de la Sociedad y de las Instituciones de México. Tiene la
Especialidad en Desarrollo de Habilidades del Pensamiento (DHP). Es
egresada del \textit{Diplomado en Formación de Profesores para Educación a
Distancia }(3ª. Generación), del Diplomado en Enseñanza y Aprendizaje
en\textit{ }Ambientes Combinados por la Universidad Autónoma de
Aguascalientes (1ª.Generación) de la UAA. Diplomado en Competencias
Docentes\textit{ }en el Nivel Medio Superior, Capacitación para formador de
instructores, ANUIES/UAA. Correo electrónico:
\href{mailto:ledavila@correo.uaa.mx}{\textstyleInternetlink{\textcolor{blue}{ledavila@correo.uaa.mx}}}}


\bigskip

{ 
\textit{María Socorro Aguayo Ceballos}}

{ 
Es Doctora en Ciencias de la Educación. Maestra en Educación con
especialidad en Comunicación. Licenciada en Ciencias de la Comunicación.
Docente de tiempo completo en la Universidad Autónoma de Ciudad Juárez.
Tiene 25 años de experiencia docente en Educación Media Superior, Educación
Superior, Maestría, Doctorado  y Formación de Docentes en las siguientes
Instituciones: Centro de Actualización del Magisterio, Facultad de Ciencias
Políticas y Sociales de la Universidad Autónoma de Chihuahua, Instituto
Tecnológico y de Estudios Superiores de Monterrey y la Universidad Autónoma
de Ciudad Juárez. Correo electrónico:
\href{mailto:socorro_aguayo@yahoo.com.mx}{\textstyleInternetlink{\textcolor{blue}{socorro\_aguayo@yahoo.com.mx}}}}


\bigskip

{ 
\textit{Ana Karent Muñoz Chávez}}

{ 
Es Estudiante del décimo semestre de la Licenciatura en Historia de la
Universidad Autónoma de Ciudad Juárez. Becaria y asistente de Investigación
desde el año 2012. Ha participado en el XIV Congreso Internacional de
Historia Regional: de fronteras y otras historias. 2013. Ciudad Juárez
Chihuahua. Octubre 2013, en el 1er Congreso Internacional Carl Lumholtz
“Los nortes de México: culturas, geografías y temporalidades” Creel,
Chihuahua. Agosto 2013. }


\bigskip


\bigskip

{ 
\textit{Belén Benhumea Bahena}}

{ 
Doctorante en Humanidades: Estudios Históricos en la Facultad de Humanidades
de la UAEMéx.  Maestra en Humanidades: Estudios Históricos en la Facultad
de Humanidades de la UAEMéx. Generación 2011-2013. graduada con Mención
Honorífica. Licenciada en Historia por la Facultad de Humanidades de la
UAEMéx. Generación 2005- 2010. Mención Honorífica. Asesora de Historia de
México en el Bachillerato Universitario Modalidad a Distancia (BUAD)  de la
UAEMéx  2011-2014. Docente de asignatura de Aprender a Aprender teoría y
práctica en la Licenciatura de Historia en la Facultad de Humanidades de la
UAEMéx. Profesora de Historia en la preparatoria del Instituto Tecnológico
de Estudios Superiores de Monterrey Campus Toluca. Diseñadora de Programas
GEI en el Bachillerato Universitario a Distancia –BUAD- de la Universidad
Autónoma del Estado de México. Correo electrónico:
\href{mailto:beli_ordenyprogreso@yahoo.com.mx}{\textstyleInternetlink{beli\_ordenyprogreso@yahoo.com.mx}}}


\bigskip

{ 
\textit{\textcolor{black}{Juliana Angélica Rodríguez Maldonado}}}

{ 
Es Maestra y Licenciada en Historia por la Universidad Autónoma de México.
Profesora de Tiempo Completo y fundadora de la Licenciatura en Historia de
la Facultad de Filosofía y Letras de la Universidad Autónoma de Tlaxcala.
Tiene perfil PROMEP. Es coautora de libros sobre Historia de Tlaxcala e
historia de las Mujeres. Correo electrónico:
\href{mailto:angelicaromal@hotmail.com}{\textstyleInternetlink{angelicaromal@hotmail.com}}}


\bigskip

{ 
\textit{\textcolor{black}{Teodolinda Ramírez Cano}}}

{ 
Es Licenciada en Historia por la Universidad Autónoma de Tlaxcala. Coordina
la Licenciatura en Historia de la Facultad de Filosofía y Letras de la
UATx. Correo electrónico:
\href{mailto:juliaju16@gmail.com}{\textstyleInternetlink{juliaju16@gmail.com}}}


\bigskip


\bigskip

{ 
\textit{Alfonso Mercado Gómez}}

{ 
Es Doctorante en Educación por la Universidad Autónoma de Durango (UAD).
Maestra en Historia por la Universidad Autónoma de Sinaloa (UAS).
Licenciatura en Historia por la Universidad Autónoma de Sinaloa
(UAS).\textbf{ }Actualmente se desempeña como Director General de Educación
Superior (UAS). Profesor e Investigador de Tiempo Completo adscrito a la
Facultad de Historia de la misma universidad. Presidente y miembro fundador
del Organismo Colegiado para la Evaluación de la Educación Media Superior
A.C. (OCEEMS A.C.), reconocido por COPEEMS. Amplia experiencia en los
procesos de gestión institucional y evaluación educativa. Autor de libro
\textit{Crónicas de Cosalá}, 2014; Galería de Rectores: Breve historia de
la Universidad Autónoma de Sinaloa, en prensa. Correo electrónico:
\href{mailto:alfonsomercado@uas.edu.mx}{\textstyleInternetlink{alfonsomercado@uas.edu.mx}}
y
\href{mailto:Alfonso.uas@hotmail.com}{\textstyleInternetlink{alfonso.uas@hotmail.com}}}


\bigskip

{ 
\textit{María de los Angeles Sitlalit García Murillo}}

{ 
\textcolor{black}{Es Doctorante en Historia, por la Universidad Autónoma de
Sinaloa. Maestra en Historia, por la Universidad Autónoma de Sinaloa.
Profesora Investigadora de Tiempo Completo en la Facultad de Historia de la
Universidad Autónoma de Sinaloa. Perfil PROMEP. Miembro del H. Consejo
Técnico de la Facultad de Historia. Reconocimiento como evaluador
“Formación de pares evaluadores de Programas Académicos, otorgado por el
Consejo de Acreditación en Ciencias Sociales Contables y Administrativas en
la Educación Superior de Latinoamérica, A.C (CACSLA).  Integrante del
Consejo Institucional para la Consolidación de la Calidad Educativa de
Nivel Superior (CICCENS)}\textit{\textcolor{black}{”}}\textcolor{black}{,
UAS. Sus publicaciones más recientes son  María de los Ángeles Sitlalit
García Murillo y Alfonso Mercado Gómez,  “Balance del proceso de re
acreditación del programa educativo de la Licenciatura en Historia de la
Universidad Autónoma de Sinaloa”, }\textit{\textcolor{black}{Currículo y
formación docente en la enseñanza de la
}}\textit{\textcolor{black}{Historia}}\textcolor{black}{, en la
}\textit{\textcolor{black}{Memoria del VIII Encuentro de la
RENALIHCA}}\textcolor{black}{, Universidad Michoacana de San Nicolás de
Hidalgo, Morelia, Michoacán, México 2013.  ISBN: 978-607-8116-30-0. CD
Electrónico: Correo electrónico:
}\href{mailto:sitlalit_77@hotmail.com}{\textstyleInternetlink{sitlalit\_77@hotmail.com}}}


\bigskip

{ 
\textit{Norma Gutiérrez Hernández }}

{ 
Es Doctora en Historia por la UNAM (titulada con Mención Honorífica).
Maestra en Ciencias Sociales (titulada con Mención Honorífica) por la
Universidad Autónoma de Zacatecas (UAZ); Especialista en Estudios de Género
por El Colegio de México y Licenciada en Historia.  Cuenta con Perfil
PROMEP desde el 2008 y es Integrante del Cuerpo Académico En Consolidación
“Enseñanza y difusión de la Historia”  en la LGAC “Historia de la
educación”.  Pertenece a la \textit{Sociedad Mexicana de Historia de la
Educación (SOMEHIDE); }a\textit{ }la \textit{Red de Especialistas en
Docencia, Difusión e Investigación en Enseñanza de la Historia} (REDDIEH);
a la \textit{Red Nacional de Licenciaturas en Historia y sus Cuerpos
Académicos A. C.} (RENALHICA); y al \textit{Sistema Nacional de
Investigadoras e Investigadores} (SNI). Actualmente es
Docente-Investigadora en la Licenciatura en Historia y la Maestría en
Humanidades y Procesos Educativos, ambos de la UAZ. Correo electrónico:
\href{mailto:ninive_17@yahoo.com.mx}{\textstyleInternetlink{ninive\_17@yahoo.com.mx}}
 }


\bigskip

{ 
\textit{Ángel Emiliano Román Gutiérrez }}

{ 
Es maestro en historia por El Colegio de Michoacán y docente-investigador en
la Unidad Académica de Docencia Superior en la Universidad Autónoma de
Zacatecas. Autor de los artículos: “Castorena y Urzua iniciador de la
educación femenina en Zacatecas. El Colegio de niñas de los mil ángeles
marianos” (2012), “La región de Zacatecas y sus fuentes”  (2012), “Historia
de la iglesia en México” (2011), “Depositadas y protección para la mujer en
Zacatecas en el siglo XVIII” (2010). Autor del libro: \textit{Clausura
femenina y educación en Zacatecas en el siglo XVIII} (2012).
 {Correo electrónico:
}\href{mailto:angelemiliano0724@hotmail.com}{\textstyleInternetlink{angelemiliano0724@hotmail.com}}}


\bigskip

{ 
G\textit{loria Pedrero Nieto }}

{ 
Es Doctora en Humanidades, especialidad Historia UAM-I. Líneas de
investigación: Historia de la Educación, Historia Agraria y del Trabajo,
Siglos XIX y XX.  Correo electrónico:
\href{mailto:gpedrero@yahoo.com}{\textstyleInternetlink{gpe}}\href{mailto:gpedrero@yahoo.com}{\textstyleInternetlink{d}}\href{mailto:gpedrero@yahoo.com}{\textstyleInternetlink{rero@yahoo.com}}}


\bigskip

{ 
\textit{Graciela Isabel Badía Muñoz}}

{ 
Es Doctora en Ciencias de la Educación ISCEEM. Líneas de investigación:
Historia y educación, educación a distancia, generación de materiales
didácticos. Correo electrónico:
\href{mailto:isabelbadia61@gmail.com}{\textstyleInternetlink{isabelbadia61@gmail.com}}}


\bigskip

{ 
\textstyleInternetlink{\textit{\textcolor{black}{Rosa María Hernández
Ramírez}}}}

{ 
\textstyleInternetlink{\textcolor{black}{Es Maestra en Educación Superior.
Líneas de investigación, historia y educación, educación superior, 
política educativa. Correo electrónico:
}}\href{mailto:rosshr11896@yahoo.com.mx}{\textstyleInternetlink{\textcolor{black}{rosshr11896@yahoo.com.mx}}}}


\bigskip

{ 
\textit{ {Lidia Medina Lozano}}}

{ 
Es Licenciada y Maestra en Humanidades en el área de Historia y Doctora en
Humanidaes y Artes por la Universidad Autónoma de Zacatecas. Es
Docente-Investigadora del Programa de Licenciatura de la Unidad Académica
de Historia y de la Maestría en Enseñanza de la historia de la Unidad
Académica de Docencia Superior de la UAZ. Desde el 2003 a la fecha tiene el
Reconocimiento nacional del Programa para el mejoramiento del Profesorado
(PROMEP). Es integrante del Cuerpo Académico en consolidación 172  “Teoría,
historia e interpretación del arte”. Sus líneas de investigación abordan la
historia de la arquitectura y el estudio de la imagen como recurso para la
enseñanza-aprendizaje de la historia. Correo electrónico:
\href{mailto:liliu8@yahoo.com}{\textstyleInternetlink{liliu8@yahoo.com}}}


\bigskip


\bigskip

{ 
\textit{ {José Luis Raigoza
Quiñonez}}}

{ 
 {Es Doctor en Historia por la
Universidad Autónoma de Zacatecas “Francisco García Salinas”,
docente-investigador de tiempo completo adscrito a la Unidad Académica de
Historia en su programa de Licenciatura. Cuenta con el reconocimiento de
Profesor con Perfil PROMEP. Pertenece al Cuerpo Académico UAZ-172 “Teoría,
historia e interpretación del arte”. Sus últimas publicaciones son el
libro: }\textit{ {La Historia del
Hospital de San Juan de Dios en Zacatecas, con
}} {ISBN 968-9099-00-0; los
artículos: “Salubridad en el Zacatecas colonial” en Edgar Hurtado Hernández
(Coord.) }\textit{ {La ciudad
ilustrada: Sanidad, vigilancia y población, siglos XVII y
XIX}} { con ISBN:
978-607-7678-64-9; “El establecimiento de la Junta de Sanidad en Zacateas”
en Rubén Ibarra Reyes, Eramis Bueno Sánchez, Rubén Ibarra Escobedo y José
Luis Hernández Suárez (coords.),
}\textit{ {Diferentes
perspectivas y posibles soluciones para la crisis en América
Latina}} {, Zacatecas, México,
UAZ, SPAUAZ, con ISBN: 978-607-8056-26-2.
} {Correo electrónico:
}\href{mailto:luisraigoza52@hotmail.com}{\textstyleInternetlink{luisraigoza52@hotmail.com}}}


\bigskip


\bigskip

{ 
\textit{ {Luis Román Gutiérrez}}}

{ 
Es Licenciado en Humanidades el área de Historia y Maestro en Filosofía e
Historia de las Ideas por la Universidad Autónoma de Zacatecas. Es
Docente-Investigador del Programa de Licenciatura de la Unidad Académica de
Historia. Tiene el Reconocimiento nacional del Programa para el
mejoramiento del Profesorado (PROMEP). Es integrante del Cuerpo Académico
en consolidación \textcolor{black}{150 “Cultura currículum y procesos
institucionales”} de la UAZ. Correo electrónico:
\href{mailto:luroma02@hotmail.com}{\textstyleInternetlink{luroma02@hotmail.com}}}