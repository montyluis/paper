\markboth{}{}
\thispagestyle{empty}
\phantom{abc}
\phantomsection{}
\addcontentsline{toc}{chapter}{PARTE I.\  LAS ÁREAS TERMINALES\\ DE LAS LICENCIATURAS EN HISTORIA} 

\vspace{0.35\textheight}
{\centering \bfseries Parte I\par}
{\centering \bfseries LAS ÁREAS TERMINALES DE LAS LICENCIATURAS\\ EN HISTORIA \par}
\markboth{}{}
\thispagestyle{empty}	
\cleardoublepage{}
%\newpage\null\thispagestyle{empty}\newpage
%\chapter*{Seminario de Catalogación\\ de documentos de archivo}
\thispagestyle{empty}	
\phantomsection{}
\addcontentsline{toc}{chapter}{Seminario de Catalogación de Documentos\\ de Archivo\newline $\diamond$
\normalfont\textit{María Elena Bribiesca Sumano y Guadalupe Zárate Barrios}}
{\centering {\scshape \large Seminario de Catalogación de Documentos de Archivo \par}}
%\addtocontents{toc}{Seminario de Catalogación de Documentos de Archivo}
\markboth{formación del historiador}{seminario de catalogación}

\bigskip
\begin{center}
{\bfseries María Elena Bribiesca Sumano\\
Guadalupe Zárate Barrios}\\
{\itshape\ Universidad Autónoma del Estado de México\/}
\end{center}

\bigskip
\phantom{abc}\par

\smallskip
\noindent {\bfseries Resumen  \par}

\noindent En este trabajo se presenta una reseña de los pasos que se
siguen en el desarrollo del seminario, el cual persigue el objetivo de
contribuir, mediante un catálogo, al rescate del contenido esencial de
las fuentes documentales, a la historiografía, a la búsqueda de
información inédita, a través del análisis del catálogo del Estado de
México, así como a generar competencias en los estudiantes para el dominio
de la lectura e interpretación de los tenores documentales para
estructurar el discurso historiográfico que les permita realizar su
trabajo de titulación: tesis, tesina o ensayo. Los puntos anteriores se han 
conformado con la experiencia obtenida por los estudiantes e investigadores 
que incursionamos en los archivos, al percatarnos que muchos de ellos están 
en franco proceso de pérdida total por el deterioro de la tinta y el soporte.

\enlargethispage{1\baselineskip}
Los escasos catálogos con que cuentan algunos son prueba fehaciente de que
además de ser la fase final del trabajo de archivo, son un recurso para
conservar la esencia del tenor documental, pues con él se reduce el manejo
de los originales, disminuyendo así su deterioro, el inevitable desgaste
por el paso del tiempo y las condiciones inadecuadas en que están los
acervos. Pese a los avances de la tecnología aplicada a los archivos, en
nuestro Estado aún estamos lejos de poder digitalizar su totalidad, pues
algunos ni siquiera se encuentran en las condiciones mínimas de
conservación.\\ 
\textbf{Palabras clave:} Catálogo, archivo, Paleografía,
Diplomática,\\ enseñanza.
%\newpage

\noindent La adquisición de competencias por parte de los 
estudiantes de la licenciatura en Historia a lo largo de cuatro 
semestres, como son las unidades de aprendizaje Métodos y Técnicas de 
Investigación, Lectura Analítica de Textos Históricos, Archivo y manejo 
de fuentes, Paleografía y Diplomática, entre otras, han sido recursos 
que les han permitido el acceso eficiente al 
\textit{Seminario de Catalogación de Documentos de Archivo}, durante el 
cual el {\itshape discentente} tiene que hacer una investigación bibliográfica para 
ubicar espacial y temporalmente el contenido del catálogo que haya 
decidido realizar, así como para comentar la versión de los autores 
que hayan escrito sobre la o las temáticas que se encuentren en el 
mismo catálogo, para de esta manera explicar lo que no se ha mencionado 
en los libros publicados. 

Generalmente, atraídos por el conocimiento de su propia memoria 
colectiva y su identidad personal y social, los estudiantes han 
seleccionado un Archivo que está en su lugar de  origen, donde 
actualmente viven o el más cercano a su domicilio. De esta manera, en el 
\textit{Seminario de Catalogación} se han abordado archivos históricos, 
escolares, municipales, parroquiales y de notarías, dando respuesta a 
las particularidades de cada caso específico.

\medskip
\noindent \textbf{Justificación}
\enlargethispage{1\baselineskip}

\noindent El rescate de fuentes efectuado durante el \textit{Seminario de 
Catalogación de Documentos de Archivo} se refiere al manejo de los 
originales, tarea con la que se logra disminuir su deterioro. Se conserva la síntesis 
representativa del texto del documento que con el tiempo pierde 
legibilidad, amén de que se registra y controla la documentación, se 
tiene conocimiento de cuántos documentos alberga el archivo y cuáles 
son sus temáticas.

\medskip
\noindent {\bfseries Objetivos}
\enlargethispage{1\baselineskip}

\noindent En el \textit{Seminario de Catalogación de Documentos de 
Archivo} nos hemos propuesto alcanzar los siguientes objetivos:

\begin{Obs} 
\item[1.] Contribuir a través de un catálogo al rescate 
del contenido esencial de las fuentes documentales del Estado de 
México. 
\item[2.] Aportar a la historiografía de este estado una 
información inédita a través del análisis del catálogo. 
\item[3.] Generar competencias en la lectura e interpretación de los tenores 
documentales para estructurar el discurso historiográfico que les permita 
realizar su trabajo de titulación. 
\end{Obs}

\medskip 
\noindent En el \textit{Seminario de Catalogación de 
Documentos de Archivo}, intentamos que el estudiante, además de 
comprender el texto de los documentos, aprenda a sintetizarlo y a
redactarlo correctamente para analizar las temáticas contenidas en 
ellos y relacionarlas con el contexto histórico que deberá obtener de 
la consulta bibliográfica; sumará al aporte del catálogo una 
información que, procesada, contribuya a enriquecer la historiografía del 
lugar; además el estudiante podrá entrenarse en las fases que todo 
historiador debe seguir, a saber: En primer lugar la selección de las 
fuentes primarias, después las bibliográficas que le permitan enterarse 
si se ha escrito o no acerca de la o las temáticas que tratan los 
documentos: leer e interpretar las fuentes, ordenar la información y 
redactar la composición que haga con ella.

Los puntos anteriores se han conformado con la experiencia obtenida por 
los estudiantes e investigadores que incursionamos por los archivos al 
percatarnos que muchos de ellos están en franco proceso de pérdida 
total por el deterioro de la tinta y el soporte. Los escasos catálogos 
con que cuentan algunos son prueba fehaciente de que  además de ser la 
fase final del trabajo de archivo, son un recurso para conservar la 
esencia del tenor documental, pues con él se reduce el manejo de los 
originales, disminuyendo así su deterioro, el inevitable desgaste por el 
paso del tiempo y las condiciones inadecuadas en que están los acervos. 
Pese a los avances de la tecnología aplicada a los archivos, en nuestro 
Estado aún  estamos lejos de poder digitalizar su totalidad, pues 
algunos ni siquiera se encuentran en las condiciones mínimas de 
conservación.
%\newpage

\noindent {\bfseries Metodología}
\enlargethispage{1\baselineskip}

\noindent El diseño de este programa es resultado del trabajo colegiado 
de los miembros del Cuerpo Académico «Historia», cuyo objetivo es 
analizar desde una óptica totalizadora los procesos históricos en sus 
diversos ámbitos de competencia, entre ellos el trabajo con fuentes 
primarias, cuya lectura entrenada le permite al historiador adentrarse 
en las diversas instituciones y analizar desde las distintas ópticas  
el devenir histórico de un lugar, en un tiempo determinado.

En la licenciatura de Historia de la Facultad de Humanidades de la 
Universidad Autónoma del Estado de México, los seminarios de titulación 
comprenden seis semestres, durante los cuales se trabaja de la 
siguiente manera:

\noindent \textsl{Primer Semestre, correspondiente a la quinta fase}\\ 
Con el nombre de \textit{Anteproyecto} el estudiante, en este semestre, 
partirá de la selección del archivo que desee trabajar, motivado por la 
temática, necesidades y expectativas que se tengan, como las mencioné 
al inicio de estas líneas, recomendamos además que el tesista escoja el 
más accesible a sus posibilidades, que tenga la seguridad de contar con 
la autorización del responsable del archivo y que se cerciore que esté 
ordenado, en virtud de que el catálogo es la fase final del proceso 
archivístico.

Cumplida esta etapa, se procede a seleccionar el Fondo, si lo hubiere; 
enseguida la serie y el periodo, el cual puede ser de uno o varios años cuando la 
documentación es abundante,  por un periodo de 
administración de gobierno municipal o estatal, un quinquenio, una 
década o una etapa trascendental de nuestra historia. Al seleccionar se 
debe tener cuidado que no haya vacíos prolongados de documentación, de 
tal manera que se pueda tener una secuencia.

El estudiante deberá cubrir los pasos anteriores en dos o tres sesiones, 
como máximo, para iniciar el catálogo y el anteproyecto. Al término del 
semestre tendrá avanzado el catálogo con el cual habrá verificado los 
puntos mencionados e iniciado el anteproyecto de investigación el cual 
queda abierto para hacer las modificaciones que van surgiendo sobre la 
marcha del trabajo.

La realización del catálogo obedecerá  al siguiente procedimiento como 
puntos generales: Selección del lugar, los nombres de las personas 
principalmente involucradas, asunto central y fechas extremas. Como 
puntos particulares: Dependerá de la información específica de la 
documentación. El formato llevará en el encabezado: el número 
progresivo de la cédula, el año, el lugar con su categoría (villa, 
pueblo, rancho, hacienda, estancia, ciudad, etc.) y título del asunto. 
En renglón aparte, la síntesis del tenor documental. También en renglón 
aparte, la fecha completa, en seguida las referencias de colocación del 
material. Si en el  archivo ya se tiene un formato, se verificará que 
los elementos estén completos; si no lo están, se completarán los 
faltantes.

\noindent \textsl{Segundo semestre correspondiente a la 
sexta fase}\\ \textit{Proyecto del trabajo de titulación}. En este 
semestre el tesista se comprometerá a alcanzar un avance sustancial del catálogo en 
proporción al tiempo de que dispondrá, en virtud de que en el $8^\circ$ 
deberá estar concluido.

Al terminar el semestre, a al vez que el estudiante continúa el trabajo, 
deberá tener terminado el proyecto del trabajo de titulación, el cual  
comprenderá el título tentativo, si tiene detectado los asuntos más 
destacados o con mayor frecuencia, la importancia de la elaboración del 
catálogo, los objetivos, el estado de la cuestión y la metodología a 
utilizar.
 
\noindent \textsl{Tercer semestre correspondiente a la 
séptima fase}\\ Denominado en el Currículum 2004 de nuestra licenciatura 
\textit{Estado de la Cuestión}. Paralelamente a la continuación del 
catálogo, el estudiante realizará las lecturas necesarias con las que 
sustentará su trabajo enriqueciéndolo no sólo con las versiones de lo 
que se ha escrito acerca de la temática que aborda el catálogo, sino 
relacionándolas de manera general con la información proporcionada por 
los documentos a manera de un marco histórico. El estudiante 
aprovechará el avance de las lecturas que utilizó para hacer el estado 
de la cuestión de su proyecto, y agregará las que hizo posteriormente.

\smallskip 
\noindent \textsl{Cuarto semestre correspondiente a la 
octava fase}\\ En el Seminario denominado \textit{Fuentes Primarias}, el 
tesista deberá presentar el catálogo ya terminado, corregido por él mismo 
y revisado adecuadamente por los o las asesoras;  asimismo deberá haber preparado los 
índices geográfico, onomástico y temático, consignando los números de 
las cédulas donde se encuentra cada uno de estos elementos, a saber: el 
lugar, el nombre de la persona y el título del tema. 

\noindent  \textsl{Quinto semestre correspondiente a la 
novena fase}\\ \textit{Redacción del Trabajo de Titulación, Primera 
Etapa}. En este semestre se realizará, en forma de borrador, la 
presentación y el estudio introductorio.
 
\noindent \textsl{Sexto semestre correspondiente a la décima fase}\\ 
\textit{Redacción del Trabajo de Titulación, Etapa final}. Con 
el índice temático en el que se  tendrán agrupadas las temáticas, se 
procederá a redactar el análisis. El estudiante comentará la 
información recabada en el catálogo y hará énfasis en lo que hasta 
ahora no se había dado a conocer. Si es pertinente, se pueden hacer 
estadísticas y gráficas, según sea el caso. Por ejemplo, si se trata de 
las defunciones se pueden seleccionar las categorías: causas de muerte, 
edad del difunto, sexo, estado civil, ocupación, casta, religión, lugar 
de procedencia, etc. Finalmente, redacción y revisión definitiva de 
todo el trabajo. Al término del semestre, el estudiante entregará el 
borrador de su trabajo de titulación totalmente terminado.

\smallskip
\noindent \textsl{Observación}. Los asesores harán una 
revisión minuciosa del contenido y redacción de las cédulas y 
resolverán las dudas de lectura o de comprensión que el alumno tenga 
acerca del documento original, ya que aun cuando la redacción sea 
dudosa, siempre conviene revisar minuciosamente el documento, pues 
suele ocurrir que lo que los alumnos escriben no corresponde al 
significado del texto. En esta fase el maestro también deberá realizar 
la revisión del impreso a fin de verificar que las observaciones se 
hayan llevado subsanado correctamente. 

La tesis contará con los siguientes contenidos:
\begin{Obs}
 \item[1.] Presentación. 
 \item[2.] Estudio introductorio y\slash{}o contexto histórico. 
 \item[3.] Análisis del catálogo (Aportación más importante de esta modalidad). 
 \item[4.] Presentación de las fichas catalográficas. 
 \item[5.] Índices. 
 \item[6.] Conclusiones.
 \item[7.] Fuentes bibliográficas consultadas. 
 \item[8.] Anexos (fotografías, mapas, gráficas, etcétera), en su caso. 
\end{Obs}

\enlargethispage{1\baselineskip}
\noindent En la \textit{Presentación}, el estudiante explicará los 
motivos personales por los que escogió el archivo y la documentación 
que manejó, describirá el lugar donde se encuentra el archivo, su 
organización, la Institución de donde proceden los documentos y la 
temática que los caracteriza, los objetivos generales y particulares, 
así mismo explicará cómo está organizado su trabajo, el método y 
técnicas de investigación empleados, las obras que consultó y para qué 
le sirvieron, así como  los pasos que siguió para elaborar el catálogo.

En el «Estudio Introductorio» desarrollará el contexto histórico que se 
hace mediante la consulta bibliográfica que el estudiante realizó en el 
estado de la cuestión descrito en el \textit{Proyecto}, más las obras 
que fue leyendo en el transcurso de su trabajo; ubicará su catálogo en 
el tiempo y en el espacio, hará una breve reseña del lugar y de los 
acontecimientos históricos más destacados del periodo manejado.
%\newpage

\noindent \textsl{Requisitos que debe reunir el catálogo}

\noindent La evaluación profesional por catálogo consiste en la 
elaboración de un trabajo escrito en el que se reporta la 
investigación, critica y análisis documental de carácter archivístico.

El trabajo escrito por catálogo se sustentará de manera individual.

El pasante deberá presentar el aval de un asesor y de dos revisores.

Para determinar la calidad del catálogo, se tomarán en cuenta los 
siguientes aspectos:
%\medskip
\begin{Obs} 
\item[1.] Mostrar originalidad en la temática y en la metodología empleada.
\item[2.] Emplear criterios de validez y confiabilidad propios de la materia.
\item[3.]  Realizar una aportación en la conservación documental.
\item[4.]  Consultar fuentes bibliográficas pertinentes, suficientes y actuales.
La Bibliografía debe ser propia de cada uno de los catálogos, por lo que
sólo se sugiere la general.
\item[5.]  Otros aspectos que contemple el reglamento interno del Espacio
Académico donde se lleve a cabo esta modalidad de titulación.
\end{Obs}

%\bigskip
El contenido de esta modalidad de titulación deberá tener los 
siguientes requisitos de redacción:
\begin{Obs}
\item[1.]  Correcto dominio del idioma español y su redacción.
\item[2.]  Extensión de 50 cuartillas mínimo para el Estudio Introductorio 
y 200 máximo para el total del Trabajo de Titulación.
\item[3.]  Interlineado de 1.5.
\item[4.]  Tamaño de letra: 12 puntos.
\end{Obs}
%\newpage

\smallskip
\noindent \textbf{Conclusiones}
\enlargethispage{\baselineskip}

\noindent El \textit{Seminario de Catalogación de Documentos de 
Archivo} ha tenido resultados importantes, al reportar un alto índice 
de jóvenes que culminan satisfactoriamente las unidades de aprendizaje 
relacionadas con el \textit{Proyecto de Titulación}, al mismo tiempo 
por presentar un considerable número de estudiantes que se titulan casi 
inmediatamente de haber culminado sus estudios de la licenciatura.

La catalogación de los documentos de archivo es propicia para la 
formación de futuros historiadores. La habilitación que ellos adquieren 
es integral, debido a que la formación por competencias se construye en 
la práctica,  pues es en ella donde se podrá constatar si han sido desarrolladas 
correctamente.

Algunas de las tesis con catálogo de documentos de archivo realizadas 
en este seminario son las siguientes:

\smallskip
\noindent \textsc{Archivos escolares}

\smallskip 
\noindent \textsl{Catálogo del Fondo Instituto Científico  
Literario Autónomo. Año 1855}. Presentada el año 2004 por Abril 
Guadalupe Flores González. El catálogo comprende los años 1851--1860, en 
el que predomina la documentación de 1855, fue elegido en atención a 
que el año 1855 fue el preámbulo del Plan de Ayutla. La documentación 
catalogada mostró los planes de estudio, contenidos programáticos, qué 
tipo de alumnos asistían ---pensionados de gracia completa y media 
gracia---, cómo se vestían y en qué consistía su alimentación, se perciben 
las formas en que el Instituto se hacía llegar los recursos monetarios 
para sostener a su alumnado, pagos a maestros, personal administrativo 
y mantenimiento del edificio, ingresos que no eran suficientes para 
solventar todos los gastos, como lo muestran la enorme cantidad de 
vales y nóminas de maestros pendientes de pago. El Instituto satisfizo 
sus necesidades durante esos años gracias a donaciones de particulares 
y a las peripecias de sus directores, quienes lucharon incansablemente 
por mantener una institución que si bien sufría muchas precariedades, a 
través del tiempo dejó huella en los hombres prominentes que formó.
\enlargethispage{\baselineskip}

\smallskip 
\noindent \textsl{Catálogo de la Centenaria y Benemérita 
Escuela Normal de Profesores de Toluca del año 1919}. Presentada en 
marzo de 2011 por María Concepción Peña Huerta. Fue seleccionado este 
año, porque aunque en los años anteriores se habían dado pasos para 
mejorar la situación educativa en el país, es en este tiempo cuando el 
gobierno trató de dar prioridad a la educación de los sectores 
desprotegidos de la sociedad, con base a los ideales de la revolución 
que pretendía por medio de la educación, crear una igualdad de 
oportunidades para todos a través de las cuales se lograría pacificar 
al país y lograr un avance social en todos los sentidos. A lo largo de la 
documentación catalogada se percibe la difícil situación política por 
la que atravesaba el país, tanto a  nivel nacional como estatal, como 
resultado del movimiento revolucionario, por lo que  necesariamente las 
instituciones se vieron impactadas; sin embargo, fue posible lograr 
dentro de la vida institucional y académica dar continuidad al quehacer 
educativo en la entidad. La documentación refleja el vínculo estrecho 
que la Escuela Normal sostuvo con el gobierno del estado, las 
vicisitudes para cubrir los gastos; también desvela la vida administrativa 
{---}~{nombramientos} y licencias del personal--- y académica ---las prácticas 
profesionales, el plan de estudio de 1919, las calificaciones, los 
exámenes ordinarios, extraordinarios, a título de suficiencia y de fin 
de cursos, el control de la disciplina. La preparación de las futuras 
maestras fue exhaustiva y estuvo de acuerdo con las características que señalaban 
las leyes, los reglamentos y planes de estudio. Se pudo observar 
también la relación de la Escuela con otras instituciones, como el 
Instituto Científico y Literario <<Ignacio Ramírez>> con el que 
compartió maestros, y la Escuela Industrial de Artes y Oficios con la 
que se entendía en cuestiones de mantenimiento de mobiliario, edificio 
y utensilios de clase.

Debido a la situación económica por la que atravesaba la entidad, fue 
necesario unir las escuelas de varones y de señoritas, lo que permitió 
dar los primeros pasos en lo que hoy conocemos como escuelas mixtas. La 
Centenaria y Benemérita Escuela Normal para Profesores se ha colocado a 
la vanguardia de la enseñanza  normalista, la disciplina y normatividad 
en la enseñanza le han permitido subsistir durante cien años con el 
mismo prestigio con que nació.
\enlargethispage{\baselineskip}

\smallskip 
\noindent \textsl{La Escuela Normal Mixta para Maestros en 
el año 1920 a través del Catálogo del Archivo Histórico de la Escuela 
Normal para Profesores}. Presentada en noviembre de 2011 por María 
del Carmen Fuentes Cruz. La autora consideró el año 1920 de particular 
interés en virtud de que marca el final de un periodo y el inicio de 
otro con referencia a la lucha armada de la Revolución Mexicana. Al 
concluir ésta,  se empiezan hacer las 
primeras gestiones en materia educativa ante el Congreso para 
crear la Secretaría de Educación Pública; la preocupación del gobierno 
era alfabetizar a la población por lo que se realizan brigadas por todo el país.

En el Estado de México existía una situación irregular en la política, de tal 
manera que se dieron frecuentes cambios de gobernador. Los documentos 
expedidos durante este año por la escuela son de índole administrativa, 
financiera, estudiantil y magisterial, lo que permitió conocer el 
funcionamiento de la Escuela, el número y tipo de alumnos, los ingresos 
y los egresos, su relación con las escuelas anexas y con las 
instituciones gubernamentales, además revelan cómo repercutieron los 
acontecimientos políticos nacionales y estatales en la Escuela.

Por medio de la documentación catalogada se pudo observar cómo en el 
Estado se decretó la municipalización de la educación. El municipio de 
Toluca se hizo cargo de nombrar y pagar el salario a los maestros de 
las escuelas rudimentarias, mientras que el gobierno estatal 
administraba desde la educación elemental hasta la superior, y se ocupó de 
incluir dentro de la escuela elemental anexa la que era dedicada a la 
instrucción rudimentaria. Cualquier movimiento del personal, tanto 
administrativo como académico,  tenía que ser informado al gobierno 
estatal. En cuanto a la vida escolar, la documentación da cuenta de los 
alimentos que se consumían, la disciplina, organización de las fiestas 
escolares, los objetos y utensilios que se ocupaban, las 
calificaciones, asistencias, las materias que cursaban las señoritas y 
las que cursaban los varones, y las becas otorgadas.

La Centenaria y Benemérita Escuela Normal Mixta se ha caracterizado por 
enseñar una conducta integrada con valores, mediante la cual  los futuros maestros 
debían mostrar académica, humana y civilmente lo aprendido en esta 
institución
\enlargethispage{1\baselineskip}

\smallskip 
\noindent \textsl{Catálogo del Archivo Histórico de la 
Escuela Centenaria y Benemérita Normal de Profesores. Caja 132. Año 
1937}. Presentada en diciembre de 2012 por Nancy Berenice Vallejo 
Jiménez. La caja fue seleccionada como parte del grupo de estudiantes 
que catalogaría la documentación generada durante el periodo 
gubernamental del General Lázaro Cárdenas. La información proporcionada 
por la documentación catalogada muestra lo difícil que se tornó la 
aplicación de la educación de corte socialista y los esfuerzos hechos 
con este fin mediante las políticas educativas tanto a nivel nacional 
como estatal. Se pudo observar también que la educación  socialista no 
se pudo aplicar  plenamente por la falta de comprensión de lo que era 
la misma, los problemas con el magisterio y los opositores a su 
aplicación como lo dictaba el artículo $3^\circ$ constitucional, sin embargo, 
se llevaron a cabo algunos puntos como:  el fomento al deporte para  a 
través de él evitar que los alumnos cayeran en el ocio y el vicio, la 
educación estuvo basada en la ciencia con el uso de laboratorios y 
encaminada a la producción e incorporación del alumno a la economía con 
la práctica de los talleres, sin descuidar la incorporación de valores 
y principios. El cuerpo documental permitió asomarse a la vida 
cotidiana de los estudiantes: cómo era su comportamiento, lo que 
sucedía en el internado, cómo vestían, cómo se divertían, cómo se 
organizaban para defender sus derechos y para recabar fondos que 
aliviaran un poco las múltiples carencias.

La Centenaria y Benemérita Escuela Normal para Profesores sigue 
tomando en cuenta, hasta hoy, los postulados de superación humana 
en todos los aspectos propuestos no sólo por el pensamiento socialista, 
sino por otras ideologías políticas, y continúa procurando formar 
maestros de calidad, responsables con la enseñanza, personas que 
conserven los principios de la disciplina y la moralidad.
\enlargethispage{\baselineskip}

\smallskip
\noindent \textsc{Archivos municipales}

%\enlargethispage{\baselineskip}
\smallskip 
\noindent \textsl{Catálogo de las Obras Públicas del 
Municipio de Ocoyoacac. 1973--1975}. Presentada en agosto de 2011 por 
Fabiola Corona Bastida. El  motivo por el que la autora decidió tomar 
el aspecto de las obras públicas en el lugar y años señalados en el 
título, se debió a que éste es su lugar de origen y le interesaba 
averiguar si los programas de urbanización implementados por el 
gobierno estatal de Carlos Hank González y del gobierno federal de Luis 
Echeverría Álvarez, habían sido aplicados también en este municipio. El 
análisis del catálogo permitió seguir los programas de desarrollo 
urbano que se observan en las obras públicas realizadas en Ocoyoacac 
durante este período: obras de limpieza y mantenimiento de 
drenaje, líneas de conducción de agua potable, arreglo y empedrado de 
calles, remodelación de fachadas de casas y del palacio municipal, 
alumbrado público, banquetas y guarniciones, instalación eléctrica. Se 
realizaron obras para el puente del río Chichipicas y del de 
Coapanoaya, también se atendió la ampliación y cuidado del jardín de  
niños, escuelas y lavaderos públicos. El Programa Echeverría de 1974 y 
el Programa Progreso de este mismo año fueron los ejes rectores para 
llevar a cabo estas obras. Estos Programas fueron hechos por medio de 
proyectos diseñados por ingenieros y arquitectos a través de convenios 
con el pueblo. En el catálogo queda constancia de que el programa 
proporcionaba una parte del dinero o el gobierno del estado conseguía 
un crédito para financiar la obra y la otra parte la daba el 
Ayuntamiento, de esta manera la política para llevar a cabo las obras 
públicas tuvo un sentido vertical: gobierno federal, estatal y 
municipal.
\enlargethispage{\baselineskip}

\smallskip 
\noindent \textsl{Catálogo \ de \ Actas \ de \ Cabildo. 
Zi\-na\-can\-te\-pec, 1872--1881}. Presentada por Marco Iván Morales Cuevas en 
septiembre de 2011. El motivo por el cual Marco Iván seleccionó este 
archivo se debe a que en el transcurso de la carrera se fue acuñando 
el interés por obtener información de la historia de este lugar de 
donde es originario y porque en esta búsqueda la bibliografía 
localizada no mencionaba específicamente los asuntos tratados en el 
municipio de Zinacantepec. Tomó el año 1872 por ser el más antiguo 
dentro del fondo  Actas de Cabildo y hasta 1881 porque fue el último  
en el que el presidente en turno registró sus sesiones de cabildo. Los 
asuntos tratados en el catálogo se agruparon de la siguiente manera: 
Agua, ganado, seguridad, deudas, armas, fiestas, cultura, el ejército, 
elecciones, escuelas, tesorería, forestación, juegos prohibidos, 
terrenos y comisiones de instrucción pública, de policía, de guerra, de 
fiel contraste, de salubridad y de hacienda. En el transcurso del 
análisis de la documentación catalogada, se observó que a pesar de la 
inestabilidad social y política, el Ayuntamiento de Zinacantepec hizo 
un esfuerzo supremo por atender las necesidades más apremiantes de la 
población, las experiencias plasmadas en las actas, dejan constancia 
del interés puesto en solucionar los problemas propios del  momento que 
se vivía.
%\enlargethispage{\baselineskip}

\smallskip 
\noindent \textsl{Catálogo de la Sección \ Presidencia \ del \ 
Archivo \ Histórico de Ixtlahuaca. 1868--1869}. Presentada por Jesús 
Guadarrama Marín en abril de 2010. El motivo que condujo al joven 
Guadarrama para escoger su trabajo de tesis se inició desde sus años de 
estudio de la carrera en Historia, al ir avanzando fue aumentando el 
deseo de conocer más y mejor el entorno donde vivía, y se percató que al 
realizar sus tareas escolares se dificultaba la posibilidad de adquirir 
información por la falta de infraestructura de los archivos, así fue 
como decidió investigar en este Archivo. La información que obtuvo 
al realizar el catálogo permitió dar cuenta de las actividades que 
desempeñaba el jefe político como funcionario público de la 
administración local. Se observan los problemas de orden socioeconómico 
que enfrentaban los ayuntamientos debido a la falta de 
recursos y de trabajos remunerados, la escasez de mano de obra en la 
agricultura, la ganadería y la minería, así como la falta de pago a los 
preceptores por lo que unos abandonaban a los grupos de escolares y 
otros estoicamente permanecían en su  misión. A pesar de la situación 
precaria se llevaron a cabo diversas obras públicas. La poca 
preparación de las personas se refleja no sólo en la redacción confusa 
de los documentos, sino en la acumulación de funciones en unos pocos 
como sucedía con el presidente municipal. Se encuentran también 
listados de las personas que, voluntaria o involuntariamente, prestaron 
ayuda pecuniaria o servicios personales a la intervención francesa y al
segundo imperio. Los obstáculos a vencer para la realización de este 
catálogo fueron, entre otros, el deterioro de los documentos, la falta de 
continuidad de los asuntos la ilegibilidad de la letra y la redacción 
confusa de los textos.
\enlargethispage{\baselineskip}

\smallskip 
\noindent \textsl{Catálogo de defunciones. Ocoyoacac. 
1886--1891}. Presentada por Rocío Nayeli Romero Ruiz, en Septiembre 
de 2008, con mención honorífica. La realización de las prácticas 
profesionales consistentes en la catalogación del Ramo Defunciones del 
Archivo Municipal de Ocoyoacac, fueron mostrando poco a poco los 
motivos que causaron más muertes en este municipio, qué grupo de la 
población fue el más propenso a contraer tales enfermedades, a qué 
edad fallecían en mayor proporción, en qué meses había mayor incidencia 
y de qué pueblo o barrio procedían los finados. Toda esta información 
dio pie para analizarla y obtener respuestas concretas que permitirían 
utilizarla en el trabajo de tesis. La autora aclara que el  motivo por 
el que el archivo consultado es municipal y no parroquial, se debe a 
que a partir de la Constitución de 1857, los asuntos considerados del 
estado pasaron a estar bajo el estricto control del Registro Civil como 
fueron los registros de matrimonios y defunciones. El resultado de este 
catálogo declara que la enfermedad que cobró más víctimas infantiles 
fue la viruela, esta situación condujo a que las personas que habían 
sobrevivido a esta enfermedad y que por ella había quedado su cara y 
cuerpo  señaladas por las cicatrices, fueran rechazadas no dándoles 
trabajo, situación que dio cabida al aumento de la pobreza provocándose 
con ella enfermedades gastrointestinales, siendo el año 1890 cuando se 
presenta el mayor número de muertes a consecuencia de la viruela. Las 
enfermedades gastrointestinales cobraron más muertes que la misma 
viruela durante el sexenio 1886--1891. Las enfermedades de origen 
respiratorio como la pulmonía aparecen en todos los años como las
responsables constantes de un alto número de defunciones. La edad de los 
niños que más morían se encuentra de los cinco años hacia abajo, la 
mayoría eran indígenas. El barrio de Santa María es el que presenta un 
mayor número de muertes. No se encontraron enfermedades como la 
diabetes y el cáncer aparece esporádicamente. La mala alimentación, la 
vivienda insalubre y la falta de vacunación impidieron un control 
efectivo de las enfermedades. 
\enlargethispage{\baselineskip}

\smallskip
\noindent \textsl{El Pueblo de Almoloya de Juárez durante 
la consumación de la Independencia de México a través del catálogo de 
su Archivo Municipal. 1820--1821}, presentada por Sarahí Margarita Nava 
Álvarez en noviembre de 2013. Al establecerse en 1820 los ayuntamientos 
constitucionales en Nueva España, se pretendía que fueran instituciones 
administradoras de los recursos de los pueblos,  a la vez que los 
gobernaran con lealtad a su responsabilidad. Fue así como el 
Ayuntamiento de Almoloya fungió como administrador encargándose de las 
contribuciones aportadas por sus pueblos, haciendas y ranchos con el 
objetivo ---en un inicio y antes de que la ciudad de Toluca fuera tomada 
por las tropas de Iturbide---, de cubrir una parte de los gastos erogados 
por el Escuadrón Urbano de Toluca; una vez que el comandante Vicente 
Filisola se apoderó de esta ciudad, los pagos se siguieron efectuando 
aunque no con la misma puntualidad y eficacia por la falta de recursos 
con que contaban las personas, pero entonces el beneficiario era el 
Ejército Imperial Mexicano.

La autora concluye que el pueblo de Almoloya no intervino 
beligerantemente, situación contraria a lo que sucedía en la zona sur 
comandada por Vicente Guerrero; es decir, el apoyo que Almoloya brindó en un 
principio a los ideales realistas, posteriormente se volcó al movimiento 
insurgente, cuando Agustín de Iturbide estableció relación con Vicente 
Guerrero, y fue de índole económica, mediante el dinero que se obtenía por 
el cobro de contribuciones, el envío de ganado vacuno y otros alimentos 
para las tropas.

%\enlargethispage{1\baselineskip}
Una vez que se puso fin a la guerra y se establecieron las bases para 
conformar el nuevo gobierno independiente, el pueblo de Almoloya de 
Juárez tuvo que acatar las disposiciones contenidas en los bandos y 
circulares expedidos por la Junta Provincial Gubernativa y la Regencia.

\noindent \textsl{Jornaleros, soldados y domésticos. Las 
causas de muerte y el Hospital de San Juan de Dios a través del 
catálogo de las actas de defunción del barrio de Santa Bárbara del 
Archivo Municipal de Toluca. 1824--1884} de Carlos Ulises Rosas 
Rodríguez, presentada el 2 de junio de 2014. Como el mismo autor lo 
manifestó, durante su carrera tuvo la inquietud por estudiar 
históricamente el barrio donde ha transcurrido la mayor parte de su 
vida. Explica que la documentación  correspondiente a los años que 
escogió se divide en dos grupos, el primero de 1824 a 1867 se refiere a 
problemas del agua sucedidos en Toluca y es en la segunda parte que va 
de 1868 a 1884  que se integran las actas de defunción expedidas por el 
hospital de San Juan de Dios, mismas que se envían al Juez de lo Civil 
para que autorice el sepelio en el panteón de Santa Bárbara. El trabajo 
muestra un recorrido histórico de la transformación de México durante 
60 años y analiza las causas que motivaron la  muerte de cada uno de 
los grupos a que hace referencia en el título, así mismo describe las 
versiones de varios autores acerca de la temática y concluye que 
ninguno expresa la información obtenida en los documentos catalogados y 
que tiene plenamente comprobada su hipótesis: <<las enfermedades 
registradas en el catálogo tienen una relación causal directa con el 
tipo y condiciones de trabajo a las que estaba sometida la población>> 
---largas jornadas de trabajo a la intemperie, la pobreza que trae 
consigo desnutrición, falta de higiene, de control de plagas, de un 
órgano gubernamental que se ocupara de la salud pública, falta de 
hospitales, desconocimiento de las enfermedades y las causas que las 
originaban.
\enlargethispage{\baselineskip}

\medskip
\noindent \textsc{Archivos parroquiales}

\smallskip 
\noindent \textsl{Catálogo de Registro de Bautizos de la 
parroquia de la \ Asunción \ de \ María, \ Ixtapan \ de \ la \ Sal. \ 1755--1757}. 
Tesis presentada por Blanca Citlali Salgado Méndez en junio de 2012. La 
inquietud por conocer qué grupos sociales habitaron este lugar de donde 
es originaria, durante los años citados fueron el motivo por el que 
Blanca Citlali decidió responder a esta interrogante investigando en 
los libros registro correspondientes a bautizos. Para la realización 
del catálogo fue necesario enfrentarse a la lectura de nombres de 
personas y lugares que no están escritos con claridad y con ortografía 
diferente o que las manchas, deterioro del soporte, mutilaciones y 
paginas faltantes dificultan su identificación. A través de la 
catalogación de este libro registro, se percibe que fue  muy escasa la 
distinción  entre los grupos sociales, pues en el centro de la 
población como en los distintos barrios o poblados cercanos se 
encontraban españoles y principalmente indios, siendo el número de estos 
últimos levemente mayor al de los primeros, aunque en menor escala también 
había mestizos, castizos, lobos y moriscos; se patentiza  la unión 
entre europeos y demás grupos. Otro aspecto que descarta tal distinción 
es que en los libros se encuentran simultáneamente anotados todos los 
grupos, y no existe un libro para cada uno como sucede en otras 
parroquias; de la misma manera se observa la participación en el bautismo, de 
padres y padrinos pertenecientes a distintos grupos. Se presentan 
también los casos en que los niños fueron dejados en las puertas de 
familias particularmente españolas que garantizaran la sobrevivencia 
del pequeño y lo incorporaban a la familia como hijo.

\medskip
\noindent \textsc{En proceso de elaboración del trabajo de titulación, 
primera etapa}

\smallskip 
\noindent \textsl{Viudas, Casadas y Doncellas en la Toluca 
Novohispana. 1706--1707, Su relación con la economía y la sociedad 
regional según el catálogo de la caja 54 del Archivo de la Notaría No. 
1 de Toluca} de Ana Lilia Aladín Díaz.

\noindent \textsl{Paisaje y urbanización en la ciudad de Toluca y una 
mirada a través de sus obras públicas y privadas, 1889--1893} de Iván 
Garduño

\noindent \textsl{Cárceles \ y \ sociedad \ en \ la \ ciudad \ de \ Toluca. \ 
1822--1824} de Sonia Alelí Arámbula.

\noindent \textsl{Circulares de la Jefatura Política de Tenango del 
Valle.1869--1876} de Fernando Domínguez.

\bigskip
\noindent {\bfseries Referencias}

\medskip 
\noindent Heredia Herrera, Antonia (1993), \textit{Archivística 
general. Teoría y práctica}, Sevilla, Servicios de publicaciones de la 
diputación de Sevilla. 

\noindent Torre Villar, Ernesto de la y Ramiro Navarro de Anda (1993),  \textit{ 
Metodología de la Investigación bibliográfica, archivística y  documental}, México, McGraw-Hill México.
\newpage
\thispagestyle{empty}
\phantom{abc}

%\clearpage\setcounter{page}{25}
\thispagestyle{empty}
\phantomsection{}
\addcontentsline{toc}{chapter}{El catálogo documental como vía de titulación en la Facultad de Historia: un instrumento\\ necesario para la investigación histórica\newline $\diamond$
\normalfont\textit{Catalina Sáenz Gallegos, María Guadalupe Carapia 
Medina\\ y Ruben Dario Nuñez Altamirano}}

\begin{center}
{\scshape\ \large El catálogo documental como vía de 
titulación\\ en la Facultad de Historia: un instrumento necesario\\ para la investigación histórica}
\end{center}
\markboth{la formación del historiador}{el catálogo documental}
\setcounter{footnote}{0}

\bigskip
\begin{center}
{\bfseries Catalina Sáenz Gallegos\\
María Guadalupe Carapia Medina\\
Ruben Dario Nuñez Altamirano}\\
{\itshape\ Universidad Michoacana de San Nicolás Hidalgo\/}
\end{center}

\bigskip
\noindent {\bfseries Resumen}

\noindent Desde el año de 1995 se aprobaron en la Facultad de 
Historia de la Universidad Michoacana de San Nicolás de Hidalgo 
diversas vías de titulación para obtener el grado de Licenciado 
en Historia, aparte de la Tesis. Una de estas es la del Catálogo 
Documental, que es la descripción de los documentos en un archivo 
histórico de un determinado tema o época, y el cual se considera 
como uno de los instrumentos más completos por tener una norma 
descriptiva que ayuda a realizar la ficha de un expediente de una 
manera clara y precisa. Para la realización de un catálogo es 
necesario tener conocimiento del archivo y de las cuestiones 
históricas que rodean al documento. Con este instrumento se abren 
nuevas líneas de investigación histórica y archivística. En los 
últimos años esta vía se ha constituido como una de las más 
importantes y a la que recurren cada vez más nuestros egresados 
apoyando a los Archivos Históricos de Michoacán para tener 
instrumentos de consulta claros que facilitan la búsqueda a los 
investigadores y evitan el deterioro de los documentos.\\ 
\textbf{Palabras clave:} Vías de titulación, Catálogo, Archivo
histórico,\\ Líneas de investigación, Instrumento de consulta.

%\newpage
\noindent {\bfseries \textenglish{Abstract}}

\noindent \textenglish{Documentary via the catalogue as degree in 
the school of history: a necessary tool for historical research.

Since the year 1995 were approved at the Faculty of History of the 
Universidad Michoacana de San Nicolás de Hidalgo various routes of 
qualification to obtain a Bachelor's degree in history apart from 
the thesis, one of this is the Catalog Documentary, which is the 
description of the documents in an archive of a particular topic or 
period, and which is considered as one of the most comprehensive 
tools for having a descriptive standard that helps make the record 
of a case of a clear and precise manner. For the realization of a 
catalog is necessary to have knowledge of file and historical issues 
surrounding the document with this instrument new lines of 
historical and archival research are opened. In recent years this 
approach has been established as one of the most important and which 
are increasingly turning our students support Archives of Michoacán 
to have clear query tools make finding investigators and prevent the 
deterioration of documents.}\\
\textenglish{\textbf{{Keywords:}} Historical research, Catalog
documentary, Archive,\\ Lines of historical and
archivalresearch, Investigators.}

\bigskip
\noindent {\bfseries Introducción}

\noindent La hoy Facultad de historia tiene su antecedente más antiguo 
en la Escuela de Altos Estudios Melchor Ocampo fundada en el año de 
1961, en la cual se ofrecían no solo la licenciatura en Historia, sino 
también las licenciaturas de Físico Matemáticas y 
Filosofía,\footnote{AHUM, fondo UMSNH, sección Facultad de Altos 
Estudios Melchor Ocampo. Subserie Planes y programas 1962--1989, caja 
116, foja 87. 12.51.} con el objetivo primordial de formar profesores 
principalmente que impartieran clases en el nivel medio superior, y 
contribuir a formar la conciencia en los problemas sociales, sin 
embargo por cuestiones políticas esta escuela fue cerrada en 1965 y 
algunos de los entonces egresados de la licenciatura de historia 
tuvieron la oportunidad de ir a cursar a la UNAM el último año si 
tenían un promedio general mayor de 8; y el resto podían titularse si 
traían a los sinodales como fue el caso del ex rector Raúl Arreola 
Cortés.%\footnote{Vargas Cabrero, Ada Estela en \url{http://www.vozdemichoacan.com.mx/columna-canciones-y-politica}, 
%accesada el día 7 de Junio de 2014.} 
 Será hasta octubre de 1973 cuando se funde la ahora Escuela de Historia con 
la necesidad de formar profesionales de la historia los cuales se dedicarían 
principalmente a la docencia.



\medskip
\noindent{\bfseries Las opciones de titulación en la Facultad\\ de Historia}

\noindent La única opción, en ese momento, válida para obtener el 
título de Licenciado en Historia era elaborar una Tesis profesional 
sobre un tema determinado, pero la eficiencia terminal de los egresados 
era muy baja porque había muy pocos titulados, por lo que un grupo de 
profesores de la escuela ante esta situación decidieron en el año de 
1995 proponer nuevas opciones de titulación aparte de la tesis 
profesional, que pudieran abatir el bajo nivel de titulados.

Entre las opciones de titulación aprobadas y reglamentadas por el 
Consejo Universitario, estaban la tesina, el examen de conocimientos, 
el Catálogo documental y el curso taller tesina, cada una con sus 
particularidades.

Si bien en un principio los titulados por Catálogo documental no fueron 
bien aceptados por la planta docente de la facultad por considerar que 
no era un trabajo que requería de gran esfuerzo de investigación como 
la tesis, poco a poco fueron admitiendo esta vía, que tiene igualmente 
su grado de dificultad por todo lo que conlleva en su elaboración.

El catálogo documental es uno de los instrumentos de descripción 
archivística más completos, por tener una norma descriptiva que ayuda a 
realizar la ficha de una manera clara y precisa; podemos decir que es 
la descripción de los documentos en un archivo histórico de un 
determinado tema o época (Núñez~Chávez~s\slash{}f,~p.~6), además de que permite 
una mayor claridad del contenido documental que se está consultando, elaborar un 
catálogo documental no solamente beneficia al egresado al obtener su 
título, sino que se convierte en un instrumento eficaz que evita el 
deterioro de la documentación se que resguarda en los acervos.

\enlargethispage{\baselineskip}
Para la realización de un catálogo es necesario tener conocimiento del 
archivo y de las cuestiones históricas que rodean al documento para 
poder hacer de buena manera un resumen de cada expediente citado por 
medio de una ficha de acuerdo a la Norma Internacional General de 
Descripción Archivística y un contexto histórico de los documentos a 
catalogar o de los asuntos que tratan los expedientes, además que el 
interesado en elaborar un catálogo documental debe contar con 
conocimientos de Paleografía y Diplomática, de archivística y tener 
habilidades de redacción y síntesis.

Al hablar de un catálogo documental nos lleva directamente a la documentación de
hechos históricos que han quedado plasmados y que nos acercan al proceso de una época 
que merece su conservación y valoración. Todo escrito es un medio de información, su 
objetivo principal es la de transmitir una información o actos determinados, por ello 
la importancia de saber cuándo fue producido y en dónde \mbox{(De~la~Torre~1991,~p.~107)}.

El catálogo documental es un punto intermedio para poder realizar 
trabajos de corte histórico, metodológico y archivístico, facilita la 
información contenida en la documentación que se cataloga y, al mismo tiempo, 
permite revisar la condición física de los documentos. Además de que es 
una herramienta disponible para los alumnos, profesores, investigadores 
y cualquier otra persona que esté interesada en el tema o algún dato en 
particular.

Este instrumento es un medio básico para dar a conocer todos los 
documentos contenidos que se encuentran en un archivo y que al mismo 
tiempo están ordenados y clasificados por fondo, sección serie y 
subserie colocados en cajas, carpetas o legajos. Los catálogos son los 
impulsores de nuevos temas de investigación, índices o guías, que 
permiten mejorar y agilizar los servicios brindados por el archivo, 
dando un resumen del estado físico de los documentos y de esta forma 
informar sobre los cuidados necesarios para su protección. Funciona 
para tener un mejor control de los expedientes y al mismo tiempo 
corroborar su correcta clasificación y orden, además de permitir la 
recuperación de datos históricos de un hecho o periodo histórico en 
particular.

Los catálogos cumplen el objetivo de poder identificar y explicar el 
contexto de los documentos a estudiar, y de esta manera permitir la 
disponibilidad al público de una forma más accesible. Además de que se 
elaboran por nuestros egresados conscientes de la necesidad creciente 
de abordar nuevas temáticas históricas, y considerando que los 
catálogos documentales poco a poco se han ido volviendo más necesarios 
e importantes dentro de las herramientas preferidas utilizadas por los 
investigadores y el mismo personal que labora en los archivos para un 
mejor manejo de información.

\begin{sloppypar}
La importancia de elaborar un catálogo se sustenta en su utilidad como 
medio de control y auxilio en la localización de la información, 
partiendo de la unidad de descripción más elemental por asunto, el cual 
nos muestra desde sus características más particulares hasta la 
complejidad de toda una masa documental sistematizada en un 
catálogo \mbox{(Villanueva~2002,~p.~133)}.
\end{sloppypar}

De acuerdo con el \textit{Reglamento de Titulación de la Facultad de 
Historia} aprobado por el Consejo Universitario en el año de 1995, en 
el capítulo IV referente al Catálogo Documental,\footnote{En \url{http://cceh.historia.umich.mx/images/Reglamentos/regla_titulacion.pdf}} el 
Artículo $12^\circ$ señala las características que debe contener 
el Catálogo:

\enlargethispage{1\baselineskip}
\begin{Obs}
\item[(1)] Se dice que comprenderá la totalidad de expedientes que conforman 
una serie o sub-ramo, dependiendo del sistema de clasificación 
implementado en el acervo a trabajar, para evitar la discontinuidad o 
fractura.
\item[(2)] La elaboración de fichas para el Catalogo se basara en el análisis 
de los documentos o expedientes.
\item[(3)] Las fichas  analíticas contendrán los siguientes datos:
\item[(a)] Signatura: ubicación física: colocación en estantes, entrepaños y cajas
archivadoras y clasificatorias (fondo, ramo o colección, número de volumen, 
folio, etc.).
\item[(b)] Número progresivo del total de expedientes que contiene el fondo.
\item[(c)]  Cronología (fechas de inicio y extremas) del expediente.
\item[(d)] Lugares de referencia, localidad y entidad a la que pertenecen y 
lugares de gestión
\item[(e)] Personajes participantes: nombres y apellidos que se registran en el 
documento.
\item[(f)] Elementos gráficos e ilustraciones, sellos, ornamentaciones, 
croquis, etc.
\item[(g)] Tipología (características físicas y documentarias del expediente) 
señalando si es original o copia; impreso o manuscrito; libro o soporte 
en que se encuentra. En caso de que el documento no este escrito en 
español, especificar la lengua utilizada.
\item[(h)] La base de datos servirá para elaborar una síntesis de cada 
expediente, usando la terminología de los documentos y registrando los 
asuntos más relevantes.
\item[(4)] El Catálogo Documental se realizara procesando una serie o sub-ramo
completos; cuando estos sean demasiado pequeños
deberán agotarse otros que tengan relación en
contenido.\footnote{Ver \url{http://cceh.historia.umich.mx/images/Reglamentos/regla_titulacion.pdf}}
\end{Obs}


Además se pide que dicho catálogo contenga un ensayo introductorio 
resultado de una investigación que se realice.

La evaluación del catálogo será igual que el de las otras opciones, un 
examen réplica ante un jurado integrado por tres sinodales.

Para la presentación en físico del catálogo, este deberá estar 
integrado por las siguientes partes:Introducción en la cual se 
conceptualiza sobre lo que es un archivo, su importancia, 
clasificación, el ciclo vital, la descripción archivística y los 
instrumentos de descripción desglosando cada uno y resaltando el valor 
del catálogo. Enseguida se redacta la justificación, donde se define lo 
que es un catálogo, su importancia, que líneas de investigación se 
abren con este trabajo y a los investigadores que les será útil el 
trabajo, después los objetivos generales y específicos.

\enlargethispage{1\baselineskip}
Un segundo capítulo aborda la historia del archivo en el cual se 
catalogara y que comprende dos apartados: la historia del 
acervo y la breve historia institucional del fondo que se trabaje.

Un tercer apartado lo constituye la metodología que son los pasos a 
seguir para elaborar dicho instrumento especificando primero el Archivo 
en el que se trabajara y la serie, luego la ficha modelo para catalogar 
utilizando la norma ISAD\,(G), la cual es una norma estandarizada empleada 
en la descripción archivística, la cual establece un criterio de 
ordenación que va de lo general a lo particular de acuerdo a su entidad 
original que lo produjo pero también de lo particular a lo general. 

\begin{Obs}
\item[$\bullet$]{Lo primero cumple con la descripción del asunto.}
\item[$\bullet$]{Lo segundo igualmente, al contemplar datos específicos: nombre, 
fecha, lugar y tema.}
\end{Obs}

%\enlargethispage{1\baselineskip}
De esta forma el catálogo reúne el principio de descripción multinivel. 
A continuación se muestra un ejemplo de una ficha modelo:

\bigskip
\begin{footnotesize}
\begin{mdframed}
\noindent {\bfseries Área de Identificación}\\
{\bfseries\textit{Código(s) de Referencia:}}MX16053AHCMO\slash\ Fondo:
Parroquial\slash\ Sección: Disciplinar\slash\ Serie:\\ Padrones\slash\ Subserie:Asientos, Fojas: 2--34.\\
C.903, Carp. 2\\
{\bfseries \textit{Título:}} Padrón, Puruándiro.\\
{\bfseries\textit{Fecha(s):}}1800
{\bfseries \textit{Nivel de Descripción}}: Unidad Documental Compuesta\\
{\bfseries\textit{Volumen y Soporte de la Unidad de Descripción:}} 32
fojas\\
{\bfseries\textit{Productor(es):}} Desconocido
\end{mdframed}
\end{footnotesize}
%\newpage

\bigskip
\noindent{\bfseries Área de contenido y estructura}
\enlargethispage{1\baselineskip}

\noindent \textbf{Alcance y Contenido:} El padrón muestra la 
calidad racial de los habitantes, hace la división de los mismos por 
casas o familias, los indios de la zona se enlistan aparte. Se numera 
un total de $6, 670$ almas en la jurisdicción.
\newpage

En segundo lugar se aborda la tipología y diplomática. La Tipología 
documental se realiza conforme a los tipos documentales encontrados en 
los expedientes catalogados, los que pueden ser dispositivos, como 
solicitudes, demandas.

Por lo que respecta a la Diplomática, como sabemos es una disciplina 
que tiene por objeto el estudio y crítica de la traducción, forma y 
elaboración de los documentos escritos resultantes de acciones 
jurídicas y actividades administrativas realizadas por personas físicas 
o jurídicas, basándose principalmente en el análisis de los caracteres 
extrínsecos e intrínsecos del documento, con el fin de que el estudio 
diplomático de los mismos, nos arrojen indicios de autenticidad 
histórica y poder entonces entrar de manera directa al trabajo 
catalográfico que corresponde

Criterios archivísticos y paleográficos: En el primero se describe el 
estado físico de los documentos, si están empastados, en cajas o libros 
y que están organizados conforme a la ordenación existente en el 
Archivo, en los criterios paleográficos tipos de letras, sus 
características, tomando el criterio de modernizar la ortografía 
excepto en los nombres propios de personas y lugares. Y asimismo se 
desenlazaran las abreviaturas

Un cuarto apartado es el contexto histórico en base al periodo 
histórico catalogado utilizando mayormente la información obtenida de 
las fichas.

\enlargethispage{1\baselineskip}
El quinto apartado es la relación de fichas catalográficas, enseguida 
las conclusiones y por último los anexos en los cuales van los índices 
onomástico que se refiere a los principales participantes de cada uno 
de los expedientes revisados y el geográfico que registrara el lugar 
donde estos se generaron. El temático en el cual se registran 
mayormente los asuntos catalogados, se incluyen gráficas que permiten 
hacer una cuantificación de la serie catalogada, un glosario de 
terminología archivística y de la época, un listado de abreviaturas 
encontradas en los documentos y por último las fuentes de consulta 
tanto bibliográficas como hemerográficas o electrónicas.

\enlargethispage{1\baselineskip}
La elaboración de los catálogos documentales se encuentra respaldada 
por un marco normativo como la ya mencionada Norma Internacional 
General de Descripción Archivística ISAD\,(G) que marca la pauta para 
elaborar las fichas catalográficas y la  del IFAI, que es el derecho a la 
Información pública. Durante el Gobierno de Vicente Fox se mostró la 
iniciativa de Ley Federal de Transparencia y Acceso a la Información 
Pública Gubernamental, la cual dio lugar a la creación del Instituto 
Federal de Acceso a la Información Pública; esta ley fue aprobada el 11 
de junio del 2002. En el primer semestre de 2010, el Congreso de la 
Unión aprobó la Ley Federal de Protección de Datos Personales en 
Posesión de Particulares, lo cual amplió sustancialmente las 
facultades, atribuciones y responsabilidades del Instituto, al ser 
considerado como autoridad nacional en la materia. Asimismo, modificó 
su nombre al de «Instituto Federal de Acceso a la Información y 
Protección de Datos», publicada el 5 de julio de 2010, que establece en el 
capítulo I que tiene por objeto la protección de los datos personales 
en posesión de los particulares, con la finalidad de regular su 
tratamiento legítimo, controlado e informado, a efecto de garantizar la 
privacidad y el derecho a la autodeterminación informativa de las 
personas. En su \mbox{artículo séptimo} menciona que los datos personales 
deberán recabarse y tratarse de manera lícita conforme a las 
disposiciones establecidas por esta Ley y demás normatividad aplicable. 
Esta Ley también nos dice que no será necesario el consentimiento para 
el tratamiento de los datos personales cuando los datos figuren en 
fuentes de acceso público como los que se encuentran resguardados en 
Archivos.  

En cuanto al procedimiento de verificación se específica en el artículo 
59 que el Instituto (IFAI) verificará el cumplimiento de la presente 
Ley y de la normatividad que de ésta derive. La Ley Federal de 
Transparencia y Acceso a la Información Pública Gubernamental fue 
divulgada el 11 de junio de 2002 teniendo como finalidad garantizar el 
acceso de toda persona a la información en posesión de los Poderes de 
la Unión, los órganos constitucionales autónomos o con autonomía legal, 
y cualquier otra entidad federal. Son objetivos de esta Ley: Proveer lo 
necesario para que toda persona pueda tener acceso a la información 
mediante procedimientos sencillos; transparentar la gestión pública 
mediante la difusión de la información que generan las instituciones 
gubernamentales; garantizar la protección de los datos personales en 
posesión de las instituciones o unidades administrativas; favorecer la 
rendición de cuentas a los ciudadanos, a fin de que puedan valorar el 
desempeño de dichas unidades; mejorar la organización, clasificación y 
manejo de los documentos, y contribuir estado de derecho de la 
ciudadanía. En el artículo 32 menciona que le corresponderá al Archivo 
General de la Nación elaborar, en coordinación con el Instituto (IFAI), 
los criterios para la catalogación, clasificación y conservación de los 
documentos administrativos, así como la organización de archivos de las 
dependencias y entidades. 
%\enlargethispage{-1\baselineskip}

\enlargethispage{1\baselineskip}
La Ley de Archivos Administrativos e Históricos del Estado de Michoacán 
de Ocampo y sus Municipios, publicada en el Periódico Oficial el 3~de~marzo 
de 2004, tiene como característica particular normar y regular 
la administración de los archivos, así como la preservación, 
conservación y difusión de los documentos y el patrimonio documental 
del Sector Público del Estado de Michoacán de Ocampo y sus Municipios, 
así como todos aquellos cuyo contenido tenga un interés histórico. En esta ley 
se dicta el empleo de las fichas catalográficas conforme a la 
norma ISAD\,(G). Define a la totalidad de documentos de cualquier época 
generados, conservados o reunidos en el ejercicio de su función por el 
sector público; y los de propiedad de personas físicas o morales 
privadas, por su valor histórico o cultural para el Estado, como 
Patrimonio Documental, lo cual resalta la importancia y el valor de los 
documentos.

Esta legislación norma que en todo el sector público deberá existir un 
archivo de trámite; y en cada uno de los Poderes del Estado y los 
Ayuntamientos, un archivo de concentración y uno histórico. En los 
archivos de Concentración e Históricos se deberán aplicar principios, 
normas y técnicas archivísticas en los procesos de ordenación, 
clasificación y catalogación de los documentos. En los Archivos 
Históricos se comprometerá a realizar índices y catálogos de la 
documentación que esté bajo su custodia. Los Archivos Históricos 
proporcionarán el servicio de préstamo y consulta pública observando 
las normas, lineamientos o disposiciones que para ello establezca el 
sector público en su reglamento interno.

Más recientemente se encuentra la Ley Federal de Archivos, la cual fue 
decretada por parte del Congreso General de los Estados Unidos 
Mexicanos y publicada en el diario oficial el 23 de enero de 2012, sin 
ninguna reforma hasta la fecha, y cuyo principal objetivo es establecer 
las disposiciones que permitan la organización y conservación de los 
archivos en posesión de los Poderes de la Unión, los organismos 
constitucionales autónomos y los organismos con autonomía legal, así 
como establecer los organismos de coordinación y concentración entre la 
Federación, las entidades federativas, el Distrito Federal y los 
municipios para la conservación del patrimonio documental de la Nación, 
así como fomentar el resguardo, difusión y acceso de archivos privados 
de relevancia histórica social, técnica, científica y cultural. Además 
de ello, dicha ley cuenta en su haber los lineamientos que se deben de 
seguir para evitar sanciones e infracciones. Aunque también es 
importante tomar en cuenta la legislación particular de cada acervo en 
que se vaya a trabajar, como la Ley de Archivos Municipales, por mencionar 
alguna.

\bigskip
\noindent {\bfseries Archivos Históricos en los que se han realizado\\ 
los catálogos documentales}

\noindent Como se indicó, los Catálogos documentales han apoyado el 
trabajo en los Archivos, facilitando un instrumento de consulta muy útil 
para los que los analizan, así como para apuntalar la labor de los que 
laboran en dichos acervos.

A través de convenios con los jefes de los Archivos, algunos de los 
cuales han fungido como asesores de estos catálogos, y las autoridades 
de la Facultad de Historia, nuestros egresados han ya elaborado un 
sinnúmero de ellos desde 1999 hasta la fecha, obteniendo algunos de 
ellos mención honorifica por el trabajo realizado y por su excelente 
defensa, se ha trabajado principalmente en los siguientes archivos: 
%\newpage

{\scshape Archivo General de Notarías: Ramo de Tierras y Aguas época\\ Colonial}.

{\scshape Archivo Histórico del Poder Ejecutivo}.\quad Siendo uno de los más trabajados 
y principalmente de los Ramos de Hijuelas, División Territorial, 
Asuntos Religiosos y más recientemente Pasaportes y Visas de los siglos 
XIX y XX\@.

{\scshape Archivo Histórico del Poder Judicial}.\quad Principalmente el Ramo Penal 
de algunos Juzgados del siglo XIX\@.

{\scshape Archivo Histórico de la Catedral de Morelia}.\quad En éste se han 
catalogado los Libros de Cabildo de la época Colonial, siglos XIX y XX\@.

{\scshape Archivo Histórico Municipal de Morelia}.\quad Las Actas de Cabildo siglo XIX 
y los Fondos Independiente I y II del siglo XIX\@.
 
{\scshape Archivo Histórico Casa de Morelos}.\quad Donde se han trabajado los Fondos de 
Inquisición, Monjas Catarinas o de Santa Rosa, la Serie Matrimonios y 
más recientemente la de Padrones poblacionales de la época Colonial.

E igualmente se han trabajado algunos {\itshape Archivos Municipales\/}, como los de 
Uruapan o Zitácuaro, e inclusive se ha elaborado un catálogo del {\itshape Archivo del 
Real de Minas de Pachuca\/}.
%\newpage

\enlargethispage{1\baselineskip}
En los últimos años ha crecido el número de egresados que optan por 
esta vía, e igualmente el número de profesores que han solicitado 
asesorarlos; sin embargo este interés desmedido ha generado algunos 
problemas, como trabajos presentados sin la rigurosidad que se pide o 
asesorías sin pleno conocimiento de la archivística o de la temática a 
tratar en los acervos, e igualmente alumnos interesados que 
posteriormente abandonan el trabajo, por lo que se han tomado una serie 
de medidas encaminadas a mejorar esta situación y además incrementar el 
número de titulados, concientizando al plantel docente de la Facultad 
sobre el valor de estos trabajos, que no solo son una serie de fichas de 
expedientes sin ningún valor, o trabajos que se realizan en solo una 
semana, sino trabajos que realmente contribuyen en gran medida a la 
investigación histórica como ya se mencionó, proporcionando nuevos temas 
para investigar y novedosas líneas de investigación, como las 
económicas, jurídicas, geográficas o sociológicas.

Hace dos años aproximadamente se instituyó un curso taller sobre Catálogo, 
coordinado por el Maestro David Eduardo Ruiz Silera, y como profesores 
el Licenciado Álvaro Marcos Martínez Director del Archivo del Poder 
Ejecutivo y el Maestro Jaime Reyes Monroy Director del Archivo Casa de 
Morelos, quienes coordinarían mesas integradas por diez alumnos, quienes 
trabajarían en los archivos ya mencionados y con la serie que 
sus asesores les designen; el compromiso que asumieron fue que en un 
periodo de seis meses, que tiene duración el curso, se titularían, con lo 
cual se iría incrementando la eficiencia terminal. Hasta la fecha se 
sigue celebrando dicho curso taller, integrándose nuevos profesores y 
nuevos archivos en los cuales trabajar, y sí se han tenido buenos 
resultados, ya que se han titulado incluso alumnos que habían egresado 
hacía 11 o 15 años. Una de las ventajas de este curso taller es que es 
sabatino, lo que no interrumpe la labor cotidiana de los egresados. 
Cuenta con una atención más personalizada del asesor, quien hace un 
seguimiento de su asesorado hasta que se titula, lo que hace en un corto 
tiempo.

Actualmente, y luego  de observar todas la problemática en torno a esta 
importante vía, se ha decidido hacer algunas adecuaciones al Reglamento de 
Catálogo. Los maestros David Eduardo Ruiz Silera, Jaime Reyes Monroy 
y Catalina Sáenz Gallegos hemos, a través de reuniones, 
acordado algunas actualizaciones al reglamento tomando en cuenta la 
normatividad reciente sobre archivística, los cambios recientes en la 
materia archivística, y procurando que realmente se realice un trabajo de calidad 
con todo lo que esto implica. Este trabajo todavía se está realizando, y 
en cuanto se termine se pasará al Consejo Técnico para su aprobación y 
posterior ejecución. Inclusive se está trabajando en la defensa que 
realizará el sustentante apoyándose en una presentación en {\it power 
point}, pero en la cual solo serán admitidas imágenes sin texto como 
apoyo.
\newpage

%\begin{footnotesize}
\noindent {\itshape\ Notas Aclaratorias\/}
\enlargethispage{1\baselineskip}

\medskip
\begin{Obs} 
\item[1.-] La ISAD\,(G) es un Instrumento para estandarizar la 
descripción archivística aprobada por el Consejo Internacional de 
Archivos y el \hbox{Archivo} General de la Nación. Constituye una guía general 
para la elaboración de descripciones archivísticas, con la idea 
fundamental de lograr uniformidad en los trabajos de descripción de 
unidades documentales, a fin de hacerlos accesibles mediante 
representaciones precisas que se organizan de acuerdo a modelos 
predeterminados, utilizando veintiséis elementos que pueden combinarse 
para constituir la descripción de una entidad archivística y las 
diferentes fases de gestión en los que se encuentre el documento. Los 
procesos descriptivos pueden ser simultáneos a la producción de los 
documentos y continuar a lo largo de su ciclo vital. Los veintiséis 
puntos se apoyan en los principios de la archivística Internacional 
aplicados (el principio de procedencia orden original) iniciando la 
descripción de lo general a lo particular. Principios de observancia 
para la aplicación de la norma internacional de descripción ISAD\,(G).
\enlargethispage{-1\baselineskip}


\item[2.-] La Ley Estatal de Archivos de Michoacán es de 
orden público e interés social y tiene por objeto normar y regular la 
administración de los archivos, así como la preservación, conservación 
y difusión de los documentos y del patrimonio documental del sector 
público del Estado de Michoacán de Ocampo y sus Municipios, así como 
todos aquellos cuyo contenido tenga un interés histórico.
\end{Obs}
%\end{footnotesize}
\newpage

\noindent {\bfseries Referencias}
%\enlargethispage{1\baselineskip}

\medskip
Alday García, Araceli (2004), \textit{Introducción a La Operación de Archivos en
Dependencias y Entidades del Poder Ejecutivo}, México, AGN.

\textit{Consejo Internacional de Archivos, ISAD\,(G)} (2000), Norma Internacional
General de Descripción Archivística, Madrid.

Corona Bustos, Martha Luz (2000), \textit{Archivo General de Notarias, Perspectivas de
investigación Multidisciplinaria}, México, UMSNH.

De la Torre Villar Ernesto (1991), \textit{Metodología de la Investigación.
Bibliográfica, archivística y documental}, México, UNAM.


Heredia Herrera, Antonia (1991), \textit{Archivística General. Teoría y 
práctica}, Sevilla, Diputación Provinvial de Sevilla. 

Villanueva Bazán, Gustavo (2002), \textit{Manual de procedimientos técnicos para
archivos históricos de universidades e instituciones de educación superior}, 
México, UNAM\slash{}CESU-BUAP. 

\medskip
\noindent {\bfseries Fuentes Hemerográficas}

\textit{Ley de Archivos Administrativos e Históricos del Estado de Michoacán de
Ocampo y sus Municipios}, Texto original, Publicado en el periódico Oficial 
el miércoles 3 de marzo del 2004, Tomo CXXXIII, Núm\@. 3.

\medskip
\noindent {\bfseries Fuentes Electrónicas}

\begin{sloppypar}
\textcolor{black}{LEY FEDERAL DE ARCHIVOS en}
\url{http://www.diputados.gob.mx/LeyesBiblio/pdf/LFA.pdf}.
\end{sloppypar}

Cruz Mundet \textit{«Manual de Archivística».}

Antonia Heredia «Archivística General». \textit{Introducción a la Archivística}.
%\url{http://www.mundoarchivistico.com.ar/?menu=articulos&accion=ver&id=299}

Lodolini Elio.\, \textit{El archivo del ayer al mañana. (La archivística entre 
tradición e innovación)}, pág. 5\@. Doc.\ electrónico en línea en 
\url{http://dialnet.unirioja.es/descarga/articulo/50952.pdf}.

\begin{sloppypar}
Núñez Chávez, Jorge (s\slash{}f), \textit{Los archivos administrativos en México}. En
\url{http://www.adabi.org.mx/content/servicios/archivistica/articulos/civilarticulos/archivosadmin.jsfx}.
Fecha consultada: 2 de junio de 2013.
\end{sloppypar}


