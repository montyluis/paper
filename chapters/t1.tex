%%\clearpage\setcounter{page}{445}
\thispagestyle{empty}
\phantomsection{}
\addcontentsline{toc}{chapter}{La tutoría: un compromiso de apoyo a lo\newline largo de la profesionalización de los\newline estudiantes de la Licenciatura en Historia\newline de la UATx\newline $\diamond$
\normalfont\textit{M.\ J.\ Angélica Rodríguez Maldonado\newline y
Teodolinda Ramírez Cano}}
{\centering {\scshape \large La tutoría: un compromiso de apoyo a lo largo\\ de la profesionalización de los estudiantes de la Licenciatura en Historia de la UATx}\par}
\markboth{la formación del historiador}{la tutoria un compromiso}
\setcounter{footnote}{0}

\bigskip
\begin{center}
{\bfseries M. J. Angélica Rodríguez Maldonado\\
Teodolinda Ramírez Cano}\\
{\itshape Universidad Autónoma de Tlaxcala} 
\end{center}

\bigskip
Ante los retos que se plantean en este nuevo siglo, en el seno de una sociedad cuya  dinámica se sustenta  esencialmente  en  el  conocimiento,  la  educación superior mexicana requiere transformarse teniendo como eje una nueva visión y un nuevo paradigma para la formación de los estudiantes, entre cuyos elementos están el aprendizaje a lo largo de toda la vida, la orientación prioritaria hacia el aprendizaje autodirigido (aprender a aprender, aprender a emprender y aprender a ser) y  la formación  integral con una visión  humanista y  responsable ante  las necesidades y oportunidades del desarrollo de nuestro país. En este escenario la atención personalizada del estudiante constituye, sin lugar a dudas un recurso de gran valor, ya  que al visualizar  al alumno como el actor  central del  proceso formativo, además de propiciar el logro de los objetivos indicados, contribuye a la adaptación del estudiante al ambiente escolar y al fortalecimiento de sus habilidades de estudio y de trabajo. Este tipo de atención puede ayudar, adicionalmente, a disminuir los índices de reprobación y rezago escolar, las tasas de abandono de los estudios universitarios y a mejorar la eficiencia terminal.

Entre los problemas que enfrentan con mayor frecuencia las Instituciones de Educación Superior 
(IES) de nuestro país, en el nivel de licenciatura, se encuentran: la deserción, el rezago estudiantil y los bajos índices de eficiencia terminal, esta última entendida como la\linebreak proporción de alumnos que habiendo ingresado en un determinado momento a la licenciatura, la concluyen en el plazo establecido en el plan de estudios; pero que no logran obtener el grado correspondiente. Por tanto la deserción como  el  rezago son  condiciones que  afectan  el  logro de  una  alta eficiencia terminal en las instituciones. Algunos datos preocupantes datan de la década de los noventa donde la eficiencia terminal de las instituciones públicas mexicanas fluctuó entre el 
51.2\,\% y el 62\,\%.

Haciendo referencia a cifras generales, como promedio a nivel nacional, de cada 100 alumnos que iniciaron estudios de licenciatura, entre 50 y 60 concluyeron las materias del plan de estudios, 5 años después, y de estos, tan sólo 20 obtienen su título. De los que se titulan, solamente un 10\,\%, es decir 2 egresados, lo hacen a la edad considerada como deseable (24 o 25 años); los demás lo hacen entre los 50 y 60 años de edad (Díaz de Cossío 1998). 

Resultado de las investigaciones realizadas en la Universidad Veracruzana en este mismo período, nos refieren que (Chaín 1999), aproximadamente 25 de cada 100 estudiantes que ingresan a realizar estudios universitarios abandonan sus  estudios  sin  haber  aprobado  las materias  correspondientes al  primer  semestre; además  la mayoría  de ellos  inician  una carrera  marcada  por la reprobación y por los bajos  promedios   en  sus  calificaciones, lo cual contribuye  a que  en  el tercer semestre   la  \textit{deserción}  alcance  un  36\,\%  de  quienes  ingresaron,  cifra  que  se incrementa, semestre con semestre, hasta alcanzar el 46\,\% al término del periodo de la formación profesional.

Los diversos  organismos  que durante  la última  década  han  analizado el sistema de educación  superior  mexicano  (CIDE,  OCDE,  SEP, ANUlES),  señalan  como  sus principales  problemas  una baja eficiencia\linebreak  terminal  a través  de sus altos  índices  de deserción  (50\,\%);  un importante  rezago  en los estudios,  resultado  de  altos índices de reprobación  y bajos  índices  de titulación 
(50\,\%).  Esta problemática no es ajena a la Universidad Autónoma de Tlaxcala.

Por todo ello el establecimiento del Sistema Institucional de Tutorías se consideró que éste, podría tener un impacto positivo en la solución de los problemas antes mencionados y elevar la eficiencia terminal, contribuyendo de manera importante a la formación integral de los estudiantes universitarios y con ello:

\begin{Obs}
\item[$\bullet$] Ayudar en el desarrollo de una metodología de estudio y trabajo apropiado para las exigencias del primer año de la licenciatura.
\item[$\bullet$] Ofrecer al alumno apoyo y supervisión en temas de mayor dificultad en las diversas asignaturas.
\item[$\bullet$] Crear un clima de confianza que permita conocer otros aspectos de la vida personal del alumno, que influyen directa o indirectamente en su desempeño académico.
\item[$\bullet$] Señalar y sugerir actividades extracurriculares que fa\-vo\-rez\-can un desa\-rro\-llo pro\-fe\-sional integral del estudiante.
\item[$\bullet$] Brindar información académica-administrativa que atienda las peticiones y necesidades del alumno.
\item[$\bullet$] Fortalecer la relación maestro-alumno
\item[$\bullet$] Apoyar la formación profesional del estudiante, abarcando aspectos científicos, humanísticos y éticos
\end{Obs}
\newpage

\textbf{Definición del sistema institucional de tutorías}

El Sistema Institucional de Tutorías está integrado por un conjunto de  acciones dirigidas a  la atención individual de  los estudiantes, y opera aunado a otro conjunto de  actividades diversas que apoyan la  práctica tutorial; puesto  que  responden  a  objetivos  de carácter general y son atendidas por un personal académico altamente comprometido con su trabajo, el cual brinda atención individualizada al estudiante.

\bigskip
\textbf{Marco normativo}

Una de las preocupaciones esenciales de la legislación mexicana señala que el desarrollo del ser humano deberá ser integral. Por eso, el Artículo 3° constitucional establece, en su fracción VII, que las universidades e instituciones de educación superior realizarán sus fines de educar, investigar y difundir la cultura de acuerdo con los principios que establece el mismo artículo en su segundo párrafo, donde  a la letra dice:  «La  educación  que  imparte  el  Estado tenderá a  desarrollar  armónicamente todas  las facultades  del  ser  humano  y fomentará en  él, a  la vez, el amor a  la Patria y la conciencia de solidaridad internacional  en la independencia y la justicia».

La Ley General de Educación señala, en el primer numeral del Artículo,  que los fines de la educación son: «Contribuir al desarrollo integral del individuo para que ejerza plenamente sus capacidades humanas».

Por otra parte,  el programa de desarrollo institucional del 2001--2006 da continuidad a lo establecido en el programa anterior, el cual señalaba como elemento estratégico para alcanzar el  objetivo  de  calidad en  la educación superior un  compromiso  de: «Efectuar acciones  que  permitan  atender  y  formar  a  los  estudiantes en los aspectos que inciden en su maduración personal: conocimientos, actitudes, habilidades, valores,  sentido  de  justicia  y  desarrollo  emocional  y  ético.  Se impulsará un aprendizaje sustentado en los principios de la formación integral de las personas». También se asume como elemento fundamental para lograr dicho propósito el desarrollo del personal académico, a través del compromiso que se desprende del Programa para el Mejoramiento del Profesorado (PROMEP), el cual tiene como propósito: «Mejorar sustancialmente, la formación, dedicación y el desempeño de los cuerpos académicos de las instituciones de educación superior como un medio para elevar la calidad de la educación superior».

En relación al desempeño académico, el programa establece claramente las actividades que debe desarrollar el profesor de educación superior, como el «Participar en actividades de apoyo a la docencia»; algunas de estas actividades mencionan  enfáticamente lo concerniente a la tutoría y a la asesoría de alumnos.

La Asociación Nacional de Universidades e Instituciones de Educación Superior (ANUlES) señala, en su propuesta de «Programa Estratégico para el Desarrollo de la Educación Superior», que para atender el «Desarrollo Integral de los Alumnos. Las lES pongan en marcha sistemas de tutoría, para que los alumnos cuenten a lo largo de toda su formación profesional con el consejo y el apoyo de un profesor debidamente preparado para tal fin.»

\enlargethispage{1\baselineskip}
Finalmente,  tanto  en el marco  nacional  como en el internacional,  se requiere  modificar y sustituir el paradigma educativo actual por otro  que  incluya  la   formación profesional de los estudiantes de manera integral; es decir, uno que  desarrolle en  él valores, actitudes, habilidades, destrezas y aprendizajes significativos. Y para alcanzar lo hasta aquí señalado, se necesita  que  el profesor  de educación  superior  asuma  un compromiso de cambio permanente,  y no sea   un  simple transmisor  del  conocimiento, sino que se convierta y se asuma como facilitador, orientador, tutor y asesor del alumno, y que a la vez le facilite el consolidar su proyecto de vida y le ayude a visualizar con claridad sus aspiraciones personales,  familiares, escolares  y profesionales.

\bigskip
\textbf{Marco conceptual}

En este siglo  las instituciones   de educación  superior  (lES)  tienen  el reto  no sólo de  hacer  mejor  lo  que  actualmente   vienen   haciendo,   sino  de reconstruirse y convertirse en    instituciones   educativas    innovadoras  con  capacidad   de  proponer   y  ensayar nuevas formas  de educación  superior  e investigación.

Por    ello,    es    importante  una    flexibilidad     curricular    que    permita     abordar interdisciplinariamente     problemas,    innovar   métodos   de   en\-se\-ñan\-za-apren\-di\-za\-je que   propicien   una   adecuada relación maestro-alumno y el desarrollo    integral   de   las capacidades   cognoscitivas   y afectivas.    Estas  son algunas  de  las  características que se espera prevalezcan  en todas  las instituciones  mexicanas  de educación superior  en este siglo.

La tutoría  como  herramienta   de  cambio  podrá, en este sentido,  reforzar  los  programas   de  apoyo integral  a  los  estudiantes   en  los  campos  académicos,   cultural   y  de  desarrollo humano,  en la búsqueda  del ideal de la atención  individualizada   del estudiante  en su proceso  formativo.

La tutoría  académica  tiene sus antecedentes  teóricos y una estrecha relación con la  Orientación Educativa.

La orientación   educativa   es una actividad  que se caracteriza   por  guiar  y conducir  a las persona de  manera  procesal, con el fin de  que logren conocerse   a  sí  mismas   y al mundo  que  les  rodea,   para  que  el  individuo   clarifique  la  esencia   de  su  vida, comprenda  que  él es una  unidad  con significado,  capaz  y con  derecho  a usar su libertad,  su dignidad  personal,  dentro  de un clima de igualdad  de  oportunidades   y actuando   con  calidad   de  ciudadano   responsable   en  los  aspectos académico, laboral etc.\ (Rodríguez  1995).
%\newpage

\enlargethispage{1\baselineskip}
\begin{sloppypar}
Esta  misma  autora  identifica  claramente  el espacio  y las funciones  del  tutor  con respecto  a las del  orientador;  y, al retomando a\linebreak Benavent~(1994),  éste  menciona  que el tutor se enmarca   en  un trabajo  de equipo  con el orientador  y que  la actividad (ANUlES 2000) varía dependiendo   del nivel  educativo  en el que se encuentre, y que originalmente es un profesor comprometido con su quehacer más allá del aula y de la función didáctica y pedagógica. La división de la orientación educativa marca la pauta para identificar  el espacio de la tutoría.
\end{sloppypar}

Para A. Lázaro y J. Asensi (1992), la tutoría se concibe como una acción de ayuda al alumno individual o en grupo y que debe ser el resultado de una labor en equipo en el que participan los profesores, el alumno y la institución educativa en general, pero que inicialmente se deposita en profesores específicamente seleccionado y formados para esta actividad. Por tanto, desde la perspectiva humanista, la tutoría académica debe estar centrada en el alumno y no ser directiva.

En el programa Institucional de Tutorías de la Universidad Autónoma de Tlaxcala se toma la definición de tutorías de la Asociación Nacional de Universidades e Instituciones de Educación Superior (ANUIES--001), el cual define a la tutoría como un proceso de acompañamiento durante la formación de los estudiantes, que se concreta mediante la atención personalizada a un alumno, o a un grupo reducido de  alumnos,  por parte de académicos competentes y formados  por  estas instituciones ,apoyándose conceptualmente en las teorías del aprendizaje más que en las de enseñanza.

Como parte de la práctica docente, la tutoría académica tiene una especificidad clara, es distinta y a la vez complementaria a la docencia frente a grupo, pero no la sustituye. Implica diversos niveles y modelos de intervención, se ofrece en espacios y tiempos diferentes a los de los programas de estudios.

La tutoría pretende orientar y dar seguimiento al desarrollo de los estudiantes, así como el apoyarlos en los aspectos cognitivos y afectivos del aprendizaje. Busca fomentar su capacidad crítica y creadora y su rendimiento académico, así como mejorar su evolución social y personal. Debe estar siempre atenta a la mejora de las  circunstancias  del  aprendizaje,  y  en  su  caso canalizar  al  alumno  a  las instancias en las que pueda recibir una atención especializada, con el propósito de resolver problemas que puedan interferir en su crecimiento intelectual y emocional (ANUlES 01).

Para que pueda realizarse una tutoría que realmente incida en el desarrollo de los alumnos, es necesario hacer trabajo en equipo, con otras entidades académicas y administrativas; unidades de atención psicológica; programas de educación continua y extension universitaria, programas de apoyo económico, etc\@.

\bigskip
\textbf{Misión}

La misión  del Sistema  Institucional  de Tutorías  es proporcionar  orientación   sistemática al estudiante,   desplegada   a lo largo  del  proceso  formativo;   desarrollar   una  gran capacidad   para  enriquecer   la práctica  educativa  y estimular   las  potencialidades para el aprendizaje   y el desempeño   profesional  de sus  actores:  los profesores   y especialmente   los alumnos.

\bigskip
\textbf{Visión}

Una organización altamente calificada para organizar eventos de tipo académico en apoyo a quienes ofrecen el servicio de tutoría.

Un sistema eficiente de   planeación y operación de proyectos con base en indicadores locales, regionales y nacionales para asegurar un servicio de calidad.

Con un equipo de trabajo altamente comprometido para participar en actividades de capacitación y actualización.

Que ofrezca un servicio educativo, flexible, digitalizado, abierto y accesible a la comunidad universitaria.

Que establezca convenios con otras instituciones que permitan el de\-sa\-rro\-llo integral de profesores  y estudiantes para responder  a las exigencias  sociales  en diferentes  escenarios  laborales.

\bigskip
\textbf{Objetivos generales}

El Sistema  Institucional  de Tutorías  contempla  una serie de objetivos   relacionados con  la integración,   la  retroalimentación   del  proceso  educativo,   la  motivación   del estudiante,   el  desarrollo   de  habilidades   para  el  estudio   y  el  trabajo,   el  apoyo académico  y la orientación   personalizada.

\begin{Obs}
\item[1.-] Contribuir   al  desarrollo   de  las  capacidades   del  estudiante   para  que  asuma responsabilidades   en el ámbito de su formación  profesional.
\item[2.-]   Fomentar  el  desarrollo   de  valores,  actitudes  y habilidades   de  integración   al ámbito  académico mediante el estímulo  del interés del estudiante   para  resolver problemáticas y  trabajar en equipo.
\newpage

\item[3.-]  Estimular  el desarrollo  de la capacidad de toma  de decisiones   del estudiante  a través del análisis  de escenarios,  opciones y posibilidades  de acción  en el proceso educativo.
\item[4.-]  Fomentar  el desarrollo  de la capacidad  de autoaprendizaje   con el fin de que los estudiantes mejoren su desempeño  en   el   proceso   educativo    y   manifiesten seguridad  y eficiencia  en su futura  práctica profesional.
\item[5.-]   Estimular en  el estudiante el  desarrollo   de  habilidades   y destrezas   para  la comunicación,   las relaciones  humanas,  el trabajo  en equipo  y la aplicación  de los principios  éticos de su profesión.
\item[6.-]  Ofrecer  orientación a los alumnos  en problemas  escolares  y\slash{}o personales,  que se  presenten   durante   el  proceso  formativo   (problemas   de  aprendizaje, en  las relaciones  maestro-alumno, entre alumnos,  problemáticas   personales y\slash{}o familiares). Si la situación lo amerita canalizarlo a instancias o especialistas para su atención (ANUlES 2001).
\end{Obs}

\bigskip
\textbf{Compromisos del tutorado}

Para  que  realmente   la  tutoría  cumpla  su  función,   es  necesario   que  se convierta  en un vínculo entre tutor y tutorado,  de forma tal que cada uno asuma su responsabilidad.
Por lo tanto, el tutorado deberá:

\begin{Obs}
\item[$\star$] Obtener información académica y administrativa necesaria para asumir la responsabilidad de diseñar su trayectoria escolar.
\item[$\star$] Cumplir  puntualmente con  las tutorías previamente calendarizadas y dadas  a conocer  por  su  tutor, en el entendido de que se encuentra en  libertad de solicitar  tutorías adicionales  cuando lo  requieran y de común acuerdo con su tutor.
\newpage

\item[$\star$] Mantener comunicación con el tutor para intercambiar puntos de vista sobre su trayectoria escolar y sobre los contenidos de los cursos remediales.
\item[$\star$] Participar en la evaluación y seguimiento del programa de tutorías.
\item[$\star$] Mostrar apertura y aceptación hacia el tutor y las acciones de la  tutoría.
\item[$\star$]Cumplir con los programas de tutoría permanente a través de la estancia que establece el plan de estudios de la carrera que haya elegido.
\end{Obs}

\bigskip
\textbf{Funciones y perfil del tutor}

Dada la importancia que tiene la tutoría académica, ésta debe recaer en un profesor que se asuma como guía del proceso formativo y que está permanentemente ligado a las actividades académicas de los alumnos bajo su tutela, por lo que se hace necesario señalar los rasgos que lo distinguen de un profesor dedicado regularmente a su actividad en el aula; por ello el tutor académico debe:

\begin{Obs}
\item[$\star$] Orientar, asesorar, apoyar y acompañar a su tutorado durante todo el proceso de formación universitaria; conduciéndolo hacia una formación integral; estimulando en él la capacidad de responsabilizarse de su propio aprendizaje y formación.
\item[$\star$] Tener un amplio conocimiento de la filosofía de la institución.
\item[$\star$] Conocer la oferta educativa de la UATx.  (Propuesta del Modelo Institucional de Tutorías.  Secretaria Académica, ANUlES. 2002)
\item[$\star$] Manejar  la modalidad  educativa  y curricular  del  área  disciplinar  en  la que  se efectúe  la práctica  tutorial.
\item[$\star$] Dominar los documentos normativos que rigen la vida académica de la institución (Reglamento de Evaluación, Servicio Social, Prácticas Profesionales, Reglamento deTitulación etc.).
\item[$\star$] Conocer los lineamientos académicos y administrativos que rigen la   vida universitaria.
\item[$\star$] Ser un profesor o investigador que posea una amplia experiencia   académica sobre todo en el área que se encuentran inscritos sus tutorados.
\item[$\star$] Tener un alto sentido de responsabilidad con respecto al servicio de tutoría.
\item[$\star$] Poseer la capacidad para expresarse con claridad, tanto en forma oral  como escrita.
\item[$\star$] Tener capacidad y dominio del proceso de la tutoría.
Contar con la capacidad para  propiciar  un ambiente  de trabajo  que favorezca la empatía  con sus tutorados.
\item[$\star$] Estar dispuesto  a actualizarse  en el campo de la tutoría.
\end{Obs}

En la Licenciatura en Historia, a partir de su creación en 1986, históricamente el ingreso ha oscilado entre 30 y 15 estudiantes. Lo que ha representado para Programa Educativo de la UATx  una baja demanda en relación con otras instituciones nacionales de educación superior; aunado a ello, hemos podido identificar una serie de factores que han incidido en la deserción y en la baja eficiencia terminal.  Entre los mas frecuentes a los cuales  se atribuye  esta  situación,   se incluyen:  la rigidez  y especialización   excesiva   de  los  planes   de  estudio;   el  empleo   de  métodos   de enseñanza   obsoletos,   la  inexistencia   de  programas   integrales   de  apoyo  a  los alumnos,  el  rol  inadecuado   del  profesor  frente   a  las  necesidades actuales del aprendizaje; una  evaluación centrada  exclusivamente en el alumno y no en los procesos a una inadecuada orientación vocacional, restricciones económicas y problemas personales\linebreak y\slash{}o familiares.

Las generaciones seleccionadas para la descripción y  análisis de este trabajo de investigación corresponden son las de 2001, 2002 y 2003. Las dos primeras compartieron plan de estudios (1999); y en el caso de la tercera, se profesionalizaron con la reestructuración del plan de estudios de 2003. 

El plan de estudios de 1999 fue concebido para cursarse a lo largo de 10 semestres;  en tanto que el Plan de Estudios 2003 de la licenciatura en Historia posee las siguientes características: es semiflexible por su organización en grupos de asignaturas distribuidas en etapas, con la menor seriación temporal obligatoria. Se integra un tronco común universitario; se enfatiza el estudio del  altiplano central, esto es la relación interregional entre Hidalgo, Puebla, Estado de México y Tlaxcala.

La licenciatura se cursa en 8 semestres de forma ideal, pudiendo reducirse el tiempo a siete ciclos o ampliarse a doce, según las necesidades e intereses de los alumnos. Se incorpora el estudio del idioma inglés de forma transversal, es decir, en tres cursos escolarizados, y se recomienda que los estudiantes se acerquen a la lectura de la bibliografía en inglés que sugiera cada una de las asignaturas.

\enlargethispage{1\baselineskip}
De acuerdo a las indicaciones de la SEP, se ha incorporado el Servicio Social al mapa curricular, de tal forma que las actividades desarrolladas en el mismo tengan un valor crediticio. También se incorporan  las prácticas profesionales, mismas que podrán realizarse  en archivos, museos, escuelas y centros de investigación. De igual manera, se ha\linebreak incorporado  la titulación. Todo ello con el propósito de sumar todas las actividades de aprendizaje y se concreten en la titulación de los alumnos al momento de egresar.

Con el Sistema Institucional de Tutorías la situación de los bajos índices se modificó; las dos primeras generaciones presentaron un repunte de permanencia del 66\,\%, y para la tercera generación objeto de este análisis y estudio,  se presentó un incremento al 
77\,\%, con  lo que el programa de tutorías nos mostró su eficiencia y  funcionalidad,  siempre y cuando las partes involucradas  procuren dar el seguimiento apropiado a las trayectorias académicas de los estudiantes (véase Anexos).

El plan de estudios 2003 nos plantea métodos y técnicas didácticas con un enfoque centrado en el estudiante ante un modelo semiflexible, por ello se  diseñan, desarrollan y proponen escenarios educativos tendientes a la construcción y reconstrucción de contenidos de aprendizaje de  acuerdo a las características y necesidades de los estudiantes.

\begin{Obs}
\item[1.] Se ha impulsado la participación activa y de autoaprendizaje independiente de los estudiantes en la construcción del conocimiento en las distintas actividades dentro y fuera del salón de clases, en la realización de prácticas de investigación de campo y el desarrollo de proyectos de investigación en las diferentes áreas de conocimiento, propiciando  la interacción  en los procesos de aprendizaje.
\item[2.] Se han creado ambientes de aprendizaje en el que se propicia la reflexión crítica acerca de los contenidos así como su aplicación en diversos escenarios con orientación interdisciplinaria, como  en  seminarios, talleres y materias que permiten este tipo de ejercicios.
%\enlargethispage{2\baselineskip}
\item[3.] Se diseña y desarrollan  actividades de aprendizaje cooperativo para fomentar la integración personal, académica, social y profesional con otros actores dentro y fuera del campus universitario.
\item[4.] Se  han planteado problemas eje  que reten a los estudiantes a proponer alternativas  de solución pertinentes, eficaces e innovadoras, para los distintos proyectos académicos en los que participen. En este sentido, se propicia que los estudiantes se incorporen a las labores de investigación y profesionalizantes en los diferentes niveles de su formación; asimismo, a partir del sexto cuatrimestre los estudiantes inician con su trabajo recepcional, de tal forma que al concluir los créditos de su plan de estudios se encuentran en  posibilidades de titularse por esta vía.
\item[5.] Se emplean  recursos educativos que ayudan al alumno a transformar las experiencias educativas por medio de comparaciones y asociaciones, evaluación de perspectivas y soluciones, elaboración de conclusiones basadas en evidencias y el empleo de las nuevas tecnologías de información-comunicación, para ello, su formación incluye aprendizaje de programas que le  permitan procesar datos  y que le conduzcan a un ejercicio de análisis e interpretación de tal información.
\item[6.] Se promueve la búsqueda de significados y valores al relacionar las temáticas y actividades con los contextos sociales y laborales de la licenciatura. Esto se lleva a cabo desde el segundo periodo de su formación, cuando deben reflexionar respecto a diferentes problemas sociales que se presentan, con temáticas que le  remiten a su realidad cotidiana, específicamente en las materias de Reflexión del Mundo Contemporáneo, Taller de Ambiente y Desarrollo y la de Taller de  Emprendedores.
%\enlargethispage{1\baselineskip}
\item[7.] Se fomenta que el proceso de evaluación formativa se sustente\linebreak en la habilidad del estudiante para demostrar el aprovechamiento,\linebreak competencias y uso del conocimiento durante las experiencias educativas, para lo cual se realizan foros de presentación de resultados de prácticas de investigación de campo y presentación de avances de investigación, como la Semana del Historiador.
\item[8.] Se ponderan las estrategias del aprendizaje  extra-clase. Para ello, se ha cuidado que los estudiantes se  presenten en foros, con  los trabajos y resultados de sus  investigaciones que desarrollan a lo largo de su formación profesional. Asimismo, la Facultad, las Secretarias de Extensión Universitaria y Autorrealizacion organizan eventos de tipo cultural, académico, deportivo y recreativo, que han contribuido a la formación de nuestros estudiantes.
\end{Obs}

El programa de tutoría tiene una cobertura del cien por ciento de los estudiantes de la Facultad y de nuestra licenciatura, lo que ha permitido construir trayectorias escolares, en las que conjuntamente tutor-tutorado se determina su carga académica en sus diversos periodos escolares  y en ciertos casos que les permitan cumplir con las exigencias de sus actividades escolares y laborales.

\medskip
\textbf{Conclusiones}

\enlargethispage{1\baselineskip}
Por lo anteriormente expuesto, es necesario  y urgente que en nuestro país las  lES  lleven  a cabo  estudios  sobre  las  características  y comportamiento de su población estudiantil, especialmente sobre los factores que influyen en su trayectoria escolar, tales como ingreso, permanencia, egreso y titulación. Ya que la información obtenida sería de gran utilidad para  identificar y atender las causas que afectan e intervienen en el éxito o en el fracaso escolar, en el abandono de los estudios profesionales, así vcomo condiciones que prolongan el tiempo establecido en los planes de estudios para concluir los mismos.

Debe ser un compromiso de todas las lES ofrecer mayor calidad en el proceso formativo, incrementar el rendimiento de los estudiantes,\linebreak reducir la reprobación y el abandono de los estudios así 
como\linebreak elevar los índices de eficiencia terminal; cumplir con objetivos claros que respondan a las exigencias sociales actuales con egresados mejor preparados para lograr su incorporación exitosa al ámbito laboral. Para lograr todo ello se requiere de una participación activa, comprometida y decidida de parte de las autoridades universitarias en su conjunto, la profesionalización y capacitación de los actores que ofrecen este servicio y de apoyos materiales y administrativos. También se hace necesario contar con la infraestructura institucional, es decir, con el equipamiento de laboratorios, centros de cómputo, bibliotecas, etc\@.

Como sabemos que los estudiantes que ingresan al nivel superior no cuentan con las habilidades, actitudes y conocimientos indispensables para utilizar de la mejor manera los recursos con que la universidad dispone,  se vuelve entonces más indispensable el acompañamiento académico que se ofrece a través del servicio de la tutoría institucional.

Por otro  lado,  las  lES  no  han detectado con  suficiente precision los factores en los periodos críticos de la trayectoria escolar  universitaria, que  pudieran desencadenar el incremento de la deserción del  estudiante,  entendida  ésta  como  el  abandono  de  los  estudios superiores.

Sin embargo, el rezago escolar afecta a los estudiantes que no acreditan las asignaturas y no pueden avanzar hasta que las aprueben. Es probable que ello se deba a la falta de información sobre los diversos PE, su perfil de ingreso, desarrollo y consolidación de saberes y competencias para su óptimo desenvolvimiento laboral. 

\enlargethispage{1\baselineskip}
En tal virtud, el establecimiento del Sistema Institucional de Tutorías de la UATx podría tener una repercusión positiva en la solución de los problemas antes mencionados, así como en el aumento de la eficiencia terminal, contribuyendo de manera importante a la formación integral de los estudiantes universitarios tlaxcaltecas.
\newpage

\textbf{Referencias}
%\enlargethispage{2\baselineskip}

\medskip
Jiménez Guillén Raúl (2003),  \textit{La planeación interactiva para la construcción de una nueva Universidad},  México, UAT.

Prawda, John (1984), \textit{Teoría y praxis de la planeación educativa en México}, México, Grijalbo.

U. A. T. (1976),  \textit{Ley Orgánica de la Universidad Autónoma de Tlaxcala}, Tlaxcala, UAT.

U. A. T. (2000), \textit{Propuesta Institucional FOMES--UAT},  Tlaxcala, UAT.

 U.A.T. (2001), \textit{Programa Integral de Fortalecimiento Integral PIFI-UATLAX 2001-2006}, Tlaxcala, UAT.
              
U. A. T. (2003), \textit{La construcción de una nueva Universidad}, Tlaxcala, UAT.

U. A. T (2008), \textit{Cuarto eje transversal y articulador. Autorrealización de las funciones sustantivas de la universidad}, Tlaxcala, UAT.

U.A.T. (2006), \textit{Plan de desarrollo institucional 2006-2010}, Tlaxcala, Colección Prospectiva, UAT.

U.A.T. (2012), \textit{Primer informe de actividades 2011-2012}. Dr. Víctor Job Paredes Cuahquentzi. Rector, UAT, Tlaxcala.

U.A.T. (2013), \textit{Segundo informe de actividades,2012-2013}. Dr. Víctor Job Paredes Cuahquentzi. Rector, UAT, Tlaxcala.

U.A.T.( 2014), \textit{Tercer informe de actividades,2012-2013}. Dr. Víctor Job Paredes Cuahquentzi. Rector, UAT, Tlaxcala.
\newpage

\textbf{Anexos}

\bigskip
\includegraphics[width=350pt,height=210pt, keepaspectratio=true]{Rodriguez-fig001.png}

\bigskip
\includegraphics[width=350pt,height=210pt, keepaspectratio=true]{Rodriguez-fig002.png}

\newpage
\thispagestyle{empty}
\phantom{abc}