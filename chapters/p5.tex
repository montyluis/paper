%%\clearpage\setcounter{page}{83}%

\thispagestyle{empty}
\phantomsection{ }
\begin{sloppypar}
\addcontentsline{toc}{chapter}{La didáctica en la formación docente: Retos\newline y perspectivas\newline $\diamond$
\normalfont\textit{Hugo Torres Salazar}}
\end{sloppypar}
{\centering{\scshape \large La did\'{a}ctica en la formaci\'{o}n docente: Retos\\ y 
perspectivas}\par\par}
\markboth{la formación del historiador}{didáctica en la formación docente}

\renewcommand*{\thefootnote}{\fnsymbol{footnote}}

\bigskip
\begin{center}
{\bfseries Dr. Hugo Torres Salazar}\footnote{Doctor en Historia por la Universidad <<Paul Valéry>>, Montpellier, Francia. Docente Investigador Titular <<C>>\@. Departamento de Historia. Centro Universitario de Ciencias Sociales y Humanidades. Universidad de Guadalajara. Coordinador de la Maestría en Historia de México.}\\
{\itshape{} Universidad de Guadalajara\/}
\end{center}

\medskip
Un profesor que tenga a la didáctica como estrategia para la enseñanza, 
correrá junto con sus alumnos la aventura del aprendizaje, mientras que 
un docente impersonal, poco amigable e indiferente, pasará inadvertido 
él y su materia, y quizás el recuerdo que les quede a los alumnos, es 
no parecerse a él.

\renewcommand*{\thefootnote}{\arabic{footnote}}
\setcounter{footnote}{0}

\medskip
\textbf{Viñeta uno}

\textit{Yo recuerdo a mi maestro [\ldots] de 4º semestre de la Preparatoria 
[\ldots], no recuerdo su nombre, lo único que recuerdo es que de los tres 
días a la semana que me tocaba la clase, dos no iba y el último se la 
pasaba hablando de fútbol, y haciendo apuestas para ver quién ganaba la 
Copa del Mundo[\ldots] Cuando no era Argentina, era Alemania, etc\ldots{} y si no 
eran los Pumas que habían ganado el campeonato nacional\ldots{}  De historia no 
conocí nada pero me hice experta en fútbol, de alguna manera tenía que 
ganarme la calificación y además tenía qué asistir a clases. 
[s.\ n.]}.\footnote{ Se tiene el relato, pero no el nombre del autor.} 
\enlargethispage{1\baselineskip}

Para concebir a la didáctica como un hacer práctico y practicado, 
debemos hacer corresponder al primero el conocimiento, al segundo la 
instrumentación, la puesta en práctica de medios, métodos e 
instrumentos de parte del enseñante. En el primero, el discurso ya está 
hecho construido por el conocimiento histórico, en el segundo el 
dis\-cur\-so es construido por el profesor de la historia, se puede decir 
que está en constante construcción, o como diría Lacan, <<no hay palabra 
sin respuesta, incluso si no encuentra más que el silencio, con tal de 
que tenga un oyente>> (Lacan 1997, p. 237). Agregaríamos, citando a Kosik, 
que <<el hombre para conocer las cosas en sí mismas, debe transformarlas 
antes en cosas para sí \ldots{}el conocimiento no es contemplación del mundo se 
basa en resultados de la praxis humana>> (Kosik 1967, p. 40). Esto nos 
lleva a sostener que la didáctica es fundamentalmente un discurso que 
incorpora al alumno, al maestro y al contenido del discurso, el 
conocimiento histórico.

Analicemos el discurso desde sus tres aristas de producción; desde el 
maestro, desde el alumno y desde el conocimiento que se enseña.

\medskip
\textbf{El maestro y la didáctica}
%\enlargethispage{1\baselineskip}

El maestro que incorpora la didáctica en su quehacer docente lo 
im\-pul\-sa\-rá necesariamente a generar relaciones significativas entre 
él y sus alumnos, a vivir un nuevo enfoque del proceso 
en\-se\-ñan\-za{--}~apren\-di\-za\-je que incluya en el alumno una 
estructura mental sustentada en el nuevo paradigma del aprendizaje y 
establecer una verdadera interacción entre seres humanos.

El vínculo didáctico construido por este profesor podrá facilitarse si 
el docente ve en sus alumnos las siguientes características:

\begin{Obs} 
\item[1.] Considerar a cada alumno como un ser humano, como un individuo que 
como sujeto cognoscente posee conocimientos, experiencias, 
sentimientos, conflictos, valores, expectativas, etcétera; en suma, que 
posea una visión integral del sujeto de la educación.
\item[2.] Aceptar al estudiante tal y como se encuentra en ese periodo de su 
desarrollo cogno-socio-biológico, y no permanezca en el rumiar de que 
el alumno <<debería saber tal cosa>>, o <<debería comportarse de tal 
manera>>.
\item[3.] Tener interés por conocer los antecedentes biográficos y escolares 
del grupo y de sus miembros que lo lleven a reconocer que el presente 
es consecuencia de su pasado, pero que a la vez esto lo faculta para 
definir estrategias de enseñanza y perfilar expectativas de logro.
\item[4.] Admitir al alumno y al grupo con las características propias del 
momento, sin perder de vista la perspectiva de que sus actitudes son 
susceptibles de modificación.
\item[5.] Preocuparse por crear un \ ambiente propicio que \ fa\-vo\-rez\-ca el 
apren\-di\-za\-je, donde se viva en una relación humana de diálogo, servicio 
y cooperación, estableciendo el vínculo educativo que produzca la 
ilusión del aprendizaje.
\item[6.] Promover el diálogo y <<construir la verdad>> como proyecto dialógico y creativo. 
<<No hay estudiantes tan intelectualmente pobres que no tengan una palabra que decir o una iniciativa que presentar>>. 
\item[7.] Logrará las metas que se registraron en el plan de trabajo pero 
además motivará al alumno a la elaboración de actividades extra clase.
\item[8.] El profesor que utiliza la didáctica reconocerá el beneficio de la 
creatividad. Generará mayor comunicación con sus alumnos dentro y fuera 
de clases, manteniendo con ellos respeto mutuo y la comprensión de sus 
necesidades. Podrá reconocer cuándo debe estar y cuando dar el pase a 
los alumnos hacia otro profesional.
\enlargethispage{1\baselineskip}
\item[9.] Si la didáctica ha sido parte de su formación, estará consciente de 
la necesidad de estarse actualizando continuamente, de adquirir día con 
día no sólo más y mejores conocimientos, sino también las técnicas para 
transmitirlos.
\end{Obs}
 
Esto puede resultar muy complicado o demasiado banal para aplicarlo en 
nuestra práctica docente, pero si no lo intentamos, ¿cómo podemos decir 
que no vale?


%\smallskip
{\bfseries Viñeta dos}

\enlargethispage{1\baselineskip}
\textit{Creo que si no recuerdo muy bien a mis maestros \{\ldots\}, es 
porque han sido más malos que cualquier cosa. Lo que más he lamentado 
es que aún en nivel licenciatura, las cosas no hayan cambiado mucho. El 
entusiasmo que experimento al iniciar un ciclo escolar, sobre todo 
cuando he elegido una materia que me interesa de sobremanera, en 
ocasiones desaparece a las primeras clases, cuando descubro que lo que 
puedo aprender (y aprehender) de las lecturas asignadas, puede ser 
interrumpido o entorpecido por la intervención del maestro. Si, el caso 
más patético fue con la clase en la que el docente, (considero que) al 
no saber ni lo más mínimo de la materia, dejó todo el <<paquete>> al 
alumnado, cuando es sabido que el proceso 
enseñanza-aprendizaje concierne a ambos polos, cada uno asumiendo su 
propia responsabilidad en lo que le corresponde, responsabilizándose 
por sus actos y sus palabras. Confieso que en un tiempo estuve de 
acuerdo con los maestros que la hacían de todólogos, dando clases de 
diferentes áreas, inclusive me tocó conocer quienes lo hacían muy bien. 
Pero luego descubro que es un riesgo mortal el que un maestro, aún 
siendo historiador, imparta una clase que no está dentro de su 
especialidad. Una de las cosas que más me sorprendió de este docente 
fue que no mostró ni el mas mínimo interés por mejorar y ampliar sus 
conocimientos, así como la metodología didáctica empleada en clase. Sé 
de antemano que es algo que no deja de ser difícil, complejo y 
complicado, pero no debe impedir que se haga un esfuerzo por superar 
tal deficiencia.}\footnote{Relato de C.D.\ para la clase de Didáctica 
de la Historia. Septiembre de 2002.}


\medskip 
\textbf{El alumno y la didáctica}
 
Por otro lado, podríamos pensar ¿qué obtiene un alumno al tener un 
profesor que enseña con didáctica?

 
Lo primero  que viene a nuestra mente es que un alumno que tiene un 
profesor que aplica la didáctica en la enseñanza, obtendrá beneficios, 
entre los cuales se pueden citar los siguientes:

 
En primer lugar, no se trataría del clásico alumno que tiene que 
memorizar mil lecturas, fechas y datos para acreditar los exámenes y 
que, al cabo de unos días, no se acuerda de nada. Por el contrario, el 
alumno obtendría conocimientos duraderos, de profundo significado en su 
formación y para cuando ejerza su profesión, continuaría acumulando 
conocimientos, pero ya no sólo para aprobar la materia, sino para 
aprobar en la vida, con conocimientos verdaderos que lo lleven a la 
formación de una conciencia y experiencia para reflexionar sobre las 
cuestiones de la vida cotidiana, a ser crítico y a la vez dar lo mejor 
de sí mismo para la sociedad.

 
Otro alumno pronunciaba lentamente el siguiente enunciado como si fuera 
un epitafio: 

\begin{quotation}
«(\ldots{}) si muchos de nuestros maestros nos hubieran enseñado con 
didáctica, quizás no hubiéramos terminado odiando tal o cual materia, 
porque el alumno enseñado con didáctica, aprende a valorar todo tipo de 
conocimiento. La escuela sería para él el lugar de crecimiento, y no de 
aburrimiento u obligación fastidiosa con la cual hay que cumplir. Un 
alumno enseñado con didáctica estaría cada vez más sediento de 
conocimiento, tendría en él la semilla de aprender día con día\ldots{}»
\end{quotation}
 
El alumno que aprende con didáctica también aprende a transferir el 
conocimiento que adquirió en el aula hacia su espacio social donde se 
desenvuelve y buscará con igual ahínco y disciplina una educación 
permanente.

Los resultados que se verán reflejados en estos alumnos, quedarán 
manifiestos en una capacidad reflexiva y crítica en el análisis de los 
problemas de la materia y de la sociedad a la cual pertenecen.


\medskip
\textbf{El conocimiento histórico y la didáctica}

 
Finalmente anotaríamos las ventajas que el profesor obtiene cuando en 
su práctica docente incluye la didáctica para la construcción del 
conocimiento, entendiendo ésta ya no sólo como una estrategia para la 
enseñanza, sino como el instrumento pedagógico artesanal que cada 
docente la da a los procedimientos didácticos en el aula promoviendo en 
el alumno el proceso de construcción del conocimiento.

 
Esta serían algunas:

 
La didáctica viene a ser en un profesor lo que determine la calidad de 
su enseñanza y el medio de ir más allá de la simple información o 
transmisión del conocimiento.

 
Tendrá la capacidad de enseñar, informando y formando alumnos de 
calidad, además de adquirir el compromiso consigo mismo y con sus 
alumnos.

 
Estará consciente de su papel en el proceso enseñanza-aprendizaje 
descubriendo bajo esa conciencia que no es el eje principal sobre el 
cual gira el aprendizaje sino únicamente parte de ese diálogo y sujeto 
activo en la formación e intercambio de vínculos sociales.

 
Un maestro con didáctica hace la diferencia entre la simple información 
y la formación.

 
Un profesor que cuenta con las bases didácticas para el aprendizaje, 
podrá generar en su práctica docente los espacios colectivos para 
construcción del conocimiento, de manera que induzca no sólo al cambio 
de conductas, sino al cambio de hombres. <<El hombre se hace 
transformándose en hombre>> (Kosik 1967, p. 39).

 
Un profesor con didáctica tendría la capacidad para preparar sus clases 
de forma variada, utilizará diversos medios y técnicas para abordar los 
temas, no se limitaría a pararse frente al grupo para dar una 
<<brillante conferencia>>, o un <<apasionado monólogo>>, o peor aún, para 
dictar con puntos y comas.

 
Se trataría de un profesor que incentivaría la participación activa de 
sus alumnos, haría que éstos dieran el máximo, que se atrevieran a 
reflexionar, criticar, proponer. Sería por tanto un profesor flexible, 
no hermético o autoritario, sino por el contrario, las clases serían 
una especie de diálogo, donde se presente el mutuo aprendizaje entre 
profesor y alumno y donde se instale como un quehacer cotidiano, la 
creatividad y la sorpresa.


\medskip
\textbf{Reflexiones finales}
\enlargethispage{-1\baselineskip}
 
Sólo dejaré a manera de colofón lo siguiente; los señalamientos que 
aquí se hacen, no son mandato, ni quien lo señala ordena;  son 
puntualizaciones que siendo importantes para la enseñanza, pasan 
inadvertidas. Y si la didáctica no lo es todo en la enseñanza, sí ayuda 
para hacerla más eficiente; o tal vez; se llegaría a la transformación 
de alumnos como Kiyo:\footnote{Escrito para la materia de Desarrollo 
de Ambientes para el Aprendizaje de la Historia. El título <<Mi maestro 
de historia>>, relato de una experiencia de aprendizaje.}

 
<<El gran corso de espaldas sobre su caballo, mira fijamente aquella 
majestuosa y exótica escultura. A lo lejos se vislumbraba su gran 
ejército. La milenaria esfinge no parece inmutarse ante este hombre que 
afirma ser un dios.

 
---¿Qué creen ustedes que está pensando Napoleón?

 
La pregunta de la maestra despejó mi mente que volaba ya bajo el sol 
abrumante egipcio. Toda la clase se encontraba absorta frente a la 
pintura de Jean-Leon Gerome, titulada <<Napoleón ante la esfinge>>. El 
lienzo tenía algo de exótico y a la vez de imposible que hacía difícil 
despegar la vista de él.

 
---¿Sabían ustedes que gracias a la campaña a Egipto de Napoleón se 
redescubrieron todas las maravillas de la arqueología del antiguo 
Egipto?

 
¡Y Claro! Entonces recordé aquella enorme cantidad de piezas egipcias 
que tanto me habían maravillado en mi visita al Louvre. Recordé haber 
estado frente a una esfinge, observándola, escrutándola como tratando 
de arrancarle un secreto, una verdad empolvada. Esa pieza tan fuera de 
contexto, dentro de una pequeña y oscura sala. Así que desde tiempos de 
Napoleón había sido condenada a estar detrás de aquella fría y ajena 
vitrina.

 
---¡Oye! ¿Estás ahí? ¡Kiyo!

 
Escuche mi nombre a lo lejos, otra vez mi imaginación me había llevado 
más allá del aula, y  mi maestra preguntaba algo que yo no había 
escuchado.

 
Pero eso era lo mágico de mis clases de historia, por eso eran mi 
materia favorita. 

 
Era como viajar en el espacio y en el tiempo, a través de personajes 
singulares y fantásticos. Comprender y relacionar cosas de mi vida por 
medio de relatos y cuentos. Así lo veía yo, pero ahora comprendo que en 
gran parte así nos lo hacía ver mi maestra. Era su habilidad para 
convertirse en esa maravillosa cuenta-cuentos y transportarnos a todos 
en una alfombra mágica, cruzando mares y batallas sin siquiera haber 
puesto un pie fuera del salón. Era mi mejor maestra, mi maestra que 
cuenta-cuentos.>>

 
<<Después de todo, dar clase de algo es decir lo que otro dijo antes que 
uno y nada más que eso. Un texto que dice otro texto>> (Ana Zavala).


\bigskip
\textbf{Referencias}
\enlargethispage{1\baselineskip}
 
Carr, Wilfred (1996), \textit{Hacia una ciencia crítica de la educación}. 
Barcelona, Laertes. 

 
Certau, Michel (1993), \textit{La escritura de la historia}. México,
Universidad Iberoamericana.

 
Florescano, Enrique; et\@.  al\@. (1992), \textit{El historiador frente a la 
historia}, México, Universidad Nacional Autónoma de México.

 
González, Antonio y Magallanes, María del Refugio (2011), \textit{Teoría y 
metodología en la enseñanza de la historia. Problemas de la educación 
básica en Zacatecas}, Zacatecas.

 
Kosik, Karel (1967), \textit{Dialéctica de lo concreto}, México, 
Editorial Gijalbo.

 
Martínez, Gloria (2014), \textit{La explicación histórica y la reconfiguración 
temporal. Un estudio a través de experiencias de aprendizaje mediado 
con alumnos de secundaria en Guadalajara}, Guadalajara.

 
Wittrock, M. (1989), \textit{La investigación de la enseñanza}. Vol I. 
Barcelona, Buenos Aires, México, Paidós.

 
Zavala, Ana (2010), \textit{Mi clase de historia bajo la lupa. Por un abordaje 
clínico de la práctica de la enseñanza de la historia}, Uruguay, CLAEH\@.

 
\_\_\_\_\_\_ (2006), \textit{Caminar sobre los dos pies: didáctica, 
epistemología y práctica de la enseñanza}, Uruguay, Práxis Educativa. 
Ponta Grossa.

%\documentclass{article}
%\usepackage{amsmath,amssymb,amsfonts}
%\usepackage{fontspec}
%\usepackage{xunicode}
%\usepackage{xltxtra}
%\usepackage{polyglossia}
%\setdefaultlanguage{spanish}
%\usepackage{color}
%\usepackage{array}
%\usepackage{hhline}
%\usepackage{hyperref}
%\hypersetup{colorlinks=true, linkcolor=blue, citecolor=blue, 
%filecolor=blue, urlcolor=blue}
%\newtheorem{theorem}{Theorem}
%\title{}
%\author{}
%\date{2014-09-04}
%\begin{document}

%\clearpage\setcounter{page}{93}
\thispagestyle{empty}	
\phantomsection
\addcontentsline{toc}{chapter}{La enseñanza de la Historia y la crisis\newline del medio ambiente\newline $\diamond$
\normalfont\textit{Gil Arturo Ferrer Vicario}}
{\centering {\scshape \large La enseñanza de la Historia y la crisis del medio ambiente \par}}

\markboth{formación del historiador}{Historia y medio ambiente}

\bigskip
\begin{center}
{\bfseries Gil Arturo Ferrer Vicario}\\
{\itshape Universidad Autónoma de Guerrero}
\end{center}

\medskip
\epigraph{Lo que nosotros denominamos la tierra es un 
elemento de la naturaleza inexorablemente entrelazado con las 
instituciones del hombre.}{\itshape — Karl Polanyi}

\bigskip
{\bfseries Resumen}

Los seres humanos somos tan sólo una especie biológica dentro de la 
diversidad natural. En este sentido, la humanidad está conformada por 
<<animales sociales>> que sobreviven en razón de sus relaciones 
societarias, pero también de sus vínculos con el medio ambiente en que 
se desenvuelven; de tal manera, la especie humana necesita de la 
Historia para que le recuerde de su caminar por su entorno natural.

%\enlargethispage{\baselineskip}
A pesar de que la Historia demuestra la importancia de la relación 
dialéctica y armoniosa que debe existir entre el ser humano y su medio 
ambiente; sin embargo, desde hace tres siglos esta relación ha sido 
sustituida por un exagerado afán de dominio. Por tal motivo, en el 
nuevo contexto en que vive el mundo se impone la necesidad de 
incorporar en la enseñanza de la Historia un proceso de reelaboración 
de conocimientos, redefinición de conceptos y modificación de las 
prioridades que generen nuevas formas de ver y entender el medio 
ambiente.\\
{\bfseries Palabras clave:} enseñanza, ambiente, naturaleza, sustentabilidad, 
ecología.

\medskip
{\bfseries Abstract} 

Human beings are only but a biological species within natural 
diversity. In this sense, humankind consists of <<social aniamals>> that 
get by through not only sociological relationships but also through 
links with the environment in which they interact; thus, the human 
species needs history in order to be reminded of the implications of 
living in such environment and the effect the latter has on it.

Even though history has demonstrated the importance of the dialectic 
and harmonic relationship that must exist between humankind and its 
environment; for the last three centuries this relationship has been 
substituted for an exaggerated desire of domination. Because of this, 
the need to incorporate a process of reproduction of knowledge in 
teaching history, surges in this new context our current world is going 
though, as well as redefinition of concepts and modification of the 
priorities that generate new ways to see and understand our 
environment.\\
{\bfseries Keywords:}  Teaching, environment, nature, sustainability, ecology.

\bigskip
{\bfseries Introducción}
\enlargethispage{1\baselineskip}

A partir de las dos últimas décadas del siglo XX, la cuestión de la 
Historia y el medio ambiente ha ocupado y preocupado a cada vez más 
estudiosos de la ciencia del pasado en todo el mundo. En varios casos, 
los estudios de la relación entre la Historia y la naturaleza se han 
abordado desde diferentes perspectivas historiográficas como son la 
Historia agraria, la Historia social y la Historia de las ideas, entre 
otras. De esta manera el conocimiento nos conduce a una concepción 
holística del mundo y, por lo tanto, se están rompiendo las fronteras 
artificiales que se habían establecido entre las diversas ciencias 
particulares. Se establece así un diálogo entre las ciencias sociales y 
las ciencias de la naturaleza (Le Goff~2005,~p.~16).

Desde la perspectiva de la Historia agraria, el proceso de 
secularización de las tierras de posesión comunal de los pueblos 
indígenas y campesinos, así como de los bienes de la Iglesia fortaleció 
la propiedad privada en detrimento de la comunal y la social, como 
sucedió en México a partir de la segunda mitad del siglo XIX y 
consolidándose con la reforma al artículo 27 constitucional a 
principios del año de 1992, intensificándose la compraventa de tierras 
y la reconfiguración de un nuevo orden territorial y con ello la 
intensificación de la explotación de los recursos naturales bajo la 
óptica de la oferta y la demanda propias del sistema capitalista. Se 
pasó de la relación armoniosa entre la sociedad y la naturaleza hacia 
una de subordinación y explotación de la segunda con respecto a la 
primera. La privatización de los recursos naturales ha ocasionado la 
dependencia de las actividades del campo con respecto a las de la 
ciudad, provocando no sólo una crisis económica en el campo, sino más 
que eso, un agudo y profundo deterioro del medio ambiente. 

Ante esta situación de grave crisis ambiental, qué nos corresponde 
hacer a los que nos dedicamos a la enseñanza de la Historia. Cómo 
podemos revertir este deterioro del espacio donde se desarrollan todas 
las acciones humanas, que son el objeto de estudio de nuestra 
disciplina. Estos son los propósitos de este trabajo. 

La enseñanza de la Historia se enmarca dentro de un sistema educativo 
permeado por el enfoque en competencias, el cual pretende lograr que el 
alumno sea competente en la realización de las actividades para las 
cuales se ha preparado pero dando prioridad al dominio de determinadas 
actividades dentro del ámbito técnico-práctico. Recordemos que las 
competencias aparecen primeramente relacionadas con los procesos 
productivos en las empresas en el campo tecnológico, en donde el 
desarrollo del conocimiento ha sido muy acelerado.

\enlargethispage{1\baselineskip}
El enfoque en competencias a pesar de su fundamento epistemológico 
constructivista, no deja de ser tecnocrático y economicista, en virtud 
que olvida la función científica, cultural y humanística que debe 
contener la educación, principalmente la educación superior, en 
particular la Historia, a favor de la mercantilización de los estudios. 
En conclusión, el propósito del enfoque pedagógico en competencias es 
implementar y lograr una relación más efectiva de la educación con la 
empresa y el mercado laboral. En pocas palabras está orientado a 
adecuar a los estudiantes al mercado laboral.

El enfoque en competencias con las características que aquí hemos 
señala\-do es particularmente nocivo para los estudios históricos; en 
virtud  de su origen eminentemente empresarial y su propósito de 
preparación exclusiva para un mercado laboral cada vez más reducido 
donde los egresados de la educación superior tienen que competir para 
acceder a un puesto de trabajo. El pensamiento posmoderno niega el 
tiempo histórico, desconociendo la cientificidad de la Historia. Hay un 
desprecio por las ciencias sociales y humanas. De ahí que la enseñanza 
de la Historia tiene la necesidad de favorecer una conciencia crítica 
en la ciudadanía y forjar y mejorar los vínculos representacionales y 
societales entre los miembros de una determinada sociedad 
(\mbox{Carretero~2004,} p.~17); pero también modificar la hasta ahora relación de dominio y 
explotación del ser humano hacia la naturaleza. Esta es la pertinencia 
de la Historia, en virtud de que ninguna actividad humana puede ser 
comprendida de forma integral al margen de los estudios históricos. La 
Historia se vuelve siempre coextensiva al hombre (Le~Goff~2005, p.~17). 

Los seres humanos somos tan sólo una especie biológica dentro de la 
diversidad natural. En este sentido, la humanidad está conformada por 
<<animales sociales>> que sobreviven  en razón de sus relaciones con sus 
semejantes, pero también de sus vínculos con el medio ambiente en el 
que se desenvuelve; de tal manera, los seres humanos necesitan de la 
Historia  para que les recuerde de su caminar por su entorno natural.

A pesar de que la Historia demuestra la importancia de la relación 
dialéctica y armoniosa que debe existir entre el ser humano y su medio 
ambiente; sin embargo, desde hace tres siglos esta relación ha sido 
sustituida por un exagerado afán de dominio. Por tal motivo, en el 
nuevo contexto en que vive el mundo se impone la necesidad de 
incorporar en la enseñanza de la Historia un proceso de reelaboración 
de conocimientos, redefinición de conceptos y modificación de las 
prioridades que generen nuevas formas de ver y entender  el objeto de 
estudio de la Historia. Fomentar la construcción de nuevos paradigmas 
de la Historia en relación con la naturaleza (González~2004,~p.~7). Por 
lo que proponemos que la Historia tenga como uno de sus propósitos 
entender el pasado de la especie humana en estrecha relación con su 
medio ambiente y, al mismo tiempo, tratar de comprender las relaciones 
estratégicas de los seres humanos entre sí y con la naturaleza, de la 
que dependen para su supervivencia y de la que forman parte como seres 
vivos.

Por último, la Historia debe posibilitar la formación de una nueva 
ética, de tipo biocéntrica, en sustitución del acendrado 
antropocentrismo que permea todas las acciones de la sociedad con 
respecto a su medio ambiente. Esta nueva ética permite no sólo pensar 
en la sobrevivencia de los seres humanos, sino en el deber moral de 
mirar y tratar de otro modo a los demás seres vivos. Esto conllevará al 
logro de más y mejor desarrollo humano para todos.  
\enlargethispage{-1\baselineskip}

%\medskip
{\bfseries El modelo educativo mexicano}

El término educación se deriva del latín \textit{educere} que significa 
guiar, conducir; de esta palabra latina se deriva otra que es 
\textit{educare} la que a su vez quiere decir formar o instruir. 
Considerando la manera en la que se ha entendido y puesto en práctica 
el proceso educativo en el transcurso de la sociedad humana, es 
evidente que la educación no ha sido mas que la  acción de guiar, 
conducir, formar e instruir.

Así, se entiende por educación todos aquellos procesos mediante los 
cuales se transmiten conocimientos, valores, costumbres y formas de 
actuar, formando con ello una cosmovisión que será propia de la 
sociedad de que se trate. Con todo ello, las nuevas generaciones sólo 
asimilan y aprenden la manera de entender el mundo de su tiempo.

En términos generales, la función de la educación ha sido lograr que 
las generaciones del presente conserven y utilicen los valores de la 
cultura en la que se desarrollan. De esta manera, los valores de una 
determinada sociedad son asimilados y reproducidos. Por ejemplo, en la 
sociedad capitalista actual se han impuesto valores tales como el 
exacerbado individualismo, el consumismo, la intolerancia, etc. En 
suma, la educación es un ingrediente fundamental en la vida del ser 
humano y de la sociedad, y se remonta a los orígenes mismos de la 
especie humana. La educación es la responsable de transmitir la 
cosmovisión en una sociedad determinada. De ahí entonces que el proceso 
educativo haya surgido junto con el propio ser humano, acompañándolo en 
su devenir histórico.  
%\newpage

Durante toda la historia de la sociedad, por lo menos hasta nuestros 
días, los intereses de los grupos sociales dominantes han determinado 
el rumbo de los sistemas educativos. En este sentido, la educación es 
la clara manifestación de la cosmovisión de una época determinada. El 
fenómeno educativo se encuentra estrechamente vinculado con la vida 
económica, política, social y cultural impuesta por los grupos sociales 
hegemónicos, siendo el Estado vigente el que normalmente se ha 
responsabilizado de la orientación y ejecución del sistema educativo. 
De esta manera, la educación, con sus características más 
sobresalientes y generales, nos indica con mayor precisión el tipo de 
sociedad que prevalece en las diversas etapas de su desarrollo 
histórico.  

Lo anterior confirma que en el transcurso del desarrollo de la sociedad 
generalmente se ha adoptado un modelo de educación que concuerda con la 
naturaleza y las características que el Estado señala, y que se refleja 
en los objetivos y en la orientación del proceso educativo.  

La educación actual tiene sus antecedentes inmediatos en el movimiento 
intelectual y cultural llamado Renacimiento, iniciado aproximadamente a 
mediados del siglo XV en Europa. Es a partir de ese momento cuando la 
sociedad empieza a buscar explicaciones apartadas de la visión 
teocéntrica del mundo característica de la Edad Media, y se establece  
una nueva cosmovisión. Se inicia el proceso de consolidación de la 
nueva sociedad moderna y su cosmovisión antropocéntrica, que derivará 
en el humanismo.

El humanismo es un pensamiento antropocéntrico según el cual el hombre 
es la medida de todas las cosas. Se enfatizan valores como el 
prestigio, el poder y la gloria. Asimismo, el ser humano en su afán de 
poder y confort empieza a ver a la naturaleza sólo como un medio para 
satisfacer sus nuevas necesidades y, por lo tanto, a explotarla de 
manera indiscriminada. Se recrudece la explotación de los recursos 
humanos y naturales. El hombre moderno, en su actitud de amo del mundo, 
pretende separarse de su entorno natural y conspirar contra él. La 
naturaleza debe subordinarse a las necesidades del  nuevo orden 
socioeconómico dominante. Se inicia el \textit{ecocidio\/}. La emergente 
organización social debe desarrollarse a partir de la búsqueda del 
bienestar material a costa de la depredación del medio ambiente. Se 
fortalece la separación del hombre con respecto a la naturaleza. En 
este escenario hace su aparición un tipo de sociedad que privilegia la 
vida material y la transformación de los satisfactores materiales y 
espirituales, así como del propio ser humano, en \textit{mercancía:} 
surge el capitalismo. Estos valores de la sociedad capitalista son 
reproducidos por los sistemas educativos nacionales, con distintos 
matices particulares, apoyados por modas pedagógicas, como el actual 
enfoque en competencias aplicado en el sistema educativo de nuestro 
país.

El Enfoque Educativo Basado en Competencias (EEBC) es el resultado de 
la aplicación del paradigma pedagógico constructivista con ciertas 
adecuaciones y algunas tergiversaciones, impuesto a nivel internacional 
a través de organismos financieros como el Banco Mundial y el Fondo 
Monetario Internacional. La manera como se ha impuesto el EEBC es la 
historia del triunfo de la visión tecnoproductiva de la educación por 
encima de la visión humanística de la misma.

Bajo esta perspectiva, la educación en competencias se manifiesta como 
la aplicación concreta en las escuelas del modelo neoliberal que busca 
la eficiencia economicista por sobre cualquier otro concepto. El alumno 
debe ser competente para la realización de las actividades que le 
demanda el proceso productivo, a eso tiende a reducirse su formación 
profesional.

La educación, como proveedora de mano de obra calificada, debe 
capacitar a la población para cumplir con las demandas de los procesos 
productivos. El estudiante debe salir al mercado laboral con ciertas\linebreak 
habilidades y conocimientos que le posibiliten desempeñarse  en un 
ambiente cada vez más tecnificado. Se forma al futuro profesionista 
para responder con eficiencia a las demandas del mercado laboral, pero 
no para cuestionarlo, mucho menos transformarlo. De esta manera, en el 
mercado de trabajo sólo son aceptados aquellos que tienen posibilidad 
de adaptarse a los mecanismos de selección impuestos, incorporándose 
aquellos que demuestren estar preparados para ello, consecuentemente 
deben competir entre sí en su afán de demostrarlo.

El EEBC pone énfasis en la actividad individualista del educando 
promoviendo actitudes egoístas e individualistas; además, tiene que ver 
más con el desempeño, por tal motivo lo importante no es la posesión de 
determinados conocimientos, sino el uso que se haga de ellos. Más que 
educación en competencias debería hablarse de capacitación en 
competencias. En fin, el propósito fundamental del enfoque pedagógico 
basado en competencias es implementar y lograr una relación más 
efectiva de la educación con la empresa y el mercado laboral. En pocas 
palabras está orientado a adecuar a los estudiantes al mercado de 
trabajo.

La educación en competencias que se ha venido impulsando en México 
pretende responder y adecuarse a la nueva sociedad posmoderna, también 
llamada «sociedad del conocimiento», y en el ámbito de políticas para 
el desarrollo nacional se manifiesta como el impulso a un modelo de 
desarrollo económico secundario y dependiente; justamente el papel que 
juega nuestro país en la actual sociedad globalizadora y neoliberal.

La política educativa mexicana optó por convertirse en simple receptora 
de los impactos del exterior, fomentando la instauración de un enfoque 
educativo que contribuye a la consolidación del nuevo paradigma 
tecnoproductivo de corte económico-empresarial, denominado 
<<competencias>> para los distintos niveles del sistema educativo 
nacional.  

El enfoque en competencias con las características que aquí hemos 
enumerado es particularmente nocivo para los estudios históricos; en 
virtud de su origen eminentemente empresarial y su propósito de 
preparación exclusiva para un mercado laboral cada vez más reducido, 
donde los egresados  de la educación superior tienen que competir para 
acceder a un puesto de trabajo.

Para el pensamiento educativo posmoderno, los historiadores no 
contribuyen al conocimiento científico. Desde esta óptica, la Historia 
no se estudia para explicar la realidad, sino que se debe construir una 
Historia acorde con las necesidades del historiador. Es decir, elaborar 
historias particulares donde cada historiador tendrá <<su verdad>>. Lo 
anterior es consecuencia del exagerado subjetivismo que caracteriza al 
enfoque en competencias. Se niega una lógica en los procesos históricos 
y se expone la idea del de la Historia con la tesis de que el cambio 
económico, social y político se ha detenido. Francis Fukuyama 
\textit{dixe}. Conviene recordar que la Historia  como ciencia humana 
se encuentra más alejada  de los estudios laborales, técnicos y 
empresariales.

Ante este panorama quienes nos dedicamos a la reconstrucción del pasado 
debemos proponer un nuevo enfoque en la enseñanza de la Historia que 
remplace a la anacrónica escuela positivista y conductista tradicional.

Bajo el enfoque pedagógico de moda en nuestro país competencias y 
valores suelen ir separados, propugnamos que vayan juntos aunque no 
revueltos, porque no es idóneo que se camuflen los valores dentro de 
las competencias. La necesidad de conciliar coherentemente el enfoque 
en competencias con la educación en valores deviene en una tarea 
urgente si consideramos los agudos problemas que tiene que enfrentar la 
educación en nuestros días: fracaso escolar, violencia, grave deterioro 
del medio ambiente, etc.; dificultades que el enfoque pedagógico en 
competencias por sí sólo no ayuda a resolver. La educación en valores 
es hoy más importante que nunca para formar ética y socialmente a las 
nuevas generaciones de modo que encuentren su papel en el mundo y 
contribuyan a su transformación. La enseñanza de la Historia también 
debe contribuir a lograr lo anterior.  


\medskip
{\bfseries La Historia y su enseñanza }

\enlargethispage{1\baselineskip}
Iniciaré este apartado planteando la siguiente interrogante: ¿por qué 
enseñar Historia? Esta es una pregunta que nos planteamos quienes nos 
dedicamos a la profesión de docentes de la ciencia del pasado. Pero 
también los alumnos. En base a nuestra experiencia profesional, la 
Historia como asignatura en los planes de estudio no goza de buena 
fama, ni tampoco es considerada como una materia interesante, atractiva 
o necesaria. Para la mayoría de los alumnos de los distintos niveles 
educativos, e inclusive para algunos de nuestros aprendentes en la 
licenciatura, consideran que nuestra disciplina no representa utilidad 
alguna. Lo anterior se explica porque la Historia que se enseña, en la 
mayoría de los casos, es cronológica, descripción de hechos de grandes 
personajes, así como simple narración de acontecimientos de carácter 
militar y político, sin ninguna relación con el presente y con el mundo 
actual. Se continúa creyendo que quien más sabe de historia es aquel 
que memoriza el mayor número de fechas y de nombres de personajes. Se 
atormenta a los estudiantes con memorizaciones que no le permiten 
comprender el proceso histórico de la sociedad.

¿Qué caso tiene que los estudiantes se aprendan de memoria una 
infinidad de fechas lejanas y actuales, nombres de personajes 
desconocidos o de lugares que ni remotamente tienen conocimiento, y que 
nada de ello entienden que tenga una relación con su vida y su entorno? 
En la mayoría de las veces, la enseñanza de la Historia  se convierte 
en cronología, descripción de acontecimientos únicos e irrepetibles, 
sin sentido ni relación con nuestro tiempo y la realidad cotidiana.  El 
problema fundamental de esta fama negativa de la Historia tiene como 
causa principal esta forma errónea de entenderla, o en la falta de otra 
concepción auténtica y vital, que permita entrelazar el conocimiento 
histórico con los problemas a que se enfrentan cotidianamente los seres 
humanos. Como profesores de Historia debemos vincular el aprendizaje 
con hechos cotidianos de modo que los alumnos puedan dominar su vida. 

Considerando lo planteado anteriormente, ¿Cuál sería el papel del 
historiador en este mundo actual de profunda crisis generalizada que 
vive la sociedad? Ante esta interrogante planteamos la siguiente 
respuesta. Debemos partir reconociendo la complejidad de la crisis del 
mundo actual y reconocer que todos los seres humanos somos parcialmente 
responsables de ella y, al mismo tiempo comprometernos en su solución. 
No se trata a simple vista de una participación directa y material ---la 
Historia no es una ciencia fáctica---, sino de una participación que 
implique conocimiento y comprensión de los acontecimientos que se 
suceden en el mundo y que afectan a la humanidad en general, y a la 
sociedad mexicana en particular. Si logramos lo anterior, los 
estudiantes cambiarán su percepción tradicional de la Historia, por una 
idea que es y deberá ser, un conocimiento vital. En otras palabras, el 
conocimiento histórico se convierte en un elemento indispensable para 
la supervivencia de la especie humana.  
\enlargethispage{-1\baselineskip}

La crisis actual de nuestro sistema educativo, y que abarca la 
enseñanza de la Historia, es el resultado de una equivocada política 
educativa y de una simulación en el financiamiento del sector 
educativo.  Aunado a lo anterior, se manifiesta una deficiencia en la 
formación docente, así como la carencia de identidad normalista y 
magisterial, lo mismo que una falta de vocación, con algunas 
excepciones, de los maestros que imparten las asignaturas de Historia 
en los diversos niveles educativos y, particularmente, en las escuelas 
de Educación Superior. Por otra parte, tomando como ejemplo la 
licenciatura de Historia en la Unidad Académica de Filosofía y Letras 
de la Universidad Autónoma de Guerrero, contamos con una planta 
académica que en su mayoría está conformada por docentes que estamos en 
edad de jubilación. Por otra parte, también se tiene otro 
inconveniente, de esta planta docente, si bien es cierto la gran 
mayoría tiene la formación de historiadores, sin embargo, muchos de 
ellos carecen de conocimientos suficientes de Didáctica en la 
disciplina, lo que deviene en otro problema para la calidad en la 
enseñanza de la Historia. Presiento que estos problemas, en mayor o 
menor medida,  se manifiestan también en otras Escuelas Superiores de 
Historia del país. De ahí la importancia de una formación académica de 
calidad en nuestras futuras generaciones de docentes de Historia. 

A continuación planteo algunas propuestas de la manera como debemos 
abordar nuestra labor docente de la Historia. Empezaremos señalando que 
los profesores de Historia deben tener bien claro cuáles son las 
diversas interpretaciones del desarrollo histórico y cuál es el sentido 
de la Historia de acuerdo a la época de que se trate. Además, se deben 
conocer y entender las fuerzas que propiciaron los cambios y 
transformaciones que ha tenido la sociedad; así como la manera como se 
ha dado históricamente la relación entre la sociedad y la naturaleza, 
en el entendido de que el medio ambiente juega un papel importante en 
la vida de las sociedades. Teniendo en claro lo anterior, los 
responsables de la enseñanza de la Historia, deben estar convencidos de 
que su disciplina no se refiere solamente a aprender cómo han pasado 
los acontecimientos, sino que indaga por qué han pasado de tal o cual 
manera. En otras palabras el porqué del movimiento histórico. Además, 
el historiador debe tener plena conciencia en la razón y en la emoción, 
de la imprescindible necesidad del conocimiento histórico para todos; 
debe conocer la realidad al que se aplica su conocimiento y en función 
de ella seleccionar los contenidos que se van a tratar en sus clases, 
de acuerdo a sus propios postulados teóricos, debidamente formulados y 
asumidos. Debe tener muy claros sus objetivos y sus fundamentos. Por lo 
tanto, enseñar Historia no significa que los alumnos adquieran 
conocimientos eruditos, los memorice y luego los recuerde en un examen; 
sino que la enseñanza se convierta en un taller en donde el alumno 
adquiera los instrumentos y las herramientas que le permitan analizar 
el pasado y el presente, desarrollar en ellos una actitud crítica, así 
como una capacidad para la comprensión y la expresión <<No darle el 
pescado, sino enseñarlo a pescar>>.                       
%\newpage

Lograr que los estudiantes comprendan que la Historia es útil no sólo 
para interpretar la realidad, sino para transformarla; nos sirve 
también para reconocer nuestros orígenes y procesos, para recuperar la 
memoria colectiva de los pueblos y para forjar una concepción 
alternativa del mundo y de la sociedad.  

El nuevo contexto en que vive el mundo con la globalización requiere en 
la enseñanza de la Historia un proceso de reelaboración de 
conocimientos, redefinición de conceptos y modificación de las 
prioridades que propicien el surgimiento de nuevos paradigmas en la 
selección\linebreak y aprovechamiento de los recursos. Así, una de las mayores 
respuestas a la globalización consiste en construir y reconstruir la 
sociedad del saber y de la cultura, que propicien nuevas formas de ver 
la realidad. El conocimiento de la Historia debe convertirse en un 
proceso permanente que logre en los educandos un espíritu de compromiso 
social. La enseñanza de la Historia constituye una labor indispensable 
dentro de la complejidad del campo educativo ya que debe aportar ideas 
que permitan la construcción de una sociedad con mejor calidad de vida, 
acorde con sus necesidades, y que contribuya a la transformación 
individual y colectiva. 

Por último es importante mencionar que el conocimiento histórico no 
sólo debe enseñarse y aprenderse en el ámbito formal de la enseñanza, 
sino mediante todas las formas de difusión que existen en este mundo 
contemporáneo. Por lo tanto, el conocimiento histórico, como toda 
ciencia, debe ser conocido y aprovechado por la mayoría de la sociedad. 

Concluyo este apartado insistiendo en que la Historia debe servirnos 
para proponer alternativas para la construcción de una sociedad justa y 
democrática, principalmente de nuestra sociedad mexicana.
\enlargethispage{-1\baselineskip}

\medskip
{\bfseries La Historia y el medio ambiente}

La crisis global que padece la sociedad capitalista en la actualidad,  también se manifiesta en el ámbito educativo y, por ende, en la enseñanza de la Historia. El sistema educativo prevaleciente en 
nuestros días, caracterizado por su función reproductora está siendo 
cuestionado. Por tal motivo, se propone una educación crítica y 
transformadora que logre incidir en la solución de los graves problemas del mundo actual, entre ellos, la aguda crisis del medio ambiente. La Historia debe enseñarse para intervenir en el cumplimiento de la nueva propuesta educativa.  

Este apartado bien pudiera titularse la naturaleza como problema 
histórico imitando el título de la obra de David Arnold. Partiendo de 
lo anterior, planteamos las siguientes interrogantes: ¿Puede la 
naturaleza ser un tema para la Historia? ¿Ha sido la naturaleza un 
problema histórico? Para muchos historiadores, la respuesta es 
negativa, pues la idea predominante no ve en ella mas que el escenario 
donde se desarrollan los acontecimientos y las acciones humanas, y 
considera lo humano como su verdadero y casi único objeto de estudio, 
tal y como lo hemos señalado en párrafos anteriores. Sin embargo, 
conquistas, expansión de fronteras, descubrimientos de nuevos mundos, 
terror por lo agreste o ensoñación por lugares paradisíacos, han 
propiciado variadas interpretaciones de lo extraño y reinterpretaciones 
de lo propio y conocido. Debemos tener en cuenta que desde las más 
antiguas visiones geográficas hasta el mundo globalizado de nuestros 
días, la naturaleza y lo que en la actualidad llamamos medio ambiente 
inciden o influyen en la interpretación y la escritura de la Historia. 
<<Dime que comes y te diré quién eres>>. Lo confortante de esta situación 
es que ya se empieza a generar un pensamiento que entiende la íntima 
relación entre la naturaleza y la cultura o para ser más precisos entre 
las naturalezas y las culturas.

Por motivos didácticos empezaremos por definir lo que se entiende por 
medio ambiente. La palabra medio se deriva del latín \textit{medium} 
como género neutro o \textit{medius} como adjetivo. El vocablo ambiente 
procede del latín \textit{ambiens, ambientis}; así también del verbo 
\textit{ambere} que significa rodear, estar a ambos lados. De esta 
manera podemos\linebreak entender como medio ambiente a todo lo que rodea a un 
ser vivo. Es el entorno que afecta y condiciona especialmente las 
circunstancias de vida de las personas o de la sociedad. Comprende el 
conjunto de valores naturales, sociales y culturales existentes en un 
lugar y en un momento determinado, que influyen en la vida del ser 
humano. En otras palabras, no se trata sólo del espacio donde se 
desarrolla la vida, sino que también comprende seres vivos, objetos, 
agua, suelo aire, etc. y las relaciones entre ellos, asimismo se 
incluyen elementos intangibles como la cultura.

El medio ambiente es un sistema formado por elementos naturales y 
artificiales que se interrelacionan y al mismo tiempo son modificados 
por la acción humana. Se trata en última instancia del entorno que 
condiciona la forma de vida de la sociedad. En resumen, podemos decir 
que el medio ambiente no es únicamente lo que rodea a las especies y a 
las poblaciones biológicas, sino una categoría sociológica relativa a 
una racionalidad social, configurada por comportamientos, valores y 
saberes (Ferrer 2014, p. 25).  Planteado lo anterior, puede quedar alguna 
duda acerca de si la naturaleza debe ser un tema de estudio para la 
Historia o bien si es un problema histórico. Más claro ni el agua.
%\newpage

Es un hecho conocido que p.la sociedad en todos los tiempos ha 
intervenido, en menor o mayor medida, en el medio ambiente en que se 
desarrolla; por lo tanto en la actualidad no debe significar una 
novedad. Entonces a qué se debe la importancia que han adquirido en la 
actualidad los estudios que tratan esta relación. El interés radica en 
el cambio de la percepción acerca de la relación hombre--naturaleza que 
se ha venido generando, principalmente, a partir de la segunda mitad 
del siglo pasado. Este cambio es el resultado del incremento del nivel 
de desarrollo y el nuevo estatus de vida alcanzado por la civilización 
posmoderna que ha ocasionado graves problemas al medio ambiente, tales 
como la contaminación, el cambio climático, el agotamiento de los 
recursos naturales y un largo etcétera, a lo cual se agrega la 
explosión demográfica; lo que por primera vez en la historia pone en 
riesgo la supervivencia de los seres humanos, así como la propia 
existencia de nuestro planeta. De ahí que la crisis ambiental sea 
considerada como el problema más importante y urgente  que tiene que 
enfrentar la humanidad en este momento.

Es importante tener presente que cuando se habla de crisis ambiental, 
se está haciendo referencia a un fenómeno de origen social: su causa se 
encuentra en el comportamiento del ser humano característico de la 
sociedad industrial del momento. De ahí que si la crisis ambiental es 
percibida como una crisis socialmente provocada, entonces su solución 
también deberá ser atendida de la misma forma, transformando su 
discusión en problema de la Historia. 

David Arnold, en su obra \textit{La naturaleza como problema histórico}, plantea lo siguiente:
%\enlargethispage{1\baselineskip}

\begin{quotation} 
El conocimiento de la subordinación y dependencia de los 
humanos respecto a la naturaleza data de hace muchísimo tiempo, pero el 
sentido de los seres humanos como guardianes y destructores de la 
naturaleza apenas acaba de nacer y, con él, la abrumadora sensación de 
nuestra responsabilidad por la destrucción pasada y la supervivencia 
futura de otras especies (Arnold~2000, p.~13).
\end{quotation}   

Es común señalar que la crisis ambiental se inicia como consecuencia de 
la Revolución Industrial, sin embargo su percepción como problema 
histórico de carácter global es propio de la historia reciente, y su 
tratamiento en perspectiva es aún poco tratado, por lo menos en nuestro 
país y, principalmente por los estudiosos de las ciencias sociales\linebreak y 
humanas, en particular los historiadores. En cuanto problema histórico, 
aparece en la segunda mitad del siglo veinte en las naciones con mayor 
desarrollo, desde donde se ha venido expandiendo y penetrando en el 
imaginario colectivo de las demás naciones, en un proceso de creciente 
socialización y toma de conciencia ambientalista, hasta posicionarse 
como un tema obligado en nuestros días. Sin embargo creo que en nuestro 
país, especialmente entre los historiadores, todavía se manifiestan 
resistencias a tratar el tema del problema del medio ambiente; a lo 
sumo se agregan algunas asignaturas, hoy llamadas unidades de 
aprendizaje, en la currícula de las licenciaturas de Historia, como es 
el caso de la Universidad Autónoma de Guerrero. Todavía prevalece la 
concepción de la Historia como disciplina  que sólo tiene por objeto de 
estudio al ser humano desvinculado de su entorno. Se nos olvida que la 
Historia debe servir para comprender y actuar sobre nuestra vida 
actual. El historiador no debe vivir en una burbuja que lo aleje de la 
realidad, sino comprometerse con  la problemática que se manifiesta en 
su espacio y su tiempo. Probablemente una de las razones de la relativa 
indiferencia puede atribuirse a que la temática apenas empieza a 
tocarse en los círculos de las ciencias sociales y de las humanidades, 
incluyendo, por supuesto, a la Historia.

\bigskip
{\bfseries Conclusión}

Considerando lo descrito en relación al sistema educativo nacional, 
podemos darnos cuenta que la educación que se imparte en los distintos 
niveles educativos, incluyendo la educación superior, es más una forma 
de conservar la cosmovisión y el \textit{status quo} prevaleciente que 
pretender modificar la realidad actual. La crítica no se encuentra 
explícitamente obstaculizada pero debe ceñirse a reglas del juego 
externas. De ahí que el sistema educativo tenga un carácter 
conservador, con resistencias al cambio y aunque se lleven a cabo 
reformas institucionales, como la recién aprobada <<reforma educativa>> y 
su <<puesta en práctica>> en nuestro país, éstas se implementan como una 
estrategia oculta para que todo continúe igual. Los cambios son 
formales y no estructurales. Pura simulación y gatopardismo educativo.  

En el nuevo contexto en que vive el mundo con la globalización con 
todos los problemas que está ocasionando, se requiere en la enseñanza 
de la Historia un proceso de reelaboración de conocimientos que 
vinculen el aprendizaje con hechos cotidianos de modo que los seres 
humanos generemos nuevas formas de ver la realidad y ser participes de 
su transformación.

Tomando en cuenta que la agudización del deterioro del medio ambiente y 
su consecuente aumento de desastres ecológicos de magnitud imaginable 
han sido, en mayor medida, generados por el ser humano; entonces la 
enseñanza de la Historia debe fortalecer las acciones tendientes a 
contribuir a la solución de este grave problema, que en nuestros días, 
constituye la amenaza más importante para la supervivencia de la 
humanidad. Por tal razón, los contenidos curriculares deben estar 
vinculados con los actores sociales pero íntimamente vinculados con el 
medio ambiente. Asimismo, integrar en el proceso de 
enseñanza-aprendizaje de la Historia, los conocimientos en relación con 
el cuidado y protección del medio ambiente. Los historiadores debemos 
encabezar la búsqueda de soluciones a la compleja problemática 
ambiental y contribuir a la transformación de la sociedad y a la 
preservación ecológica del planeta. La enseñanza de la Historia debe 
constituirse en un proceso permanente de aprendizaje basado no sólo en 
el respeto de todas las formas de vida, sino en un factor de compromiso 
social fundamental: sensibilizar al ser humano relacionándolo con su 
ambiente. Pero además, en este proceso dialéctico 
enseñanza--aprendizaje, se deben generar ideas para reconstruir una 
sociedad con mejor calidad de vida, acorde con sus necesidades; 
confrontar nuestra labor docente con los valores que la guían y 
reforzar las acciones que contribuyan a la transformación individual y 
social, pensando siempre en la relación armoniosa que demos establecer 
con nuestro medio ambiente. Este es el reto para la nueva enseñanza de 
la Historia. Este es el compromiso que debemos asumir quienes nos 
dedicamos a la enseñanza de Clío en un mundo globalizado que está 
poniendo en riesgo la supervivencia de la especie humana.  


\bigskip 
{\bfseries Referencias}
\enlargethispage{1\baselineskip}

\medskip
Arnold, David (2000), \textit{La educación como problema histórico}, 
México, FCE.

Batllori Guerrero, Alicia (2008), \textit{La educación ambiental para la sustentabilidad:un reto para las universidades}, México, Universidad Nacional Autónoma de México.

Carreto, Mario y James F. Voss (2004) \textit{Aprender y pensar y la 
historia}, Buenos Aires-Madrid, Amorrortu Editores.

Ferrer Vicario, Gil Arturo, \textit{et. al. } (2014), \textit{Educación para la sustentabilidad}, México, Ediciones Eón-Universidad Autónoma de Guerrero.

Glazman, Raquel (1986), \textit{La docencia; entre el autoritarismo y la igualdad}, México, SEP-Ediciones El Caballito.

González de Molina, Manuel (2004), \textit{Historia y medio ambiente 
}, Morelia, Michoacán, México, Jitanjáfora.

González Gaudiano, Édgar J. (2008), \textit{Educación, medio ambiente y sustentabilidad}, México, Siglo XXI Editores. 

Le Goff, Jacques (2005), \textit{Pensar la historia. Modernidad, presente, progreso}, Barcelona, Ed. Paidós.

Nieto López, José de Jesús (2001), \textit{Didáctica de la Historia 
}, México, Ed. Santillana.

Zubiria Remy, Hilda Doris (2004), \textit{El constructivismo en los 
procesos de enseñanza-aprendizaje en el siglo XXI}, México, Plaza y 
Valdés.

