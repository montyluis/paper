%\documentclass{article}
%\usepackage{amsmath,amssymb,amsfonts}
%\usepackage{fontspec}
%\usepackage{xunicode}
%\usepackage{xltxtra}
%\usepackage{polyglossia}
%\setdefaultlanguage{spanish}
%\usepackage{color}
%\usepackage{array}
%\usepackage{hhline}
%\usepackage{hyperref}
%\hypersetup{colorlinks=true, linkcolor=blue, citecolor=blue, filecolor=blue, urlcolor=blue}
%\usepackage{graphicx}
%% Text styles
%\newcommand\textstylehps[1]{#1}
%\newcommand\textstyleappleconvertedspace[1]{#1}
%\newtheorem{theorem}{Theorem}
%\title{}
%\author{Alfonso}
%\date{2014-08-20}
%\begin{document}
%%\clearpage\setcounter{page}{541}

\thispagestyle{empty}
\phantomsection{}
\addcontentsline{toc}{chapter}{Programas de Estudio en Modalidad\newline no
Convencional\newline $\diamond$
\normalfont\textit{María de los Ángeles Sitlalit García Murillo,\newline Beatriz Rico Álvarez  y
Sonia Bouchez Caballero}}
{\centering {\scshape \large Programas de Estudio en Modalidad no
Convencional}\par}
\markboth{la formación del historiador}{modalidad no convencional}
\setcounter{footnote}{0}
\renewcommand*{\thefootnote}{\fnsymbol{footnote}}
\setcounter{footnote}{0}

\bigskip
\begin{center}
{\bfseries María de los
Ángeles Sitlalit García Murillo\\
Beatriz Rico Álvarez\\
Sonia Bouchez Caballero}\\
{\itshape Universidad Autónoma de Sinaloa}\footnote{Profesoras de
la Facultad de Historia de la UAS. Este trabajo se realizó basándose en el documento del  proyecto de la modalidad de la Licenciatura en Historia,  proporcionado por  la Coordinación de la Licenciatura en Historia de dicha facultad, y en el Diplomado (llevado por sus servidoras) en Formación Docente en Modalidades No Convencionales  para la implementación de la modalidad semiescolarizada.}
\end{center}

\renewcommand*{\thefootnote}{\arabic{footnote}}
\setcounter{footnote}{0}

\bigskip
\textbf{Resumen}

El diseño de ambientes virtuales de aprendizaje para
apoyar los procesos de aprendizaje en las modalidades educativas no
convencionales es una necesidad sentida, ya que la tendencia de la
educación superior en nuestros tiempos va hacia el uso de las TIC y de la
plataforma educativa, utilizadas en la formación académica de los
estudiantes y de los docentes.

El presente trabajo es el resultado obtenido de un diplomado en Formación Docente
en Modalidades no convencionales, en el cual participaron los docentes de
la facultad de Historia, y donde se abordó el uso de plataformas utilizando
la Internet con fines educativos, es decir, el  uso de las tecnologías en el
área del conocimiento. Además, se elaboraron  programas de estudio, cartas
descriptivas, guías del estudiante y antologías de los diferentes cursos
con el fin de ofrecer la licenciatura en historia en modalidad
semipresencial.\\ 
\textbf{Palabras Claves:} Calidad Educativa, Ambientes para el aprendizaje,
Modalidades Virtuales, Modelo Educativo.


\medskip
\textbf{Abstract}

The design of virtual
learning environment to support learning processes in unconventional
educational methods is a felt need, as the trend of higher education in our
time tends to the use of TIC and educational platform for academic
training of students and teachers.

This work is the result of
a Diploma in Teacher Education in unconventional methods, with the
participation of teachers of the faculty of history and where we discussed
the use of platforms using the Internet for educational purposes, namely,
the use of technology in the area of knowledge. In addition, we developed
curricula, descriptive letters, student guides and anthologies of different
courses, in order to offer a degree in history in a semipresencial mode.\\
\textbf{Keywords:} Quality Education, Learning Environments,\\ Virtual Modalities,
Educational Model.


\medskip
Las nuevas generaciones viven intensamente el uso de las tecnologías
digitales, al punto que esto podría estar incluso modificando sus destrezas
cognitivas ya que están desarrollando algunas destrezas distintivas como la
facilidad de  adquirir  gran cantidad de información fuera de la escuela,
toman decisiones rápidamente y están acostumbrados a obtener respuestas
casi instantáneas frente a sus acciones e interactúan en espacios
diferentes.


\medskip
\textbf{Antecedentes}

\enlargethispage{1\baselineskip}
La educación superior en México ha tenido un importante
avance en los últimos años, logrando mayores niveles de calidad y una mayor
cobertura.  Sin embargo, su desarrollo no es todavía suficientemente amplio
para enfrentar los retos que le plantea el crecimiento demográfico y las
necesidades del país en materia de formación de científicos, técnicos y
profesionistas. 

En el documento \textit{La Educación Superior en el Siglo XXI, líneas
estratégicas de  desarrollo}, destaca la ANUIES que <<el crecimiento de la
demanda de educación superior que se avizora  para las dos primeras décadas
del siglo veintiuno difícilmente podrá ser atendido  adecuadamente con los
sistemas tradicionales>> (ANUIES~2000, p. 215). Por ello, sugieren impulsar
nuevas formas de educación a través de modelos abiertos y a distancia. 

Esta necesidad de ampliar la cobertura a través de modalidades alternativas
de educación ha sido retomada por la Universidad Autónoma de Sinaloa
dentro del \textit{Plan de Desarrollo Institucional: Consolidación 2017},
en el que se plantea como uno de sus objetivos estratégicos el <<Contribuir
a la ampliación de la cobertura a partir de la consolidación de modalidades
educativas no escolarizadas>> (PDIU~AS~2013, p. 39). 

Por su parte, la Facultad de Historia, consciente de este reto, busca ofrecer
una opción para aquellos estudiantes que, teniendo interés de adquirir una
formación profesional en historia, no han podido hacerlo en el modelo
presencial escolarizado que se oferta actualmente.  En este sentido, se
ofrece la licenciatura en historia en modalidad semipresencial, en la que
se retoman el plan de estudios y el diseño curricular por competencias
vigente en la Facultad.

%\enlargethispage{1\baselineskip}
Para su operación, se busca complementar las sesiones presenciales con el
uso de antologías y guías didácticas, que guíen los procesos de autoestudio
de los alumnos, así como a través de actividades y ejercicios en línea que
coadyuven al logro de las competencias necesarias para su formación
profesional, y que permitan la retroalimentación entre maestros y
estudiantes. 
\newpage

\textbf{Modalidad semiescolarizada}

En este sentido, esta la nueva modalidad, que  es una combinación de enseñanza presencial y de autoformación, de alta calidad académica, que permite la relación entre el
asesor y alumno, sobre la base de un porcentaje de presencialidad menor al
modelo escolarizado y fomenta en el alumno la iniciativa sobre su
aprendizaje, ritmo y circunstancias en que lo llevará a cabo, sin que esto
sea impedimento para su aprendizaje, ya que el estudiante contará en los
momentos programados para ello con asesorías presenciales, así como con el
apoyo de guías de autoestudio. En esta nueva modalidad el alumno deberá
convertirse en el principal actor de su educación, buscando ante todo
incrementar su capacidad cognoscitiva, con base en el autoaprendizaje, la
mejora continua de sus técnicas de estudio y el acudir a las asesorías con
el personal docente para exponerle sus dudas, intercambiar ideas,
experiencias y opiniones y presentar sus trabajos. La necesidad de la
presencia del alumno en el campus es sustituida por la institución,
proporcionándole todos aquellos elementos que les permiten lograr su
formación a distancia. Por lo que el grado de apertura y flexibilidad del
modelo depende de los recursos didácticos de acceso, del equipo de
informática y telecomunicaciones del cuerpo académico y de una adecuada
administración escolar.

El carácter no presencial del alumno no debe significar la ausencia de los
elementos de infraestructura institucionales necesarios para la
administración del modelo, incluyendo el apoyo al desempeño del personal
académico. Las habilidades a desarrollar por parte del alumno es el trabajo
cooperativo como un recurso muy  importante para generar procesos de
autonomía. Además, se  requiere la colaboración y responsabilidad personal.

En el caso de La Licenciatura en Historia, el Sistema de Torre Académica de
la Universidad Autónoma de Sinaloa, basándose en el eje estratégico
institucional Docencia, cuyo objetivo es mejorar la calidad, pertinencia y
equidad de los programas educativos y servicios de la Universidad, a través
de la evaluación permanente y la sistematización de sus procesos
implementando una política para consolidar la calidad y pertinencia en los
programas educativos, llevó a cabo el Diplomado \textit{Formación Docente
en Modalidades no Convencionales}, el cual surge de la necesidad de
perfeccionar y enriquecer una experiencia docente que se transforme y
se reformule constantemente.

El objetivo general de dicho diplomado fue formar una nueva generación de
docentes-asesores capaces de realizar una planeación didáctica de acuerdo
al modelo educativo implementando en la institución, y además de diseñar y
operar un paquete didáctico utilizado en diversos ambientes para el
aprendizaje, aplicando estrategias de intervención pedagógica en los
programas educativos en modalidades presencial y mixta.

\enlargethispage{1\baselineskip}
Los objetivos específicos fueron formar facilitadores-asesores capaces de
brindar a los participantes los elementos teórico-metodolólogicos que les
permitan resolver la problemática del aula y proponer alternativas de
solución. Así como impulsar la superación profesional de los docentes
involucrados en la tarea de la educación, el intercambio de experiencias
relativas a la práctica docente, lo que les permitirá cuestionar su
cotidianidad en el aula con la finalidad de que el usuario del diplomado
proponga estrategias didácticas en su espacio de trabajo más acordes a la
sociedad del conocimiento y al momento actual en que vivimos. Otro objetivo
es utilizar y propiciar el uso de ambientes de aprendizaje para
estudiantes y docentes, a fin de que reconozcan los roles de los diferentes
actores del proceso educativo de un curso.

El diplomado consistió en cuatro módulos: 1) modelo educativo y enfoque por
competencias, cuyo contenido fue \textit{las competencias, actuación eficiente en un
contexto determinado y las competencias, un enfoque socioformativo}; 2)
planeación y evaluación, cuyo contenido fue \textit{las modalidades de la
educación, planeación didáctica y estratégica y evaluación en el enfoque
por competencias}; 3) paquete didáctico, cuyo contenido fue \textit{las modalidades
educativas, diseño de planes de estudio, como diseñar el programa de una
asignatura, elaboración de antología, elaboración de guías didácticas,
diseño del paquete didáctico y ambientes de aprendizaje con el contenido de
ambientes para el aprendizaje, fuentes de información y de expresión, redes
de colaboración y su interacción} y 4) competencias docentes necesarias para el
trabajo en aula. 

Para la evaluación se consideró el producto de los planes de estudio,  carta
descriptiva, guía didáctica del estudiante y la antología para las
asignaturas del primer semestre (en nuestro caso diseñamos dos  programas
de estudio: la ética profesional del historiador e Historia del Noroeste).

La duración del diplomado fue de un cuatrimestre, tiempo durante el que los docentes
asistieron a las asesorías grupales o sesiones presenciales de 4 horas de
duración, una vez por semana.

\medskip
\textbf{Fundamentación: El entorno social y educativo}

La sociedad actualmente experimenta cambios importantes en todos sus
ámbitos. La globalización y liberalización de la economía, la
competitividad, pero también la colaboración internacional en materia de
capitales, bienes y servicios, han conducido a la necesidad de modificar la
forma en que funcionan los sistemas e instituciones, entre ellas las relacionadas
con la educación.

La Educación Superior, en particular, ha participado a la vez que está siendo
transformada por las grandes revoluciones en la ciencia y la tecnología; la
diversificación de estilos de vida individual y familiar demandan nuevas
modalidades y ambientes educativos que faciliten a todas las personas,
jóvenes y adultas, continuar educándose a lo largo de la vida. 

La formación de profesionales para esta nueva sociedad del conocimiento y la
innovación, también tiene que considerar las diversas problemáticas que
actualmente experimentamos, entre ellas: la inestabilidad económica y
social; la pobreza, que deja a millones de personas sin cubrir sus
necesidades básicas y sin acceso a trabajo, educación y salud; la
inseguridad producto de la violencia manifestada de diversas maneras; los
problemas relacionados con el medio ambiente y la sustentabilidad.

\enlargethispage{1\baselineskip}
La ANUIES (2012) orienta a las
instituciones de educación superior mexicanas a coadyuvar en garantizar la
inclusión de los jóvenes en programas de formación avanzada; a incrementar la
calidad y responsabilidad social de las instituciones y sus actores en los
procesos de transmisión, generación y divulgación del conocimiento, y a 
promover  la seguridad, los derechos humanos y el cumplimiento de sus
obligaciones, así como priorizar el desarrollo sustentable.

Entre los aspectos innovadores en educación superior que han aparecido de
manera incipiente en el último lustro, destacan los esfuerzos  por atender
alumnos con necesidades educativas especiales, así como a estudiantes con
talentos, para apoyarles a fin de disminuir el fenómeno de fuga de cerebros
y aprovechar el potencial de los mexicanos para impulsar el desarrollo de
nuestro país. En este rubro, nuestra Universidad está siendo pionera con los
programas de Formación de Doctores Jóvenes y el de Atención a la
Diversidad.

Uno de los elementos importantes a considerar en educación superior es lo
referente al paradigma de la evaluación, calidad y rendición de cuentas.
Respecto a ello, diversos organismos e instancias supervisan,  evalúan,
certifican y acreditan la calidad educativa, tanto a nivel nacional como
internacional. Como parte de las orientaciones generales de dichos
organismos, destacan en la actualidad la responsabilidad social, la
inclusión educativa y la calidad, que implican contribuir al desarrollo
social, económico, científico y cultural, orientándose al bienestar de las
personas y a preservar la naturaleza.

En nuestro país se ha promovido diversificar la oferta formativa y ampliar
la cobertura con equidad, solidez académica y optimización de los recursos,
así también, se ha promovido un sistema educativo más abierto, flexible y
articulado, lo que se ha traducido en  acciones de
vinculación, intercambio académico, movilidad de estudiantes y académicos,
y formación de redes de cooperación  a nivel nacional e internacional.
\enlargethispage{2\baselineskip}

%\smallskip
\textbf{Modelo académico de la UAS, base del diseño\\ curricular}

El modelo académico de nuestra institución considera seis ejes para
desarrollar las funciones sustantivas y el diseño de los programas
educativos de la institución. Estos ejes son:
%\smallskip
\begin{Obs}
\item[$\star$] {Integración de funciones sustantivas}
\item[$\star$] {Desarrollo social y natural sustentable}
\item[$\star$] {Atención equitativa a necesidades y talentos}
\item[$\star$] {Internacionalización}
\item[$\star$] {Incorporación de tecnologías}
\item[$\star$] {Innovación}
\end{Obs}

\medskip
El modelo académico de la institución también orienta a flexibilizar la
formación profesional, de diversas y diferentes formas. Las posibilidades
de flexibilidad  factibles de considerar  se representan en el esquema
siguiente:

\vspace{6pt}
\begin{figure}[H]
\centering
\includegraphics[scale=0.61]{cul.png}
\end{figure}

\medskip
El Modelo Académico se orienta a diseñar los planes de estudio con base en el
Modelo Curricular por Competencias Profesionales Integradas. El concepto de
competencias <<hace referencia a la estructura de atributos que permiten a
un profesional movilizar sus recursos teóricos, prácticos y actitudinales,
para desempeñarse de manera contextualizada y efectiva al solucionar
problemas o situaciones en un área específica de actividad.>> (PDI UAS,
2013).

\enlargethispage{1\baselineskip}
Las competencias que debemos contemplar en el currículo se dividen en dos
grupos: \textit{genéricas y específicas}. Las primeras se
identifican con los elementos compartidos en el proceso de formación de
cualquier perfil profesional, tales como la capacidad de aprender, tomar
decisiones, diseñar proyectos. Es decir, \textit{son
comunes a todas las carreras profesionales}.
Las competencias específicas son aquellas propias de una
profesión; incorporan conocimientos, métodos, técnicas, reglamentos y
comportamientos que conforman el núcleo básico para desarrollar el
ejercicio profesional. Las competencias genéricas pueden convertirse en
competencias sello, es decir, las que otorgan identidad
a una comunidad educativa y sus egresados.  Con base en un proceso de
consulta interna y externa, en la UAS  se han
identificado orienta diez competencias sello que son consideradas base para
la formación y que deben ser incluidas en función de las prioridades del
programa de la Licenciatura en Historia.

El compromiso social de la disciplina histórica se expresa a través de un
discurso narrativo-explicativo que pretende brindar un aporte al
autoconocimiento de las sociedades humanas. Este conocimiento histórico ha
mostrado algunas limitantes, incapacidad para difundirse entre la totalidad
de la sociedad y, de manera más reciente, la pérdida de su vigencia, la que
ha sido provocada en parte, por las transformaciones sociales y culturales
promovidas por la globalización de las comunicaciones, experimentadas en
los últimos veinticinco o treinta años, las cuales han inducido a una
ruptura entre el pasado y el presente, el tiempo y el espacio. Esto también
tiene explicación en el mercantilismo que dinamiza la revolución científica
actual y explica que algunas disciplinas o campos interdisciplinares, como
las humanidades o ciencias sociales, dispongan de cada vez menos peso en el
currículum escolar, de menos presupuesto para la investigación, mientras
que las ciencias con aplicación directa en el mundo empresarial y militar
disponen con mayor frecuencia de presupuestos ilimitados (Torres~2008).

\enlargethispage{1\baselineskip}
Las críticas hacia la Historia han cuestionado la pertinencia de sus
problemas y de los métodos a los que recurre para desentrañarlos, con lo
cual <<han puesto en tela de juicio los presupuestos en los que se basaba la
ciencia histórica desde su fundación como disciplina histórica en el siglo~XIX>> (Iggers 1995, p.103). En la segunda mitad del siglo~XX, el
postmodernismo, el giro lingüístico y los cuestionamientos sobre la
relación entre el discurso narrativo y la representación histórica (White~1992) 
o el debate suscitado sobre el <<fin de de la historia>> (Fukuyama~1993) 
tras la caída del socialismo, constituyen algunas de las corrientes
que han cuestionado la posibilidad de una historia <<objetiva>> (Noiriel~1997), al tiempo que han discutido las concepciones científicas,
hermenéuticas y analíticas de la Historia (Iggers~1995; Prost~2001).

La comunidad internacional de historiadores y de historiadoras profesionales
es consciente de que su labor parte del estudio de aquellos discursos
construidos socialmente los que, al ser sometidos a criterios de
selectividad más rigurosos y tras un proceso de validación, se transforman
en conocimiento social. Este conocimiento histórico, como todo saber social,
es reflexivo, crítico e inacabado. De igual modo, los historiadores han
percibido la necesidad del trabajo inter, multi y transdisciplinario en
aras de construir vínculos que conduzcan al mutuo enriquecimiento de las
áreas de trabajo, cuyo objeto de estudio es la sociedad (Acuña~2009). Estas
críticas han propiciado que los historiadores reflexionen sobre la función
social de la disciplina, al tiempo que han revalorizado el \textit{oficio del
historiador}: <<comprender el presente por el pasado>> y, correlativamente,
<<comprender el pasado por el presente>> (Burker~2001, p.22) No obstante,
para que este proceso sea efectivo, se hace indispensable que el
profesional en Historia domine una serie de elementos propios de su
quehacer como lo son <<el conocimiento del presente, la discusión
teórico-filosófica-temática-metodológica de la Historia, el trabajo en
equipo y la preocupación por la difusión histórica>>
(Viales~2010, p.9). En la Facultad de Historia de la Universidad Autónoma
de Sinaloa se ha optado por un modelo que propicie la criticidad, sin dejar
de lado el perfil humanista, al tiempo que se ha aspirado a la formación de
investigadores de alto nivel, capaces de comunicar y difundir por diversos
medios un conocimiento histórico relevante. 
 
En su devenir, la disciplina histórica ha tenido una tendencia a mostrar un
mayor interés por el estudio de ciertas áreas, ya sea por influencias
foráneas (entiéndase europeas, estadounidenses y latinoamericanas), como
por las circunstancias e intereses particulares, ya sea de tipo personal o
por la influencia de factores regionales o locales, los que en su conjunto
han incidido en el predominio de una u otra área de especialización
histórica. De esta manera, se han establecido algunas divisiones o
especializaciones para el abordaje de la Historia como son: la historia
económica, la historia social, la historia política y la historia cultural.
A estas divisiones se han venido sumando una serie de especializaciones
surgidas de las intersecciones, entre ellas las que han efectuado una
contribución al conocimiento. No obstante, algunos de estos nuevos campos
han tropezado con dificultades, debido a su poca visión de los procesos
históricos generales. 

\enlargethispage{1\baselineskip}
Respecto a la Historia política o la Historia del poder, en la actualidad
predominan dos áreas de investigación. La primera de ellas se enfoca en
cómo interpretar las construcciones hegemónicas e ideológicas, lo que ha
ocasionado una discusión teórica muy profunda entre tres corrientes de
izquierda: la visión del poder como análisis de discurso  (encabezada por
Ernesto Laclau), la que lo conceptúa desde la psicología marxista (Zizek en
el caso europeo) y una tercera visión, cuya propuesta la sustenta Tony Negri,
sobre cómo interpretar diversos fenómenos sociopolíticos tomando en cuenta
la relación entre las clases, los movimientos sociales y los dirigentes,
los líderes sindicales y la gran masa. 

La segunda área versa sobre la cuestión de la memoria, y de ella se
desprenden dos visiones: una que apuesta por romper los límites entre
Historia y Memoria, como forma de recuerdo personal o colectivo del pasado,
para tratar de entender la memoria del individuo y la colectividad; y otra
que sigue apostando por conceptualizar la memoria solamente como una
interpretación subjetiva del pasado y no como una posibilidad para
interpretarlo. De este modo se constata que la Historia Política ha
adquirido una dimensión más amplia, gracias a la heterogeneidad presente en
la discusión teórica, así como por los problemas y preguntas que se le han
hecho al poder, los que han permeado otras áreas y han facilitado la
fundación de nuevos campos como la historia de género, historia de la niñez
e historia de los nacionalismos entre otros.

La Historia económica se ha caracterizado por una práctica compartida de
historiadores y economistas, hay economistas historiadores e historiadores
económicos. Existe un predominio en un sector, integrado por
estadounidenses, heredero de la New Economic History y de la Cliometría,
que se sigue practicando ---al igual que algunos europeos y latinoamericanos---
preocupados por el crecimiento económico, los modelos económicos
alternativos y los modelos econométricos con el fin de explicar el
desempeño de las economías en el largo plazo (como sucede con los estudios
de Angus Maddison). Otra corriente importante está siendo practicada en
Europa y América Latina por quienes están más vinculados con la Historia,
los cuales han incorporado una visión más socioeconómica o de una historia
económico-social, semejante a la Escuela Francesa convencional o la
Sociohistoria. Estas líneas de investigación abordan principalmente el estudio
de la historia de los ciclos productivos, la reconstrucción de la
economía nacional y la circulación de bienes y servicios, la historia de la
globalización, así\linebreak como la historia empresarial, la historia de la colonización, del
bienestar, la historia rural y agraria, la historia industrial, la historia
económica ecológica, y un importante grupo de subtemas, originados en las
intersecciones de estos.

En cuanto a la Historia social y cultural, existen tres grandes períodos de
desarrollo: uno es el iniciado en la década de 1970, caracterizado por
gestar una historia social y cultural muy relacionada con la historiografía
británica y con la Escuela Francesa de los Annales o la historiografía
francesa en general, la cual planteaba un tipo de ciencia comprometida con
las transformaciones sociales, que introdujo a nuevos actores y reconoce
que en todo hecho social se entrelazan hechos culturales y económicos, tal
y como se evidenció en el abordaje de temáticas propias de la historia
económica, los movimientos sociales y la conflictividad. Posteriormente,
surgió un movimiento que se ha llamado postmodernismo, culturalismo o giro
lingüístico, con una fuerte influencia de los estudios culturales
estadounidenses. Este movimiento ha sido predominante, se ha diferenciado
un enfoque sustentado en lo relativo, especialmente en el concepto de
búsqueda de la verdad, de cómo se construye la historia y con qué fin.
Enfatiza en los sujetos, lo simbólico, la semiótica, la Antropología, las
construcciones sociales, los imaginarios y las representaciones. En ese
sentido, es una historia más relacionada con los segmentos académicos, que
con los del compromiso social, tal y como sucedía en el período anterior.

%\enlargethispage{1\baselineskip}
Un tercer momento, un tanto reactivo a la corriente del giro histórico, ha
convertido a la Historia en un factor interpretativo de las realidades,
asociada con las Ciencias Sociales y con los compromisos políticos y
culturales. En esta línea de interpretación han surgido nuevos campos más
allá de lo cultural, como la historia ambiental, la ecológica, de la salud
y de la ciencia.

\enlargethispage{1\baselineskip}
La historia ambiental, la historia ecológica, la historia
eco\-nó\-mi\-co-eco\-ló\-gi\-ca, la ecología histórica y la historia agroecológica,
sólo para mencionar algunos de los principales puntos de partida de los
estudios de las relaciones sociedad-naturaleza en perspectiva histórica, se
han desarrollado alrededor del interés común de mostrar las múltiples
interacciones entre la <<naturaleza>> y el medio social, como una constante
en la historia de la humanidad, y no como dos realidades inconexas, tal y
como habían sido concebidas en la historiografía tradicional, donde el
mundo natural se ha presentado como un simple telón de fondo sobre el cual
se han tejido las interacciones de las sociedades humanas en ausencia de la
naturaleza (Worster~2000).

Los estudios históricos del ambiente parten de la premisa de que <<la
naturaleza no es pasiva\ldots{} es un socio inseparable de la cultura humana en la
historia del planeta>> (McEvoy 1993, p.13) por lo que
visibilizar el papel de la naturaleza en la historia de la humanidad, así
como el de las sociedades humanas en la historia de la naturaleza, más que
una necesidad se torna en un imperativo. En el caso de América Latina, esta
subdisciplina o campo de trabajo historiográfico, es indisociable del
intercambio ecológicamente desigual que históricamente se ha construido
entre el norte y el sur y entre pobres y ricos, análoga y propia de la
<<economía de rapiña>> instaurada desde tiempos coloniales. Es por esto que
en el subcontinente, el estudio de las relaciones entre las sociedades
humanas y el mundo natural, se ha acompañado de un interés y compromiso en
visibilizar el carácter desigual, en responsabilidad e impacto, de las
transformaciones planetarias del ambiente. Es por esto que referentes
teórico-conceptuales como la historia económico-ecológica y del ecologismo
de los pobres, se han constituido en herramientas básicas para hacer
historia ecológica desde América Latina.

Las áreas del conocimiento disciplinario de la Historia Política o la
Historia del Poder, la Historia Económica, la Historia Social y Cultural,
la Historia Ambiental, la historia ecológica, la historia
económico-ecológica han incorporado las diferentes problemáticas que se
abordan en el mundo académico internacional. Asimismo, han producido una
valiosa inserción en los desarrollos teórico–metodológicos, al introducir
nuevas tendencias metodológicas, tanto de tipo cuantitativo como
cualitativo. El desarrollo de la investigación especializada y su
aplicación a nivel de Plan de Estudios de Bachillerato y Licenciatura en
Historia es uno de los grandes desafíos que plantea la transposición
didáctica para la formación de profesionales en Historia. Esta es la
dirección que pretende seguir el nuevo Plan de Estudios de Bachillerato y
Licenciatura en Historia, a fin de eliminar la brecha que existe en la
actualidad entre la Escuela y el Posgrado en la formación de este
profesional.

\bigskip
\textbf{Consideraciones finales}

%\enlargethispage{1\baselineskip}
El desarrollo histórico de la educación es ilustrativo respecto al esfuerzo
por la cobertura a las necesidades educativas de los diferentes grupos
poblacionales de nuestro país. La educación abierta y a distancia se ha
planteado en uno de sus enfoques como una alternativa más para la búsqueda
de la democratización y socialización  del conocimiento, y, de manera
específica, la educación  superior no está al margen de ello. En las
instituciones de educación  superior se ha alternado,  junto con su oferta
educativa tradicional, el diseño e implementación de modalidades no
convencionales de educación que, en búsqueda de la equidad formativa de grupos
específicos de la población, se formulan como una respuesta concreta.

En la Universidad Autónoma de Sinaloa, desde hace algunos años, se ha
ofertado, en las áreas de educación y humanidades y 
las de ciencias sociales y administrativas, la modalidad semiescolarizada. La
Facultad de Historia se integra a esta modalidad, partiendo
del interés existente en algunos sectores de la
población, especialmente en profesores del área de las ciencias sociales, y
humanidades en general, por cursar la Licenciatura, opción que resuelve a
dichos interesados la problemática que representa acudir de forma continua
a las aulas.

La modalidad semiescolarizada acrecienta su viabilidad
por ser incluyente con un mercado potencial que ha quedado sin la
posibilidad de integrarse al estudio de la historia, debido a las
distintas actividades que desarrollan, siendo este grupo
de interesados parte significativa para la Licenciatura en Historia,
justamente por la modalidad de la misma.

La singularidad de la modalidad en que se implementa la Licenciatura en
Historia involucra a los estudiantes para que acudan a sesiones
presenciales el día sábado, y continúen su formación en forma
complementaria, vía Internet. El modelo y organización curricular de la
licenciatura se compone de seis ejes bien definidos: Competencias
genéricas, Historiográfico, Histórico, Metodológico, Acentuaciones y
Optativas. 
%\newpage

\medskip
\textbf{Referencias}

Acuña O., V.H. (2009), Estudiar Historia. Manuscrito no publicado.
Universidad de Costa Rica.

ANUIES (2000), Plan Maestro de Educación Superior abierta y a distancia.
México.

ANUIES (2012), Inclusión con responsabilidad social. Una nueva generación de
políticas de educación superior, México, ANUIES.

Barros  Carlos (1995), <<La contribución de los terceros Annales y la
historia de las mentalidades>>. 1969--1989,  en Iztapalapa, n. 36, México,
enero-junio, pp.\ 73--102.
%\newpage

Burke, Peter (2001), \textit{Formas de hacer historia}, Madrid, Alianza
Editorial.

Cabra Torres, Fabiola (2008), <<La evaluación y enfoque de competencias:
Tensiones, limitaciones y  oportunidades para la innovación docente en la
universidad>>. Revista Escuela de administración de negocios, Núm. 63,
mayo-agosto, pp. 91--105, Universidad EAN, Colombia.

Corrales Burgueño, Víctor Antonio (2009), \textit{Plan de Desarrollo
Institucional Visión 2013}, Culiacán, Sinaloa, Universidad Autónoma de
Sinaloa.

Fukuyama  Francis (1992).  \textit{El fin de la historia y el último hombre}, Barcelona, Editorial Planeta.

Iggers  Georg (1995),  \textit{La ciencia histórica en el siglo XX. Las tendencias
actuales}, Barcelona, Labor.

Noiriel, Gérard (1997), \textit{Sobre la crisis de la historia}, Madrid,
Frónesis-Cátedra-Universitat de València.

\begin{sloppypar}
Viales H., R. (2010), <<Mitos, corrientes y reflexiones. El oficio del
historiador en la Costa Rica del siglo XXI>>. Reflexiones, Segunda Época. En
\url{http://reflexiones.fcs.ucr.ac.cr/images/edicion_78_99/mitos.pdf}. Consultado el 29
de enero del 2010. 
\end{sloppypar}

White, H. (1992), \textit{El contenido de la forma. Narrativa,
discurso y representación histórica}, Buenos Aires, Paidós.

%White, Hayden (1992), \textit{El contenido de
%la forma. Narrativa, discurso y representación histórica}, Buenos Aires,
%Paidós.

Worster, D. (2000), <<Haciendo Historia Ambiental>>. En Castro, Guillermo
(selección, traducción y presentación), \textit{Transformaciones de la Tierra. Una
antología mínima de Donald Worster}, Panamá.