%\documentclass{article}
%\usepackage{amsmath,amssymb,amsfonts}
%\usepackage{fontspec}
%\usepackage{xunicode}
%\usepackage{xltxtra}
%\usepackage{polyglossia}
%\setdefaultlanguage{spanish}
%\usepackage{color}
%\usepackage{array}
%\usepackage{hhline}
%\usepackage{hyperref}
%\hypersetup{colorlinks=true, linkcolor=blue, citecolor=blue, filecolor=blue, urlcolor=blue}
%% Text styles
%\newcommand\textstyleFootnoteSymbol[1]{\textsuperscript{#1}}
%\newtheorem{theorem}{Theorem}
%\title{}
%\author{Saúl Jerónimo}
%\date{2014-09-04}
%\begin{document}
%\clearpage\setcounter{page}{1}\section[La teoría de la historia, la
%historiografía y la enseñanza tradicional ]{  La teoría de la
%historia, la historiografía y la enseñanza tradicional }
{\centering La historiografía como eje articulador de la enseñanza de la
historia.La historiografía como eje articulador de la
enseñanza de la historia}

{\raggedleft
Dr. Saúl Jerónimo Romero
\par}
{\raggedleft
Departamento de Humanidades 
\par}


\bigskip
Hace tres años en la VI reunión de RENALIHCA realizada en la UABC 
presenté un trabajo relativo a la formación del historiador en el siglo 
XXI. En dicha conferencia sostenía que a pesar de los avances en la 
teoría de la historia, el giro lingüístico, el giro espacial y ahora el 
giro cultural, en México se seguía enseñando la disciplina histórica de 
manera muy similar como lo harían los historiadores del siglo XIX o en 
última instancia como a principios del siglo XX. También mencioné que 
había un fuerte divorcio entre la práctica histórica y su teorización, 
lo que limitaba la reflexión historiográfica y daba por resultado la 
realización de trabajos de investigación descriptivos, de recorta y 
pega. En ese diagnóstico suponía que el problema radicaba en una 
enseñanza memorística y repetitiva; basada en técnicas y no en 
reflexiones teóricas sobre el saber histórico. Este tipo de enseñanza, 
decía en aquel momento, provoca serios problemas, entre otros, que 
nuestros estudiantes sean poco competitivos comparados con otras 
disciplinas y con historiadores de otras latitudes, que están 
preparados con un enfoque multi y transdisciplinario, lo que les 
permite abordar problemas complejos, tanto del pasado como del 
presente; los prepara con mayor éxito para el ingreso a los posgrados y 
los proveen de herramientas para ser investigadores y no solo 
compiladores de información, lo que en principio los orienta hacia uno 
de los campos de trabajo más codiciados, la investigación y por el otro 
lado, los prepara para ser mejores docentes en la medida en que no 
están repitiendo libros de texto, sino que constantemente están 
investigando sobre las materias que imparten, etcétera.

Al tratar de ubicar en dónde reside el problema, he planteado la 
siguiente hipótesis de trabajo: A pesar de que todos los programas de 
estudio de las licenciaturas en historia se han modificado en los años 
recientes, la mayoría de estas modificaciones se han realizado 
obedeciendo criterios exógenos a la disciplina y a las discusiones 
teóricas en torno a las ciencias sociales, y salvo excepciones el eje 
teórico que debería articular el mapa curricular. Esta desarticulación 
provoca que no haya correspondencia entre objetivos, perfil de egreso y 
el mapa curricular. 

En las siguientes líneas se abunda en esta discusión. A partir del 
análisis de treinta y dos planes y programas se exponen algunas de las 
transformaciones relevantes, se señalan algunas limitaciones y 
finalmente, se hace una propuesta sintética de algunos contenidos 
mínimos. 

\textbf{Las evaluaciones}

Los criterios de evaluación que imponen las agencias evaluadoras y 
acreditadoras son mecanismos de medición estándar, aplicable para 
cualquier carrera que se imparte en el nivel superior, por lo que no se 
ocupan de plantear mecanismo de evaluación de calidad. Por ejemplo, las 
estrategias de CONACyT es impulsar la vocación de investigación entre 
los jóvenes del bachillerato y fortalecer los doctorados, por lo tanto 
las licenciaturas y las maestrías no son prioritarias, por lo que los 
recursos escasean para las licenciaturas y maestrías. Además priorizan 
la eficiencia terminal sobre cualquier otro indicador de calidad. Estas 
instancias están más preocupadas por resultados numéricos como: índices 
de reprobación, uso intensivo o no de las tecnologías de la 
comunicación,  eficiencia de egreso, eficiencia terminal, 
diversificación de las formas de titulación, mecanismos de evaluación 
para profesores y alumnos; número de profesores de tiempo completo, 
cuántos de ellos pertenecen al SNI, cuántos al PROMEP, etcétera; si las 
aulas tienen o no suficiente equipo, cuántos libros hay en la 
biblioteca, etcétera. Las instituciones han procurado cumplir con estos 
criterios porque son políticas nacionales vinculadas a financiamiento. 
Así que esos indicadores numéricos hay que cumplirlos.

Celebro que estas políticas no se hayan ocupado de la orientación 
académica de los planes y programas, pues creo que sería un enorme 
desastre que desde alguna oficina gubernamental se decidiera la vida 
académica de las universidades. También me parece bien que estos 
mecanismos de evaluación hayan servido para que en las universidades 
tengan más y mejor acceso a internet, para que haya más computadoras y 
acceso a bibliotecas físicas y virtuales; para que se impartan cursos 
de aprender a aprender, de escritura de géneros universitarios, de 
diversos tipos de software, etcétera. No me alegra tanto, que en la 
búsqueda de mejor eficiencia terminal se titule a los alumnos por 
promedio, pues a nuestros mejores alumnos les evitamos realizar una 
pequeña investigación, que es una de la labores relevantes del quehacer 
del historiador. En algunos casos se ha procurado la aprobación de 
seminarios que más o menos cumplen esa función.  No obstante estos 
apoyos, considero que en muchos casos no se ha logrado proponer un 
currículo que permita tener egresados con criterio, que entiendan las 
dimensiones históricas de tiempo y espacio, que estén ciertos de 
necesidad de interpretar y del papel que juega el lenguaje en las 
construcciones historiográficas. 

Incluso intentos como el proyecto Tunning Latinoamérica que ha tratado 
de ofrecer una guía de lo que debían contener planes y programas en 
materia de capacidades, habilidades y conocimientos básicos para cada 
una de las carreras no se ocuparon del diseño del mapa curricular. Sus 
recomendaciones son tan genéricas, que se prestan a confusión o llegan 
a ser poco prácticas. En un trabajo anterior hice una crítica a esa 
propuesta, en particular lo relativo al espacio, donde quedaba claro 
que la noción de espacio propuesta no se enfocaba de manera histórica, 
si no como algo ya dado, ahistórico, y cargado de ideología 
nacionalista, regionalista o estatal.

\textbf{  Objetivos declarados:}

Los planes y programas de historia en su mayoría tratan de formar
profesionales dedicados a la investigación y problematización de asuntos
históricos, en algunos casos con una visión crítica, en otros desde una
perspectiva racional; otros más hacen hincapié en recuperar la función que
juega la historia en vincular presente, pasado y futuro y en otros, se
enfatiza en las funciones docentes, de difusión y preservación del
patrimonio histórico cultural\footnote{Véase los programas de BUAP, UNAM,
Aguascalientes, etcétera. El de la Universidad Juárez de Tabasco dice lo
siguiente: Formar profesionales que sean capaces de interpretar
científicamente los procesos históricos realizando estudio de las diversas
dimensiones del hombre y la sociedad a través del tiempo, para revalorar el
presente y participar en la transformación del futuro. En Licenciatura en
historia, Universidad Juárez de Tabasco, en 
\url{http://www.ujat.mx/interioradentro.aspx?ID=157&NODO=78} }. 

En todos ellos se puede apreciar la intención de formar profesionales 
que puedan manejar información con fines investigativos y el manejo de 
resultados varía de acuerdo a las circunstancias sociales en las que 
opera cada programa. En todos ellos se hace mención en la necesidad de 
formar a los futuros historiadores con los fundamentos teóricos y 
metodológicos necesarios para llevar a cabo estas tareas. Estas 
pretensiones me parecen fundamentales del para el quehacer histórico; 
sin embargo considero que todavía estamos distantes de lograr la 
consecución de estos objetivos con eficiencia. Vale la pena destacar 
que en cualquiera de las diversas ocupaciones que tienen los egresados 
en todas se requiere, saber allegarse conocimiento y tener claridad 
sobre la responsabilidad sobre lo que se está publicando, difundiendo o 
enseñando. Así entonces, dos elementos son básicos en la formación del 
historiador saber investigar y tener claridad sobre su responsabilidad 
sobre su producción\footnote{ En el Plan de estudios de la UAEM se dice 
al respecto: Como características colaterales ser capaz de investigar, 
insertarse en el paradigma del aprendizaje con el manejo de recursos 
didácticos y técnicos innovadores, así opios procedimientos. Esta 
capacidad le permite no sólo apropiarse permanentemente de nuevos 
conocimientos sino también de ser capaz de transmitir en su actividad 
como docente, esta misma capacidad. En Plan de Estudios de la UAEM 
2004, p. 2}

A propósito, recupero una artículo de Hayden White publicado en el 
número de Nexos de mayo de 1982, nótese que es un texto de hace 32 
años, en el que provocativamente decía de la profesión histórica no era 
ni ciencia ni arte, pero que sin embargo los historiadores desde el 
siglo XIX habían reclamado los privilegios del artista y el científico, 
pero al mismo tiempo se negaban a someterse a los rigores críticos y 
creativos del arte y la ciencia.\footnote{ Hayden White, “El peso de la 
historia en Nexos”, Mayo de 1982, consulta en línea en www.nexos.com} 
Explica que ello se debe a dos causas: 

Una es la naturaleza de la profesión histórica misma, quizá la 
disciplina conservadora  par excellence. Desde mediados del siglo XIX 
los historiadores acostumbran fingir una especie de ingenuidad 
metodológica voluntaria, que originalmente servía a un buen propósito: 
protegerlos de la tendencia a abrazar los sistemas explicativos 
monistas, el idealismo militante en la filosofía o el positivismo 
militante en la ciencia.\footnote{ Hayden White, “El peso de la 
historia en Nexos”, Mayo de 1982, consulta en línea en www.nexos.com}

White después retoma la crítica que se le hizo a la disciplina 
histórica a principios del siglo XX; la cual, le acusaba de ser 
acomodaticia y políticamente servil. Me parece que esa crítica ya no es 
aplicable a todos los tipos y formas de hacer historia que se hacen 
desde el último tercio del siglo XX y lo que va de este siglo. Sobre 
todo, porque los emisores de discurso histórico se han multiplicado y 
las versiones oficiales de la historia, son unas más entre muchas 
otras. Basta considerar desde dónde y con qué fines se investiga, por 
mencionar algunos ejemplos: universidades públicas y privadas, 
estatales y nacionales; centros de investigación privados y públicos; 
organizaciones no gubernamentales y asociaciones diversas; centros 
culturales e incluso religiosos. Además, los fines de estas historias 
varían en función de los intereses sociales desde los que se 
construyen, quisiera poner como ejemplo las investigaciones que tienen 
que ver con justicia transicional, que evidentemente no reproducen la 
historia desde el punto de vista gubernamental. 

En lo que si estoy de acuerdo con White es la idea de que:

La historia tiene hoy la oportunidad de allegarse las nuevas 
perspectivas del mundo que ofrecen una ciencia y un arte igualmente 
dinámicos. La ciencia y el arte han trascendido los conceptos 
tradicionales que les exigían ser copias literales de una realidad 
presumiblemente estática; y ambas han descubierto el carácter 
esencialmente provisional de las construcciones metafóricas destinadas 
a comprender un universo dinámico. Así confirman implícitamente la 
verdad a la que llegó Camus cuando escribió “Antes fue cuestión de 
saber si la vida tenía o no un significado para vivirla. Ahora está 
claro, al contrario, de que será vivida mejor si no tiene sentido”. 
Podemos corregir la afirmación para leer: será vivida mejor si tiene 
muchos sentidos en lugar de un solo sentido.\footnote{Ibidem}


\bigskip 
Si este es el tipo de historiador necesario en el siglo XXI, 
vale la pena reflexionar sobre cuáles serían los contenidos necesarios 
para que los historiadores tuvieran esas posibilidades de innovación y 
de responsabilidad sobre lo que investigan, publican y enseñan. 

\textbf{Los Planes de Estudio}

Ante este panorama, creo que es esencial que el futuro historiador 
curse un eje teórico sólido, con materias teóricas, tales como: 
nociones básicas del conocimiento historiográfico, teoría de la 
historia, epistemología histórica, historiografía general, etcétera. De 
tal suerte que pueda reconocer: cuál es su función como productor de un 
tipo de conocimiento; cuál es la naturaleza del conocimiento que 
produce el historiador a lo largo de su carrera profesional; el papel 
que juega el lenguaje en la construcción del conocimiento humano y en 
particular del histórico; las diferencias entre filosofía de la 
historia y teoría de la historia; las posibilidades de conocer el 
pasado; la historicidad de la práctica histórica y de su teoría. 
Asimismo, el reconocimiento de los conceptos históricos claves como 
tiempo, espacio, discurso. Así como el conjunto elecciones sobre las 
que se estructura el discurso histórico y la responsabilidad ética y 
profesional que conlleva su práctica.

¿Todo esto cuándo? Desde el primer día en la universidad y a lo largo 
de toda la carrera. Este tipo de materias le servirá para entender y 
analizar la críticamente la formación que están recibiendo; así podrán 
valorar adecuadamente historiografía que utilizaran durante su proceso 
formativo y tendrán consciencia de que el objetivo central de su 
formación es generar conocimiento histórico y divulgarlo y de ninguna 
manera obtener un conocimiento enciclopédico. Es decir que no están 
aprendiendo todo lo que ha ocurrido a la humanidad desde los griegos 
hasta nuestros días y que en su momento a ellos les tocará hacer 
selecciones similares. Proponer problemas de investigación de su 
localidad, de México o del mundo global y estar preparados para 
seleccionar las fuentes y métodos para resolverlos.  

Al momento de diseñar el plan de estudios los profesores encargados 
siempre encuentran con el dilema, de qué incluir y qué dejar fuera. Es 
decir, que también el currículum es una selección que hacen un grupo de 
profesionistas y expertos. Por ejemplo si necesitamos que el alumnos 
tengan una ubicación, en términos histórico espaciales, de la dimensión 
local hasta los grandes asuntos mundiales, que le den un bagaje 
histórico-cultural suficiente para entender sus objetos de 
investigación. Nos vamos a encontrar con la imposibilidad de incluir 
todo ese conocimiento en el currículo de los planes de estudio, ya que 
en cuatro los alumnos tienen que familiarizarse con la historia mundial 
o universal, que cabe mencionar siempre es un problema delimitar lo que 
es historia mundial o universal, sin caer en europeocentrismo o en 
algún otro ismo.\footnote{ Por ejemplo el perfil de egreso de la 
Universidad de Ciencias y Artes de Chiapas es el siguiente: - 
Comprenderá los procesos sociales, políticos, culturales, geográficos y 
económicos que han caracterizado a la historia mundial, americana y 
nacional, desde la prehistoria hasta nuestros días. \par \ \ - Conocerá 
la historia de Chiapas y sus regiones, desde la época prehispánica 
hasta el pasado más reciente, y en el contexto de la historia nacional 
y mundial. \par \ \ - Adquirirá conocimientos acerca de las 
características y métodos de las principales tradiciones 
historiográficas a nivel mundial, y en particular de la historiografía 
de México y de Chiapas. \par \ \ - Contará con un amplio conocimiento 
de los planteamientos teóricos y metodológicos de las ciencias 
sociales. \par \ \ - Obtendrá conocimientos teóricos-prácticos para 
realizar actividades de investigación, docencia y difusión del 
conocimiento histórico. En Folleto informativo, consultado en 
\url{http://www.unicach.mx/_/descargar/licenciatura/historia.pdf} } 
Siempre resulta complicado determinar qué tanto se incluye de los 
procesos que fueron transformando el espacio en: la época de los 
griegos y romanos, en la edad media, de los ocurridos en México y en 
los Estados de la república y en las localidades. Si se intentara, el 
plan de estudios quedaría como un rompecabezas con piezas de diferente 
escala que no encajan entre sí. Creo que sería más fácil una materia en 
dónde se explique la noción de espacio histórico, que permita al alumno 
comprender las diversas formas en que se divide el espacio, de acuerdo 
a los sistemas políticos, interacciones sociales, relaciones 
comerciales, interacciones culturales, etcétera y que para 
comprenderlas y delimitarlas es necesarios tomar en cuenta su 
historicidad. Es decir, un planteamiento más teórico que descriptivo y 
por supuesto en la enseñanza se puede hacer uso de todos los ejemplos 
que se quiera. 

Esto aplica para todas las áreas de conocimiento que se quieran 
abordar: por ejemplo para las historiografías, las relativas a la 
historia de la historiografía, la manera tradicional de enseñarlas es 
cronológicamente desde los griegos hasta lo más cercano a nuestros 
días, muy pocas son las facultades que cuando menos han cambiado a 
enseñarla a través de corrientes historiográficas o través de las 
metodologías de cada tipo de historiografía.\footnote{ Mediante 
corrientes lo hacen: la ENAH y la Universidad de Sinaloa y a través de 
metodologías enseñan, la Universidad de Aguascalientes y la del Estado 
de México. Todas las demás siguen el método tradicional. }  La mayoría 
sigue siendo en estricto orden temporal y por autores. Ninguna ha 
cambiado a enseñarla por problemas, tales como sujeto, discurso, 
géneros discursivos, memoria, procesos de significación, identidades 
históricas, usos de la historia, etcétera. Por ejemplo La Benemérita 
Universidad Autónoma de Puebla tiene como primer párrafo de su perfil 
de egreso: “El historiador egresado estará comprometido con la 
preservación de la memoria histórica y recuperará la historia de su 
espacio regional, nacional y universal, manteniendo una visión crítica 
del presente vinculándolo con el pasado.”\footnote{ El BUAP, 
Licenciatura en Historia, en 
\url{http://cmas.siu.buap.mx/portal_pprd/wb/EDUCATIVA/licenciatura_en_historia_1} 
} Y no hay en su mapa curricular ninguna materia en la que directamente 
se trate la problemática de la memoria histórica. Quizá en las 
relativas en Área de Sistematización de Documentos y Patrimonio 
Históricos, pero la comprensión de los fenómenos asociados a Memoria 
histórica no son exclusivos de los referidos a patrimonio. 

Un enfoque por grandes problemáticas historiográficas permitiría exponer
estos problemas de manera comparativa, alentaría la curiosidad por
descubrir cómo se han tratado estos problemas en diferentes momentos,
culturas o situaciones. Es decir que estas materias dejarían de ser
informativas para ser un elemento más de formación, de tal suerte que los
alumnos de historia tendrían un contacto permanente con el pensamiento
histórico y su problematización. 

De otra forma, la información que reciben los alumnos es sumamente 
parcial y parece no tener otra justificación esa selección de hechos y 
acontecimientos que se les enseñan, más que la costumbre. Donde parece 
que lo central, por ejemplo, es que lo alumnos recuerden que Tucídides 
escribió Guerra del Peloponeso y no la discusión, sobre evidencia, 
testigo, crítica etcétera que abrió su texto, preocupaciones que serán 
importantes entre los historiadores de los siglos XIX y XX. Si no se 
quiere cambiar el modelo de enseñanza por problemas, estas materias 
deberían de incluir discusiones sobre horizonte, cultural, enunciación, 
recepción, análisis de textos, etcétera. La historiografía llamada 
clásica, griega y romana se presta adecuadamente para estos debates y 
muestra con claridad que no hay una sucesión evolutiva de los modos, 
formas y métodos de hacer historia. Hay momentos que se retoman 
problemáticas que se discuten con perspectivas distintas, propias del 
momento de enunciación o recepción del contexto del momento. Un enfoque 
de esta naturaleza integraría los elementos teóricos con las diversas 
prácticas historiográficas. 

Al revisar los planes y programas de estudio encuentro que las universidades
la Universidad de Coahuila, Morelos, de San Nicolás de Hidalgo, Baja
California, Baja California Sur, FES Acatlán, Facultad de Filosofía y
Letras, UNAM, y la ENAH han incluido cursos tales como: introducción a la
historia, introducción a la teoría de la historia, al pensamiento
histórico, a la investigación histórica, la interdisciplina, lo cual me
parece un avance significativo, aunque en algunos casos habría que ver el
contenido exacto de las materias de introducción, pues algunos de estos
cursos se ocupan de temas tales como introducción al uso de archivos y
fuentes documentales, lo cual tiene su importancias pero no está enfocado
en el mismo sentido que esta propuesta. 

Sin embargo en la mayoría de los planes no hay continuidad de esta 
primera introducción al ámbito teórico del quehacer historiográfico; en 
casi todos los planes incluyen dos cursos relativos a Teoría de la 
historia, salvo en los de las universidades Iberoamericana y de San 
Luis Potosí que contienen tres cursos. La mayoría de las instituciones 
los imparten entre el segundo y quinto semestre. En otras instituciones 
no hay una propuesta sistemática de estos contenidos. Pongo algunos 
ejemplos: en Universidad de Tamaulipas que se cursan esas materias en 
el cuarto año; la Universidad de Nuevo León solo ofrece un curso de 
teoría de la historia en quinto semestre y en sexto semestre se ofrece 
como optativa la materia de Filosofía de la historia; la Universidad de 
Sinaloa solo brinda en el primer semestre Introducción a la teoría de 
la historia y después ya no hay materias teóricas sino únicamente 
historiografías  específicas. La Universidad Juárez Autónoma de Tabasco 
es obligatorio cursar en tercer semestre Filosofía y teoría de la 
historia y después no hay más materias teóricas. En la UAM Iztapalapa 
se imparten en el eje teórico tres materias denominadas, Teoría y 
Problemas Sociopolíticos Contemporáneos, nada que ver con la propuesta 
teórica de la historiografía. En la universidad de Guadalajara el curso 
de Teoría de la Historia es optativo. En fin no es el caso, agotar 
todos los ejemplos; pero sí mostrar, que en el currículo de la mayoría 
de las licenciaturas este no es un eje central, o tiene poca 
integración con el resto el de las materias que cursan. 

Otro asunto que me parece relevante comentar es, que algunas 
instituciones han optado por enseñar una forma específica de hacer 
historia, ya sea por corriente historiográfica, autor, o pensamiento 
político. En el plan de estudios de la Universidad Juárez de Tabasco se 
reconoce:

Aunque en el caso de las Ciencias Sociales es difícil mantener una 
posición de neutralidad, ya que los que se ocupan de estas ciencias se 
inclinan por alguna corriente en particular ya sea por su formación y 
posición de clase, resulta peligroso establecer en una estructura 
curricular una determinada orientación hacia alguna de las diversas 
corrientes de la historia, ya que no todos coinciden en esa determinada 
corriente o en muchos casos pueden diferir completamente, por lo cual 
es adecuado plantear una posición abierta e 
interdisciplinaria.\footnote{ Universidad Juárez Autónoma de Tabasco 
División Académica de Ciencias Sociales y Humanidades 
\textit{Reestructuración del Plan de Estudios de la Licenciatura en 
Historia}. En 
\url{http://www.archivos.ujat.mx/2014/dacsyh/plan_estudios/LIC.\%20EN\%20\%20HISTORIA.pdf} 
}

A pesar de este reconocimiento de la importancia de ofrecer diversas 
opciones, solo se ofrecen al final un taller de historia oral y una 
materia específica de historia de género. Seguramente, es un problema 
relacionado con la planta docente. En otros momento se priorizo el 
materialismo histórico. La BUAP por ejemplo, menciona que desde la 
creación de la licenciatura ha habido siete planes de estudio, con 
orientaciones distintas, una tradicional, otra el materialismo 
histórico y la actual que sigue a la Escuela de los Annales.\footnote{ 
BUAP, Facultad de Filosofía y Letras Programa de de Licenciatura en 
Historia. Informe de autoevaluación para la acreditación del Programa 
Educativo de la Licenciatura en Historia de la Facultad de Filosofía y 
Letras de la  Benemérita Universidad Autónoma de Puebla  En 
\url{http://historia.dosmildiez.net/revisionplan/wp-content/uploads/2007/10/acreditacion_colegio-de-historia-buap.pdf} 
, pp.28-29} Evidentemente desde la aparición de los Annales en 1929 a 
la fecha ha corrido mucha tinta y muy diversas formas de hacer 
historia, pero a pesar de ello no es la única corriente. (¿Y si un 
alumno quiere hacer historia intelectual?). Quizá con los elementos 
teóricos adecuados los alumnos podrían escoger problemas de 
investigación y luego elegir la conceptualización teórica que les 
permita comprender ese fenómeno, en donde no solo las corrientes de 
pensamiento histórico le brinden las herramientas para comprender esos 
fenómenos, sino los de las ciencias sociales en su conjunto. Esto le 
permitiría elegir el método adecuado a su proyecto de investigación y 
buscar allegarse los conocimientos necesarios para desempeñar su 
actividad. 

Estoy consciente que las plantas académicas a veces están formadas en la
misma escuela histórica, son seguidoras de uno o varias figuras señeras,
todo eso está bien, pero sí creo que debe haber la apertura y la
disposición para apoyar a los alumnos en sus indagaciones y apoyarlos
metodológicamente, en estructura investigativa y no condicionarlos a seguir
determinada corriente histórica. 

No es mi intención agobiarlos con esta descripción de planes y programas y
tampoco que pensemos que le historiador se tiene que dedicar a la teoría o
filosofía de la historia y retomo aquí las palabras de Luis Villoro, quien
en 1960 escribió un artículo en Historia Mexicana y llegaba a la siguiente
conclusión: 

Creemos que los historiadores americanos necesitan plantearse con mayor 
gravedad el problema del  objeto y métodos de su ciencia. Con ello no 
pedimos que hagan filosofía. Quien tal pensara sólo demostraría tener 
una pobre idea del historiador, al reducirlo al papel de simple técnico 
o ingenuo narrador. Al historiador compete reflexionar sobre los 
fundamentos y fines humanos de su ciencia. Sólo él puede formular 
nuevas hipótesis de trabajo y aplicarlas en procedimientos completos; 
mientras no haga esto, todas teorías filosóficas acerca de la historia 
serán vacías especulaciones. Por eso, las grandes reformas de la 
historiografía un nunca fueron resultado de los filósofos de la 
historia en cuanto tales, sino de los mismos historiadores: Solo si el 
historiador cobra cabal conciencia de la especificidad de su objeto y 
redescubre en él la vida creadora del hombre en toda su riqueza, sólo 
si se percata de la dignidad de su función humana, podrá recuperar el 
papel de director en la sociedad que antaño le correspondiera. 
\footnote{ Luis Villoro. “La tarea del historiador desde la perspectiva 
mexicana”, en \textit{Historia Mexicana}, vol. IX, núm. 3, enero-marzo 
de 1960, pp. 329-339, El Colegio de México. Compilado en Evelia Trejo. 
\textit{La historiografía del siglo XX en México. Recuento, 
perspectivas teóricas y reflexiones}. México, UNAM, p. 290}

Este enfoque parece ir ganando adeptos en las licenciaturas de San Luis
Potosí, la UNAM y la ENAH y en algunas otras instituciones, ya tenemos
cuando menos cinco materias que apuntan a comprender los ejes teóricos pero
considero que todavía no es suficiente. Las propuestas contemporáneas de
trabajo historiográfico requieren: una cultura histórica, que se puede
formar únicamente si hay una cabal comprensión de la naturaleza del aporte
que hace la historiografía al conocimiento historiográfico. Este enfoque se
reconoce por ejemplo en la presentación del plan de estudios de la Escuela
Nacional de Antropología: 

Las grandes transformaciones experimentadas por las sociedades en el siglo
XX han tenido como consecuencia la ampliación del horizonte intelectual de
la historia: la disciplina ha incorporado nuevos enfoques, temáticas y
metodologías. Este cambio se ha plasmado en el terreno de la teoría, pues
los cambios sociales y culturales del último siglo dieron paso a la
comprensión de los fenómenos sociales con una perspectiva más crítica, a la
vez que se sostiene la necesidad del conocimiento riguroso del acontecer
histórico. \footnote{ }

\textbf{Propuesta:}

Analizar la pertinencia de integrar un eje teórico a lo largo de toda la
licenciatura

Este eje debe comprender desde cursos introductorios, epistemología
histórica, filosofía de la historia, teoría de la historia y ética 

Las materias de historiografía deberían estar integradas por problemas tanto
las generales, como las nacionales y locales. 

El alumno tiene derecho a elegir el tipo de problemas que desea investigar,
los métodos que necesita utilizar y es responsabilidad de las plantas
académicas y de la institución proporcionarle las condiciones adecuadas
para el aprendizaje de su profesión.

Debe haber cursos de actualización para profesores para que puedan impartir
estos cursos. 

Gracias. 

%\end{document}