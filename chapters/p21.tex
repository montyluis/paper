%\documentclass{article}
%\usepackage{amsmath,amssymb,amsfonts}
%\usepackage{fontspec}
%\usepackage{xunicode}
%\usepackage{xltxtra}
%\usepackage{polyglossia}
%\setdefaultlanguage{spanish}
%\usepackage{color}
%\usepackage{array}
%\usepackage{hhline}
%\usepackage{hyperref}
%\hypersetup{colorlinks=true, linkcolor=blue, citecolor=blue, filecolor=blue, urlcolor=blue}
%\usepackage{graphicx}
%% Text styles
%\newcommand\textstylehps[1]{#1}
%\newcommand\textstylefootnotereference[1]{\textsuperscript{#1}}
%\newcommand\url[1]{\textcolor{blue}{#1}}
%\newcommand\textstyleappleconvertedspace[1]{#1}
%\newtheorem{theorem}{Theorem}
%\title{Calidad educativa en el nivel superior y reforma universitaria.}
%\author{}
%\date{2014-09-07}
%\begin{document}
\markboth{}{}
\thispagestyle{empty}
\phantom{abc}
\phantomsection{}
\addcontentsline{toc}{chapter}{PARTE IV.\ ACREDITACIÓN, CERTIFICACIÓN\newline E INNOVACIÓN EN LOS PROGRAMAS\newline EDUCATIVOS DE HISTORIA} 

\vspace{0.35\textheight}
{\centering \bfseries Parte IV\par}
{\centering \bfseries ACREDITACIÓN, CERTIFICACIÓN E INNOVACIÓN EN LOS 
PROGRAMAS EDUCATIVOS DE HISTORIA \par}
\markboth{}{}
\thispagestyle{empty}	
\cleardoublepage{}

%\clearpage\setcounter{page}{467}
\thispagestyle{empty}
\phantomsection{}
\addcontentsline{toc}{chapter}{Evaluación de PE: estrategia de gestión\newline
institucional para mejorar la calidad de la\newline educación\newline $\diamond$
\normalfont\textit{Alfonso Mercado Gómez\newline y María de los Ángeles Sitlalit García Murillo}}
{\centering {\scshape \large Evaluación de PE: estrategia de gestión
institucional para mejorar la calidad de la educación}\par}
\markboth{la formación del historiador}{dimensión didáctica}
\setcounter{footnote}{0}

\bigskip
\begin{center}
{\bfseries Alfonso Mercado Gómez\\
María de los Ángeles Sitlalit García Murillo}\\
{\itshape Universidad Autónoma de Sinaloa}
\end{center}

\epigraph{No cabe duda de que el término
«evaluación» es hoy moneda de uso común en cualquier discurso educativo.\\
Con una u otra acepción, asociada a una diversidad de prácticas e
impulsada por distintas estrategias políticas,
la evaluación suscita un creciente interés en los sistemas
educativos contemporáneos.}{(Tiana 1996)}


\bigskip
\textbf{Resumen}

La evaluación y la acreditación en México han sido planteadas como un mecanismo 
para fomentar la calidad de la educación superior. A nivel internacional, la 
evaluación y la acreditación son procesos reconocidos como medios idóneos para el
mejoramiento de los sistemas de educación superior.

Muchos de los problemas en la educación superior que dieron origen a la
evaluación, no serán solucionados  únicamente por esta vía. Evaluar
solamente tiene sentido si existen las condiciones para consolidar o
remediar los problemas principales, la evaluación debe estar acompañada de
políticas creativas para mejorar aspectos como la eficiencia terminal, la
preparación de profesores, los sistemas administrativos, o la producción
científica. Estos son aspectos que requieren políticas y fondos a largo
plazo, que muchas veces ameritan continuidad aún cuando la evaluación no
demuestre mejoramientos a corto plazo.\\ 
\textbf{Palabras clave:} evaluación, acreditación, calidad,\\ educación
superior.


\bigskip
\textbf{Abstract}

PE Assessment: Institutional management strategy to
improve the quality of education.

The assessment
and
accreditation in
Mexico
have been
raised
as a
mechanism to
enhance
the quality of
higher
education.
Internationally,
assessment and
accreditation are
recognized
as
suitable for
the
improvement of
higher
education systems
means
processes.

Many of the
problems
in higher
education
that gave
rise to
the
assessment not
  only 
  be
solved 
  this
way. 
  Evaluate 
  only 
  makes
sense if 
  there are
conditions to 
  consolidate 
  or
remedy 
  the major
  problems ,
the assessment must
  be
accompanied by 
  creative
policies 
  to
improve 
  terminal
efficiency 
  aspects ,
preparation of
  teachers, 
  administrative, 
  or 
  scientific
production. 
  These
are 
  issues
that require 
  policies
and 
  long-term
funds , which often
  warrant 
  continuity 
  even
if 
  the
assessment does not 
  demonstrate 
  short-term 
  improvements.\\
\textbf{Keywords:} assessment,
accreditation,
quality 
higher
education.
\newpage

\textbf{A manera de introducción}

La evaluación y la
acreditación en México han sido planteadas como un mecanismo para fomentar
la calidad de la educación superior. A nivel internacional, la evaluación y
la acreditación son procesos reconocidos como medios idóneos para el
mejoramiento de los sistemas de educación superior. Además, los
planteamientos en esta materia se han venido haciendo con el interés
creciente de que éstos puedan responder a las circunstancias históricas,
sociales y educativas de nuestro país.

\begin{sloppypar} 
La acreditación, en su connotación institucional e individual, implica una búsqueda de
reconocimiento social y de prestigio por parte de los individuos que
transitan por las instituciones educativas. En ese sentido, los procesos de
acreditación se han constituido en un requerimiento en nuestros días, ya
que están destinados a garantizar calidad y proporcionar credibilidad
respecto a un proceso educativo y sus resultados (Pallán 1995, p. 12).
\end{sloppypar}

En la medida en que la
acreditación institucional y especializada representa un mecanismo para
orientar las tareas educativas de la formación profesional, de acuerdo con
prácticas y resultados ampliamente reconocidos, nacional e
internacionalmente, se convierte en un medio indispensable para el
mejoramiento general en la calidad de los sistemas de educación superior.
De ahí que la \textit{acreditación
tenga un papel estratégico dentro de las políticas
educativas orientadas a promover cambios relevantes en la organización, eficiencia y
eficacia de los sistemas de educación
superior}.
\newpage

\textbf{Desarrollo}

Los procesos de
acreditación, evaluación y calidad están relacionados entre sí, y resulta
muy difícil considerarlas separadamente. La acreditación se realiza
conforme a un proceso de evaluación y de seguimiento, con el fin de
disponer de información fidedigna y objetiva sobre la calidad relativa a
instituciones y programas universitarios, sea que estén en su fase de
reconocimiento inicial o en pleno desarrollo de su proyecto
institucional.

\enlargethispage{1\baselineskip}
Algunas reflexiones
respecto a los procesos de la evaluación y la acreditación inducen a
considerarlos no como un fin en sí mismos, sino medios para promover el
mejoramiento de la educación superior. En la actualidad ha resultado usual
asociar ambas actividades con el mejoramiento de la calidad, la generación
de información para la toma de decisiones, la garantía pública de la
calidad de las instituciones y de los programas que ofrece.

La difusión de los
resultados de la evaluación contribuye a que los diversos sectores
interesados en la educación adquieran un criterio sobre la calidad de tales
desempeños y programas.

 
 Si bien la acreditación y
la evaluación guardan estrecha relación, son, a la vez, procesos
diferenciables y complementarios. En el caso de México (ANUIES 1984),
\textit{la
evaluación}  ha sido
definida como un proceso ---continuo, integral y participativo--- que permite
identificar una determinada circunstancia educativa, analizarla y
explicarla mediante información relevante. Con ésta se busca el
mejoramiento de la institución, programa o individuo evaluado,
 \textit{constituyéndose en la base para la acción del mejoramiento
correspondiente}.

 
 Mientras que \textit{la acreditación} se trata
de un procedimiento cuyo objetivo es comparar el grado de acercamiento del
objeto analizado con un conjunto de normas previamente definidas e
implantadas como deseables. Al mismo tiempo, la acreditación implica el
reconocimiento público de que una institución o un programa satisfacen
determinados criterios de calidad y, por lo tanto, son confiables.

\begin{sloppypar} 
En México hay una
preocupación temprana por la evaluación, no correspondida con acciones,
disposiciones y planteamientos hechos rea\-li\-dad. En el seno de la Asociación
Nacional de Universidades e Instituciones de Educación Superior (ANUIES),
desde principios de los años setenta, fueron aprobados dos resolutivos
correspondientes al establecimiento de exámenes nacionales para el ingreso
de estudiantes a licenciatura, para egresados de la misma, al igual que un
centro nacional de exámenes que se abocara a dichas tareas. Los resolutivos
nunca se transformaron en un elemento de política educativa real y
quedaron, en todo caso, como una muestra de las preocupaciones que
campeaban en aquella época, tal como sucedía en el conjunto de países que
más habían avanzado en esa materia. Trece años después se da un nuevo
impulso a la idea de evaluación; una asamblea nacional de la misma ANUIES
es dedicada enteramente a dicho tema; se dan definiciones, se enmarca el
asunto de la evaluación en el particular, diversificado y complejo sistema
mexicano de educación superior, se deslindan caminos para instrumentar la
idea, pero ésta no puede implantarse; era, en todo caso, un tema sólo
perteneciente a las instituciones agrupadas en dicha Asociación. Habría que
esperar otros años para que la evaluación formase parte de las políticas
públicas de la educación superior (ANUIES 1984, 1991).
\end{sloppypar}

\enlargethispage{1\baselineskip} 
Fue a partir de 1989
cuando un tercer intento llega a reconocer en la evaluación un instrumento
fundamental para el mejoramiento de las casas de estudio. En las asambleas
de 1990 y 1991, en Cuernavaca y Tampico, se acuerda
\textit{<<participar
decididamente con el Gobierno federal en un proceso de evaluación de la
educación superior, tanto para proponer y acordar criterios y formas de
evaluación, como para participar en las instancias idóneas de
decisión\ldots}>>. En la
segunda de las asambleas se aprobaron lineamientos para la evaluación de la
educación superior, los cuales fueron en dirección de crear un sistema
nacional en esa materia, para impulsar actividades en las áreas de
evaluación institucional, interinstitucional y sobre el propio sistema de
educación superior. En un lapso muy breve, fueron establecidos: una
comisión nacional de evaluación, comités de pares académicos, un centro
nacional de evaluación de la educación superior y un conjunto de
dispositivos y mecanismos tanto a escala del gobierno federal como de las
instituciones que veían en la evaluación un instrumento útil en sí mismo
para el mejoramiento del sistema de educación superior.

 
 Hasta la fecha, subsisten,
con mucho vigor, instituciones como los Comités Interinstitucionales para
la Evaluación de la Educación Superior (CIEES), El Consejo Nacional para la
Acreditación de la Educación Superior (COPAES), el Centro Nacional para la
Evaluación (CENEVAL),
el Programa de Mejoramiento del Profesorado (PROMEP), entre otros;
convirtiéndose la evaluación en un componente esencial como indicador
institucional de resultados y la base  para el financiamiento en los
ámbitos de la educación superior.

 
Las décadas que van de los 70 al 2000, en nuestro país, han sentado las
bases y creado las condiciones y la  cultura de \textit{la evaluación
educativa} con el firme propósito de garantizar una \textit{educación de
calidad}. En otras palabras, la evaluación y la acreditación de la
educación superior es un punto estratégico en el logro de tal objetivo y
sobre todo al propiciar  la mejora continua y el aseguramiento de la
calidad de los programas educativos y coadyuvando con ello a la detección
de desigualdades e impulsar la igualdad de oportunidades educativas. 
\enlargethispage{1\baselineskip}
 
Es en este sentido que a nivel nacional e internacional los sistemas de
evaluación y acreditación forman parte de la agenda gubernamental, ya que
con ello, se ha propiciado al mejoramiento de los sistemas educativos en
nuestro país y en el mundo. 

 
En México, dichos procesos de evaluación y acreditación son realizados por
diferentes organismos que a través del análisis de \textit{indicadores},
\textit{criterios}, \textit{estándares}, \textit{instrumentos de medición},
entre otros, buscan detectar  el universo existente en los procesos de
enseñanza aprendizaje;  atendiendo los diferentes ámbitos como son: los
alumnos, los programas educativos, su planta académica, su infraestructura,
su financiamiento, la administración del programa, etcétera. Mismo que
podemos concentrar y\slash{}o resumir en  tres categorías: 1. \textit{Recursos del
programa}, 2. \textit{Procesos}, y 3. \textit{Resultados}.


\medskip
\textbf{Resultados}

 
Es así, que surgen ---como dos mecanismos de regulación de la educación
superior--- los procesos de evaluación y acreditación que conlleva a la
autoevaluación interna y evaluación externa, considerando que la calidad
educativa debe ser primordial en la gestión universitaria. Con el programa
para la modernización educativa 1989--1994, se da paso a la
institucionalización y consolidación de la \textit{cultura de la
evaluación} en el nivel superior; concibiendo para ello, la eficacia,
cobertura e innovación como un medio para reconocer y asegurar la calidad 
de la educación superior.

 
En este contexto, en 1991, surgen los Comités Interinstitucionales para la
Evaluación de la Educación Superior (CIEES), integrando los nueve comités:
arquitectura, diseño y urbanismo; ciencias agropecuarias; ciencias
naturales y exactas; ciencias de la salud; ciencias sociales y
administrativas; educación y humanidades; ingeniería y tecnología;
administración y gestión institucional y difusión y extensión de la
cultura; quienes a través de una \textit{evaluación diagnóstica}, donde
maneja los componentes de: integral, objetiva, contextual, analítica,
constructiva y trascendente; para después presentar una serie de
valoraciones y\slash{}o recomendaciones para mejorar la integración,
funcionamiento y calidad educativa del programa evaluado ((Moheno 2010;
Tuirán 2010).

 
Posteriormente, como una política emanada del \textit{Plan de Desarrollo
Educativo 1995--2000},  surge a finales del 2000, el Consejo para la
Acreditación de la Educación Superior, A.\ C., (COPAES), con el firme
propósito de dar continuidad a los procesos de evaluación y con la
finalidad de responder a la necesidad de regular a los organismos que hasta
ese momento realizaban  los procesos de acreditación de los PE, con
objetivos muy precisos ---que vinculados con los de la SEP--- buscan promover
la calidad educativa a través de procesos verificados por medio de la
\textit{evaluación externa}, y  coadyuvando con ello a que las autoridades
educativas definan políticas, estrategias y acciones que permitan elevar y
asegurar la calidad de la educación superior (Rubio 2007).

 
En sí, el objetivo principal de  COPAES es el de <<regular los proceso  de
acreditación y garantizar que los programas educativos  acreditados tengan
un nivel apreciable de desarrollo y consolidación>>\linebreak (García~2005). 

\enlargethispage{1\baselineskip} 
El trabajo realizado tanto por los CIEES, alrededor de $4,666$ PE desde 1991
(\url{www.ciees.org.mx}), y por los organismos reconocidos por  COPAES, han
contribuido significativamente a mejorar la calidad de los programas,
(Rubio 2007). Donde éste último a septiembre de 2013, cuenta con un padrón
de programas acreditados de $2,810$ tanto a nivel de Licenciatura como de
Técnico Superior Universitario,
(\href{http://www.copaes.org.mx/}{\url{www.copaes.org.mx}}).
\newpage
 
El Programa Nacional de Educación 2001--2006,  reconoce que la evaluación
externa y la acreditación de la educación superior son la vía para fomentar
 la mejora continua y el aseguramiento de la calidad; y donde se plantea de
manera específica uno de los ejes estratégicos  <<Educación de Buena
Calidad>>,  por lo que se implementaron una serie de estrategias y acciones
relacionadas con los procesos de evaluación.


\enlargethispage{2\baselineskip} 
En el Plan Nacional de Educación 2006--2012, dichos objetivos y propósitos
se vuelven a plantear y a consolidar con los procesos de acreditación con
la finalidad de desarrollar, asegurar, mejorar y consolidar todo lo
relacionado con la educación.

 
El Plan Nacional de Desarrollo 2013--2018, plantea impulsar una serie de
políticas tendientes a lograr una <<educación de calidad para todos>> con la
finalidad de que propicie el <<desarrollo de las capacidades y habilidades
integrales de cada ciudadano, en los ámbitos intelectual, afectivo,
artístico y deportivo, al tiempo que inculque los valores por los cuales se
defiende la dignidad personal y la de los otros>>.

 
 Como podemos observar, la
búsqueda de la calidad ha sido el tema, preocupación y meta expresados en
planes nacionales e institucionales desde hace más de una década. La
necesidad de lograr una mayor calidad de los procesos y resultados de la
educación ha sido también una inquietud planteada cada vez con mayor
intensidad, hasta el punto de considerar que la calidad es un atributo
imprescindible de la propia educación; toda educación debe ser de
calidad.

 
Los ejercicios de evaluación tanto interna (autoevaluación) como externa
(evaluación realizada por los CIEES y\slash{}o organismo reconocido por COPAES)
son indispensables y complementarias; mientras que la primera permite ver
hasta el detalle  más mínimo y plantear acciones  para su mejoramiento; la
segunda, es importante porque contempla a la interna enriqueciendo sus
resultados; ya que valida a la evaluación interna que puede ser objetiva,
pero también parcial, y permite realizar una evaluación comparativa y mucho
más enriquecedora.

 
Los procesos de evaluación y acreditación contribuyen a una mayor
información sobre sistemas, instituciones y unidades, para diseñar
políticas a largo plazo. Hacia el exterior, la evaluación brinda
información al consumidor. Hacia el interior, proporciona los datos
necesarios para poder relacionar la función particular con los incentivos,
las finanzas y la gestión apropiada. En este aspecto, la acreditación no
frena la diversificación, la cual es un factor crucial para la innovación
de la educación superior (Clara 1992, 1995). Más bien, garantiza que cada
parte cumpla con requisitos mínimos y brinda información confiable sobre la
diversidad de funciones.


\medskip
\textbf{Consideraciones finales}

\enlargethispage{1\baselineskip} 
La evaluación y la acreditación son herramientas necesarias para la
determinación de la calidad educativa. Muchos de los problemas en la
educación superior que dieron origen a la evaluación, no serán solucionados
 únicamente por esta vía. Evaluar solamente tiene sentido si existen las
condiciones para consolidar o remediar los problemas principales, la
evaluación debe estar acompañada de políticas creativas para mejorar
aspectos como la eficiencia terminal, la preparación de profesores, los
sistemas administrativos, o la producción científica. Estos son aspectos
que requieren políticas y fondos a largo plazo, que muchas veces ameritan
continuidad aún cuando la evaluación no demuestre mejoramientos a corto
plazo. En otras palabras, los procesos de evaluación y acreditación no sólo
deben indicar los niveles deseables para el sistema, sino también
proporcionar incentivos o condiciones para que las instituciones puedan
planificar estrategias para lograr las metas.

La evaluación y la acreditación han tenido un breve y sinuoso camino dentro del sistema de
educación superior en México. Se llegó tarde frente a lo que, con
frecuencia, se erigen como paradigmas: los sistemas que en ese mismo ámbito
tienen establecidos Estados Unidos y Canadá, por ejemplo.

La celebración del Tratado de Libre Comercio con ambos países y su vigencia, 
a partir de 1994, estimuló notablemente los acercamientos que permitieron observar 
con mayor detalle el funcionamiento de esos sistemas y las comparaciones fueron
inevitables. Para muchas instituciones y directivos la sincronía en materia
de comercio con ambos países debía de corresponderse con algo semejante en
educación superior. Afortunadamente el debate fue normando los enfoques en
torno a esta materia y todo indica que México está siguiendo un camino
propio, acorde con su circunstancia, historia y condicionantes culturales.
Sin embargo, no es un asunto totalmente terminado: dos capítulos del
Tratado de Libre Comercio con Estados Unidos y Canadá hacen referencia al
intercambio e ingreso de profesionales y las condiciones para los
reconocimientos recíprocos, muchos de los cuales tienen que ver con
acreditación, certificación y evaluación.
\enlargethispage{1\baselineskip}
 
A escala mundial, la
evaluación y la acreditación son procesos reconocidos como medios idóneos
para el mejoramiento de los sistemas de educación superior. Además, en el
caso de México, al igual que en muchos otros países latinoamericanos, los
planteamientos en esta materia se han venido haciendo con el interés
creciente de que éstos puedan responder a sus
propias circunstancias históricas, sociales y educativas; y ésta preocupación 
por aumentar la calidad radica en la importancia que la educación superior tiene en el
desarrollo económico y socio-cultural de las
naciones (Pallan 1996).
\newpage

\textbf{Referencias}
\enlargethispage{1\baselineskip}
 
Asociación Nacional de Universidades e Instituciones de Educación Superior
(ANUIES)  (2013), \textit{Inclusión con responsabilidad social. Elementos
de diagnóstico y propuestas para una nueva generación de políticas de
educación superior},  México: ANUIES.

\begin{sloppypar} 
\_\_\_\_\_\_ (1999), \textit{El sistema
nacional de evaluación y acreditación. Un proyecto de visión al 2010 y
propuestas para su consolidación}, México,
ANUIES.
\end{sloppypar}
 
Chapela Castañares, G., (1993), <<Notas sobre el proceso de creación de un sistema de acreditación de las instituciones de educación superior en México>>.
En \textit{Acreditación
universitaria en América Latina. Antecedentes y
experiencia}, México, ANUIES--CINDA--OUI. 

 
Delors, J. \textit{et al.} (1996), \textit{La educación encierra un tesoro.
Informe a la UNESCO de la Comisión Internacional sobre la educación para el
siglo XXI}, Santillana\slash{}Ediciones UNESCO.

 
Grindle M. S. (2000), \textit{La paradoja de la reforma educacional:
pronosticar el fracaso y encontrarnos con el avance}. Presentado en el
Seminario Internacional Reformas  Educativas y Política en América Latina.
Santiago, Chile: Centro de Investigación y Desarrollo de la educación
(CIDE)

 
García Garduño, J. M. (2005), \textit{El avance de la evaluación en México
y sus antecedentes}, México, Revista Mexicana de Investigación Educativa,
octubre-diciembre, año\slash{}vol. 10, número 027.

 
Moheno Padrón, M. G. (2010), \textit{Evaluación de la educación. Calidad
educativa mediante algunos mecanismos de evaluación: Las experiencias de la
BUAP}.

 
Pallán Figueroa, C. (1995). <<Sistemas de
acreditación y acreditación para la profesión de abogado>>.
En \textit{Revista de la Educación Superior},
ANUIES, n.\ 96: 8--12, octubre--diciembre.

 
\_\_\_\_\_\_ (1996), <<Evaluación, acreditación
y calidad de la educación en México. Hacia un sistema nacional de
evaluación de la educación superior>>. En
\textit{Universidades},  Unión de Universidades
de América Latina, n.\ 12, pp.\ 9--17, julio-diciembre.
\enlargethispage{2\baselineskip}
 
Peña Nieto, E. (2013), \textit{Plan Nacional de
Desarrollo 2013-2018}, México, Gobierno de la República.

 
Rubio Oca, J. (2007), \textit{La evaluación y acreditación de la educación
superior en México: Un largo camino aun por recorrerse}, México,
Reencuentro, Universidad Autónoma Metropolitana-Xochimilco. 

 
Tadesco, J.C. (1999), \textit{Educación de la Información. Encuentro
Internacional de Educación Media}, Bogotá, Secretaria de Educación.

 
Tuirán, R. (2010), \textit{La Educación Superior en México: avances,
rezagos y retos}. 

 
Tiana, A. (1996), \textit{La evaluación de los sistemas educativos}.
Revista Iberoamericana de Educación, Número 10, Enero--Abril 1996,
Evaluación de la Calidad de la Educación.


UNESCO--IESAL  (2006), \textit{Informe sobre la educación superior en
América Latina y el Caribe. 2000-2005., y  Metamorfosis de la
educación superior}, Caracas, IESALC--UNESCO.

\begin{sloppypar} 
Vargas Jiménez, I. (2008) <<Análisis de cinco desafíos en el ejercicio de
la administración educativa>>. En \textit{Revista Electrónica Actualidades
Investigativas en Educación}, Costa Rica, Instituto de Investigación en
Educación Universidad de Costa Rica. Recuperado el 5 de octubre de 2013 en 
\url{http://revista.inie.ucr.ac.cr/uploads/tx_magazine/desafios.pdf}
\end{sloppypar}
 
\href{http://www.ciees.org.mx/}{\url{www.ciees.org.mx}}\newline 
\href{http://www.copaes.org.mx/}{\url{www.copaes.org.mx}}
\newpage

\begin{center} 
\textbf{Anexos}

\vspace{2in}
\includegraphics[scale=0.222]{p22-img001.pdf} 
%\includegraphics[scale=0.215]{p22-img001}
\end{center}
{\scriptsize {\bfseries Fuente:} SEP--PIFI 2002--2012}
\newpage

\begin{mdframed}[userdefinedwidth=3.4in,align=center,
linecolor=blue,linewidth=3pt]
\begin{figure}[H]
\begin{minipage}{3in}

\centering{ 
\footnotesize\textbf{Logros y avances}\par  
\footnotesize\textbf{(Competitividad Académica)}\par

\footnotesize{Comparativo de  la evolución de la matricula}\par 
\footnotesize{en PE de calidad en las}\par 
\footnotesize{Universidades Públicas (2003--2012)}}
\end{minipage}
\end{figure}
\end{mdframed}

\smallskip
\includegraphics[scale=0.615]{p22-img002.pdf} 

\scriptsize{{\bfseries Fuente:} Dirección General de Educación Superior\slash{}UAS, 2014}

\medskip
\includegraphics[scale=0.625]{go2-img001.pdf}

\scriptsize{{\bfseries Fuente:} Dirección General de Servicios Escolares\slash{}UAS, Ciclo Escolar 2012--2013\\
Dirección General de Educación Superior\slash{}UAS, 2014}
\newpage
\thispagestyle{empty}
\phantom{abc}