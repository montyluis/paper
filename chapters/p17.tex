%\documentclass{article}
%\usepackage{amsmath,amssymb,amsfonts}
%\usepackage{fontspec}
%\usepackage{xunicode}
%\usepackage{xltxtra}
%\usepackage{polyglossia}
%\setdefaultlanguage{spanish}
%\usepackage{color}
%\usepackage{array}
%\usepackage{hhline}
%\usepackage{hyperref}
%\hypersetup{colorlinks=true, linkcolor=blue, citecolor=blue, filecolor=blue, urlcolor=blue}
%% Text styles
%\newcommand\textstyleInternetLink[1]{\textcolor{blue}{#1}}
%\newcommand\textstyleappleconvertedspace[1]{#1}
%\newtheorem{theorem}{Theorem}
%\title{}
%\author{Adriel Rodríguez}
%\date{2014-06-24}
%\begin{document}
%%\clearpage\setcounter{page}{355}

\thispagestyle{empty}
\phantomsection{}
\addcontentsline{toc}{chapter}{Historiografía de la guerra de castas en\newline Campeche: una historia 
fragmentaria\newline $\diamond$
\normalfont\textit{Miriam  Edith León Méndez}}
{\centering {\scshape \large Historiografía de la guerra de castas en Campeche:\\ una historia fragmentaria}\par}
\markboth{la formación del historiador}{guerra de castas}
\setcounter{footnote}{0}

\bigskip
\begin{center}
{\bfseries Miriam  Edith León Méndez}\\
{\itshape Universidad Autónoma de Campeche}
\end{center}

\bigskip
\textbf{Resumen}
%\enlargethispage{1\baselineskip}

A partir del análisis historiográfico se pretende establecer un avance de
investigación que nos permita conocer lo que se ha escrito sobre la Guerra
de Castas en Campeche. Siendo ésta un acontecimiento que cimbró gran parte
de la península de Yucatán, resalta el interés por rescatar esta historia,
pues la consideramos trascendental, si la enmarcamos a partir de la
erección de Campeche como Estado. 

La historia de la Guerra de Castas ha sido escrita desde el espacio
territorial de Yucatán, quedando muy poco estudiado lo que aconteció fuera
de sus límites territoriales; este es precisamente el caso de Campeche. Es
por ello, que en la historiografía existente en Campeche encontramos muy
poca información sobre esa temática, hecho que nos induce a plantear que
los escritos en función a la Guerra de Castas en Campeche fue manejado de
una manera parcial, favorable, para el grupo en el poder. Los que
escribieron en aquella época  formaban parte importante del grupo elitista
en el estado y fueron los autores que -de una u otra forma- participaron en
ese proceso histórico, tuvieron un enfoque plagado de prejuicios e
influencias de su propio horizonte y del contexto que les tocó vivir.

Relevante es mencionar que esta ponencia forma  parte de una investigación
más amplia que pretende establecer un análisis del conjunto de textos que
se han escrito en función al proceso conocido como Guerra de Castas, en el
espacio de la península de Yucatán en México. Proyecto inserto en el CA y
financiado por el PROMEP.\\
\textbf{Palabras clave:} historiografía, guerra, castas, contexto,
prejuicios.


\bigskip
\textbf{Abstract}

\begin{sloppypar}
From the historiographical analysis aims to establish a research
breakthrough that allows us to know what has been written about the Caste
War in Campeche. Since this is an event that rocked much of the Yucatan
Peninsula, highlights the interest in recovering this story, because we
consider it vital, if we frame from the erection of Campeche as a state.
\end{sloppypar}

The history of the Caste War has been written from the Yucatan territorial
space, leaving little studied what happened outside its territorial limits;
This is precisely the case of Campeche. Therefore, in the existing
historiography in Campeche we found very little information on this
subject, a fact that leads us to argue that the writings according to the
Caste War in Campeche was handled in a partial manner favorable for the
group power. The writers at the time were an important part of the elite
group in the state and authors-one way or another participated in this
historic process were, had a fraught approach biases and influences of his
own horizon and context they he lived.

Is relevant to mention that this paper is part of a broader research
analysis aims to establish a set of texts that are written according to the
process known as Caste War, in the space of the Yucatan Peninsula in
Mexico. Insert Project in CA and funded by the PROMEP.\\
\textbf{Keywords:} historiography war breeds, context, prejudice.
\newpage

Realizar un análisis historiográfico  implica reflexionar en la construcción
del conocimiento histórico, cómo se construye el discurso, qué herramientas
y fuentes utiliza para crear esa historia que nos conlleva a conocer el
pasado y comprender el presente que hoy nos ocupa. Tiene que ver también
con el porqué y para qué se escribe. Cuestionamiento que resultan por más
interesantes cuando se estudian los procesos históricos, inmersos desde un
horizonte de enunciación y un contexto.

Es desde esta perspectiva que se pretende analizar el proceso histórico
conocido como Guerra de Castas en Yucatán, delimitando su estudio y
ubicando el espacio a  Campeche como Estado.
 
Valga mencionar que este trabajo forma parte de un proyecto más amplio que
lleva por título <<La construcción Sociopolítica en la Guerra de Castas
desde la perspectiva histórica, lingüística y literaria>>, Proyecto
del Cuerpo Académico Problemas de Teorías del Lenguaje, Historiografía  y
Exégesis del Discurso Literario, aprobado y financiado por el Programada de
Mejoramiento al Profesorado (PROMEP-SEP, Convocatoria para Cuerpos
Académicos 2013). 


Interesante ha sido combinar las disciplinas de la historia y la literatura,
toda vez que para difundir el conocimiento histórico, se ha recurrido a la
obra de la novela histórica, como un ejemplo, destaca la  novela histórica
de Silvia Molina llamada \textit{Ascensión Tun},  donde se denota
una estructura paralela con dos visiones del conflicto que suscitó la
Guerra de Castas: la maya y la blanca. 

La maya es representada
por los personajes de Juan Bautista, indígena maya, quien recrea sus
memorias y las comparte al niño Ascensión Tun, quien, a la vez, las
asimila y se siente orgulloso de sus ancestros por la guerra iniciada y
justificada; mientras que la otra parte, la
visión blanca la revive
Consuelo, una niña que sufrió los desmanes y desdichas de la Guerra de
Castas: fue abandonada por sus padres en su apresurado salir de su casa,
por la guerra que los alcanzaba, quedó sola y escondida, presenciando las
tragedias de la guerra: violación y muerte, actos que la siguieron  y
marcaron su propia vida (Molina 1981, pp. 31--32). Personajes reales, no
ficticios, que nos permiten refigurar el proceso vivido, narrado e
imaginado.
\enlargethispage{1\baselineskip}

Silvia Molina maneja dos
visiones que, sin lugar a dudas, han  caracterizado la historiografía sobre
la Guerra de Castas: pues por una parte se encuentran los textos que nos
hablan sobre el enfoque de los mayas, que defienden su causa y la
justifican; y por el otro lado, están los blancos, los que no comprendían
el comportamiento de los mayas ni el estallido de la guerra.

Es precisamente ésta última visión y representación la que se pretende
analizar, pero solo dentro de la historiografía campechana; toda vez  que
la historia de la Guerra de Castas ha sido escrita mucho más desde el
espacio territorial de Yucatán, quedando muy poco estudiado lo que
aconteció fuera de sus límites territoriales: este es precisamente el caso
de Campeche.


Asimismo, nos encontramos que uno de los problemas más agudos de la historia
de Campeche es que presenta enormes lagunas históricas que no permiten
comprender totalmente su propia historia. 


Ante esto se ha tratado de motivar a los alumnos de la Licenciatura en
Historia para que incursionen sobre diferentes periodos de la historia de
Campeche. Cabe mencionar que este proyecto incluye a tesistas con beca de
apoyo para sus investigaciones; no obstante, poco hemos avanzado al
respecto. 
 
Es así como iniciamos clasificando los textos que se han escrito  sobre
Guerra de Castas o  de aquellos que contienen alguna información sobre esa 
historia en Campeche; partimos  del siglo XIX, mediados de ese siglo cuando
irrumpe la Guerra en la península de Yucatán.


De tal forma que, este estudio, representa  un primer acercamiento a la
temática a estudiar, y muestra un análisis  de los textos  de  Justo
 Sierra O´Reilly (\textit{Los Indios de
Yucatán. Consideraciones históricas sobre la influencia del elemento
indígena en la organización social del
país}), Serapio
Baqueiro (\textit{Ensayo Histórico de las Revoluciones en Yucatán, de 1840
a 1864}), Eligio Ancona (\textit{Historia de Yucatán}), y J. F. Molina
Solís (\textit{Historia de Yucatán desde la independencia de España hasta
la época actual)}, discurriendo en el análisis de  la historiografía
crítica. De igual manera explica la referencia del libro de Tomás
Aznar Barbachano y Juan Carbó Álvarez
(\textit{Memoria sobre la conveniencia. Utilidad y
necesidad de erigir constitucionalmente en estado de la confederación
mexicana el antiguo distrito de Campeche, constituido de hecho en Estado
Libre y Soberano desde mayo de 1858, por virtud con los convenios de
división territorial que celebró con el estado de Yucatán, de que era
parte)}, donde se registra el Convenio territorial entre
Campeche-Yucatán.

Campeche en esa época (1857),  a una década de haber iniciado la guerra, se
 separa de Yucatán y se conforma como Estado de la Federación Mexicana. De
hecho las discordias que existieron entre los grupos del poder se apuntan
como elementos causales, que motivaron el estallido de la Guerra de Castas.
A decir de Don E. Dumond:

\begin{quotation}
El que las expectativas de las clases más bajas, sobre todo la de los
indígenas, se hubieran frustrado abruptamente, y que la rebelión empezara
precisamente cuando lo hizo, tenía que ver con la situación política
general  de la península, incluida una chocante distribución de impuestos,
las rivalidades extremas entre las facciones de Mérida y Campeche y las
maquinaciones de políticos particulares (Dumond~2005,\linebreak pp.~631\textendash{}632)
\end{quotation} 

Instalado el Gobierno de Pablo García y <<deseando terminar la guerra civil
que aniquila a Yucatán,  eliminar el elemento de discordia que ha servido
en todas las épocas de arma poderosa y fratricida a los ambiciosos y
enemigos de la pública tranquilidad\ldots>> (Aznar y Carbó 1994, p.
 139), se establece el compromiso de
cuidar y mantener cubierta la línea fronteriza de los Chenes, además de que
adquiere el deber y el compromiso de apoyar al gobierno de Yucatán,
económicamente, a fin de sostener la guerra contra los indios: <<Este
subsidio será una cantidad igual  a la que importe la tercera parte del
presupuesto de todos los gastos del Estado de Campeche>> (Aznar y Carbó~1994, 
p.~140).


Lo anterior como parte del Convenio de División territorial, realizado 
entre Mérida y Campeche,  en al año de 1858, que se contempla en la
\textit{Memoria sobre la conveniencia. Utilidad y
necesidad de erigir constitucionalmente en estado de la confederación
mexicana el antiguo distrito de Campeche, constituido de hecho en Estado
Libre y Soberano desde mayo de 1858, por virtud con los convenios de
división territorial que celebró con el estado de Yucatán, de que era
parte}. Documento  escrito para justificar la separación Campeche-Yucatán 
ante el Congreso de la Unión en 1861.


De esta manera, es a partir de la \textit{Memoria}, escrita por Tomás Aznar
y Juan Carbó, que el gobernador de Campeche (Pablo García) implementa
medidas para tratar el caso de la Guerra dentro de los límites
territoriales del nuevo Estado. Se puede
asentar que esta representa la primera visión de los campechanos sobre la
Guerra de Castas.


Tomás Aznar Barbachano nació en Mérida en el año 1825 y perteneció a una
prestigiosa familia española. Estudió en el colegio de don Miguel Casares
(Mérida), en el Seminario Clerical de <<San Miguel de Estrada>> (Campeche) y
en la Universidad de Yucatán, donde logró el título de  abogado en 1850;
formó parte del ilustre  Colegio Nacional de abogados. Fue catedrático de
filosofía y álgebra y ocupó diferentes cargos públicos. Fue vicegobernador
de Campeche de 1862 a 1864 durante el gobierno de Pablo García, y asumió el
compromiso de encauzar al reciente Estado de Campeche; además, fue Fiscal
de Hacienda,  Juez de lo Criminal, Juez de Primera Instancia de lo Civil,
Diputado en el Congreso de la Unión, Agente del Ministerio Público y
Presidente del Consejo de Gobierno (Pérez 1979, pp. 32--33).
\enlargethispage{1\baselineskip}
 

Considerando que el medio óptimo para difundir la ideología que
representaba, junto con  García, fundó varios periódicos: \textit{El Hijo
de la Patria,  La Ley, El Chisgarabis, Los Primeros Ensayo, La Nueva Época} 
y \textit{Las Mejoras Materiales}. Colaborando también en el
\textit{Espíritu Publico}, órgano oficial de gobierno y en el periódico
\textit{La Alborada}.  Ahí se publicaron y se dio  difusión a las ideas
liberales del grupo garciista.

Juan Carbó Álvarez, hijo de padres españoles, nació en Campeche y se dedicó
principalmente a la política. Participó activamente en la separación
Campeche-Yucatán, a lado del grupo garciista y selló su lealtad a Pablo
García: fue un colaborador muy cercano, ya que fungió como Secretario
General de Gobierno; ocupó cargos políticos  de importancia como Diputado
Federal, Jefe Político del Partido de Campeche, Jefe de las Colonias
Militares en el Sur, Jefe Político del Partido de Champotón y Subdirector
del Ferrocarril Campechano. Incursionó en el periodismo escribiendo
artículos en el periódico \textit{El Pensamiento} y colaboró como
editorialista en el periódico \textit{El Espíritu Público} (Sierra~1991, 
p.~79).
\newpage

Ambos autores de la \textit{Memoria} pertenecieron a familias reconocidas en
la sociedad y ocuparon cargos claves en la administración pública, jugando
un papel predominante en la separación de Campeche-Yucatán. Políticos que
incursionaron en el  periodismo y que,  por tanto, influyeron en la
ideología del pueblo.


Su propia formación y entorno les permitió ver la situación que se suscitaba
en la península de Yucatán como una guerra injustificada puesto que sus
mismas plumas los delatan al llamar  a los indígenas <<enemigos de la
tranquilidad pública>>. Posición que se asume junto con la visión de los
blancos, de aquellos que condenaron la guerra y apoyaron al grupo de los
españoles.

 
Otros escritos de la época de la segunda mitad del siglo XIX, se unifican
con la postura anterior,  ya que quienes escribieron fueron precisamente
los testigos de los hechos. Así encontramos que los autores del siglo
decimonónico se sumaron a escribir desde su perspectiva y posición social;
de tal forma que sus representaciones sobre guerra de Castas coincidían con
la visión de los blancos.

 
La Guerra de Castas de 1847 es conocida en la historia nacional de México
como un acontecimiento social y político que cimbró todas y cada una de las
esferas de la vida del sureste peninsular;  identificada como una lucha de
sublevación, donde el indígena maya resulta ser el  principal protagonista,
la guerra de castas se marca como el suceso más importante del siglo XIX,
un conflicto que conmocionó a la sociedad yucateca y que permeó en todos
los escritos de la época como una referencia obligada para los autores del
contexto histórico que les tocó vivir. 
\enlargethispage{1\baselineskip}
 

Estudiada como rebelión indígena, lucha de clases, guerra étnica,  etc.,
nos muestran diferentes interpretaciones, visiones y representaciones que
esbozan distintos enfoques y conjeturas, que tienen como interés explicar
las causas por las que los mayas yucatecos se levantaron en armas en el
dramático año de 1847. 
\enlargethispage{1\baselineskip}

Un primer autor que escribió sobre este acontecimiento fue Justo Sierra
O’Reilly, quién nació en el pueblo de Tixcacaltuyú del antiguo Distrito de
los Chenes en 1814, campechano de nacimiento, fue abogado de formación.
Hizo sus estudios en el Colegio de San Idelfonso (Mérida), en el Nacional
Colegio de San Idelfonso (México) y en el Colegio de Abogados se matriculó
como tal y recibió tituló de Doctor en la Nacional y Pontificia Universidad
del Estado, donde también impartió sus enseñanzas como catedrático.

Casó con la hija del gobernador de Yucatán, Santiago Méndez, y se posicionó
en un lugar importante dentro dela administración pública al ocupar cargos
relevantes en el gobierno: Diputado, Senador del Congreso de la Unión y Juez
de Distrito; también figuró como Secretario de una Comisión que tuvo como
objetivo firmar los tratados del 28 de diciembre de 1841 para la
reincorporación de Yucatán a la República.

Al estallar la guerra de Castas el 30 de junio de 1847, el gobernador Dr.\
Domingo Barret, lo envió como agente del gobierno de Yucatán y Comisionado
especial cerca de los Estados Unidos, para gestionar, la desocupación de la
Isla del Carmen que los norteamericanos habían ocupado en su guerra contra
México, conseguir un reconocimiento especial para Yucatán entonces separado
de la República y solicitar auxilio para rechazar la sublevación indígena
(Pérez~1979, p.~597).
\enlargethispage{-1\baselineskip}

De  su persona se ha escrito, a decir de Gustavo Martínez Alomía y Juan de
Dios Pérez Galaz, que fue el padre de la literatura peninsular, el primero 
en intentar un estudio serio de la historia peninsular.

Destacó como periodista, puesto que la mayoría de sus escritos fueron
difundidos en diferentes periódicos: \textit{El amigo del pueblo, La razón,
la Unión Liberal, El Museo Yucateco, El Registro Yucateco}; y fue en el
periódico \textit{El Fénix}, publicación que circuló en el antiguo Distrito
de Campeche de 1848 a 1850, donde inició sus escritos de corte regional que
se pueden considerar como las primeras interpretaciones sobre la historia
general de Yucatán. Es precisamente en éste último periódico donde 
publicó, con el título \textit{Consideraciones sobre el origen, causas y
tendencias de la sublevación de los indígenas, sus probables resultados y
su posible remedio}, sobre la situación inicial de la Guerra de Castas 
(Sierra O’Reilly 1994).


\begin{quotation}
buscaba las causas de la rebelión indígena en defectos innatos de la 
«raza enemiga» y en los efectos dela administración colonial sobre las
relaciones entre ambas razas; para él no existió el problema de combinar
las posibles virtudes de una raza que formó parte de los antepasados de los
yucatecos contemporáneos, con los vicios ingénitos que causaron la rebelión
de los mayas (Molina~1992, p.~185).
\end{quotation}

\medskip
Posturas que se reproducen más tarde,  en el libro titulado \textit{Los
Indios de Yucatán} y en donde se denota una posición contra los indios y
contra todo aquello que violentó la tranquilidad de un pueblo próspero, 
declarando a aquellos como la \textit{raza enemiga}. De tal forma que
apunta:
\enlargethispage{1\baselineskip}

\begin{quotation}
Aquella guerra salvaje y sin cuartel; la saña implacable con que la llevaba
a efecto un enemigo fuerte por su número y por su ardor ciego y brutal; el
desgarrador gemido de las mujeres, ancianos y niños; la terrífica barrera
del mar impidiendo el paso franco a los fugitivos que sentían sobre sus
ijares el machete del indio; el frenesí delirante con que los bárbaros
reducían a escombros las aldeas, villas y ciudades, destruyendo los templos
y monumentos de nuestra civilización; la sangre, el humo las pavesas, el
estruendo que traían en su rápido curso aquel desbordado torrente,
poderosos motivos eran por cierto para difundir la angustia y la desolación
entre los descendientes de los antigua raza colonizadora. ¡Días del luto y
de dolor supremo, que jamás pueden olvidar los hombres de esta generación,
y que pasarán a la posteridad dejando en su tránsito una huella profunda!\linebreak
(Sierra O’Reilly~1994, pp.~17\textendash{}18).
 
(\ldots) lo que había sido obra de más de tres siglos de penosa labor estaba
convertido en ruinas inmensa, destruida la industria, muerta la riqueza,
mermada la población\ldots (\textit{ibid.}, p.~18).
\end{quotation}

Su opinión hacia la guerra de Castas seguía siendo totalmente inclinada
hacia los blancos, visión parcial y condenatoria hacia la raza indígena.

Otras obras de historia general se escribieron y se detuvieron en el proceso
conocido como guerra de Castas, sin embargo éstas ---como la anterior--- se
refería más al espacio territorial de Yucatán que a Campeche.


Ejemplo de esto es la obra Serapio Baqueiro Preve,  autor del texto
\textit{Ensayo histórico sobre las revoluciones de Yucatán}, quien fue
un abogado, historiador y periodista. Nació el 14 de
noviembre
de \href{http://es.wikipedia.org/wiki/1838}{\textstyleInternetLink{\textcolor[rgb]{0.0,0.0,0.039215688}{1838}}} en Dzitbalchén, poblado de
\href{http://es.wikipedia.org/wiki/Campeche}{\textstyleInternetLink{\textcolor[rgb]{0.0,0.0,0.039215688}{Campeche}}},
que entonces formaba parte de
\href{http://es.wikipedia.org/wiki/Yucatán}{\textstyleInternetLink{\textcolor[rgb]{0.0,0.0,0.039215688}{Yucatán}}}.
Murió en
\href{http://es.wikipedia.org/wiki/Mérida_(Yucatán)}{\textstyleInternetLink{\textcolor[rgb]{0.0,0.0,0.039215688}{Mérida}}} el 17 de marzo de \href{http://es.wikipedia.org/wiki/1900}{\textstyleInternetLink{\textcolor[rgb]{0.0,0.0,0.039215688}{1900}}}.
Fue \href{http://es.wikipedia.org/wiki/Gobernantes_de_Yucatán}{\textstyleInternetLink{\textcolor[rgb]{0.0,0.0,0.039215688}{gobernador
provisional de Yucatán}}} en \href{http://es.wikipedia.org/wiki/1883}{\textstyleInternetLink{\textcolor[rgb]{0.0,0.0,0.039215688}{1883}}}.


Hizo sus primeros estudios en las escuelas que dirigían los maestros José
María Ruz, José María Morano y Margarita Mora, para después trasladarse a
\href{http://es.wikipedia.org/wiki/Mérida_(Yucatán)}{\textstyleInternetLink{\textcolor[rgb]{0.0,0.0,0.039215688}{Mérida
(Yucatán)}}} a fin de estudiar en el
Seminario Conciliar de San Ildefonso, en donde cursó el bachillerato.
Después ingresó al curso de Derecho de los maestros Antonio Mediz y Vicente
Solís Rosado con quienes obtuvo su título de abogado en el año
de \href{http://es.wikipedia.org/wiki/1863}{\textstyleInternetLink{\textcolor[rgb]{0.0,0.0,0.039215688}{1863}}}.
Fue Juez de primera instancia en las ciudades
de \href{http://es.wikipedia.org/wiki/Tekax}{\textstyleInternetLink{{Tekax}}} y Mérida, en Yucatán, Fiscal de Hacienda, Magistrado del Tribunal
Superior, diputado al Congreso local y consejero del Gobierno de %
%Citar correctamente las referencias de internet.
%Revisar los párrafos del texto donde aparecen los links de internet y no están citados a lo largo del texto.
%Autor desconocido
%16 de agosto de 2014 20:00
Yucatán. Se hizo cargo del poder ejecutivo yucateco
en \href{http://es.wikipedia.org/wiki/1883}{\textstyleInternetLink{\textcolor[rgb]{0.0,0.0,0.039215688}{1883}}} por ausencia del gobernador
\textstyleInternetLink{\textcolor[rgb]{0.0,0.0,0.039215688}{Octavio Rosado
}}(Martínez 2010, pp.~299\textendash{}300).
\enlargethispage{1\baselineskip}
 

En el campo de la instrucción pública se distinguió como
profesor de historia universal y como Director de
la \href{http://es.wikipedia.org/w/index.php?title=Escuela_Normal_de_Profesores_de_Yucatán&action=edit&redlink=1}{\textstyleInternetLink{ {Escuela
Normal de Profesores de Yucatán}}}. Durante un año fue
también Director del Instituto Literario del Estado de Yucatán. Afiliado al
Partido Liberal mexicano, redactó con Manuel
Peniche, \href{http://es.wikipedia.org/wiki/Eligio_Ancona}{\textstyleInternetLink{ {Eligio
Ancona}}} y Manuel
Oviedo, \textit{La Sombra de Cepeda},
periódico que combatió a \href{http://es.wikipedia.org/wiki/Segunda_Intervención_Francesa_en_México}{\textstyleInternetLink{ {la
intervención francesa}}} y al \href{http://es.wikipedia.org/wiki/Segundo_Imperio_Mexicano}{\textstyleInternetLink{ {Imperio}}} de
\href{http://es.wikipedia.org/wiki/Maximiliano_I_de_México}{\textstyleInternetLink{ {Maximiliano
de Habsburgo}}} {en} \href{http://es.wikipedia.org/wiki/México}{\textstyleInternetLink{ {México}}}.

Su obra \textit{Ensayo histórico sobre las revoluciones en
Yucatán} fue tal vez su principal aporte historiográfico a la
convulsiva circunstancia de la \href{http://es.wikipedia.org/wiki/Península_de_Yucatán}{\textstyleInternetLink{\textcolor[rgb]{0.0,0.0,0.039215688}{península
de Yucatán}}} en el \href{http://es.wikipedia.org/wiki/Siglo_XIX}{\textstyleInternetLink{\textcolor[rgb]{0.0,0.0,0.039215688}{siglo
XIX}}}. Se  publica en 1865 por la Imprenta literaria de Eligio Ancona con
el nombre de \textit{Ensayo histórico sobre las revoluciones de Yucatán
desde el año de 1840 hasta 1864} y se difunde también en el Periódico <<El
instructor>>, con fecha del 8 de enero de 1866. Más tarde ---en 1878---, es
editada  por  la Imprenta de don Manuel Heredia Argüelles. La edición que
hoy analizamos se clasifica en cinco volúmenes, y utiliza una narrativa
cronológica  a fin de explicar los acontecimientos alrededor de los frentes
de combate que se generaron en torno a la Guerra de Castas.


Empleando términos como <<la divina providencia>>, deja entrever su creencia
religiosa cargada de providencialismo, ya que todo lo que explica sucede
por la voluntad de Dios: <<Es que la divina providencia  cuando prepara uno
de esos grandes acontecimientos, prepara también hombres que lo igualen en
magnitud por sus esfuerzos\ldots>> (Baqueiro~1990, p.~102). Se valió, por tanto, 
de la divinidad como\linebreak explicación del suceso narrado.


Parece que Serapio Baqueiro adopta una postura imparcial frente a los
acontecimientos que recrea en su obra, pues lamenta y censura la acción de
crueldad por ambos grupos que peleaban en la mal llamada guerra de castas.
Cito:

\begin{quotation}
(\ldots) sensible fue sin embargo un acto de crueldad ejercido por algunos jefes y
oficiales, a cuya cabeza se encontraba el jefe mismo de la división, cuyos
instintos de sangre no nos son desconocidos. En una hermosa casa de altos
de las que circundan la plaza principal de Tekax, fueron encontrados varios
indios que no habiendo podido salir se refugiaron en ella. Pues bien, a
estos desgraciados los cogieron de los pies y de las manos entre dos
personas, y después de fuertes mecidas que les daban, los arrojaban desde
los elevados corredores del edificio hasta abajo,  en donde eran recibidos
a punta de bayoneta por los soldados. Entre estos infelices estaba un pobre
niño, que derramando lágrimas se abrazaba en una de las rodillas de los
oficiales, pidiendo que lo salvaran, pero ni para ese ángel de inocencia
hubo misericordia, pues fue arrojado como los demás a la plaza, y recibido
como ellos a punta de las bayonetas. ¡Horror infunden estos bárbaros
procedimientos!\linebreak (\textit{ibid.}, pp.~22\textendash{}23)
\end{quotation}

Por contraparte, también condena al indígena porque la
<<guerra se extendía igualmente a todas las poblaciones de la comarca,
cayendo bajo el hacha y la tea incendiaria de los bárbaros, ricos
establecimientos e inocentes victimas que acababan con furor.>>

No obstante, hay que apuntar que el padre de Serapio
Baqueiro fue Coronel, don Cirilo Baqueiro, quien participó activamente en
la Guerra de Castas, por ello en su escritura se dirige al grupo de los
blancos que combatían a los indígenas como <<nosotros>>. No es de dudar,
entonces,  que su interés por escribir y dejar plasmada la historia fue
influenciada por la línea paterna, de lo que escuchó, conoció y vivió.
Además, siendo niño le tocó vivir el traslado de residencia precisamente
por los hechos y acontecimientos acaecidos por la Guerra de Castas.

Otro autor que escribe historia general es Eligio Jesús
Ancona Castillo, a la que titula \textit{Historia de Yucatán desde la
época más remota hasta nuestros días}. Ancona nació en  Mérida, Yucatán, en
1835, en cuna de una familia española,  estudió el
bachillerato en la ciudad de Mérida en el Seminario Clerical de San
Ildefonso y en la Universidad Literaria del Estado, donde  se tituló como
abogado en 1862. Se destacó como maestro, abogado, novelista, historiador,
dramaturgo, periodista y político debido a que ocupó puestos importantes
como Gobernador de Yucatán en 1868 y entre los años de 1875--1876, Regidor
del Ayuntamiento de Mérida (1866), Magistrado del Tribunal de Circuito de
Yucatán. En 1891,  por su experiencia como jurisperito, fue designado
Magistrado de la \href{http://es.wikipedia.org/wiki/Suprema_Corte_de_Justicia_de_la_Nación}{\textstyleInternetLink{ {Suprema
Corte de Justicia}}}. En el ocaso de su vida
%Texto de internet, citarlo
%Autor desconocido
%16 de agosto de 2014 20:02 
 fue diputado al congreso federal en representación de
Yucatán. Fue miembro de la Sociedad de Geografía y Estadística. Muere en
1893.

\enlargethispage{1\baselineskip}
Su obra, \textit{Historia de Yucatán desde la época más remota hasta
nuestros días}, representa una consulta obligada para conocer la historia
de Yucatán, escrita en 4 partes, se imprimió por primera vez en Mérida en
1878 y posteriormente en Barcelona en 1889; dedica la última parte a la
Guerra de Castas, donde expone las causas de la sublevación, describiendo
el conflicto bélico y la vida política en torno al suceso. Las causas las
atribuía a las  injusticias cometidas en contra de la población indígena
desde la conquista española, que generaron sentimientos contrarios,
rencores crecidos que se convirtió en una guerra de razas (la criolla y la
indígena). Sin embargo,  enfatiza:


\begin{quotation}
La raza indígena se sublevó precisamente en el momento en que se habían dado
los pasos más avanzados para hacer cambiar su condición. (\ldots) comenzaban á
abrirse escuelas para nivelarla en instrucción con el resto de sus
compatriotas; sus impuestos habían disminuido considerablemente, y aquellos
pocos de sus individuos que habían logrado educarse o adquirir otra clase
de méritos, habían ocupado puestos honrosos en la administración pública,
en la carrera militar y en el sacerdocio (Ancona~1889, pp.~14\textendash{}15).
\end{quotation}

Por ello, considera que las causas pueden explicar la insurrección, pero
nunca justificarla,  porque el objetivo inicial  de la guerra, bajo el
mando de Cecilio Chí, fue exterminar a todos los individuos que no
pertenecieran a la raza blanca con el objeto de que los descendientes mayas
fueran los dueños absolutos de todo.  


\begin{quotation}
(\ldots) los indios  se arrojaron repentinamente sobre las casas de todos los
vecinos que no pertenecían a su raza, y cumpliendo con las órdenes de su
sanguinario jefe, asesinaron sin piedad a blancos, mestizos y mulatos,
perdonando solamente a algunas mujeres para saciar su concupiscencia
(\textit{ibid.}, p.~24)
\end{quotation}

Además, Ancona pinta con detalle escenas que caracterizan al indígena
sublevado como sanguinario, bárbaro y salvaje, a decir de su propia pluma:
\enlargethispage{1\baselineskip}


\begin{quotation}
Hubo un hecho, sobre todos, que con razón excitó la indignación general.
Habiendo ocupado los indios el rancho Yaxché, á ocho leguas de Tihosuco,
sorprendieron en él a la Sra. D. Dolores Padrón, dueña de la finca, y á una
hija suya; les robaron sus alhajas y dinero, las atacaron, las desnudaron y
cometieron con ella todo género de excesos. A los gritos que daba la
desdichadas, acudió un adolescente, hijo de la primera, á quien los indios
derribaron, desde luego, dándole un fiero machetazo en la  cabeza. La Sra.
Patrón y su hija intentaron aplacar a los asesinos, pero éstos las mandaron
 callar, y arrojándose sobre el  joven, que todavía se agitaba en el suelo
con las últimas convulsiones de la agonía, le abrieron el pecho de una
puñalada, como habría hecho un sacerdote maya con la victima destinada al
sacrificio, le arrancaron el corazón y bebieron con salvaje alegría la
sangre que brotaba con abundancia de su heridas (\textit{ibid.}, p.~41).
\end{quotation}

Es  bajo la narración que podemos comprender por qué Eligio Ancona se añade
al grupo de los blancos y enfatiza el aspecto racial de la guerra.
Suscitada la guerra, Ancona era un adolescente y, sin lugar a dudas, que
resultó afectado por los efectos de este movimiento, tanto así que su
crítica es determinante y condenatoria.

 
Se suma a la anterior la obra \textit{Historia de Yucatán desde la
independencia de España hasta la época actual} de  Juan Francisco Molina
Solís, quien se destacó como periodista, historiador y abogado.Nació en 
Hecelchakán, Campeche, en el año de 1850, en una familia que era
reconocida en la alta sociedad yucateca. Estudió en la Escuela mixta,
dirigida por doña María Nájera y en la escuela primaria de don Faustino
Franco e ingresó al Colegio Comercial en la ciudad de Campeche y al Colegio
Católico  de la ciudad de Mérida y, posteriormente, a la Escuela de
Jurisprudencia, obteniendo su título como abogado en 1874. Ocupó cargos
importantes en la administración como Juez de Distrito interino, diputado
suplente a la Legislatura local, Magistrado Supernumerario de los H.H.
Tribunales Superiores de Justicia del Estado (Martínez~2010, pp.~337--338). 

\enlargethispage{1\baselineskip}
Molina Solís es considerado como un
importante~\href{http://es.wikipedia.org/wiki/Mayista}{\textstyleInternetLink{\textcolor[rgb]{0.0,0.0,0.039215688}{mayis}}}\href{http://es.wikipedia.org/wiki/Mayista}{%
%Citar la referencia de internet
%Autor desconocido
%16 de agosto de 2014 20:03
}\href{http://es.wikipedia.org/wiki/Mayista}{\textstyleInternetLink{\textcolor[rgb]{0.0,0.0,0.039215688}{ta}}} 
por su interés en el estudio de los
textos~\href{http://es.wikipedia.org/wiki/Lengua_maya}{\textstyleInternetLink{{mayas}}}{,
y su dedicación por conocer y hablar la lengua maya.} Se 
reconoce y valora su escritura de la historia por el rigor y lo sistemático
de sus investigaciones históricas. Sus escritos sobre Yucatán implicaron
una investigación que lo llevó
al~\href{http://es.wikipedia.org/wiki/Archivo_de_Indias}{\textstyleInternetLink{ {Archivo
de Indias}}}~{en}~\href{http://es.wikipedia.org/wiki/España}{\textstyleInternetLink{ {España}}}  {.}


Su texto \textit{Historia de Yucatán desde la independencia de España hasta
la época actual} representa el trabajo más importante, ya que presenta un
juicio cercano a la posible realidad histórica que refigura, adoptando una
postura imparcial frente a los acontecimientos que describe. Es una obra
clásica y de consulta obligatoria: en el primer tomo, analiza  la
estructura social y económica de la península antes de la guerra  de
castas, ubica al indígena entre las pugnas entre centralistas y liberales,
reseña la toma de Valladolid, la rebelión de Cecilio Chi, la conspiración
de Bonifacio Novelo y Jacinto Pat;  es en el segundo tomo, donde  estudia
detalladamente la guerra de castas y los problemas políticos inherentes a
la sublevación.

Para Molina Solís, la guerra representó el trastoque social que arruinó a
Yucatán, cito:


\begin{quotation}
Y no fue sino hasta julio del dicho año que se tuvo la prueba palpitante de
que una conspiración se tramaba por individuos mayas de oriente con el
objeto de arruinar el orden social, sustituyéndole con otro en que el
gobierno quedase en manos de rebeldes (Molina~1921, p.~4).
\end{quotation}

Condena la guerra por las consecuencias que de ella se generaron:

\begin{quotation}
La verdad neta es que en la noche del 30 de julio de 1847,  cuando menos
esperado era (Cecilio Chí), se abalanzó como un tigre fiero sobre 
inocentes víctimas, y allanando con su gente casas y edificios, asesinó a
cuantos de raza distinta a la suya estaban domiciliados en Trapiche: hombres
y mujeres, niños y ancianos, todos inertes cayeron bajo el terrible
machete. Se dice que fueron todos quemados vivos, dentro de sus casas
(\textit{ibid.}).
\end{quotation}
%\newpage
 
La lucha se diseñaba con sus verdaderos colores: no era la guerra civil: era
la guerra de incendio, de robo, de matanza, de exterminio\linebreak (\textit{ibid.}, p.~14).

Cada uno de los autores a que hemos hecho referencia, nace, crece y se desarrolla en el
siglo XIX, bajo la influencia del liberalismo decimonónico y fueron
testigos de la suscitada Guerra de Castas; su contexto social, político,
económico, les permitió plasmar lo que vivieron. Su formación educativa
 ---como abogados--- les permitió escalar a posiciones importantes de la
estructura social, así como les permitió incursionar en el periodismo ---el
medio más utilizado en esos años--- para promover y difundir su ideología,
sus creencias. Por  ello, la escritura y la narración está plagada de
intereses y prejuicios personales. 


Es una historiografía eminentemente política, toda vez que los que la
escriben ---como es el caso de nuestros autores--- ocupan un puesto clave e
importante en la sociedad que les tocó vivir, un cargo administrativo en el
gobierno de estado; su actuación política y su investidura les permitió
escribir para ser leídos y escuchados, no sólo en el entorno de la
península de Yucatán sino también fuera de los espacios territoriales.

Las obras estudiadas que tratan sobre la guerra social en Yucatán defienden
a la clase  blanca, y aunque reconocen las causas por la que se gestó, no
conciben a la clase indígena como poseedora del poder político; razón por
la que, siguiendo sus propios intereses personales y de grupo, se
circunscriben a la narración de la lucha por el poder en la política
regional. Es una representación que hemos llamado la Visión de los
blancos.
%\enlargethispage{-1\baselineskip}
%\newpage 

\enlargethispage{1\baselineskip}
Preciso es mencionar que el discurso de esta época se caracteriza por poseer
una escritura lineal, cronológica, con diversidad de géneros literarios
utilizados para llegar al público lector, busca (pasando la mitad del siglo
XIX), establecer la verdad y la objetividad en la historia; y para ello
utiliza la documentación como prueba de lo consideran  que realmente
sucedió.

Los textos analizados se valoran por su importancia, ya que fueron las
primeras versiones que se escribieron en torno al estallido de la guerra,
por testigos que vivieron el momento histórico reseñado. Fueron, por así
decirlo, las primeras tesis del acontecimiento.

La historia de la Guerra de Castas ha sido escrita desde el espacio
territorial de Yucatán, quedando muy poco estudiado lo que aconteció fuera
de sus límites territoriales; este es precisamente el caso de Campeche. Es
por ello, que en la historiografía existente en Campeche, en el siglo XIX, 
encontramos muy poca información sobre esa temática.  Lo revisado y hasta
aquí analizado nos permite plantear que los escritos en función a la Guerra
de Castas en Campeche fue manejado de una manera parcial, favorable, para
el grupo en el poder. Los que escribieron en aquella época  formaban parte
importante del grupo elitista en el estado y fueron los autores que ---de una
u otra forma--- participaron en ese proceso histórico, quienes tuvieron un
enfoque plagado de prejuicios e influencias de su propio horizonte y del
contexto que les tocó vivir.

Con lo anteriormente planteado podemos notar que la historiografía
campechana carece de obras que nos permitan comprender y entender el
proceso conocido como Guerra de Castas. Surgen diferentes interrogantes
alrededor de esta temática que resultarían relevantes investigar, y que
queda como invitación abierta para las nuevas generaciones de historiadores
campechanos.
\newpage

%\bigskip
\textbf{Referencias}

Ancona, Eligio (1889), \textit{Historia de Yucatán, desde la época más
remota hasta nuestros días, Parte Cuarta: Guerra Social}, Barcelona,
Imprenta de Jaime Jesús Roviralta.

Aznar Barbachano, Tomás y Juan Carbó Álvarez (1994),
\textit{Memoria sobre la conveniencia. Utilidad y
necesidad de erigir constitucionalmente en estado de la confederación
mexicana el antiguo distrito de Campeche, constituido de hecho en Estado
Libre y Soberano desde mayo de 1858, por virtud con los convenios de
división territorial que celebró con el estado de Yucatán, de que era
parte}, Campeche, Congreso del Estado de Campeche.

Baqueiro Preve, Serapio (1990), \textit{Ensayo histórico
sobre las revoluciones en
Yucatán}, tomo II y III, Mérida, Universidad Autónoma de Yucatán.

Dumond, Don E. (2005), \textit{El machete y la cruz. La sublevación de
campesinos en Yucatán}, México, Universidad Nacional Autónoma de México,
Plumsock Mesoamerican Studies, Maya Educational Foundation.


Martínez Alomía, Gustavo (2010), \textit{Historiadores de Yucatán, apuntes
biográficos y bibliográficos de los historiadores de la península desde su
descubrimiento hasta fines del siglo XIX}, Campeche, Gobierno del Estado de
Campeche.

Molina, Ludy V. (1992),
<<La imagen del indio maya en los historiadores yucatecos del siglo
XIX>>. \textit {Mayab}, núm. 8, pp. 183--191.

Molina, Silvia (1981), \textit {Ascensión Tun}, México, INBA\slash{}Martín
Casillas editores.
\newpage

Molina Solís, Francisco J. (1921), \textit{Historia de Yucatán desde la
independencia de España hasta la época actual}, tomo II,  Mérida, Talleres de la
Compañía Tipográfica Yucateca.

Pérez Galaz, Juan de Dios (1979),
\textit {Diccionario
geográfico,  histórico y biográfico de
Campeche}, Campeche, Gobierno del estado de Campeche. 

\begin{sloppypar}
Sierra Brabatta, Carlos (1991), \textit{Diccionario biográfico de Campeche},
Campeche, Gobierno del Estado de Campeche.
\end{sloppypar}

Sierra O´Reilly, Justo (1994), \textit{Los
Indios de Yucatán. Consideraciones históricas sobre la influencia del
elemento indígena en la organización social del
país}, tomo I, Mérida, Universidad Autónoma de Yucatán.
\newpage
\thispagestyle{empty}
\phantom{abc}

%\documentclass{article}
%\usepackage{amsmath,amssymb,amsfonts}
%\usepackage{fontspec}
%\usepackage{xunicode}
%\usepackage{xltxtra}
%\usepackage{polyglossia}
%\setdefaultlanguage{spanish}
%\usepackage{color}
%\usepackage{array}
%\usepackage{hhline}
%\usepackage{hyperref}
%\hypersetup{colorlinks=true, linkcolor=blue, citecolor=blue, filecolor=blue, urlcolor=blue}
%\newtheorem{theorem}{Theorem}
%\title{}
%\author{Laurita}
%\date{2014-06-12}
%\begin{document}
\markboth{}{}
\thispagestyle{empty}
\phantom{abc}
\phantomsection{}
\addcontentsline{toc}{chapter}{PARTE III.\  LOS MODELOS Y PROGRAMAS\newline DE TUTORÍAS EN LAS 
LICENCIATURAS\newline DE HISTORIA} 


\vspace{0.35\textheight}
{\centering \bfseries Parte III\par}
{\centering \bfseries LOS MODELOS Y PROGRAMAS DE TUTORÍAS\newline EN LAS LICENCIATURAS DE HISTORIA\par}
\markboth{}{}
\thispagestyle{empty}	
\cleardoublepage{}
%%\clearpage\setcounter{page}{379}
\thispagestyle{empty}
\phantomsection{}
\addcontentsline{toc}{chapter}{El Modelo Institucional de Tutorías en la\newline Universidad Autónoma de
Aguascalientes y sus\newline implicaciones en la Licenciatura en Historia\newline $\diamond$
\normalfont\textit{Laura Elena Dávila Díaz de León}}
{\centering {\scshape \large El modelo institucional de tutorías en la Universidad Autónoma de
Aguascalientes y sus implicaciones en la Licenciatura en Historia}\par}
\markboth{la formación del historiador}{modelo institucional de tutorías}
\setcounter{footnote}{0}

\bigskip
\begin{center}
{\bfseries Laura Elena Dávila Díaz de León}\\
{\itshape
Universidad Autónoma de Aguascalientes}
\end{center}

\bigskip
{\bfseries Resumen}

En este trabajo se analizan los postulados e implicaciones del Modelo
Institucional de Tutorías propuesto para la Universidad Autónoma de
Aguascalientes  y las implicaciones que ha tenido su implementación en los
estudiantes de la Licenciatura en Historia.

Se analiza en tres apartados por un lado se examinan los postulados del
Modelo propuesto por la Dirección General de Docencia de Pregrado, en
segundo lugar se revisa la implementación  para los estudiantes de la
Licenciatura en Historia y finalmente se exploran los resultados obtenidos.

\medskip
{\bfseries Abstract}

In this paper we analize the assumptions and implications of the
Institutional Model  of Tutorials proposed for the Autonomous University of
Aguascalientes and the implications of its implementation on the students
of the BA in History.

This work is discussed in three sections: on one side the model postulates
are examined, on a second place the implementation for students in the
Bachelor of History is reviewed and finally the results are explored.


\medskip
{\bfseries Introducción}

Los cambios en el contexto en donde opera la Educación Superior están
obligados a redirigir el tipo de ofrecimiento que sobre orientación
académica y apoyo al estudiante ofrecen las instituciones universitarias.
Los modelos de tutoría académica y los servicios de orientación a los
estudiantes quedan así sujetos a revisión en las instituciones.

El aprendizaje a lo largo de la vida aparece como una necesidad si se quiere
garantizar una mayor profesionalización y competitividad de los estudiantes
en un mundo globalizado, cambiante y tecnificado. Ello exige una educación
relacionada con el contexto, una nueva oferta formativa de acuerdo con las
demandas sociales y, en definitiva una nueva forma de enseñar y aprender
basada en competencias personales y profesionales (ligadas a la capacidad
de resolver problemas profesionales propios de su ámbito de actuación y
concebidos como estructuras dinámicas que se desarrollan a lo largo de la
vida) que la nueva situación demanda. 

En este trabajo se analizan los postulados e implicaciones del Modelo
Institucional de Tutorías propuesto para la Universidad Autónoma de
Aguascalientes  y las implicaciones que ha tenido su implementación en los
estudiantes de la Licenciatura en Historia.

Se analiza en dos apartados por un lado se examinan los postulados del
Modelo propuesto por el Departamento de Formación Integral de la Dirección
General de Docencia de Pregrado, en segundo lugar se revisa la
implementación para los estudiantes de la Licenciatura en Historia y
finalmente se exploran los resultados obtenidos. 
\newpage

{\bfseries 1. Los Modelos Institucionales de Tutorías\\ en la  UAA}

La tutoría en la UAA inició a partir del año de 1997, a través de un
proyecto de información e instrucción con un grupo de profesores de los
diferentes centros académicos. Para el desarrollo e implementación del
programa se requirió definir el perfil específico del tutor universitario,
y desarrollar metodologías que ofrecieran estrategias para el desempeño
adecuado de los tutores.

El Programa Institucional de Tutoría (PIT) fue aprobado por el H. Consejo
Universitario en su sesión ordinaria de junio de 2003. Para la
implementación de este programa se tomó en cuenta a  la Organización de las
Naciones Unidas para la Educación, la Ciencia y la Cultura (UNESCO), a
través del Informe de la Comisión Internacional sobre Educación para el
Siglo XXI que propone una educación permanente que responda al reto del ser
humano de adecuarse a un mundo que cambia rápidamente. La educación que se
promueve en el ámbito internacional desde la perspectiva del documento
referido implica que las personas a través de la educación puedan <<afrontar
las novedades que surgen en la vida privada y en la vida profesional>>
(Jacques 1997, p.16), a partir de lo cual los rasgos que deben promoverse a
través de la educación son aprender a aprender, para discernir su actuar y
conocer a los sujetos con los cuales se relaciona, así como entender mejor
el mundo del que forma parte.

En México, la Asociación Nacional de Universidades e Instituciones de
Educación Superior (ANUIES), que en el apartado que corresponde a los
<<Programas de las Instituciones de Educación Superior>>, incluye una sección
que atiende al <<Desarrollo Integral de los Alumnos>>, en donde señala que
<<como componente clave para dar coherencia al conjunto, se requiere que las
IES pongan en marcha sistemas de tutoría, gracias a los cuales, los alumnos
cuenten a lo largo de toda su formación con el consejo y el apoyo de un
profesor debidamente preparado>> (ANUIES 2000), la Ley General de
Educación, en el primer numeral de su artículo 7° que dictamina los fines
de la educación, describe que deberá contribuir al desarrollo integral de
cada individuo, para que pueda ejercer de forma plena sus capacidades
humanas.

\enlargethispage{1\baselineskip}
Asimismo, el Plan Nacional de Desarrollo 2007--2012, en el Programa Sectorial
de Educación, en lo referente a la enseñanza superior, manifiesta claras
tendencias hacia el ejercicio de la tutoría, pues señala que se debe:
<<contribuir al impulso de programas de tutoría y de acompañamiento
académico de los estudiantes a lo largo de la trayectoria escolar para
mejorar con oportunidad de aprendizaje y rendimiento académico>> (SEP
2007). También el Programa para el Mejoramiento del Profesorado (PROMEP),
establece en el renglón de dedicación, aquellas actividades que debe
desarrollar el profesor de educación superior, entre ellas se encuentra el
participar en actividades de apoyo a la docencia. Entre estas últimas
actividades, se menciona puntualmente a la <<tutoría y la asesoría a los
alumnos>>.

Cabe señalar que de igual forma, a nivel estatal el Plan Sexenal de Gobierno
del Estado 2010-2016, dentro de sus Estrategias Generales, establece
algunas líneas a las que la tutoría pretende atender, de esta forma en la
número cinco <<Educación de Calidad>> se plantea mejorar la calidad de la
educación de manera que se generen planes enfocados a mejorar el desempeño
de los estudiantes, estando al alcance de todos, <<(\ldots) garantizando \textit{las
condiciones para asegurar la permanencia de los estudiantes dentro del
Sistema Educativo, con la disminución de la deserción y el
mejoramiento en la eficiencia terminal}>> (Plan Sexenal de Gobierno del
Estado 2010, p.182).

En todas estas referencias, es observable que coinciden en señalar la
importancia estratégica de impulsar y operar una educación integral en las
instituciones de educación superior, siendo la tutoría una estrategia de
apoyo útil para favorecer el desarrollo de los estudiantes, para ello se
hace necesario que el rol que ha desempeñado el docente del nivel superior
pase de ser un transmisor de conocimientos, a transformarse en un
facilitador, tutor o asesor del alumno, a fin de que alcance una formación
que le prepare para un desenvolvimiento acorde con su proyecto de vida.


\medskip
{\bfseries 1.1. El Programa Institucional de Tutorías\\ 2003--2013}

En la Universidad Autónoma de Aguascalientes, desde sus inicios, la
formación integral de sus estudiantes ha sido su principal objetivo. En
este sentido, han sido varias las estrategias adoptadas para coadyuvar a
tal fin, y aunadas al desarrollo de las funciones sustantivas de docencia,
investigación, difusión y extensión apoyadas por actividades
administrativas, en los últimos años se ha desarrollado la tutoría como una
modalidad educativa que tiene como objetivo la atención personalizada del
estudiante en dimensiones como el rendimiento y aprovechamiento académico
en diversas materias curriculares, su incorporación y desarrollo como
estudiante universitario, el desarrollo de estrategias de aprendizaje,
entre otras.

En el 2003 en la sesión ordinaria de junio del H. Consejo Universitario
aprobó el Programa Institucional de Tutoría, el cual presentaba una
metodología acorde con las orientaciones y objetivos de esta casa de
estudios. Para su difusión se celebraron reuniones con las comisiones
académicas y con los grupos de tutores de cada Centro Académico. Asimismo,
se impartieron cuatro cursos de inducción a 16 profesores de reciente
incorporación al programa y se distribuyeron folletos y carteles con
información alusiva al tema. Paralelamente se implementó el Sistema
Electrónico de Información de Tutorías, que permitiera el acceso de
profesores a la información académica de los estudiantes. (Ávila Storer
2003).

En el año 2005 se estableció un procedimiento de registro de tutores, a
través de la elaboración de un Programa de Acción Tutorial (PAT) y un
Reporte Final de Tutoría (RFT), en el Sistema Integral de Información y
Modernización Administrativa (SIIMA), con la intención de dar seguimiento
al trabajo tutorial de los profesores, mismo que incluiría la evaluación a
los docentes. (Urzúa Macías 2005). En este proceso y con el objetivo de
mejorar las estrategias de atención para el estudiante, se diseñó un
instrumento de evaluación, que se aplica al finalizar cada periodo
académico, a través del SIIMA, el cual se realiza de manera electrónica por
parte del Jefe de Departamento Académico al cual pertenece el tutor y por
el Departamento de Orientación Educativa.

En la actualidad, el Programa Institucional de Tutoría responde a las
características de una educación de calidad planteadas en el Modelo
Educativo Institucional (MEI) en donde se concibe a la educación enfocada
al estudiante y su aprendizaje <<ubica al educando como el eje central y
principal protagonista de su quehacer, es más responsable de su propia
educación. La institución, a través de distintas estrategias, le brinda el
apoyo necesario para transitar con éxito en su proceso formativo>> (UAA
2006, p. 7), de esta manera el PIT a través de las acciones y estrategias que
implementa pretende proporcionar la atención personalizada que contribuya a
que el estudiante logre un desempeño académico satisfactorio y además de
ello sea responsable de su propio proceso de formación y aprendizaje.

Dentro de este contexto se espera que la tutoría se considere un logro
conjunto que realizan todos los actores involucrados en la docencia, con un
impacto positivo en los principales indicadores que miden los alcances
institucionales. 

Este programa comprende acciones de identificación, atención y seguimiento
de alumnos, por parte de un profesor tutor que fomente en el alumno su
capacidad crítica y creadora, tomando en cuenta no sólo su rendimiento
académico sino también su evolución social y conocimiento de su
individualidad. 
\enlargethispage{2\baselineskip}

La responsabilidad del Programa Institucional de Tutoría está adscrita al
Departamento de Formación Integral de la Dirección General de Docencia de
Pregrado. En los Centros Académicos existe una Coordinadora General como
responsable. En el Departamento de Historia el PIT se integra por un
Coordinador General y 4 PTCs, uno por cada grupo de la carrera de Historia
y así reasegura la atención de todos los estudiantes.

La eficacia del programa se atiende con la realización de un Plan de Acción
Tutorial inicial y un Reporte final de actividades de los tutores. Cada
tutor dispone de una base de datos y del expediente individualizado de cada
estudiante para el mejor desarrolla de su actividad

En el año 2011 se planteó la necesidad de evaluar el Programa Institucional
de Tutoría para lo que el Departamento de Orientación Educativa en el
semestre Agosto-Diciembre de este año, elaboró y aplicó un instrumento con
la intención de conocer la percepción del estudiante en cuanto al Programa
Institucional de Tutoría, y el desempeño de sus tutores, además de obtener
algunos indicadores cuantitativos y cualitativos. La muestra se conformó
con 421 estudiantes de manera aleatoria controlada de todas las carreras de
los 7 centros\linebreak académicos.

Las conclusiones que arrojó este estudio fueron que el  Programa
Institucional de Tutoría no ha tenido un impacto positivo en el índice de
retención de los estudiantes de nivel licenciatura. La estructura actual de
304 tutores para la atención de 12,133 estudiantes, es decir 40 estudiantes
en promedio por tutor, homologa las características de la tutoría,
dedicando el mismo tiempo de atención y tareas a cada estudiante,
independientemente de su situación académica, empero, es importante señalar
que la población estudiantil no requiere del mismo tipo de atención
tutorial, siendo plausible que en su trayectoria escolar un gran número de
alumnos no requiere de atención especializada, pues tienen la capacidad de
resolver las situaciones que pudieran presentarse. Además, los datos
presentados muestran un total de 1,995 estudiantes de la matrícula que se
encuentran en alto riesgo académico, es decir, podrían causar baja pues
tienen materias en última o penúltima oportunidad.

De lo anterior, se derivó la importancia de implementar un Programa
Institucional de Tutoría que respondiera a las necesidades actuales que
presenta la institución y contribuyera a la retención de los estudiantes,
así como disminuir sus riesgos académicos. Se detectó una necesidad de
capacitación continua enfocada a los tutores que les permitiera identificar
y atender las situaciones problemáticas que atraviesan los estudiantes
durante su trayectoria universitaria. Además, se consideró relevante
definir las funciones y actividades que los tutores deben realizar en pro
de los fines ya mencionados.
\newpage

{\bfseries 1.2. El Programa Institucional de Tutorías\\ 2013--2014}

{\bfseries 1.2.1. Conceptualización de la Tutoría}

En el modelo implementado para el año 2013,  la tutoría es parte esencial de
las actividades de apoyo a la formación integral de los estudiantes de
pregrado, nivel técnico superior universitario y licenciatura, en especial
a aquellos que presentan un perfil de alto riesgo académico. Para algunos
autores, como Ayala citado en Sánchez, Mora y Sánchez (2004), la
connotación de tutoría es ayudar y orientar a un alumno o a un pequeño
grupo de alumnos principalmente en sus actividades relacionadas con el
aprendizaje, ayudarles en la resolución de sus tareas y facilitarles la
localización oportuna y rápida de información.

Desde esta visión, la Universidad Autónoma de Aguascalientes reconoce a la
tutoría como un proceso individual o grupal que se brinda al estudiante, a
través del tutor durante su permanencia en la UAA, con el propósito ofrecer
espacios de apoyo para una trayectoria universitaria guiada y con ello
coadyuvar en el abatimiento de los índices de deserción, reprobación,
rezago y elevar el de la eficiencia terminal, en pro del desarrollo
integral del estudiantado.

Bajo esta perspectiva, el modelo tutorial de la UAA se centra en atender a
los estudiantes y su formación integral, de ahí que se plantee llevar a
cabo una tutoría diferenciada de acuerdo a las características y
necesidades de la comunidad estudiantil; el PIT está enfocado a dos
vertientes:

\begin{Obs}
\item[1)] \textit{Incorporación del estudiante a la vida universitaria}. Se dirige
a los estudiantes desde su ingreso a la Universidad y hasta el cuarto
semestre, a fin de facilitar su proceso de integración a la vida
universitaria y establecer estrategias preventivas y/o correctivas que
eficiente su rendimiento académico, contribuyendo así a reducir los índices
de deserción.

\item[2)] \textit{Seguimiento al proceso de titulación}. Está dirigida a
estudiantes de quinto semestre en adelante, su intención es dar seguimiento
a su proceso de titulación para que se dé en tiempo y forma, brindar
atención a las situaciones académicas que pudieran presentarse y favorecer
el establecimiento de vínculos del estudiante con el sector laboral.
\end{Obs}

\medskip
{\bfseries 1.2.2. Objetivo general}

Apoyar y dar seguimiento a la trayectoria de los estudiantes de pregrado,
nivel técnico superior universitario y licenciatura, a través de la
atención académica y de los servicios educativos institucionales para
incidir positivamente en los índices de rezago, reprobación, deserción y
eficiencia terminal, elevando así la calidad de los procesos formativos a
favor del desarrollo integral de los universitarios.


\medskip
{\bfseries 1.2.3. Objetivos particulares}

\begin{Obs}
\item[$\rhd$] Contribuir en el proceso de integración y adaptación de los estudiantes a
la actividad universitaria.
\item[$\rhd$] Orientar al estudiante en problemáticas académicas y\slash{}o personales que
resulten durante el proceso formativo, y de ser necesario canalizarlo a las
instancias especializadas para su atención.
\item[$\rhd$] Fomentar una actitud positiva del estudiante hacia su autoaprendizaje.
\item[$\rhd$] Orientar a los estudiantes sobre el ámbito socio-laboral con el propósito
de facilitar su integración.
\item[$\rhd$] Promover en los alumnos la búsqueda de información oportuna que les
permita la toma de decisiones académicas, el uso adecuado de los servicios
educativos y apoyos institucionales, así como la realización de trámites y
procedimientos acordes a su situación académica.
\item[$\rhd$] Orientar, apoyar y dar seguimiento al estudiante en el cumplimiento
oportuno de los requisitos de titulación así como en otros procesos de
egreso.
\end{Obs}

\medskip
{\bfseries 1.2.4. Justificación}

Entre los problemas más complejos y frecuentes que enfrentan las
Instituciones de Educación Superior (IES) del país, en el nivel pregrado,
nivel técnico superior universitario y licenciatura, se encuentran la
deserción, el rezago estudiantil y los bajos índices de eficiencia
terminal. Tal situación implica que las IES enfoquen su atención en
favorecer el rendimiento académico de los estudiantes con el fin de
disminuir la reprobación y la deserción e incrementar así los índices de
eficiencia terminal, de acuerdo con las demandas sociales.

Con este panorama, impera la implementación de objetivos estratégicos en la
operación de un Programa Institucional de Tutoría en la Universidad
Autónoma de Aguascalientes acorde con el Marco Institucional de Formación
Integral (UAA 2011), que articule los esfuerzos institucionales para lograr
el desarrollo académico oportuno y pertinente de los estudiantes; en ese
sentido, la tutoría será una de las estrategias claves para apoyar y dar
seguimiento al desempeño de los universitarios y al cumplimiento de los
requisitos de titulación.

Asimismo, la tutoría en la institución, es fundamental para contribuir al
logro de las metas establecidas a nivel internacional y nacional para la
educación superior, como son igualdad, equidad y cobertura, si se toma en
cuenta que el promedio nacional de estudiantes que inician licenciatura es
que de cada 100, entre el 50 y 60 logran concluir las asignaturas de su
plan de estudios y tan sólo 20 obtienen tu título (ANUIES 2000).

Por lo tanto, el Programa Institucional de Tutoría, apoya en la resolución
de las problemáticas educativas y se identifica como parte fundamental de
los procesos de integración educativa.


\medskip
{\bfseries 1.2.5. Modalidades de tutoría}

La tutoría en la Universidad Autónoma de Aguascalientes, es de carácter
individual, grupal, presencial, virtual, o de pares, de acuerdo a las
características de los estudiantes que se atienden en cada centro académico
y programa educativo correspondiente.


\medskip
{\bfseries 1.2.6. Actores del Proceso Tutorial}

\medskip
{\bfseries 1.2.6.1. Tutor}
\enlargethispage{-1\baselineskip}

Para la Real Academia Española (2012), el tutor es la persona encargada de
orientar a los alumnos de un curso o asignatura. Si se parte de ese
concepto, se identifica la parte esencial de la labor tutorial: orientar.
Ampliando este premisa, encontramos que el tutor establece una relación
pedagógica diferente a la que establece la docencia ante grupos numerosos
(Canales 2004) y que asume de manera individual la guía del proceso
formativo del estudiante y está permanentemente ligado a las actividades
académicas, orientando, asesorando y acompañando al mismo durante el
proceso educativo con la intención de conducirlo hacia su formación
integral, estimulando su responsabilidad por aprender y alcanzar sus metas
educativas (ANUIES 2000).

En los conceptos anteriores, se ensalza la figura del tutor en la formación
de universitarios y se reconoce de manera significativa el proceso de
acompañamiento pedagógico formal que éste le brinda al estudiante tutorado.
Lo mismo sucede con los tutores de la Universidad Autónoma de
Aguascalientes, quienes son esenciales para la formación integral del
estudiante.


\medskip
{\bfseries 1.2.6.1.1. Perfil del tutor}

Si bien la tutoría es una de las actividades inherentes a las funciones
docentes, es importante recalcar que para ser tutor, también es necesario
conformar un perfil; Canales (2004) considera principalmente tres aspectos:

\begin{Obs}
\item[$\star$] Cualidades Humanas: se refieren a la definición del ser del tutor como la
empatía, la madurez, el compromiso y la responsabilidad con la labor.

\item[$\star$] Cualidades Científicas: se refieren al saber del tutor, es decir, que
tenga conocimientos disciplinares que le permitan desempeñarse como tal,
principalmente, en los campos de la sociología, economía, pedagogía y
filosofía.

\item[$\star$] Cualidades Técnicas: definen el saber hacer del tutor, como la
organización, la elaboración de diagnósticos, conocimiento de técnicas de
motivación y de estudio.
\end{Obs}

\medskip
{\bfseries 1.2.6.1.2. Funciones generales del tutor}
\enlargethispage{1\baselineskip}

El Programa Institucional de Tutoría busca contribuir en la formación
integral de los estudiantes, por ello, sus funciones están estrechamente
ligadas a propósitos de transformación institucional y comprenden:
\newpage

{\itshape Favorecimiento del desarrollo personal}

\begin{Obs}
\item[$\diamond$] Orientar al estudiante en la adaptación a la vida universitaria.
\item[$\diamond$] Promover actividades que apoyen la formación integral del estudiante.
\item[$\diamond$] Promover en el estudiante la toma de decisiones autónomas y responsables.
\end{Obs}

\medskip
{\itshape Apoyo académico}

\begin{Obs}
\item[$\diamond$]  Atender la problemática relacionada con el proceso de
enseñanza-aprendizaje.
\item[$\diamond$] Promover el aprendizaje autónomo de los estudiantes.
\item[$\diamond$] Monitorear la trayectoria académica del estudiante.
\item[$\diamond$] Monitorear el cumplimiento de los requisitos de titulación.
\end{Obs}

%\enlargethispage{1\baselineskip}
\medskip
{\itshape Desarrollo profesional}

\begin{Obs}
\item[$\diamond$] Orientar a los estudiantes en la identificación de sus objetivos
profesionales y promover una actitud ética.
\item[$\diamond$] Orientar a los estudiantes en lo relacionado con los requerimientos para
la inserción al campo laboral.
\item[$\diamond$] Orientar a los estudiantes en la transición de la Universidad al trabajo.
\end{Obs}

\medskip 
{\bfseries 1.2.6.2. Estudiante}


Considerando los planteamientos del Modelo Educativo Institucional (UAA
2006), se concibe a los estudiantes de esta institución como seres humanos
en formación, en constante crecimiento y desarrollo de capacidades que
pueden desarrollar los valores de la cultura humanista lo que les permite
entender el mundo que les ha tocado vivir e influir positivamente en éste.
Se caracterizan por ser reflexivos y críticos, capaces de interactuar con
los contenidos de aprendizaje y lograr los objetivos de manera exitosa,
además de ser constructores de su propio conocimiento al seleccionar,
elaborar, organizar, utilizar y dar significado a la información para
actuar en su entorno, a través de las variadas tareas que han sido
diseñadas para ellos.

Sin embargo, vale la pena señalar que el PIT se enfoca por un lado a los
estudiantes antes descritos, pero que en un primer momento está dirigido a
aquellos alumnos que por presentar una situación de rezago, irregularidad o
de riesgo académico, recibe apoyo y orientación por parte de un tutor
durante los primeros cuatro semestres de la carrera, a través de
actividades que le permitirán su desarrollo integral.

\medskip
{\bfseries 1.2.7. Operatividad del programa institucional\\ de tutoría}
%\enlargethispage{1\baselineskip}

Tal como indica el Estatuto de la Ley Orgánica de la Universidad Autónoma de
Aguascalientes, la Dirección General de Servicios Educativos, a través del
Departamento de Orientación Educativa, coordinará el Programa Institucional
de Tutoría como parte de las actividades que promueven la formación
integral del estudiante universitario, en conjunto con los Centros
Académicos. Por lo tanto, la operatividad general de este programa se
describe a continuación:

\begin{Obs}
\item[1)] Designación de los tutores por parte del Centro Académico
\item[2)] Capacitación de los tutores
\item[3)] Implementación de la acción tutorial
\item[a.] Funciones de los tutores en cada vertiente
\item[b.] Elaboración del Plan de Acción Tutorial
\item[c.] Elaboración del reporte final
\item[4)] Evaluación del PIT
\end{Obs}


\medskip
{\bfseries 1.2.7.1. La designación de tutores}

Los Decanos y Jefes de Departamento de los Centros Académicos serán los
responsables de realizar la designación de los tutores, esto dependerá de
la población estudiantil de cada programa educativo, siguiendo la acción
tutorial requerida para cada semestre y las necesidades propias del
alumnado.


\medskip
{\bfseries 1.2.7.2. Capacitación de tutores}

\enlargethispage{1\baselineskip}
Es indispensable establecer un proceso de capacitación y actualización con
los profesores que ejercerán las funciones de tutoría con el propósito de
que puedan dar cumplimiento con los objetivos del PIT. Esto debido a que,
en la mayoría de los casos, los profesores no poseen una formación que
contribuya apropiadamente al desempeño de las funciones tutoriales, por lo
cual resulta indispensable brindar oportunidades para conocer y desarrollar
habilidades, que les permitan intervenir adecuadamente en esta situación y
apoyar al desarrollo integral de los estudiantes y a la atención
individualizada durante el proceso formativo, que exige al tutor una
capacitación permanente en modelos centrados en el alumno y el aprendizaje.

La capacitación de profesores se plantea con el objetivo de proporcionar
conocimientos y herramientas para el ejercicio de la tutoría, en los tres
aspectos del perfil del tutor: ser, saber y saber hacer. Para ello se
establecen tres áreas de formación y actualización:

\begin{Obs}
\item[{\bf Área básica:}] busca desarrollar los conocimientos, habilidades y
actitudes básicas e indispensables que el tutor de la UAA requiere para el
desempeño de sus funciones, esta área se dirige a profesores noveles en el
ejercicio de la tutoría. Se enfoca a que los profesores identifiquen la
estructura del PIT, sus objetivos, sus metas y las funciones que deben
desarrollar.

\item[{\bf Área intermedia:}] Brindar cono\-cimien\-tos \ y \ de\-sa\-rro\-llar \ ha\-bi\-li\-da\-des \ que le faci-li\-ten al tutor contribuir en los aspectos que se refieren a la
formación integral del estudiante; se refiere a un espacio para profundizar
en las tendencias, necesidades e innovaciones de la tutoría.

\item[{\bf Área especializante:}] Esta área ofrecerá a los tutores
metodologías para la elaboración de diagnósticos y evaluaciones de
problemas especiales que incidan en el rendimiento académico de los
estudiantes y poder asesorar adecuadamente. Cabe señalar que en esta área
corresponde a los Jefes de Departamentos Académicos realizar la detección
de necesidades, para ser comunicados y canalizados en cursos especiales a
través del departamento de Orientación Educativa.
\end{Obs}

\medskip
{\bfseries 1.2.7.3. Implementación de la acción tutorial}
\enlargethispage{1\baselineskip}

Para la ANUIES, la acción tutorial es la <<ayuda u orientación que ofrecen
los profesores-tutores a los alumnos en un centro educativo, organizados en
una red o equipo de tutorías. Se concreta en una planificación general de
actividades, una formulación de objetivos y en una programación concreta y
realista>> (ANUIES 2000). Así, atendiendo al modelo tutorial de la UAA,
cada tutor tendrá sus propias funciones dependiendo de la vertiente de
tutoría en la que corresponda su actuación, de ahí que la asignación a cada
tutor de sus funciones específicas sea básica para realizar adecuadamente
este proceso.

Considerando que en la primera vertiente de la tutoría denominada
\textit{incorporación del estudiante a la vida universitaria} se dirige a
brindar atención de estudiantes en los primeros semestres de ingreso a la
UAA y a aquellos con perfil de alto riesgo académico con el fin de
evitar la reprobación y la deserción, las funciones específicas del tutor
son:

\enlargethispage{1\baselineskip}
\begin{Obs}
\item[$\bullet$] Establecer un diagnóstico o perfil general sobre la situación de los
estudiantes, atendiendo a lo establecido para el Plan de Acción Tutorial
(PAT).
\item[$\bullet$] Detectar a estudiantes con perfil vulnerable y de alto riesgo.
\item[$\bullet$] Canalizar a las instancias correspondientes a los estudiantes que así lo
requieran
\item[$\bullet$] Brindar información sobre los servicios universitarios disponibles para
los estudiantes, así como notificar sobre eventos y actividades
universitarias.
\item[$\bullet$] Apoyar a los estudiantes en la identificación de su plan de estudios
(materias, campo laboral, requisitos de titulación, entre otros).
\item[$\bullet$] Trabajar actividades de socialización e intercambio de experiencias entre
los estudiantes.
\item[$\bullet$] Identificar a los estudiantes idóneos para impartir la tutoría de pares y
dar seguimiento a este proceso.
\item[$\bullet$] Orientar y dar seguimiento al cumplimiento del Programa Institucional de
Formación Humanista.
\item[$\bullet$] Orientar y dar seguimiento al cumplimiento del Programa Institucional de
Lenguas Extranjeras.
%\newpage
\item[$\bullet$] Organizar estrategias de apoyo y asesoría académica para los estudiantes
irregulares.
\item[$\bullet$] Responder a las dudas e inquietudes de los estudiantes sobre situaciones
académicas particulares, ya sea de manera individual, grupal o virtual.
\item[$\bullet$] Elaborar el reporte final del Programa Institucional de Tutoría.
\item[$\bullet$]Mantener constante comunicación con el Centro Académico y la Dirección
General de Servicios Educativos.
\end{Obs}

En la otra vertiente, sobre el \textit{seguimiento al proceso de titulación} que s
e dirige a estudiantes de quinto semestre en adelante, para atender a
aquellas situaciones enfocadas al proceso de titulación y contribuir al
establecimiento de vínculos que favorezca la inserción laboral del futuro
profesionista; para ello los tutores tienen las siguientes funciones:

\begin{Obs}
\item[$\bullet$] Definir las actividades a realizar de acuerdo a lo establecido para el
Plan de Acción Tutorial (PAT), para dar cumplimiento al PIT.
\item[$\bullet$] Dar seguimiento a los estudiantes con alto riesgo académico y con
situación irregular.
\item[$\bullet$] Canalizar a las instancias correspondientes a los estudiantes que así lo
requieran.
\item[$\bullet$] Orientar y dar seguimiento al cumplimiento del Programa Institucional de
Servicio Social
\item[$\bullet$] Orientar y dar seguimiento al cumplimiento del Programa Institucional de
Prácticas Profesionales
\item[$\bullet$] Identificar a los estudiantes idóneos para impartir la tutoría de pares y
dar seguimiento a este proceso.
\item[$\bullet$] Implementar una diversidad de estrategias para acercar a los estudiantes
a la realidad laboral.
\item[$\bullet$] Responder a las dudas e inquietudes de los estudiantes sobre situaciones
académicas particulares, ya sea de manera individual, grupal o virtual.
\item[$\bullet$] Elaborar el reporte final del Programa Institucional de Tutoría
\item[$\bullet$] Mantener constante comunicación con el Centro Académico y la Dirección
General de Servicios Educativos
\end{Obs}

Cabe señalar que la acción tutorial no es exclusiva de algún tipo de
modalidad de la tutoría, sólo se privilegia la individual para los perfiles
vulnerables y del quinto semestre en adelante, el estudiante podrá
solicitar atención individual por parte del tutor, a quien le corresponderá
orientar y/o canalizar a las instancias de apoyo institucionales
correspondientes.


\medskip
{\bfseries 1.2.7.4. Elaboración del Plan de Acción Tutorial}

Uno de los factores que permite que la acción tutorial refleje el impacto y
los resultados esperados, es que se trata de una actividad que no debe
realizarse de manera espontánea o casual, se requiere de un trabajo de
reflexión, análisis y visión para planear, organizar y realizar esta
actividad formativa.

\enlargethispage{1\baselineskip}
Dentro de la acción tutorial la planeación es una estrategia que permite
visualizar y sistematizar lo que se pretende alcanzar, a través de las
diversas acciones grupales e individuales que deberán responder a las
necesidades detectadas a partir del diagnóstico realizado por el tutor de
grupo. El Plan de Acción Tutorial se encuentra integrado por los elementos
que a continuación se mencionan:

\begin{Obs}
\item[$\bullet$] {\slshape Datos generales del grupo tutorado:} Se identifican el grupo,
semestre y programa educativo del nivel técnico superior universitario y
licenciatura, correspondiente.
\item[$\bullet$] {\slshape Diagnóstico o perfil general de grupo tutorado:} En el primer
semestre, se realizará a partir de la detección de los factores relevantes
que se obtengan de los resultados del Examen Médico Automatizado, EXANI II,
promedio de bachillerato y EXHCOBA del grupo; el tutor destacará aquellos
aspectos a los cuales es necesario atender y, en caso de ser necesario,
realizará las canalizaciones a las instancias universitarias
correspondientes.

A partir del segundo y hasta el último semestre, se realizará a partir de
los resultados de la evaluación del trabajo realizado durante los semestres
anteriores, con el fin de ofrecer seguimiento a los factores de atención.
También se considerarán relevantes en el perfil del grupo, el seguimiento a
los requisitos de titulación, el desempeño escolar de los estudiantes y
procurar relacionar a los estudiantes con el campo laboral a través de
actividades específicas:
%\enlargethispage{1\baselineskip}
\item[$\bullet$] {\slshape Planeación de las actividades:} El tutor, a partir del
diagnóstico o perfil general del grupo, deberá registrar aquellas
estrategias tutoriales a llevar a cabo: sesiones grupales, individuales,
registro de eventos, conferencias, congresos o actividades que fomenten la
formación integral del estudiante y que estén programados para el semestre
escolar, con el fin de conminarlos a participar. Asimismo, organizar las
sesiones para orientar e informar acerca del seguimiento a los requisitos
de titulación, de continuidad con su integración a la vida universitaria o
de incorporación al sector laboral.
\end{Obs}
%\newpage

Los elementos anteriores se podrán integrar gracias a las orientaciones de
la acción tutorial, a las modalidades de la tutoría y a la capacitación
constante del tutor.
\newpage

\medskip
{\bfseries 1.2.7.5. Elaboración del Reporte Final}

Los resultados obtenidos y el logro de las metas planteadas al inicio, así
como su evaluación general serán registrados en un apartado del Pan de
Acción Tutorial al finalizar el semestre reportado; los elementos a
considerar son los siguientes:

\begin{Obs}
\item[$\star$] \textit{Resultados cuantitativos}. Se reportarán número de estudiantes regulares,
en cumplimiento de los requisitos de titulación, con situaciones de alerta
académica o personal.
\item[$\star$] \textit{Resultados cualitativos}. Se reportarán resultados del proceso tutorial,
de las actividades planeadas y de la autoevaluación de la tutoría.
\end{Obs}

El Reporte final, se entregará al finalizar el semestre, al jefe de
Departamento Académico correspondiente, así como a su Decano y a la
Dirección General de Servicios Educativos, con el fin de realizar las
acciones pertinentes que a cada área competen, a favor de la optimización
de recursos y actualización de la tutoría en beneficio de los estudiantes.


\medskip
{\bfseries 1.2.7.6. Evaluación del PIT}

La evaluación del Programa Institucional de Tutoría se considera un proceso
dinámico, sistemático y permanente de retroalimentación que permite valorar
el cumplimiento del objetivo propuesto, los alcances y el funcionamiento
del programa, para poder así realizar ajustes y adecuaciones para la mejora
continua del programa. Todo esto con miras a obtener insumos para
determinar de forma general el impacto del PIT en los índices de
reprobación, deserción y rezago de los estudiantes.
\enlargethispage{1\baselineskip}

Desde esta perspectiva, y con el fin de lograr los objetivos institucionales
respecto a la calidad en la educación se plantea realizar dentro del
Programa Institucional de Tutoría un esquema de evaluación integrado por
tres grandes procesos:

\enlargethispage{1\baselineskip}
\textit{1) Evaluación del desempeño del tutor:} Entendido como un proceso
formativo que permite al docente valorar su desempeño en la acción
tutorial, a fin de fortalecer su quehacer docente; en este proceso se
pretende el reconocimiento de los logros obtenidos y el mejoramiento de las
áreas de oportunidad detectadas en los tutores. Dicha evaluación se llevará
a cabo a través de la recuperación de las opiniones por parte de los
estudiantes tutorados y de la evaluación que realice el propio Departamento
de Orientación Educativa respecto al cumplimiento de los lineamientos
establecidos. Algunos de los tópicos particulares a evaluar son:

\begin{Obs}
\item[a.] Conocimiento de la normatividad institucional.
\item[b.] Compromiso del tutor con la actividad tutorial.
\item[c.] Disposición para atender a los estudiantes.
\item[d.] Actitud empática en la atención al estudiante.
\item[e.] Capacidad para orientar a los alumnos en decisiones académicas.
\item[f.] Capacidad para orientar al estudiante en metodología y técnicas así como
para resolver dudas académicas.
\item[g.] Satisfacción del estudiante con la relación establecida con el tutor.
\end{Obs}

%\medskip
\textit{2) Evaluación de la planeación del programa:} Considerado como un
proceso de retroalimentación enfocado a la rendición de cuentas y al\linebreak
establecimiento de una cultura de transparencia a través de la presentación
de resultados y la valoración de los objetivos alcanzados, para determinar
el costo-beneficio del propio programa y tomar decisiones respecto a
posibles ajustes.


%\bigskip
\textit{3) Evaluación del impacto del programa:} Se busca recuperar
información respecto a la difusión que se tiene del programa, el índice de
capacitación de tutores, la utilidad de la metodología empleada y el
mejoramiento en los indicadores académicos institucionales.

Con la implementación de estos tipos de evaluación se ofrece la posibilidad
de visualizar de forma holística los resultados que se obtienen de la
implementación del Programa Institucional de Tutoría a fin de obtener
insumos que permitan identificar la eficiencia y pertinencia del programa a
nivel institucional y con ello tomar las decisiones que se consideren
necesarias para el logro de las metas establecidas.


\medskip
{\bfseries 2. Implicaciones para los estudiantes\\ de la Licenciatura en Historia}

El Centro de Ciencias Sociales y Humanidades trabajó durante el año 2013  el
PIT como grupos piloto para su posterior implementación en los diversos
Centros Académicos. En particular en el departamento de Historia se vivió
este proceso de tal manera que nos permitió realizar las siguientes
observaciones de acuerdo al esquema de fortalezas, debilidades y áreas de
oportunidad:

\enlargethispage{1\baselineskip}
{\itshape Fortalezas}

\begin{itemize}
\item Estructura del Programa Institucional de Tutoría en dos vertientes:
Incorporación del estudiante a la vida universitaria (1º--4º semestre) y
seguimiento al proceso de titulación (5º--9º semestre)
\item Información resguardada en SIIMA y con identificación de colores
diferenciados para estudiantes en situación de riesgo.
\item Enfocar atención especial a estudiantes en situación de riesgo.
\item Entrega de 3 reportes que conforman el PAT.
\end{itemize}
%\newpage

{\itshape Debilidades}

\begin{itemize}
\item La entrega de reportes en el SIIMA tiene un límite de palabras y no es
posible completar el reporte.
\item La fecha de entrega de la fase de evaluación del PAT está
completamente desfasada ya que se debe entrega el reporte con anterioridad
a la presentación de exámenes extraordinarios desconociendo el resultado
obtenido por el estudiante lo que impide hacer una valoración del trabajo
del tutor y del estudiante.
\item Tiempo asignado a tutoría de manera  insuficiente para proporcionar
tutoría individualizada a los estudiantes en situación de riesgo ya que es
necesario realizar una estrategia de atención para cada uno de ellos y otra
para el grupo en general. 
\item Cuando el tutor no imparte clase a los estudiantes es muy difícil
localizarlos y en general no asisten a la entrevista individual porque
tienen miedo.
\item Comunicar con tiempo las fechas para la entrega del PAT en sus tres
fases (Diagnóstico, seguimiento y monitoreo y Evaluación)
\end{itemize}

\medskip
{\itshape Áreas de Oportunidad}

\enlargethispage{1\baselineskip}
\begin{itemize}
\item Negociar con el departamento de redes para que \textit{no se cierre el
SIIMA} ya que hay momentos clave para la tutoría de los estudiantes, en
particular al inicio y final del semestre.
\item Ampliar el espacio en el SIIMA para la redacción del diagnóstico y las
estrategias de atención a los estudiantes.
\item Capacitar a las personas responsables de dar apoyo y seguimiento del
PIT adscritas a la Dirección General de Servicios Educativos.
\item Asignar al profesor un tiempo \textit{adecuado y real} para realizar
las actividades necesarias.
\item Coordinarse con el departamento de Control Escolar ya que a los
estudiantes que revalidan materias se les ubica en situación de riesgo
llegando a adeudar hasta 10 materias e incrementando el porcentaje de
estudiantes en esta situación y bajando el promedio general del grupo.
\end{itemize}
%\newpage

%
\bigskip
{\bfseries Referencias}

\medskip
Asociación Nacional de Universidades e Instituciones de Educación Superior
(2000), \textit{Propuesta para la organización e implantación de programas
institucionales de tutoría en las instituciones de educación superior},
México, ANUIES.

Ávila Storer, A. (2000), \textit{Segundo Informe}, México, UAA.

Ávila Storer, A. (2003), \textit{Quinto Informe}, México, UAA.

Canales, E. M. (2004), \textit{El perfil del tutor académico}, Universidad
Autónoma de Tlaxcala\slash{}Universidad Autónoma del Estado de Hidalgo.

Calderón Hernández, J. (1998), \textit{Programa Institucional de Tutoría}.
México, Universidad Autónoma del Estado de Hidalgo.

Castillo y García (1996), El tutor y la tutoría en el modelo UNED. En
García Aretio, Lorenzo (comp.), \textit{La educación a distancia y la
UNED}, Madrid, Universidad Nacional de Educación a Distancia.

Castillo A. Santiago, Torres G. José A, Polanco G. Luis. (2099),
\textit{Tutoría en la enseñanza, la universidad y la empresa}, Madrid,
Pearson Prentice Hall.

Cortés, M.T.P.I. et. al. (1997), El papel del profesor-tutor dentro de los
grupos de alta exigencia académica en la Facultad de Medicina de la UNAM.
Experiencias de profesores y alumnos. Ponencia presentada en el III
Congreso Nacional de Investigación, COMIE. En: \textit{Documento de
lecturas para la tutoría académica. Cuadernos de apoyo a la
docencia para Sistema de Créditos}, México, U de G.

Delors, Jacques (1997), «La Educación Encierra un Tesoro». \textit{Informe a
la UNESCO de la Comisión Internacional sobre la Educación para el Siglo
XXI}, México, Correo de la UNESCO.

Díaz de Cossío, R. (1998), «Los desafíos de la educación superior mexicana».
En \textit{Revista de la Educación Superior Nº 106, abril-junio}, México,
ANUIES.

Diccionario de la Real Academia de la Lengua Española (2012).

Gobierno del Estado de Aguascalientes (2010), \textit{Plan Sexenal de
Gobierno del Estado}, México, Gobierno del Estado.

\begin{sloppypar}
H.\ Congreso de la Unión (1917), \textit{Constitución Política de los Estados
Unidos Mexicanos}. Última reforma publicada DOF 09--02--2012. Documento en
línea consultado el 20 de mayo de 2012. Disponible en
\url{http://www.diputados.gob.mx/LeyesBiblio/pdf/1.pdf}
\end{sloppypar}

\begin{sloppypar}
H.\ Congreso de la Unión (1993), \textit{Ley General de Educación}. Última
reforma publicada DOF 09--04--2012. Documento en línea consultado el 20 de
mayo de 2012. Disponible en
\url{http://www.diputados.gob.mx/LeyesBiblio/pdf/137.pdf}
\end{sloppypar}

\begin{sloppypar}
Poder Ejecutivo Federal (2007), \textit{Plan Nacional de Desarrollo\linebreak
2007--2012}, México, Presidencia de la República.
\end{sloppypar}
\newpage

Programa de Mejoramiento del Profesorado (2006), \textit{Un primer análisis
de su operación e impactos en el proceso de fortalecimiento académico de
las universidades públicas}, México, SEP.

Sánchez, M. G., Mora, L. A. y Sánchez, J. F. (2004), \textit{Tutoría, algunos
elementos para su conceptualización}, México, UNAM.

Secretaría de Educación Pública (2007), \textit{Programa Sectorial de
Educación, 2007--2012}, México, SEP.

\begin{sloppypar}
The International Encyclopedia of Education (1994) Editors in Chief:
Torsten Husén and T. Second edition, Vol. 11. Pergamon, Neville
Postlethwaite.
\end{sloppypar}


Tinto, V. (1992). \textit{El abandono de los estudios superiores: una nueva
perspectiva de las causas del abandono y su tratamiento. México}.
UNAM-ANUIES.

Universidad Autónoma de Aguascalientes (2007), «Modelo Educativo
Institucional». En \textit{Correo Universitario}, sexta época, núm.16,
publicado el 15 de marzo del 2007, México, UAA.

Universidad Autónoma de Aguascalientes (2008), \textit{Plan de Desarrollo
Institucional 2007--2015… Hacía un Renovado Horizonte}, México, UAA.

Universidad Autónoma de Aguascalientes (2008), \textit{Reglamento para la
operación del programa de estímulo al desempeño del personal docente}.

Universidad Autónoma de Aguascalientes (2009), \textit{Folleto Informativo},
México, UAA.

Universidad Autónoma de Aguascalientes (2011), \textit{Folleto Informativo},
México, UAA.
\newpage

Universidad Autónoma de Aguascalientes (2011), \textit{Marco Institucional
de Formación Integral}.

Universidad de Guadalajara (1992), Estatuto del personal académico, México,
UDG.

Universidad Nacional Autónoma de México (1996), \textit{Reglamento General
de Estudios de Posgrados}, México, UNAM.

Urzúa Macías (2005), \textit{Primer Informe}, México, UAA.
\newpage
\thispagestyle{empty}
\phantom{abc}
