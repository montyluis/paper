%%\clearpage\setcounter{page}{187}%
\thispagestyle{empty}
\phantomsection{}
\addcontentsline{toc}{chapter}{Trayecto de una línea académica en peligro de\newline extinción. El caso de la difusión de la historia\newline en la Universidad Autónoma de Zacatecas\newline $\diamond$
\normalfont\textit{Antonio F. de Jesús González Barroso,\newline María R. Magallanes Delgado y Ángel Román Gutiérrez}}

{\centering{\scshape \large Trayecto de una línea académica en peligro de extinción\newline El caso de la difusión de la historia en la Universidad\newline Autónoma de Zacatecas}}
\markboth{la formación del historiador}{trayecto de una línea académica}
\setcounter{footnote}{0}


\bigskip
\begin{center}
{\bfseries Antonio F. de Jesús González Barroso\\
María R. Magallanes Delgado\\
Ángel Román Gutiérrez}\\
{\itshape\ Universidad Autónoma de Zacatecas\/}
\end{center}

\bigskip
\textbf{Resumen}

El tema de la divulgación de la historia en la Universidad Autónoma de 
Zacatecas es relativamente nuevo. Un grupo de profesores, en particular 
los interesados e integrantes de la línea de difusión de la historia 
del Cuerpo Académico 184 llamado \textit{Enseñanza y difusión de la 
historia},  han visto en esta línea una excelente alternativa para sus 
estudiantes. Los resultados de los encuentros de egresados y 
empleadores dan cuenta de aspectos interesantes con respecto a lo 
anterior, incluso dentro del gremio se ha colocado como la segunda 
fuente de empleo, después de la docencia y por encima de la 
investigación. De esta manera, el presente trabajo pretende resaltar, 
por un lado, la difícil labor que se ha hecho para sostener esta línea 
ante la adversidad y desdén de propios y extraños inmersos en el mundo de 
Clío. Por otro lado pretendemos ver cómo se ha venido posicionando en 
la vida institucional, a tal grado que hoy por hoy es un eje terminal 
del reciente plan de estudios puesto en marcha en agosto del 2013. 
Finalmente, analizaremos el papel que han venido desempeñando los estudiantes 
a propósito de esta línea, ya que en su mayoría desconocen el horizonte 
de posibilidades que tienen frente a ellos como futuras oportunidades 
dentro del mercado de trabajo de la difusión y la divulgación de la 
historia.


\bigskip
\textbf{Abstract}

The subject of the spreading of history in the Autonomous University of
Zacateas, is relatively new. A group of professors, in particular the
interested ones and members of the line diseemination of the history of
Academic Body 184 \textit{Teaching and dissemination of history}, they have seen in
this line an excellent alternative for its students. The results of the
encounter of graduates and employers give account of interesting aspects with
respect to the previous thing, within the profession is even had placed like
second source of employment, after teaching and over the investigation. This
way, the present work tries on the one hand, to emphasize the difficult work
that has been made to maintain this line beforethe adversity and own and
strange disdain of immersed in world of Clio. On the other hand, we try to see
how that line has come positioning in the institutional life to such degree
that at the present time is a terminal axis of the recent plan of studies put in
march in August of the 2011. Finally we will analyze the role that has come
playing the students with regard to this line, because in his majority they do
not know the horizon of possibilities that they have in front of them like
future opportunities within the market of work of the dissemination and the
spreading of history.

\bigskip
\textbf{Presentación}

En una ocasión un maestro de licenciatura comentó que aquel historiador que en
su vida laboral no consulta un archivo histórico y que se aleja de los
documentos, no es y\slash{}o deja de ser historiador. Si bien esta reflexión puede
resultar muy romántica para algunos, hay otros que no la compartimos y con esto
no nos referimos a que no estemos de acuerdo con el historiador documentalista,
por el contrario,\linebreak reivindicamos esa parta tan indispensable, pero no única, para
los historiadores y para la historia, y todavía quisiera ir más allá: para la
sociedad.

Está claro que la función principal de la historia es recuperar los aconteceres
de una cultura determinada. El punto de discusión sigue siendo cómo se recupera
y de qué manera se da a conocer y es presentado ese acontecer, en el presente.
Sin duda alguna el oficio del historiador hasta cierto punto sigue siendo un
tanto egoísta, ya que en su mayoría se preocupa más por seguir rigurosamente un
marco teórico y una metodología que en la forma y estilo que pudiera ser
presentado su producto histórico. En concreto, difícilmente considera a su
destinatario o mejor dicho a su público.


En ese sentido el presente trabajo intenta demostrar que la divulgación de la
historia es una alternativa viable, vigente y procedente para difundirla a
través de actividades, medios, representaciones, etc., y que haría de esta
disciplina un conocimiento más atractivo y hasta cierto punto rentable, no
nada más para la comunidad académica, sino para el público en general.


No obstante, en este trabajo también hacemos una crítica constructiva a esos
mismos círculos académicos que de alguna manera se oponían y se siguen
oponiendo, viendo el oficio de la difusión de la historia como una actividad
simple y muy limitada, aclarando que nos referimos al caso de la  de la
Universidad Autónoma de Zacatecas.


\bigskip
\textbf{El origen del eje específico de difusión\\ de la Historia en la UAZ}

En agosto de 2004 la licenciatura en historia de la Universidad Autónoma de
Zacatecas arrancó con un nuevo plan de estudios, abrigado por el modelo
académico UAZ siglo XXI\@. En este escenario, el programa educativo recibía a una nueva generación de estudiantes bajo nuevos lineamientos y requisitos de
admisión con miras a mejorar los indicadores de su eficiencia terminal. Su
reforma curricular se confeccionó en un lapso aproximado de 12 meses con la
participación activa de su claustro docente, estudiantes, egresados y
empleadores. Una vez hecho el análisis de las necesidades académicas de los
estudiantes y después de haber concluido la metodología para echar a andar esta
nueva propuesta, se concluyó que el licenciado en historia de la Universidad
Autónoma de Zacatecas optaría por transitar en uno de los tres ejes específicos
como formación pertinente para los estudiantes: 1) Docencia, 2) Investigación y
3) Difusión de la historia. 

La implementación de estos tres ejes se convirtió en la punta del iceberg
para detonar y diversificar el tan noble oficio del historiador, es decir, se
dio el paso a formar historiadores in\-ves\-ti\-ga\-do\-res-do\-cu\-men\-ta\-lis\-tas y\slash{}o de archivo, a historiadores con una formación profesional más completa con un
horizonte más amplio, pensado para que los estudiantes de historia pudieran
competir en el mercado laboral con aquellos licenciados de otras disciplinas
que ocupaban empleos principalmente de docentes y divulgadores de la historia y
la cultura. A continuación podemos ver como ya desde un principio, el colectivo
de profesores consideró elementos que relacionaban al estudiante con aspectos
de la difusión de la historia:


\begin{quote}
Al concebir que la formación de los estudiantes debe ser integral, se promoverá
la asistencia a eventos tales como: conferencias, exposiciones, cine, debates,
presentación de libros, obras de teatro, etcétera. Asimismo, se le introducirá
al ámbito de su profesión y se le instará a proponer proyectos en las
diferentes esferas de su competencia; además, será capaz de coexistir en un
mundo complejo caracterizado por la paradoja, la contradicción y la
incertidumbre, y si bien la ética performativa es la que impera  en la
actualidad (eficiencia\slash{}ineficiencia), se le instruirá en la denotativa
(verdad\slash{}falsedad) y en la prescriptiva (justicia\slash{}injusticia); es decir,
compromiso ético. (Medina 2011, p. 4)
\end{quote}

\bigskip
Cuando el documento se empezó a familiarizar dentro de la comunidad
universitaria, pero principalmente en el Área de Humanidades y Educación, causó
revuelo por dos razones: primeramente, los programas académicos de Filosofía,
Letras y Antropología se resistieron a transitar al modelo académico, lo que
significa que siguen manteniendo un plan de estudios rígido,  además de que
nunca aceptaron la integración de esta Área. Hasta la fecha sigue siendo una
controversia. 


El problema no concluyó ahí ya que los cuestionamientos no se hicieron esperar,
incluso dentro de la misma Unidad Académica de Historia,  cuando el plan de
estudios 2004 fue presentado ante el H. Consejo Académico no
fue bien visto por parte de algunos consejeros, al punto de que cuestionaron la
implementación de los ejes de Docencia y el de Difusión de la Historia. En su
opinión, textualmente decían:

\begin{quotation}
(\ldots) el nuevo plan de estudios propuesto por los compañeros de la licenciatura (de la Unidad Académica de Historia) han cortado[sic] materias y pusieron otras nuevas\ldots aquí no se forma docentes para eso está la escuela Normal Manuel Ávila
Camacho. Aquí se deben formar solamente
investigadores\ldots\footnote{Sesión del Consejo Académico de Unidad del 3 de marzo de 2004. Archivo de la Unidad Académica de Historia de la Universidad Autónoma de Zacatecas. Actas del Consejo de Unidad.}
\end{quotation}

\bigskip
Aprovechamos para comentar que según el encuentro de egresados que se llevó a
cabo en febrero de 2011, colocaba a la docencia en\linebreak primera posición como
receptora de empleo seguida de la divulgación de la historia y en tercer sitio
la investigación. Incluso desde que salió la primera generación en 1993 hasta
nuestros días, según los registros existentes, solamente un egresado trabaja
como investigador de tiempo completo.\footnote{Es
el caso del licenciado Limonar Soto, quien se desempeña desde 1999 como
investigador en el Instituto Nacional de Antropología e Historia (INAH).}

Consideraban que se «\ldots había mutilado el plan de estudios\ldots» y que, por supuesto,
resultaba en perjuicio de los estudiantes. Con respecto al eje específico de
Difusión de la Historia, opinaban que 


\begin{quotation}
(\ldots) parece que están pensando en que los estudiantes (de la licenciatura en
historia) van a ser cuidadores de museos, periodistas, guías de turistas\ldots se
está abandonando su formación que debe ser la de investigadores\ldots un historiador
debe de estar en el archivo consultando fuentes documentales\ldots (\textit{ibid.})
\end{quotation}

\bigskip
En concreto resultó un tanto desconcertante el promover esta orientación frente
a críticas, obstáculos y opiniones vertidas por parte del mismo gremio. Desde
nuestro punto de vista, lo más grave de esta situación, resultó ser  la
información que se generaba y llegaba hasta los estudiantes, a tal punto de
confundirlos y predisponerlos en contra de este campo de acción. Aún así, los
esfuerzos y empeño por parte del colectivo de profesores de la licenciatura en
historia, siempre  estuvo presente, y aunque ha venido trabajando ante estas
adversidades, se ha impuesto y día a día intenta ganar más territorio. De paso
podemos mencionar que temas como historia de las mujeres y\slash{}o de género
atrajeron la atención por ser temas innovadores y poco explorados entre la
comunidad académica.\footnote{Para este tema, ver Gutiérrez (2013).}

Cuando se desarrolló la metodología para la reforma del plan de estudios en el
2004 se elaboró un detallado estudio comparativo con al menos 24 licenciaturas
de historia del país.\footnote{El estudio comparativo se realizó con las
Licenciaturas de Historia de las universidades de la UNAM, Tlaxcala, Sinaloa,
Cd. Juárez, Chiapas, UDLA, Morelos, Michoacán, UAM, Baja California Sur,
Sonora, Guerrero, Universidad Iberoamericana, Guanajuato, Escuela Nacional de
Antropología e Historia, Tabasco Nuevo León, Campeche, Universidad Autónoma del
Estado de México y Benemérita Universidad  Autónoma de Puebla.} El resultado de
este estudio fue que muy pocas de ellas incorporaban asignaturas de divulgación
de la historia, y en ese sentido coincidimos con Naranjo Chacón cuando se
refiere a que


\begin{quotation}
(\ldots) En concreto, Naranjo Chacón señala como las escuelas de historia
costarricenses no han integrado la divulgación y la difusión como parte
esencial de sus proyectos académicos; dejándolos, en el mejor de los casos, a
los mismos en manos de algunos profesores, que actúan según su propio criterio;
cuando no, aceptando sin ningún reparo la reconstrucción de la memoria a
novicios, aficionados y a periodistas, la cual por cierto se hace desde la
perspectiva tradicional, ideologizante y personalista. (Naranjo
2004, p. 2)
\end{quotation}

\bigskip
Si bien la cita anterior hace alusión al caso de Costa Rica, no resultaba ni
resulta ser tan extraño a la realidad mexicana. De cualquier manera, una vez
elaborada la metodología del plan de estudios 2004 para la licenciatura en
historia de la Universidad Autónoma de Zacatecas se demostró la necesidad
imponderable de implementar sus tres ejes específicos: investigación, docencia
y divulgación. Estos dos últimos ejes sentaron las bases para detonar otras
acciones en pro de ampliar la oferta educativa, la capacitación e innovación,
los servicios, la actualización, la profesionalización y demás temas que tienen que
ver con la docencia y la divulgación de la historia. Lo que significó que estas
necesidades para la formación de los estudiantes, se expusieran en espacios, foros,
talleres, congresos, coloquios, etcétera.

El efecto de todo esto fue positivo, de entrada porque se creó un Cuerpo
Académico bajo la denominación «Enseñanza y Difusión de la Historia», en el que
hasta la fecha se sigue cultivando una Línea de Generación y Aplicación del
Conocimiento en la Difusión de la Historia, cuya principal tarea era 
«(\ldots{}) innovar
formas de difusión del conocimiento histórico así como de las mejoras en los
procesos de la enseñanza-aprendizaje de la historia\ldots » (González 2013, p. 3)

Pero no nada más eso, para el 25 y 26 de Septiembre de 2009 en la ciudad de
Guadalajara se celebró el \textit{IV Coloquio sobre Docencia de la Historia},
 auspiciado por el Centro Universitario de Ciencias Sociales y Humanidades y la
Academia de Docencia de la Historia. Aunque en dicho evento se presentaron 28
investigaciones referentes al tema, lo más importante de todo esto y con la
intención de atender  problemáticas generales de las instituciones educativas
de la enseñanza y difusión la historia, nace en este evento la propuesta de 
conformar una \textit{Red de Profesionales de la Docencia y
Difusión de la Historia}, misma que tendría como objetivo central el abonar a
la reflexión sobre la enseñanza-aprendizaje de la Historia en México en todos
los niveles educativos, por lo que la inscripción a la Red no estaría sujeta
solamente a universidades o centros de educación de superior, sino a cualquier
persona interesada, formada o que labore como docente de Historia. Tal empresa
fue turnada a la Universidad Autónoma de Zacatecas por conducto del Cuerpo
Académico \textit{Enseñanza y Difusión de la Historia}, su tarea sería convocar
a un Encuentro Nacional y conformar la Red (González 2010, p. 6).


\bigskip
Finalmente podemos decir que la implementación del eje específico de Difusión de
la Historia está pensado en la profesionalización de este campo; de hecho se
puede ver, incluso en su plan de estudios, la tutoría, la asesoría y demás
servicios académicos que se le brindan al estudiante en su trayectoria escolar.
Un ejemplo de lo anterior fue cuando en el 2010 un grupo de estudiantes de la
orientación de Divulgación de la Historia realizaron su servicio social y
práctica profesional bajo una supervisión y monitoreo que permitió a los
estudiantes realizarlos de una manera exitosa y eficaz, sin que el trabajo del
estudiante se devaluara por la unidad receptora o bien se convirtiera en mano
de obra barata por la misma (González 2012, p. 8). Dos prestadores del eje 
de difusión viajaron a partir de febrero de
2009 a casi todos los municipios del estado a hacer trabajo de campo para dar
sustento empírico al Proyecto Sistema de Inventario de Artesanías de México,
inscrito en el Programa Nacional de Arte Popular, porque fueron asignados al IDEAZ\@.


\bigskip
\textbf{La divulgación de la historia y sus resultados\\ concretos}

El \textit{boom} que se dio al interior de la Universidad Autónoma de Zacatecas
con el fomento y la profesionalización de la divulgación de la historia se ha
venido reflejando a través de los años. Lo más interesante de todo esto, es que
el objetivo principal de la creación de este eje específico se logró, ya que los
propios estudiantes han sido los que lo han sabido aprovechar de muy buena
manera. 

Existen muchos casos de éxito de estudiantes que se han decidido a transitar
en el eje de la divulgación de la historia, y que se han insertado en el mercado
laboral. Si bien resultaría imposible enumerarlos todos, solamente haremos mención
de algunos ejemplos. Hacia el 2009, un grupo de estudiantes se pusieron de
acuerdo y conformaron una asociación civil titulada \textit{Leyendas de
Zacatecas}, en la que emprendieron un negocio rentable y exitoso basado en
recorridos nocturnos por el centro histórico de la ciudad de Zacatecas. Estos
jóvenes no perdieron el vínculo con la licenciatura, por el contrario, 
recibieron asesoría por parte del colectivo de profesores para la elaboración
de su vestuario. Por otro lado, su formación en las asignaturas de Historia del
Virreinato e Historia de Zacatecas les ayudó a conocer el contexto de la
época y ubicar al asistente que los escuchaba y acompañaba entre las calles,
callejones, monumentos y edificios más representativos del casco histórico de
la ciudad de Zacatecas.

Sus integrantes por supuesto que no eran personas improvisadas o espontáneos al
momento de sus presentaciones, por el contrario, hacían valer su condición de
historiadores,  acompañada de cualidades artísticas y dotes escénicas adquiridas
extracurricularmente. Debido a su éxito, prontamente surgieron grupos imitadores
sin ser incluso estudiantes de historia, con la finalidad de copiar la idea y
trabajar dentro de sus mismos espacios, circuitos, agencias y operadoras de
viaje que los contrataban. No obstante, el prestigio del grupo Leyendas de
Zacatecas dirigido por Moisés López Cid,  de inmediato se hizo notar y la
Secretaria de Turismo del Estado de Zacatecas les dio la titularidad y les
otorgó una certificación, además de incluirlos en su lista de verificación. 


Con lo anterior se evitaba que los grupos apócrifos 


\begin{quotation}
sin certificación ni capacitación instalen módulos, en los que incluso 
pegan logotipos falsos, para imitar sus servicios durante las 
vacaciones. López Cid declaró en días pasados que la dependencia 
estatal se enfocaba a la verificación sólo en algunos restaurantes, 
hoteles y comercios, pero dejaba de lado los recorridos de leyendas; 
debido a ello, la Secturz les informó que serán apoyados 
\end{quotation}

El director del grupo aseguró que Leyendas de Zacatecas tiene un 
elevado estándar de calidad, pues está certificado por la Secretaría de 
Turismo estatal con el número de folio 5803016; además, cuentan con el 
aval 3780 de la Secretaría de Turismo federal.\footnote{Entrevista 
periodística a Jesús Moisés López Cid director del grupo Leyendas de 
Zacatecas, el 5 de agosto de 2012, en el Periódico NTR, Noticieros en 
Tiempo Real.}

\bigskip
Un gran mérito que tenía el grupo Leyendas de Zacatecas estribaba en la
reconstrucción histórica del ambiente. Había hasta 10 personajes en escena
todos con vestuario de la época, narraban la historia y las leyendas de esta
ciudad minera. El recorrido duraba aproximadamente hasta 30 minutos y arribaban
a las casonas más antiguas del virreinato y el siglo XIX\@. A los asistentes se
les ofrecía una degustación de mezcal y dulces típicos, como el rollo
de guayaba, entre otros postres de la región. Su cobertura alcanzaba los sitios
más significativos de la ciudad: la Mina del Edén, La Bufa, museos y
algunos municipios cercanos como Guadalupe y Jerez.

Ante esta marea y expansión de la divulgación de la historia en la licenciatura
también ha habido casos individuales en donde vemos a los estudiantes
practicando la fotografía profesional, como lo es el caso del licenciado en
historia Juan Carlos Basabe, quien además de dedicarse a esto, también trabaja
en la fototeca del Estado, montando exposiciones de fotografías antiguas
alusivas a periodo de la historia de Zacatecas. Asimismo, es auxiliar del
fotógrafo Pedro Valtierra, ganador del premio Internacional de Periodismo Rey de
España en 1998, quien le ha facilitado el acceso a sus archivos de imagen para
ordenarlos y clasificarlos. Involucrarse en este medio, le facilitó desarrollar
un trabajo de investigación para su titulación en la licenciatura. Su trabajo
de tesis consistía en el análisis de las imágenes de las bienales de fotografía
en México durante las décadas de 1980 y 1990.

Para finalizar este apartado se debe mencionar que la divulgación de la historia,
conocida también en Europa como la historia pública, patentiza la idea central
de nuestro plan de estudios referente a insertar una historia de calidad en la
extensa gama de los medios de comunicación, pero también en el cine, la radio,
la prensa, la televisión, Internet, etcétera. Todo ello con la finalidad de
incrementar la cultura entre las sociedades, es decir, llevar la historia de
calidad hacia las grandes masas. 

\bigskip
\textbf{Preferencias y selecciones de los estudiantes}

El plan de estudios de la licenciatura se reformó para 2011, y se integraron
otros tres ejes específicos, además de la investigación, la docencia y la difusión,
también se ofertan los de organización y administración de acervos, historia
del arte e historiografía. A continuación presentamos unas gráficas que
muestran las tendencias de los estudiantes con respecto a sus preferencias para
la elección de algún eje específico:

\begin{figure}[H]
\centering
\includegraphics[scale=0.55]{p11a-img001.png}
\caption{Grado de conocimiento sobre los ejes terminales de la licenciatura en Historia}
\end{figure}


\bigskip
\begin{figure}[H]
\centering
\includegraphics[scale=0.60]{p10-img001.pdf} 
\caption{Ejes terminales seleccionados por los estudiantes}
\end{figure}

\bigskip
\begin{figure}[H]
\centering
\includegraphics[scale=0.55]{p10-img002.pdf} 
\caption{Número de estudiantes encuestados y semestre al que pertenecen}
\end{figure}

\bigskip 
Sin duda alguna, el plan de estudios resultó atractivo para la mayoría de los estudiantes, pues desean cursar tres de los seis ejes que se ofertan en la licenciatura, como lo muestra
la siguiente gráfica:


%\medskip
\begin{figure}[H]
\centering
\includegraphics[scale=0.50]{p10-img003a.pdf} 
\caption{Los 3 ejes terminales predilectos}
\end{figure}

\bigskip
Con la intención de tener un diagnóstico lo más completo posible, se les
preguntó a los estudiantes del por qué habían elegido ese eje, y así contestaron:


\bigskip
\begin{figure}[H]
\centering
\includegraphics[scale=0.60]{p10-img004.pdf} 
\caption{Razones para la elección de los ejes}
\end{figure}


\bigskip
Finalmente, se les preguntó sobre la temporalidad de elección de su eje:


\bigskip
\begin{figure}[H]
\centering
\includegraphics[scale=0.60]{p10-img005.pdf} 
\caption{Tiempo para la elección del eje terminal}
\end{figure}



%\bigskip
%\begin{figure}[H]
%\centering
%\includegraphics[scale=0.94]{p10-img006.pdf} 
%\caption{ }
%\end{figure}

\bigskip
\textbf{Comentario final}

El eje de Divulgación de la Historia en la licenciatura de la Universidad
Autónoma de Zacatecas sirvió para que los estudiantes egresaran con mejores
competencias y herramientas de trabajo, y así también tuvieran mejores
oportunidades en el mercado laboral. Como resultado de lo anterior, también dio
pie a la creación de un Cuerpo Académico que hasta nuestros días fomenta esta
línea en la comunidad académica.

A pesar de ciertas actitudes detractoras por parte de aquellos académicos de nuestro
mismo gremio que siguen empeñados en que el archivo es el santuario para el
historiador o, mejor dicho, que el documento «habla por sí mismo»  ---pero que
además mantienen una postura inamovible de que el historiador que no es investigador no
es historiador---, la divulgación de la historia es una realidad, es una
oportunidad, incluso un campo todavía en gran medida inexplorado. Si observamos
con cuidado las gráficas, podemos darnos cuenta que, hoy por hoy, los
estudiantes de la licenciatura tienen en primer lugar como preferencia,
formarse en el eje de docencia, y después vendría casi un empate entre el eje de
investigación y el de divulgación, tan solo por una diferencia de dos puntos.

Después de diez años la licenciatura en historia ha evolucionado a pasos
agigantados. A pesar de los números mostrados en las gráficas, seguimos
considerando que el eje de difusión sigue estando en peligro de extinción,
puesto que debería ser el eje que estuviera en primer lugar como preferencia de
los estudiantes, dado que la divulgación de la historia se visualiza como un
gran campo de acción para sus egresados. 
\newpage

\textbf{Referencias}

González, Antonio (2009), \textit{Red Nacional de Profesionales de la 
Enseñanza y Difusión de la Historia. El proceso fundacional y la 
búsqueda de las acciones colectivas}. Ponencia presentada en Encuentro 
Iberoamericano de redes y grupos de investigación, 26, 27 y 28 de mayo 
de 2010 en el Puerto de Mazatlán, Sinaloa, México.

\_\_\_\_\_\_ (2012), Los «laboratorios» del 
historiador: servicio social y prácticas profesionales en la UAZ\@. En 
\textit{7º Encuentro de la Red Nacional de Licenciaturas en Historia y 
Cuerpos Académicos (RENALIHCA) y 1er\@. Encuentro americano de 
licenciaturas en Historia}, San Cristóbal de las Casas, Chiapas, 
México.

\_\_\_\_\_\_ (20013), «Enseñanza y difusión de la historia». Un cuerpo académico en vías de la 
consolidación. \textit{En VIII Encuentro de la Red Nacional de la 
Licenciaturas en Historia y Cuerpos Académicos y 2º Encuentro 
Iberoamericano de Licenciaturas en Historia}, Facultad de Historia, de 
la Universidad Michoacana de San Nicolás de Hidalgo, Morelia, 
Michoacán, México.

Gutiérrez, Norma (2013), «Los estudios de género en la 
enseñanza-aprendizaje de la historia: de la invisibilidad a su 
integración en el Plan de estudios de la Licenciatura en Historia de la 
Universidad Autónoma de Zacatecas», en Memoria. 
IV Encuentro Nacional de Docencia, Difusión y Enseñanza de la Historia. 
Segundo Encuentro Internacional de Enseñanza de la Historia y Tercer 
Coloquio entre Tradición y Modernidad, Querétaro, 
México, Universidad Autónoma de Querétaro.
\newpage

Medina, Lidia y Roman, Luis (2011), \textit{Plan de Estudios 2004 del Programa de 
Licenciatura en Historia de la UAZ}, Primer Encuentro Nacional de 
Profesionales de la Historia, Cuerpo Académico Enseñanza y Difusión de 
la Historia, Universidad Autónoma de Zacatecas, México. 
 
Naranjo, Gustavo (2004), «La Divulgación Científica Aplicada a la 
Historia a través de las Nuevas Tecnologías de la Información» 
\textit{Revista de Historia}, No\@. 48, Heredia, Costa Rica, EHUNA-CIHAC\@. 

Solar Cubillas, David (2009), «La Divulgación de la Historia y 
otros estudios sobre Extremadura»,  en Felix Iniesta Mena (coord.) 
\textit{Actas de las X Jornadas de Historia sobre la Divulgación de la 
Historia y otros estudios sobre Extremadura}, España,
Llerena: Sociedad Extremeña de Historia.
\newpage



{\centering ANEXO \par}

{\centering Cuestionario aplicado a los estudiantes \par}


\bigskip
Nombre del alumno: \hrulefill 
%\_\_\_\_\_\_\_\_\_\_\_\_\_\_\_\_\_\_\_\_\_\_\_\_\_\_\_\_\_\_\_\_\_\_\_\_


Grado y grupo: \hrulefill
%\_\_\_\_\_\_\_\_\_\_\_\_\_\_\_\_\_\_\_\_


Marque con una X la respuesta seleccionada.


1.- ¿Sabe qué son los ejes terminales de la licenciatura en Historia?


Sí  (\phantom{abc})  No (\phantom{abc})  Explíquelo brevemente:

\hrulefill{}

\_\_\_\_\_\_\_\_\_\_\_\_\_\_\_\_\_\_\_\_\_\_\_\_\_\_\_\_\_


2.- De los seis ejes terminales que ofrece la licenciatura en Historia, ¿cuál ha
elegido?


Investigación  (\phantom{abc})


Docencia  (\phantom{abc})


Difusión  (\phantom{abc})


Organización y administración de acervos  (\phantom{abc})


Historia del arte  (\phantom{abc})


Historiografía  (\phantom{abc})


3.- ¿Por qué ha elegido este eje?


\hrulefill{}

\hrulefill{}

\_\_\_\_\_\_\_\_\_\_\_\_\_\_\_\_\_\_\_\_\_\_\_\_\_\_\_


4.- Si aún no ha elegido un eje terminal, ¿cuáles serían sus tres primeras
opciones?\\ 

\hrulefill{}

\_\_\_\_\_\_\_\_\_\_\_\_\_\_\_\_\_\_\_\_\_\_\_\_\_\_\_\_\par


\par
5.- ¿Para qué fecha podría tener definido un eje terminal?\par


\hrulefill{}


%%\clearpage\setcounter{page}{207}%

\thispagestyle{empty}%
\phantomsection{}
\addcontentsline{toc}{chapter}{El periodo virreinal de México y su alcance en nuestros días: La difusión del conocimiento\newline histórico\newline $\diamond$
\normalfont\textit{Marco Antonio Peralta Peralta\newline y Marcela Janette Arellano González}}

{\centering{\scshape \large El periodo virreinal de México y su alcance en nuestros días: La
difusión del conocimiento histórico}\par}
\markboth{la formación del historiador}{el periodo virreinal de México}
\setcounter{footnote}{0}


\bigskip
\begin{center}
{\bfseries Marco Antonio Peralta Peralta}\\
{\itshape\ Universidad Autónoma de Querétaro\/}\\
{\bfseries Marcela Janette Arellano González}\\
{\itshape\ Universidad Autónoma del Estado de México}
\end{center}

\bigskip
\textbf{Resumen}

La presente ponencia tiene el objetivo de explicar, cuál es la
relevancia académica y social que tiene el estudio del periodo virreinal de
México y a través de qué medios es posible lograr su difusión en los espacios
no académicos. Centramos nuestra discusión en dos ejes. Por un lado, la
relevancia del periodo novohispano como punto de partida de la identidad que
después del siglo XIX, dio sus características culturales y sociales a la
llamada <<nación mexicana>>. Por el otro  lado, para ejemplificar  nuestra tesis
consideramos dos factores: la multietnicidad y la religiosidad, elementos que
han permeado en el imaginario social y en la vida cotidiana de nuestros días y
que sin duda, tienen su punto de arranque en  la época colonial. En síntesis,
argumentar la validez, pertinencia e importancia que debe tener el estudio del
periodo virreinal que ha sido opacado por los <<grandes procesos decimonónicos>>
o bien por la polémica del siglo XX.\\
\textbf{Palabras clave:} Difusión histórica, multietnicidad,
religiosidad,\linebreak historicidad, periodo novohispano.
\newpage

\textbf{Abstract}
 
This communication has the objective to explain the relevance that has the
Novohispanic period of Mexico, and through what ways must spread in-non
academics areas. We consider two points. Firstly, we talk about the relevance
that has the novohispanic period because, we considerate that this period was
fundamental to make and build the identity social and cultural of mexican
society, after the 19th century. The other point; to exemplify our thesis, we
take two elements: the multi-ethnicity and the religiosity. These aspects have
permeated in the social imaginary mexican and the everyday life. In short, we
try to argue the validity, relevance and importance of the colonial age;
because it’s has despised front to the biggest processes of the 19th century,
or the polemical 20th century.\\
\textbf{Keywords:} Historical spread, multi-ethnicity, religiosity, historicity,
novohispanic period.


\bigskip
\textbf{Consideraciones previas}

En todas las épocas y en todas las sociedades, las relaciones sociales han
marcado la convivencia entre los grupos humanos. La religión por su parte, ha
estado presente en esa dinámica social y cultural y en ocasiones, ha sido
protagonista de los procesos históricos. Hablar de la religiosidad y de la
multietnicidad en nuestro siglo XXI, no es algo novedoso ni tampoco algo ajeno
a nuestra vida cotidiana; nuestro mundo moderno y ansioso del progreso
tecnológico ha tratado de sustituir en cierto sentido, el pensamiento religioso
del hombre contemporáneo y frente a la multiculturalidad, ha creado una
sociedad masificada y unificada (Peralta 2013, p. 299).

\enlargethispage{1\baselineskip}
Según lo anterior, no sólo corresponde a las ciencias sociales y humanísticas
explicar los cambios que se han acelerado a finales del\linebreak siglo XX y principios del XXI.
Se trata de comprehender y explicar las transformaciones no sólo desde el
ámbito académico sino también tomando conciencia de nuestro papel como
difusores del conocimiento histórico-cultural (Fontana 1992). En atención a lo
anterior, la presente ponencia tiene dos objetivos. Por un lado, centramos
nuestra atención en explicar la relevancia que ha tenido la religiosidad
popular y la multiculturalidad (entendida como producto de la multietnicidad)
en el pensamiento mexicano, con base en el análisis del mundo virreinal. Por el
otro lado, demostrar que esa religiosidad y esa multiculturalidad permean en
nuestro siglo XXI y por ello su difusión es vital para comprehender, entender y
explicar el nuevo paradigma social.


En este tenor, nuestra ponencia se divide en dos partes. En primer lugar hacemos
un esbozo general de la construcción de la religiosidad novohispana y la
multietnicidad. Asumimos el hecho de que  en este periodo histórico se forjó la
identidad americana del mundo hispánico en general, y la identidad del mexicano
en particular (Rubial 2010, pp. 13--16); es decir, si bien en los siglos XIX y XX
se construyó la mexicanidad que hoy llega a nosotros, ésta fue producto directo
de la tradición virreinal. Decidimos ubicarnos en el periodo novohispano porque
desde hace ya varios años, hemos trabajado esta época y en este sentido,
tenemos mayor certeza de explicar nuestra propuesta con base en una temática de
la que tenemos mayor conocimiento. 

 
No obstante, más allá de los intereses personales, está claro  que hoy en día,
en las escuelas de educación básica y media superior, e inclusive en algunas
Universidades Públicas, se han reducido los contenidos informativos del mal
llamado <<periodo colonial>> a favor de otras etapas históricas más cercanas a
nuestro presente, lo que finalmente ha hecho que cada vez la juventud preste
menor interés hacia\linebreak nuestra historia virreinal. En definitiva, consideramos
justificable regresar al periodo novohispano porque su difusión en los
ambientes no académicos se ha visto reducido a pesar de ser un proceso
histórico clave de la <<Historia Nacional>>.


En la segunda parte de nuestro trabajo, explicamos a grandes rasgos las
innovaciones tecnológicas y en particular el uso de las Tecnologías de la
Información y la Comunicación (TIC) con el propósito de explicar cómo a través
de las redes sociales, comunes y muy cercanas a la sociedad del siglo XXI, los
profesionistas de la Historia podemos hacer asequible el conocimiento científico
para comunicarlo a través de estas plataformas virtuales. De la misma manera,
Recuperamos los foros académicos y no académicos como otra vía segura para
lograr la difusión de nuestras investigaciones. 

Lo anterior significa que asumimos que aún y con las ventajas que nos ofrecen
las redes sociales, el contacto humano; es decir, el compartir un espacio con
distintas personas, es fundamental para el intercambio de ideas y el diálogo
académico, por eso, los profesionistas de la Historia debemos procurar estos
foros de análisis y discusión (Blázquez, Latapí y Torres 2013). Por todo ello,
nuestro trabajo es una reflexión del quehacer del historiador; es decir, la
relevancia de su trabajo y las vías más comunes en nuestro siglo XXI para dar a
conocer el producto de nuestros trabajos. 


\bigskip
\textbf{Primera Parte}

\textbf{La configuración de la religiosidad en Nueva España}

En los albores del siglo XVI el mundo occidental, vinculado a la tradición
Católica Romana Apostólica, reformaba su concepción del mundo como consecuencia
del advenimiento de nuevas ideologías\linebreak religiosas arropadas, principalmente,
bajo los pensamientos calvinista y luterano (Kamen 1999, pp. 88--89). Esta reforma
promovida por el pontífice de Roma y apoyada por los imperios de corte
católico, intentó frenar el avance de estas doctrinas, toda vez que buscaba
consolidar la hegemonía que hasta entonces tenía el catolicismo. 


Al interior del mundo católico, ya en el siglo XVII, imperios como el español
abanderaron una campaña de difusión y expansión del catolicismo por medio del
arte y la cultura barroca, la cual sirvió como medio pedagógico para enseñar la
doctrina cristiana y además, a través de su ostentación y luminosidad,
intentaba poner en evidencia que el catolicismo era la vía única para salvar el
ama (Fernández 2010).  Por ello, se llevaron a cabo empresas de difusión no
sólo al interior del viejo continente sino también, en las posesiones
ultramarinas de los estados católicos. 


El movimiento reformador y las campañas de difusión y expansión de la doctrina
católica, en España se vieron reflejados en los principios religiosos,
políticos y económicos del movimiento barroco (Maravall 1998,
\textit{passim}). El barroco significó un periodo de contrastes para la
monarquía hispánica en donde los excesos y la heterodoxia religiosa no sólo se
manifestaron en Europa sino en sus posesiones ultramarinas (Rubial 2010, pp.
210--211).

\enlargethispage{1\baselineskip} 
Según lo anterior, a partir de Trento la Iglesia hizo un cambio profundo en el
dogma que hasta entonces regía el pensamiento católico de Occidente. Esto
significa que el Concilio tridentino estableció la base de los nuevos deberes
religiosos; es decir, el bien espiritual y la nueva conducción de vida
cristiana. Alberto Carrillo Cázares afirma que la reunión ecuménica de Trento
fue el paradigma religioso que marcó el nacimiento de la Iglesia Americana y
por tanto de la cristiandad del Nuevo Mundo (Cázares 2006, p. XVIII).


Esta nueva cristiandad tenía como fin justificar el modo de vida cristiano que
caracterizaba a la sociedad española de esa época; esto era, en palabras de
Marcelin Defurneaux (1964)  un cristianismo regido por el honor y la fe
católica, y es que los monarcas españoles se pronunciaron como vicarios de Dios
en sus territorios y a su vez, como defensores de la fe y del clero. 


En las posesiones ultramarinas de España (y en particular en el virreinato de la
Nueva España), previo a las resoluciones de Trento, es muy probable que la
religiosidad popular ganara terreno sobre la ortodoxia católica por dos razones
principalmente. En primer lugar, la prolongación de las sesiones tridentinas
obligó a que Alonso de Montúfar, arzobispo de México en 1555, convocara a sus
obispos a celebrar un cónclave provincial para hacer frente a los problemas
locales que tenía que enfrentar la introducción de la nueva fe por lo que dejó
al margen las disposiciones de Trento (Chávez 1996, p. 50).

En segundo lugar, el sincretismo religioso entre la cultura mesoamericana y
europea, hizo que los misioneros y evangelizadores <<adaptaran>> la doctrina
católica a las circunstancias y referentes conceptuales de los
naturales.\footnote{\textit{Cfr.} Gonzalbo
Aizpuru, Pilar (2008), \textit{Historia de la Educación en la época Colonia. El
mundo indígena}, México, El Colegio de México, 2ª reimp.} Esta condición de
permisibilidad fue aprovechada por los católicos venidos del viejo mundo al
nuevo para poder expresar, a su manera, su fe y devoción (Rubial 1999, pp. 57--58)
un ejemplo de ello fue el florecimiento de las devociones a los santos, cuya
práctica en ocasiones, era objeto de latría.\footnote{De acuerdo a lo reglamentado por
los Concilios, solamente Dios debía ser objeto de latría; es decir, sólo a él
se le debía rendir adoración; a los santos se les debía venerar pues ellos eran
una parte del mismo Dios.}
\enlargethispage{-1\baselineskip}
 
El santoral por ejemplo, entendido como la manifestación de la religiosidad
popular, nos sirve para explicar por qué durante el siglo XVII se popularizó la
práctica de expresiones devocionales  poco ortodoxas, cuyo fin era la
manifestación pública de la fe para la salvación del alma; acciones que
finalmente transitaron entre las devociones y desviaciones al dogma católico.


Una vez terminadas las sesiones del ecuménico Concilio de Trento hacia 1563, la
Nueva España se regía (en la práctica) por las normas que se dictaminaron en el
Concilio de 1555. Sin embargo, una década más tarde, el segundo Concilio
Mexicano se celebró en primer lugar, para ratificar las disposiciones del
Concilio tridentino (Lorenzana~1769, p.~188) y, en segundo lugar, para <<adecuar>>
las resoluciones de su predecesor de 1555 al nuevo contexto religioso impuesto
por Trento. Por ejemplo, en la sesión XIV del segundo Concilio se estableció
que los oficios divinos debían realizarse conforme a las tradiciones sevillanas
y dejar a un lado las prácticas heterodoxas que el primer concilio había
permitido (\textit{ibid.}, p.~196).
\enlargethispage{1\baselineskip}


A pesar de estas nuevas disposiciones (durante la segunda mitad del siglo XVI),
la sociedad novohispana aún practicaba una piedad popular poco relacionada con
las disposiciones conciliares. Para contrarrestar este abanico de expresiones
<<multi-religiosas>>, dos décadas más tarde en 1585, Pedro Moya de Contreras,
sucesor de Montúfar,  apoyado por los obispos de Guatemala, Michoacán,
Tlaxcala-Puebla, Yucatán, Nueva Galicia y Antequera celebró el Tercer Concilio
Provincial Mexicano.

En definitiva, Montúfar en 1555 y 1655, y después Moya de Contreras en 1585
pusieron por escrito, la forma en que debía conducirse la nueva sociedad
católica que apenas colonizaba la Nueva España; es decir, dejaron por escrito
el <<tipo ideal>> que la sociedad debía alcanzar.\linebreak Ambos tomaron como parámetros
las disposiciones que se decretaron en Trento, porque el alcance de la reforma
católica era de envergaduras nunca antes vistas en el mundo católico. Sin
embargo, la realidad cultural y religiosa de los habitantes del naciente
virreinato se había complejizado más allá de lo que los propios Concilios
suponían y por eso, la piedad católica popular, ajena a los discursos oficiales
postridentinos encontró cobijo y legitimidad en la época barroca del siglo
XVII.

\enlargethispage{1\baselineskip}
Entre las reformas más relevantes que produjeron las sesiones tridentinas y los
Concilios Mexicanos destacan, como ya se dijo, el papel del purgatorio y el uso
de la imagen. El papel del primero consistió en promover la piedad cristiana
para que a través de ésta, el creyente alcanzara la gracia divina (Wobeser
2010, p. 43), en tanto que el uso de la imagen jugó un papel pedagógico para la
enseñanza de la doctrina católica (Bribiesca~2010, p.~355). En este sentido se
entiende la tesis que sostiene que en siglo XVII, la sociedad novohispana se
configuró con base en la imagen y el discurso religioso (Rubial 2010, p. 14).

Ya en el siglo XVIII, con el cambio dinástico en la monarquía española, esta
religiosidad barroca se oponía a la modernización ideológica que se fraguaba en
la península ibérica. El despotismo ilustrado, racionalizó la idea de crear un
imperio regalista unificado tanto en la religión como en el pensamiento y la
política. Quedaba claro que después de la reforma tridentina en el mundo
hispánico, seguía un cambio político profundo entre la <<laxitud>> de la era  de
los Habsburgo y el despotismo de los Borbones. En Nueva España, las reformas
borbónicas comenzaron a tener efectos prácticos ya muy entrado el siglo XVIII,
después de 1750, pero éstos fueron contundentes y trascendentales para la vida
cotidiana de la sociedad novohispana y muy particularmente para la Iglesia
(Zoraida 1992). 


Una figura clave del siglo XVIII novohispano fue sin duda el Arzobispo Antonio
de Lorenzana y Butrón quien a principios de 1770 celebró el cuarto Concilio
Provincial Mexicano; la intención era erradicar de raíz la heterodoxia
religiosa que se había permitido con la dinastía de los Habsburgo. Se Trataba
de una denuncia pública  a las formas de practicar la religión  y en palabras
de David Brading (2006)  se hizo lo posible por sustituir el barroquismo por un
pensamiento más cercano al dogma y los principios religiosos de Roma.


Desde luego que la religiosidad practicada hasta ese momento por la sociedad
novohispana hizo una resistencia frente a los principios reformistas del Cuarto
Concilio, es decir, de ninguna manera podemos pensar que a partir de 1759  (con
el ascenso de Carlos III al trono Español) se cambió la forma de pensar de los
novohispanos; es más, sólo hasta las acciones reformistas de Gálvez hacia
1786-1789 el pensamiento religioso se vio afectado por las políticas borbónicas
(Zoraida~1992). Sin embargo si ponemos en la balanza los cambios ocurridos de
1789 a 1810, momento en el que da inicio el movimiento insurgente,
evidentemente nos daremos cuenta que de ninguna manera, esa religiosidad
barroca y multiétnica se abandonó pues al menos a finales del siglo XVI y
durante todo el XVII se consolidó y configuró como la forma religiosa del
pensamiento novohispano; es decir de del pensamiento mexicano.


\medskip
\textbf{La multietnicidad y la conducción de vida\\ novohispana}
%\enlargethispage{1\baselineskip}

La Nueva España, y en particular el centro del virreinato, durante los tres
siglos de dominación española fue un espacio de convivencias armoniosas y
conflictivas, un lugar de apariencias, relaciones económicas, disputas
políticas y religiosas, en fin, un espacio en donde se\linebreak desarrolló una sociedad
dinámica y compleja; sin embargo ¿cómo se originó esa dinámica?, ¿hubo
variaciones en ella? y quizá algo más significativo ¿quiénes fueron los
protagonistas de los cambios y permanencias?


En los albores del siglo XXI, en México, ha crecido el número de investigaciones
e investigadores que centran su atención en el análisis de a multietnicidad;
sin embargo, cada día pesa más la tesis de la tercera raíz (Velázquez e
Iturralde 2011) frente al binomio indio-español. No es nuestra intención hacer
una revisión historiográfica al respecto,\footnote{ Véase por ejemplo: Velázquez
G., María Elisa (coord.) (2012) \textit{Debates históricos contemporáneos.
Africanos y afrodescendientes  en México y Centroamérica}, México,
INAH/CEMCA/IRD/UNAM/AFRODESC, 2011, (Serie Africanías 7).} pero está claro  que
a finales del siglo XVI, en la Nueva España la clasificación de los grupos
sociales se complejizó con el arribo masivo de población africana (Alberro y
Gonzalbo~2013, pp.~37--39).


La multietnicidad la entendemos como el resultado de la mezcla entre las
diferentes calidades sociales que convivieron en la Nueva España (Gonzalbo
2009). La relevancia que tuvo en la vida cotidiana de la Nueva España no sólo
se debe entender como la simple agrupación entre unas calidades y otras. Se
trató más bien de un intercambio de saberes, costumbres, formas de actuar,
maneras de practicar la religión, tradiciones y otras manifestaciones
culturales entre unos grupos y otros y entre unas comunidades y otras; es
decir, la multiculturalidad en el mundo hispánico echó raíz en el imaginario
colectivo como consecuencia de esa convivencia cotidiana, producto de las
manifestaciones multiétnicas.


Solange Alberro y Pilar Gonzalbo Aizpuru (2013) recientemente publicaron la obra
\textit{La sociedad novohispana. Estereotipos y realidades}. Se trata de una
investigación de primer orden que hace un análisis del estado actual sobre el
conocimiento que tenemos en relación al mundo novohispano. En primer lugar
Gonzalbo nos expone la complejidad de los conceptos (Alberro y Gonzalbo 2013,
pp. 23--27). En segundo lugar, Alberro hace un estudio esclarecedor sobre la calidad
de los indígenas frente a la dinámica cultural y social del virreinato y la
metrópoli. En particular, el trabajo de Alberro da luz acerca de la percepción
que tenemos sobre la supuesta docilidad de los indios (Alberro y Gonzalbo
2013, pp. 197--201).


Para entender la multietnicidad como un proceso complejo y dinámico, partimos de
la idea de que en el siglo XVI, después de la conquista, en teoría, existían la
República de Indios y la República de Españoles; sin embargo, conforme avanzaba
la centuria frente a estas agrupaciones comenzaron a aparecer otros grupos,
producto de la migración de mano negra africana, entonces, aquella población
indio española empezó a relacionarse con personas que no eran de una calidad
específica. Por ejemplo, a partir de 1585, en el Tercer Concilio Provincial
Mexicano se decidió que los párrocos debían llevar libros separados para
registrar los bautizos (Rubial 2013, p. 221). Desde luego que no se tuvo
conocimiento de una clasificación social ni a finales del siglo XVI ni a
principios del XVII; fue hasta mediados del siglo barroco cuando se comenzó a
tener conciencia de que había grupos entre el negro y el indio, entre el
español y el negro o entre indio y español; es decir que no eran ni una cosa ni
otra, frente a la complejidad multiétnica. Es decir, a lo largo del siglo XVII
se comenzó a construir una sociedad de apariencias y realidades (Alberro y
Gonzalbo 2009) que complejizó la conformación de una sociedad uniforme. Lo que
finalmente se consolidó después de la segunda mitad del siglo XVII fue un mundo
configurado por calidades sociales y no por razas o <<castas>>.


Ya en el siglo XVIII, con los avances científicos de Europa y con el avance de
la botánica y la zoología, seguramente la sociedad se preguntó ¿por qué no
clasificar también a las personas? Es entonces cuando, con el uso de las
técnicas modernas de la pintura se comenzó a pintar una sociedad pintoresca y
estratificada. En este sentido, los cuadros de castas son una evidencia
histórica de la clasificación social que las personas tenían de sí mismas. 


Es aquí en donde debemos tener cuidado de no dar por hecho que esas pinturas
reflejan en realidad la vida cotidiana de la Nueva España. Decimos esto porque
los registros parroquiales y algunas fuentes notariales nos dicen algo muy
distinto, es decir, en ellos no hemos encontrado todo ese abanico de grupos que
presentan los cuadros de castas. Hemos identificado a Españoles, mestizos,
castizos, indios, negros, mulatos, lobos, y en menor medida moriscos y demás
calidades o castas. 


Lo anterior significa que esa secuencia de imágenes grabadas en los cuadros de
castas nos hablan de calidades sociales, más no de razas por lo que su valor
recae en la información que nos brinda acerca del vestido, la comida, los
oficios, las modas, los paisajes, pero no necesariamente de una sociedad
tajantemente dividida.

\enlargethispage{1\baselineskip}
La multietnicidad como categoría opuesta a la sociedad de castas nos ofrece
mayor información en relación a la identidad novohispana porque no se ciñe a
simplemente asegurar que una casta permaneció siempre en el mismo nivel social
y cultural en el que nació, hay casos abundantes en donde los mulatos o
indígenas bautizados como tales, a lo largo de sus vidas tuvieron posibilidad
de cambiar de calidad social a partir del prestigio familiar, el reconocimiento
social y también el aclaramiento de la piel, lo mismo ocurrió con la población
española y \mbox{criolla}.


\bigskip
\textbf{Segunda Parte}

Hasta aquí, hemos intentado explicar de manera muy general, cómo se configuró la
identidad religiosa de la Nueva España y cómo se construyó una religiosidad
popular ajena a los dogmas católicos ortodoxos. De igual forma hemos hecho un
esbozo amplio de la configuración étnica que hizo posible la formación de una
sociedad multicultural. Sin embargo, cuál es la relevancia de todo ello. En
primer lugar, no podemos negar que en el siglo XXI, existen en México grupos
indígenas que aparecen en un padrón social como evidencia de la multietnicidad
mexicana. De Igual forma, en varios estados de la república como Guerrero y
Oaxaca (por mencionar a los más representativos) existen poblaciones
afrodescendientes que nos ofrecen la evidencia de la tercera raíz (Flores y
otros 2014). 


Así como existen estos grupos humanos mal llamados <<sociedades autóctonas>>,
existen evidencias contundentes de nuestra herencia virreinal en el ámbito del
pensamiento religioso. Por ejemplo, la celebración de las fiestas patronales en
las comunidades rurales y semiurbanas o bien, las peregrinaciones a los
santuarios religiosos en donde se congrega gente de todas las clases sociales y
de todos los niveles socio profesionales. En definitiva, estos ejemplos también
nos deben llamar la atención no sólo por el impacto social que hoy en día
pueden llegar a causar sino porque al igual de lo anterior (los grupos
multiétnicos) son improntas de la era virreinal que a pesar del largo proceso
formativo de <<lo mexicano>> sigue permeando en nuestros días.

Nuestra lista de ejemplos puede continuar con aspectos culinarios, del vestido,
de los instrumentos de uso cotidiano o inclusive de la lengua; sin embargo, lo
que está claro es que de ninguna manera podemos asumir la idea de que el
paradigma neoliberal a eliminado los rasgos identitarios de la sociedad
mexicana, desde luego que los ha modificado y ha hecho que éstos se adapten a
la realidad del siglo XXI. No obstante, es esta problemática de adaptación lo
que nos da pie a pronunciarnos, ahora más que nunca, en favor de una mayor
difusión del conocimiento virreinal no sólo en las escuelas sino en los medios
y espacios que la propia modernidad del siglo XXI ha abierto para comunicarnos;
es decir, no ser ajenos y reacios a trabajar, como científicos sociales y
humanísticos, al margen de las innovaciones tecnológicas sino por el contrario
explotarlas a fin de comunicar y difundir la cultura histórica y por tanto,
promover una conciencia histórica. 


\bigskip
\textbf{El reto de la Historia en el nuevo milenio}


El Plan de Desarrollo Nacional  (PND) de los Estados Unidos Mexicanos enfatiza
cinco metas para el logro de los objetivos del gobierno Federal y los
estatales. En cuanto a la Educación, la tercera meta <<México con Educación de
Calidad>> supone:


garantizar un desarrollo integral de todos los mexicanos y así contar con un
capital humano preparado, que sea fuente de innovación y lleve a todos los
estudiantes a su mayor potencial humano. Esta meta busca incrementar la calidad
de la educación para que la población tenga las herramientas y escriba su
propia historia de éxito. El enfoque, en este sentido, será promover políticas
que cierren la brecha entre lo que se enseña en las escuelas y las habilidades
que el mundo de hoy demanda desarrollar para un aprendizaje a lo largo de la
vida. En la misma línea, se buscará incentivar una mayor y más efectiva
inversión en ciencia y tecnología que alimente el desarrollo del capital humano
nacional, así como nuestra capacidad para generar productos y servicios con un
alto valor agregado (PDN 2014).


Está claro que frente a este reto de envergadura nacional, los profesionales de
la Historia tienen que desarrollar las herramientas y estrategias necesarias
para promover y desarrollar una enseñanza de calidad. De manera más específica,
consideramos que en lo que respecta a los Historiadores que investigan y
enseñan el periodo virreinal, esta meta nos obliga a recurrir a las nuevas
herramientas tecnológicas para lograr hacer asequible la complejidad del mundo
novohispano.


Sabemos que en México son numerosos los Centros de Investigación Histórica y de
igual manera el gremio de historiadores-investigadores crece constantemente.
Sin embargo ¿en qué medida, las investigaciones que se generan responden a las
necesidades primordiales de la sociedad mexicana? 


Cuando hablamos de las funciones sociales que debe cumplir un Historiador, el
listado es tan extenso como complejo pues algunos especialistas aseguran que su
rol dentro de la sociedad es el de <<reconstruir la historia, investigando y
difundiendo sus diferentes aspectos para construir  la historia total>>
(Sánchez 2008, p. 30). Otros  aseveran que el papel del Historiador varía según
su área de especialización ya que adquiere las competencias para investigar,
enseñar, difundir el conocimiento histórico o bien, para conservar el
patrimonio humano; de la misma manera hay quienes aseguran que más allá de
estas tareas su papel es el de inculcar la conciencia histórica y lograr un
cambio en el comportamiento humano.


Podemos presumir que algunas de las razones por las cuáles la sociedad que está
fuera de la academia no está interesada en la Historia es porque en ocasiones
las novedades del conocimiento histórico se presentan con lenguajes
especializados y de hecho la finalidad de estas contribuciones se limita sólo a
la crítica de los pares especialistas, dejando a un lado la labor comunicativa
de difundir de manera sencilla pero concreta la disciplina (Peralta 2013).

 
Este fenómeno de distanciamiento entre investigadores y sociedad se puede
explicar a través de la noción de historicidad de la siguiente manera. Si el
historiador-investigador es plenamente consciente que su papel consiste en
aportar a la sociedad el ingrediente de la conciencia histórica y que su
trabajo estará en función de las necesidades sociales, entonces podríamos
asumir que él, como ser histórico tiene un grado de historicidad determinado;
no obstante en la mayoría de los casos, los especialistas de la historia
dedicados a investigarla centran su atención en los procesos a investigar y
soslayan la parte de difusión; es decir, no prestan atención a si lo que hacen
e investigan puede realmente afectar y mejorar la calidad de vida de la
sociedad externa a su gremio o en el mejor de los casos, si su trabajo
contribuirá a la formación de la consciencia histórica (Peralta 2013,
\textit{passim}).


Asumimos que  la mayoría de las Universidades públicas del país insertas en el
paradigma de la educación por competencias han desarrollado sus programas
profesionales con la finalidad de que los estudiantes, a lo largo de sus
estudios universitarios, se vinculen con el mercado laboral, a través de
estancias o prácticas profesionales. Las competencias se han traducido en una
competitividad incesante, en donde para mantener vigente la condición de
investigador, se requiere realizar investigaciones en periodos de tiempo muy
breves, lo que conlleva a no mirar el problema desde un estadio más amplio; es
decir, parece ser que importa más la cantidad que la calidad.
\newpage

\textbf{La difusión de la Historia a través de las redes\\ sociales y los foros}


Entre las múltiples TIC, el siglo XXI ha visto popularizarse a las plataformas
virtuales y de manera específica a las comunidades virtuales que se aglutinan
en torno a un interés común. Estas agrupaciones o redes sociales virtuales no
son ajenas a los investigadores de tal suerte que Instituciones de primer
orden, como la Universidad Nacional Autónoma de México o el Colegio de México,
por citar dos ejemplos conocidos,\footnote{Podemos advertir que por lo menos,
más del 50\% de las Universidades del país, entre el año 2012 y 2014, han
abierto espacios virtuales para hacer difusión de sus actividades e
investigaciones y además, casi todas emplean el Internet como una plataforma de
intercambio y diálogo de ideas entre académicos y la sociedad en general.} han
recurrido a últimas fechas a hacer uso de la Internet para dar a conocer sus
avances científicos y tecnológicos.


Rebeca Valenzuela (2013) asegura que en el ámbito educativo del siglo XXI, no se
puede pasar por alto el uso de las redes sociales como herramienta de
enseñanza-aprendizaje. En este sentido, asumimos que nuestra tarea como
difusores del conocimiento histórico, ya no sólo del periodo virreinal sino de
cualquier otra época requiere echar mano de estos medios electrónicos.


De acuerdo a lo anterior, nos interesa traer a colación algunos ejemplos de los
que tenemos conocimiento y que sirven para demostrar como los historiadores
hemos logrado reducir la brecha entre el mundo académico y la sociedad en
general. En este sentido, la labor que han hecho las instituciones educativas y
los centros de investigación especializados son alicientes para que otras
agrupaciones académicas y no académicas comiencen a mirar de manera distinta a
las redes sociales.

En primer lugar, podemos traer a colación un grupo virtual administrado en una
red social conocida por la mayoría de la juventud mexicana ({\itshape Facebook}) llamada
<<Amantes de Clío>>. Se trata de una agrupación administrada por estudiantes de
posgrado de diferentes Instituciones académicas nacionales. Este grupo tiene
como objetivo:


\begin{sloppypar}
divulgar el conocimiento de
la disciplina histórica y proporcionar herramientas para su estudio. En Amantes
de Clío eres completamente libre para compartir recursos, difundir información,
colgar convocatorias, debatir o cuanto sea útil a nuestra formación\slash{}profesión.
No se discrimina a nadie, ya sean estudiantes de pregrado, posgrado, ya
investigadores, cronistas, o sencillamente aficionados a la historia. Nuestras
temáticas versan sobre la Historia de México\slash{}Nueva España\slash{}Mesoamérica
principalmente, pero cualquier otra es bien recibida (Amantes de Clío 2014).
\end{sloppypar}

Otro ejemplo de mayor alcance
a nivel nacional es la página electrónica de <<H-México>>. Su relevancia recae en
que es un medio electrónico que se actualiza día con día y que su objetivo es
hacer del conocimiento, tanto a investigadores especializados como al público
en general, y particularmente a sus asociados,\footnote{Para ser asociado de
H-México lo único que debe hacerse es crear un perfil y una cuenta a partir de
un correo electrónico
específico.}  acerca de las
novedades editoriales de la disciplina, congresos, convocatorias, ofertas de
trabajo, exposiciones y demás actividades de los centros de investigación e
instituciones académicas o bien de asociaciones distintas.


Para el caso de la difusión
del conocimiento sobre el periodo virreinal, en 2013, la Universidad Autónoma
del Estado de México (UAEMéx) a través de su Facultad de
 Humanidades, y el Instituto
de Investigaciones Filológicas de la Universidad Nacional Autónoma de México
convocaron a un Congreso de paleografía. Para el evento, se organizó a la par
un portal virtual en Facebook en donde se colocaron algunos fragmentos de las
mesas e informaciones relacionadas con el evento (Congreso de Paleografía
2013). Hasta donde tenemos conocimiento, el portal sigue vigente y en él se
sigue subiendo información relacionada con el trabajo archivístico y
paleográfico de México.


En noviembre de 2013 se
celebró el XXVI Encuentro de Investigadores del Pensamiento Novohispano en la
Ciudad de México. En el evento se trataron asuntos tanto historiográficos como
informativos acerca del estado actual que guardan los estudios del mundo
virreinal, En este sentido, también a través de las redes sociales dieron
difusión y promoción al congreso de tal suerte que la celebración estuvo
concurrida y enriquecida por distintos profesionales de diferentes partes del
país.

Un último ejemplo de
envergadura internacional y del cual tenemos conocimiento es el Congreso
Internacional de Familias y Redes Sociales con sede en la Universidad de
Sevilla, el cual se estará realizando en octubre de este año (2014). Esta
convocatoria a los investigadores tiene por objetivo discutir y reflexionar en
torno a la multietnicidad y la multiculturalidad en el mundo Atlántico desde el
siglo XVI al XXI.


\medskip
\textbf{A manera de cierre}

En definitiva, consideramos
que aún nos queda un camino por recorrer lleno de retos y obstáculos para poder
difundir una conciencia acerca de la relevancia que tiene el periodo virreinal
entre la sociedad mexicana; sin embargo, las TIC del siglo XXI y los nuevos
medios de comunicación masiva que se han popularizado en no más de un lustro se
prestan más que nunca para cerrar la brecha que nos queda. Sin embargo, debemos
ser plenamente conscientes los profesionales de la historia que aún con estas
ventajas tecnológicas nuestro compromiso para con la sociedad debe priorizar el
contacto humano; es decir, comunicar de manera
adecuada por qué debemos
considerar a la Historia no sólo como una <<materia>> escolar sino como un
proyecto social en el que estamos insertos todos.


Hemos ofrecido sólo dos
ejemplos de una tradición que nos ha acompañado a lo largo de nuestra historia
nacional: la religiosidad y la multietnicidad. Esta lista puede ampliarse
conforme caminemos en las grandes etapas históricas. Por ejemplo habría que
abrir un análisis para entender al indigenismo de los siglos XIX y XX o más
cercano a nuestros tiempos, valdría la pena explotar las redes sociales y los
foros  para reflexionar sobre la transición de poderes políticos en las
primeras décadas de nuestro siglo XXI.


Lo que está claro es que como
profesionales de la Historia y de la difusión y enseñanza de la misma no
podemos sólo centrarnos en nuestras investigaciones sino que debemos abrir
nuestros intereses a las nuevas innovaciones culturales y tecnológicas para dar
a conocer nuestro trabajo.
\newpage

%\medskip
\textbf{Referencias}

\medskip
Alberro, Solange y Pilar Gonzalbo (2013), \textit{La Sociedad novohispana.
Estereotipos y realidades}, México, El Colegio de México.


Arce Sáinz, María Marcelina, Jorge Velázquez Delgado y Gerardo de la Fuente Lora
(coords.) (2010), \textit{Barroco y cultura novohispana. Ensayos
interdisciplinarios sobre filosofía política, barroco y procesos culturales:
cultura novohispana}. México, Ediciones León\slash{}Benemérita Universidad Autónoma
de Puebla.


Berenzon Gorn, Boris (1993), <<La difusión de la Historia en México: La identidad
imaginaria>> en \textit{Anales de Antropología}, Revista del Instituto de
Investigaciones Antropológicas de la Universidad Nacional Autónoma de México,
vol. 30, núm. 1, pp. 145-181. Disponible en:
\url{http://www.revistas.unam.mx/index.php/antropologia/article/view/16980}


Blázquez Espinoza, José Carlos, Paulina Latapí Escalante, Hugo Torres Salazar
(comps.) (2013), \textit{Memoria del Cuarto Encuentro Nacional de Docencia,
Difusión y Enseñanza de la Historia, Segundo Encuentro Internacional de
Enseñanza de la Historia, Tercer Coloquio entre Tradición y Modernidad} [disco
compacto], México, Universidad Autónoma de Querétaro\slash{}Benemérita Universidad
Autónoma de Puebla.


Brading, David (2006), \textit{Orbe Indiano, de la monarquía católica a la
república criolla 1492-1867}, México, Fondo de Cultura Económica. 7ª reimp.


Bribiesca Sumano, María Elena (2010), \textit{La religiosidad popular en el
valle de Toluca a través de los testamentos, 1565-}1623, México, [Tesis de
Maestría], Universidad Pontificia de México.
\newpage

Carrillo Cázares, Alberto (2006), \textit{Manuscritos del concilio tercero
provincial mexicano (1585). Edición, estudio introductorio y traducción de
textos latinos por Alberto Carrillo Cázares, }Tomo 1, Vol. 1, México, El
Colegio de Michoacán\slash{}Universidad Pontificia de México.


Chartier, Roger (2005), \textit{El presente del pasado. Escritura de la
Historia, Historia de lo escrito}, México, Universidad Iberoamericana.

 
Chávez Sánchez, Eduardo (1996), \textit{Historia del Seminario Conciliar de
México}, Vol.1, México, Editorial Porrúa.


Defurneaux, Marcelin (1964), \textit{La vida cotidiana en España durante el
siglo de oro}, Argentina, Harchete.

\begin{sloppypar}
Fernández, Martha (2010), <<El retablo barroco: sus tipologías y su mensaje
simbólico>> en María Marcelina Arce Sáinz, Jorge Velázquez Delgado y Gerardo de
la Fuente Lora (coords.), \textit{Barroco y cultura novohispana. Ensayos
interdisciplinarios sobre filosofía política, barroco y procesos culturales:
cultura novohispana}, México, Ediciones León\slash{}Benemérita Universidad Autónoma
de Puebla.
\end{sloppypar}

Flores García, Georgina, María Elene Bribiesca Sumano y otros (2014),
\textit{Azúcar, esclavitud y enfermedad en la hacienda de Xalomolnga, siglos
XVII}, Toluca, Universidad Autónoma del Estado de México.


Fontana, Josep (1992), \textit{La Historia después del fin de la Historia}, Barcelona, Crítica.


Gonzalbo Aizpuru, Pilar  y Verónica Zárate Toscano (coords.) (2007),
\textit{Gozos y sufrimientos en la Historia de México}, México, El Colegio de
México\slash{}Instituto de Investigaciones Dr. José María Luis Mora.


Gonzalbo Aizpuru, Pilar (2009), \textit{Vivir en Nueva España. Orden y desorden
en la vida cotidiana}, México, El Colegio de México.

Kamen, Henri (1999), \textit{La Inquisición española. Una revisión Histórica},
España. Crítica.


López de Ayala, Ignacio (comp.) (1785), \textit{El sacrosanto y ecuménico
Concilio de Trento traducido al idioma Castellano por don Ignacio López de
Ayala. Agregase el texto latino }\textit{corregido según la edición auténtica
de Roma publicada en 1564}, Madrid, Imprenta Real, 2ª ed.


Lorenzana y Butrón, Francisco Antonio (comp.) (1769), \textit{Concilios
provinciales Primero y Segundo celebrados en la muy noble y muy leal Ciudad de
México, presidiendo el Ilustrísimo y reverendísimo Señor Don Fray Alonso de
Montufar en los años 1555 y 1565. Dalos a luz el Ilustrísimo Señor Don
Francisco Antonio Lorenzana. Arzobispo de esta Santa Metropolitana Iglesia},
México, Imprenta del Superior Gobierno de el Bachiller D.~Joseph Antonio de
Hogal.


Maravall, Antonio (1998), \textit{La cultura del barroco, Análisis de una
estructura histórica}, España, Editorial Ariel.


Mazín, Oscar (2007), \textit{Iberoamérica. Del descubrimiento a la
independencia}, México, El Colegio de México.


Peralta Peralta, Marco Antonio (2013), <<La Historicidad en la formación de
profesionales de la Historia de la Universidad Autónoma del Estado de México>>
en José Carlos Blázquez Espinoza, Paulina Latapí Escalante, Hugo Torres Salazar
(comps.) \textit{Memoria del Cuarto Encuentro Nacional de Docencia, Difusión y
Enseñanza de la Historia, Segundo Encuentro Internacional de Enseñanza de la
Historia, Tercer Coloquio entre Tradición y Modernidad} [disco compacto],
México, Universidad Autónoma de Querétaro\slash{}Benemérita Universidad Autónoma de
Puebla, pp. 298--307. 
\newpage

Rubial García Antonio (1999), \textit{La santidad controvertida. Hagiografía
criolla alrededor de los venerables no canonizados de Nueva España}, México,
Universidad Nacional Autónoma de México\slash{}Fondo de Cultura Económica.


Rubial García, Antonio (2010), \textit{El paraíso de los elegidos, Una lectura
de la Historia cultural de Nueva España (1521-1804)}, México, Fondo de Cultura
Económica /Universidad Nacional Autónoma de México.


Rubial García, Antonio (coord.) (2013), \textit{La Iglesia en el México
Colonial,} México, Ediciones Educación y Cultura\slash{}Universidad Nacional
Autónoma de México\slash{}Benemérita Universidad Autónoma de Puebla.


Sánchez Reyes, Gabriela (2007), <<Entre el dolor y la curación: la relación entre
los milagros y las imágenes religiosas como remedio de enfermedades>> en Pilar
Gonzalbo Aizpuru y Verónica Zárate Toscano (coords.), \textit{Gozos y
sufrimientos en la Historia de México}, México, El Colegio de México\slash{}Instituto de Investigaciones Dr. José María Luis Mora.


Valenzuela Argüelles, Rebeca (2013), <<Las redes sociales y su aplicación en la
educación>> en \textit{Revista Digital Universitaria}, Coordinación de Acervos
Digitales. Dirección General de Cómputo y Tecnologías de la Información y
Comunicación de la Universidad Nacional Autónoma de México, vol. 14, núm. 4,
abril, pp. 1--14,  disponible en 
\url{http://www.revista.unam.mx/vol.14/num4/art36/art36.pdf}


Velázquez G., María Elisa (coord.) (2011), \textit{ Debates históricos
contemporáneos. Africanos y afrodescendientes  en México y Centroamérica}, México, INAH/CEMCA/IRD/UNAM/AFRODESC,  (Serie Africanías 7).


Wobeser, Gisela von (2010), \textit{Cielo, infierno y purgatorio en el
virreinato de Nueva España}, México, Universidad Nacional Autónoma de México\slash{}Editorial Jus.


Zoraida Vázquez, Josefina (coord.) (1992), \textit{Interpretaciones del siglo
XVII mexicano: el impacto de las reformas borbónicas}, México, Nueva Imagen.


\bigskip
\textbf{Referencias electrónicas}


<<Amantes de Clío>>, Plataforma virtual de difusión, disponible en:
\url{https://www.facebook.com/groups/325047715911/?fref=ts}


<<Centro de Estudios Históricos Ceh>> del Colegio de México, Plataforma virtual de difusión
disponible en: \url{https://www.facebook.com/centrodeestudioshistoricos.ceh?fref=ts}

\begin{sloppypar}
<<Coloquio Nacional sobre
Paleografía y diplomática>>, Plataforma virtual de difusión disponible en:
\url{https://www.facebook.com/1er.Coloquio.Paleografia?fref=ts}
\end{sloppypar}


<<H-México>> Plataforma virtual de divulgación y difusión especializada en
Ciencias Sociales y Humanidades, disponible en:
\url{https://www.facebook.com/pages/H-MEXICO/84628605603?fref=ts}


<<Plan de Desarrollo Nacional>>, Presidencia de la República, disponible en:
\url{http://pnd.gob.mx/}
\newpage
\thispagestyle{empty}
\phantom{abc}
