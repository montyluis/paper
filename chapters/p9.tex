%\documentclass{article}
%\usepackage{amsmath,amssymb,amsfonts}
%\usepackage{fontspec}
%\usepackage{xunicode}
%\usepackage{xltxtra}
%\usepackage{polyglossia}
%\setdefaultlanguage{spanish}
%\usepackage{color}
%\usepackage{array}
%\usepackage{hhline}
%\usepackage{hyperref}
%\hypersetup{colorlinks=true, linkcolor=blue, citecolor=blue, filecolor=blue, urlcolor=blue}
%\newtheorem{theorem}{Theorem}
%\title{}
%\author{Vanessa Moreno}
%\date{2014-06-17}
%\begin{document}

%%\clearpage\setcounter{page}{151}
\thispagestyle{empty}
\phantomsection{}
\addcontentsline{toc}{chapter}{Estrategias de Enseñanza desde el Aula\newline $\diamond$
\normalfont\textit{Vanessa Magaly Moreno Coello,
Patricia Gutiérrez Casillas\newline y
Mario Heriberto Arce Moguel}}
{\centering {\scshape \large Estrategias de Enseñanza desde el Aula}\par}
\markboth{la formación del historiador}{estrategias de enseñanza}
\setcounter{footnote}{0}


\bigskip
\begin{center}
{\bfseries Vanessa Magaly Moreno Coello\\
Patricia Gutiérrez Casillas\\
Mario Heriberto Arce Moguel}\\
{\itshape\ Universidad Autónoma de Chiapas\/} 
\end{center}

\bigskip
{\bfseries Resumen}

El objetivo es compartir desde nuestra experiencia como docentes de la 
licenciatura en historia, y en específico de las materias: debates 
historiográficos actuales e historiografía moderna y contemporánea las 
estrategias de enseñanza que hemos implementado y los resultados 
obtenidos. 


En debates historiográficos actuales consideramos prioritario que el 
estudiante maneje el análisis y la discusión sobre los debates actuales 
alrededor de la historiografía, tanto a nivel local como internacional. 
En historiografía moderna y contemporánea se plantea que el estudiante 
haga uso del análisis historiográfico como método para estudiar e 
interpretar las fuentes históricas, por medio del conocimiento de los 
procesos de pensamiento que se dieron en la producción historiográfica 
en las épocas moderna y contemporánea.


Tomando en cuenta las particularidades y la diversidad de los alumnos 
de historia (UNACH), aunado a que el equipo tecnológico y tiempo para 
trabajar con ellos son insuficientes, hemos ejecutado varios recursos 
didácticos, de acuerdo con la secuencia y las características de las 
materias. En la práctica combinamos herramientas con las que los 
estudiantes pongan en práctica lo aprendido en clase, ya sea por medio 
del debate de las lecturas propuestas para cada sesión, exposición de 
temas elegidos o la redacción de un pre ensayo que les permita generar 
conocimiento y; a nosotros visualizar los temas que necesitan ser 
reforzados. 

Como parte sustancial de la enseñanza en la materia se instruye al 
alumno a construir un ensayo, como producto final, esto nos permitirá 
evaluar no sólo los conocimientos adquiridos, sino el uso de las 
herramientas implementadas durante el semestre.\\
{\bfseries Palabras clave:} estrategias de enseñanza, experiencia docente, alumnos, 
recursos didácticos y herramientas de aprendizaje.


\bigskip
\textbf{Desde nuestra experiencia}


En la actualidad es necesario buscar y ensayar estrategias de enseñanza 
que nos faciliten brindar apoyo respecto al aprendizaje de los alumnos 
de la carrera de Historia. Como profesores necesitamos aplicar métodos 
y técnicas que nos permitan trascender la clase y no solo limitarnos al 
discurso teórico. Convencidos de que mejorar la manera en que ejercemos 
la práctica didáctica en el aula puede dar resultados motivantes, para 
la investigación de los temas en clase. 


Nuestro objetivo se centra en guiar al estudiante en el camino a la 
cons\-truc\-ción de una historia consciente y humanista, para evitar que se 
pierda en la búsqueda del conocimiento total de la realidad y centre 
sus esfuerzos en reconstruir y analizar un fragmento del pasado a 
partir de un conocimiento transversal, dando paso a una nueva 
percepción sobre éste. 
 
El proceso de inducción del estudiante se realiza a través del contacto 
con lecturas que correspondan al periodo historiográfico que abarca el 
curso. A partir del material seleccionado se les proporciona una serie 
de categorías que tienen que analizar en cada lectura. Con este 
ejercicio el estudiante se ve en la necesidad de realizar la lectura 
del texto y ubicar cada categoría solicitada.

En principio nos damos a la tarea de identificar el periodo histórico 
al que se refiere la obra, el autor, sus obras principales, sus 
influencias y el periodo histórico al que corresponde la obra a tratar 
en clase. Las categorías a trabajar están enfocadas al análisis del 
texto histórico, como el origen del texto, cuestionándonos la 
circunstancia en que fue escrito, explicando el momento histórico en 
que se escribió, ya sea una obra individual o colectiva y a partir de 
ello discutir sobre la intencionalidad del texto.

Posteriormente cuestionamos el objetivo de la obra, respecto a lo que 
el autor resalta y cuál es el objetivo de ello. Identificamos al o a 
los sujetos de la historia, el marco teórico ideológico en el que está 
la propuesta del autor. 

\enlargethispage{1\baselineskip}
El alumno presenta en clase el análisis de la lectura y el profesor 
realiza una revisión de cada categoría, confirmando o corrigiendo cada 
respuesta. Con este ejercicio el estudiante compara sus respuestas con 
las proporcionadas por el profesor y sus otros compañeros, de tal forma 
se aclaran cada categoría y se asimila cada concepto analizado.

Al momento de realizar el análisis en clase es necesario que el 
profesor induzca a los alumnos a la reflexión de cada respuesta que 
proporcionan, el ejercicio está enfocado a generar una actitud crítica 
del estudiante ante las distintas corrientes que se estudian.

La evaluación se da en cuatro niveles, la primera es de diagnóstico la 
cual permite identificar los conocimientos previos del alumno, 
asegurando el punto de partida sin dejar vacíos de información. En 
segundo lugar está la evaluación formativa que se lleva a cabo por 
medio de ejercicios de retroalimentación, en este caso mesas de 
discusión, al término de cada tema para que los alumnos expresen sus 
dudas y discutan con sus compañeros las posibilidades y los límites de 
cada corriente historiográfica.

Otra forma de evaluación es que los estudiantes preparen la exposición 
de un tema con el fin de que desarrollen el análisis crítico e 
interpretación de textos y que puedan comparar las diversas posiciones 
de los autores con respecto a una corriente. Para finalizar y como 
producto de las materias, cada alumno entrega un ensayo sobre un tema 
específico, o el análisis historiográfico de un texto, según su 
elección.


\textbf{Desarrollo de las materias}

Con lo anterior pretendemos que la materia de historiografía moderna y 
contemporánea que se imparte en cuarto semestre, sirva para que los 
alumnos sean capaces de utilizar el análisis historiográfico como 
método para estudiar e interpretar las fuentes, y que conozcan los 
principales procesos de pensamiento que tuvieron lugar en el campo de 
la producción historiográfica en las épocas moderna y contemporánea.

\enlargethispage{1\baselineskip}
Con este fin, iniciamos el curso con el pensamiento cartesiano las 
interrogantes que generan la discusión en el aula tienen como fin 
dilucidar la influencia que tuvo sobre las ciencias y las humanidades, 
y en particular sobre la disciplina histórica, en el siglo XVII\@. Lo que 
se pretende es fomentar el interés y apertura del pensamiento para 
entender el pensamiento filosófico desarrollado por René Descartes y 
cuya influencia se dejó sentir en el desarrollo de la historiografía a 
partir de su escepticismo. 


El debate en el aula no sólo se centra en la polémica discusión que se 
generó entre Descartes y Collingwood y, que dio paso a una corriente 
crítica de donde nace la escuela historiográfica cartesiana, también se 
enfoca en analizar el uso y abuso de las fuentes en las que caen 
algunos historiadores que mal fundamentan su trabajo en el 
cartesianismo.

%\enlargethispage{1\baselineskip}
Examinamos las principales características del racionalismo y la 
Ilustración en el siglo XVIII, tomamos en cuenta que es a partir de 
este siglo que los conceptos en la historia van a cambiar y se empieza 
a construir la historia de los Estados Nacionales. La influencia en el 
pensamiento histórico de la Ilustración y sus principales exponentes 
son analizados en esta unidad temática. 

Mediante los conocimientos correspondientes pretendemos que los 
estudiantes generen un pensamiento crítico a través del análisis que 
les permita comprender la formación de los Estados Nacionales modernos 
a partir de la Revolución Francesa, el ascenso de la clase burguesa y 
la diferencia con el estado nacional identificado con la monarquía.
 
La tercera unidad del curso plantea las principales tendencias 
historiográficas del siglo XIX, en particular el romanticismo, el 
idealismo filosófico, el liberalismo, el positivismo y la escuela 
científica alemana. Formando una dinámica de trabajo que nos permite 
generar habilidades de sistematización, clasificación y síntesis.

%\enlargethispage{1\baselineskip}
Para la última unidad trabajamos los principales aspectos de las 
tendencias historiográficas desarrolladas en la primera mitad del siglo 
XX, en particular el materialismo histórico, el historicismo, el 
neoprovidencialismo y el existencialismo, mediante los conocimientos 
correspondientes, con las habilidades de pensamiento crítico, capacidad 
de análisis, síntesis y evaluación.

%\enlargethispage{1\baselineskip} 
Respecto a la materia debates historiográficos actuales el propósito es 
introducir al estudiante al análisis y discusión sobre los debates 
actuales alrededor de la historiografía, tanto a nivel local como 
internacional.

Las unidades de estudio son diversas, iniciamos el recorrido con la 
crisis e impacto de la posmodernidad, la microhistoria italiana, la
historia de las representaciones y la nueva historia cultural, la historia 
de los {\itshape\ Subaltern Studies\/} y el poscolonialismo, la historia de las mujeres, 
la historia del género y finalmente con la historia a Debate.

Iniciamos el curso partiendo de la exposición del contexto general del 
surgimiento de la posmodernidad, con ello conceptualizamos el término e 
identificamos las principales características de la posmodernidad, el 
Impacto de estos posicionamientos en torno al discurso historiográfico 
y los principales representantes. La discusión en clase nos presenta 
las diversas interpretaciones que los estudiantes realizan al comparar 
las diversas posturas en torno a la posmodernidad. 

Con ello identifican los puntos relevantes que la posmodernidad 
cuestiona y trastoca en la historiografía. El objetivo es generar 
interés por el diálogo y la reflexión para la construcción de 
conocimientos significativos. De tal manera que el alumno obtenga la 
información necesaria para conocer el contexto en el cual surgió la 
posmodernidad, la crítica desde este planteamiento a la modernidad, así 
como el impacto que tuvieron sus representantes en la construcción no 
sólo del discurso histórico, sino en general de las ciencias sociales.

Al trabajar la microhistoria italiana, en clase presentamos los 
orígenes y representantes, sus principios generales y posicionamientos 
teórico metodológico, la situación actual, su relación con la Escuela 
de los Annales y otras disciplinas sociales. Exponemos también los 
conocimientos generales sobre la historia regional, para poder 
diferenciarla de la microhistoria. Para ello es importante que se 
interrelacione el contexto de surgimiento de la microhistoria hasta el 
momento actual, ello para poner en perspectiva la relevancia de esta 
forma historiográfica.

Con la historia de las representaciones y nueva historia cultural el 
estudiante podrá definir los términos cultura y representaciones, no 
sólo como conceptualizaciones abstractas sino como categorías de 
análisis del discurso historiográfico. Ello le ayudará a percibir las 
diferencias y las interrelaciones con la antropología y la sociología, 
y lo que distingue a la historia cultural y a los estudios 
antropológicos.

La historia de los {\itshape\ Subaltern Studies\/} y el poscolonialismo 
se presenta en clase a partir de los orígenes e impulsores de los estudios 
subalternos. Los conceptos claves que se definen son hegemonía y 
subalterno. Se definen las características de la corriente crítica 
poscolonial. A partir del cuestionamiento de la historia «tradicional» 
bajo los posicionamientos metodológicos de los estudios subalternos. Se 
desea que el alumno conozca diferentes posturas en la construcción del 
conocimiento histórico desde puntos de referencia y contextos 
distintos, es decir propuestas «alternativas» a los conocimientos 
hegemónicos derivados del viejo mundo. 
%\enlargethispage{-1\baselineskip}

Con la historia de las mujeres, historia del género, acercamos a los 
estudiantes al género como una categoría de análisis, colocamos en el 
debate el papel de las mujeres como nuevos sujetos en la historia. 
Discutimos el impacto del giro lingüístico en la construcción de una 
historia de género, las relaciones de género, su interrelación y 
construcción histórica. Ello ayuda a diferenciar los distintos 
posicionamientos teórico{-}metodológicos de la historiografía con enfoque 
de género. Se busca que el alumno comprenda la importancia de la 
inserción de las mujeres como nuevos sujetos históricos.


Para finalizar el curso discutimos respecto a la propuesta 
historiográfica de historia a debate, iniciamos con la presentación de 
sus representantes y las actividades que realizan. Llevando la 
discusión en clase sobre la interdisciplina que trabaja historia a 
debate e Identificamos los posicionamientos de la historia a debate con 
respecto a la construcción del conocimiento historiográfico. 


\bigskip
\textbf{Reflexiones finales}

A partir de las particularidades de cada materia y de la intención de 
cada docente sobre como impartir el conocimiento y como se proporciona 
a los estudiantes, es que generamos nuestras estrategias de enseñanza 
que contienen una lógica en el orden de las actividades a realizar.


Consideramos que el avance en las ciencias sociales y sobre todo en la 
disciplina de la historia nos obliga a desarrollar dinámicas 
metodológicas que encaminen a los estudiantes a reformular nuevas 
formas discursivas para acrecentar la epistemología de esta ciencia. La 
enseñanza de esta disciplina histórica debe fortalecerse y lograr que 
al egresar de la licenciatura el alumno cuente con las herramientas 
teóricas-metodológicas necesarias que le permita desarrollarse en un 
posgrado en cualquier institución del país.

Nuestro compromiso es presentar al alumnado una propuesta que contribuya a su
proceso formativo profesional, de manera sencilla y flexible donde se incluya
los nuevos aportes epistemológicos en el campo de la Historia, que les facilite
los retos a los que se enfrentarán como egresados. En palabras de Luis González
y González (1999), los nuevos historiadores encuentran que las exigencias
modernas obligan no sólo a escalar más grados (maestría, doctorado), sino que
ahora se busca que se especialicen en un área de la historia, se busca la
profesionalización de la historia al exigir que el historiador delimite
escrupulosamente su espacio geográfico y temporal.


Las exigencias, como sabemos no quedan ahí, es preciso y necesario que el
historiador presente una propuesta teórica que enriquezca y consolide la
construcción de la Historia como una ciencia, el estudiante debe de ser
consciente de que la tradición de la historiografía urge de esto. 

Sin embargo estamos conscientes de que el egresado de la licenciatura 
de Historia de la UNACH, no debe de adquirir herramientas que lo 
enfoquen solamente en la investigación o la elaboración de proyectos 
históricos, se busca además que desarrolle habilidades para la 
docencia, la divulgación de la historia o que desde la institución se 
desenvuelva como gestor del patrimonio sociocultural. 


Con el perfil trazado a lo largo de la carrera se pretende que el 
Licenciado en Historia al graduarse tenga una amplia gama de opciones 
donde pueda laborar y se desarrolle en espacios de acorde a formación, 
desde técnico (museos, archivos históricos, fototecas, hemerotecas) 
hasta centros de investigación, hay diversas opciones a las que puede 
aspirar. Sin olvidar que la docencia en historia en sus diferentes 
niveles siempre es una elección muy loable.

Las materias de debates historiográficos actuales e historiografía 
moderna y contemporánea, son parte medular en la cimentación de esta 
disciplina, el estudiante que no logre comprender su aporte en la 
construcción del conocimiento histórico no podrá ejercer el análisis y 
la crítica sobre las fuentes, por lo tanto no fracasará en su aporte a 
la riqueza historiográfica de México y de Chiapas.
\newpage

\textbf{Referencias}

\medskip
Alcoff, Linda (1989), «Feminismo cultural versus Post{-}estructuralismo: la crisis
de la identidad de la teoría feminista» en Feminaria, Año II, Núm. 4 Buenos Aires, Noviembre. 

Anderson, Perry (1980), \textit{El Estado absolutista}, México, Siglo XXI
Editores.

Barret, Michéle y Phillips, Anne (comp.) (2002), \textit{Desestabilizar la
teoría}, Debates feministas contemporáneos, Programa Universitario de Estudios
de Género, Universidad Nacional Autónoma de México, Colección Género y
Sociedad. 

Barros, Carlos (ed.) (1995), \textit{Historia a debate}, Santiago de
Compostela, Tomo I, II, III\@. 


Barroso Acosta, Pilar {\itshape et al.\,} (coord.) (1994), \textit{El pensamiento
histórico: ayer y hoy. II\@. Del Iluminismo al Positivismo}, México, UNAM
(Coordinación de Humanidades, Lecturas Universitarias, 37).


Berlín, Isaiah (1986), \textit{Contra la corriente}, México, FCE\@.


Burkhardt, Jacob (1961), \textit{Reflexiones sobre la historia universal},
México, FCE\@.


Burkhardt, Jacob (1945),\textit{Del paganismo al cristianismo: la época de
Constantino el Grande}, México, FCE\@.


Berman, Marshall (1989), Todos los sólidos se desvanecen en el aire. La
experiencia de la modernidad, México, Siglo XX\@. 


Burke, Peter (1997), \textit{Historia y teoría social}, México, Instituto Mora.

 
Carbonell, Charles{-}Oliver (1986), \textit{La historiografía}, México, FCE\@. 


Cassirer, Ernst (1972), \textit{ Filosofía de la Ilustración}, México, FCE\@.

Chartier, Roger (1992), \textit{El mundo como representación. Historia
cultural: entre práctica y representaciones}, España, Gedisa Editorial. 


\_\_\_\_\_\_ (1994), \textit{Lecturas y lectores en la Francia del
Antiguo Régimen}, México, Instituto Mora. 

\_\_\_\_\_\_ (1994), \textit{El orden de los libros. Lectores, autores y
bibliotecas en Europa, entre los siglos XV y XVIII}, Barcelona, Gedisa
Editorial. 


Childe, V. Gordon (1974), \textit{Teoría de la historia}, Buenos Aires,
Pléyade. 

Colmenares, Germán (1989), \textit{Las convenciones contra la cultura}, Bogotá,
Tercer Mundo Editores. 


Comte, Augusto (1942), \textit{Primeros ensayos}, México, FCE\@.

Corcuera, Sonia (1997), \textit{Voces y silencios en la Historia}, México,
FCE.

Croce, Benedetto, (1960) \textit{La historia como hazaña de la libertad},
México, FCE\@.

Cusicanqui, Silvia Rivera y Rossana Barragán (comps.) (1998), \textit{Debates post
coloniales: Una Introducción a los estudios de la subalternidad}, La Paz,
Historias, Sephis, Aruwiyiri. 


Darton, Robert (1987), \textit{La gran matanza de gatos y otros episodios en la
historia de la cultura francesa}, México, FCE. 


\_\_\_\_\_\_ (2002), «Intellectual an Cultural History» en Michael de
Baecque, Antoine, «La princesa de Cambelle o el sexo destrozado» en Graphen,
Revista de Historiografía, Jalapa, INAH, Veracruz, núm\@. 1, año 1. 

\enlargethispage{1\baselineskip}
De Certau, Michel (1993), \textit{La escritura de la Historia}, México,
Universidad Iberoamericana. 


Descartes, René (1968), \textit{Discurso del método}, Buenos Aires, Aguilar.

Dilthey, Wilhelm (1978), \textit{Hombre y mundo en los siglos XVI y XVII},
México, FCE. 

\_\_\_\_\_\_ (1990), \textit{Teoría de las concepciones del mundo}, México,
Alianza\slash{}CNCA\@. 

Droysen, Johann Gustav (1988), \textit{Alejandro Magno}, México, FCE\@. 

Duchet, Michele (1984), \textit{Antropología e historia en el Siglo de las
Luces}, México, Siglo XXI Editores. 

Foucault, Michel (1977), \textit{La arqueología del saber}, México, Siglo XXI\@.

 
\_\_\_\_\_\_ (2012), \textit{Vigilar y Castigar}, Madrid, Editorial Biblioteca Nueva\@. 


\_\_\_\_\_\_ (1981), \textit{Las palabras y las cosas}. \textit{Una
arqueología de las ciencias humanas}, México, Siglo XXI\@. 


García Cantú, Gastón (1971), Antología. \textit{Textos de historia universal de
fines de la Edad Media al siglo XX}, México, UNAM\@. 


Geertz, Clifford (1999), «Descripción densa: hacia una teoría interpretativa de
la cultura», en Perla Chinchilla Pawling
(comp.), {\itshape Historia e interdisciplinariedad}, México, Universidad Iberoamericana.

\begin{sloppypar}
Ginzburg, Carlo (1993), «\textit{Just one witness}», en Saul Friedlander
(comp.), {\itshape Probing the limits of representation. Nazism and the final solution},
Cambridge, Harvard University Press. 
\end{sloppypar}

 
\_\_\_\_\_\_ (1997), \textit{El queso y los gusanos. El cosmos según un
molinero del siglo XVI}, México, Editorial Océano. 

\_\_\_\_\_\_ (2004), \textit{Tentativas}, México, Prohistoria Ediciones. 

Gonzalbo Aizpuro, Pilar (2004), \textit{Historia de la vida cotidiana en
México}, México, FCE, Tomos I y II\@. 


González y González, Luis (1999), \textit{El oficio de historia}, México, El Colegio de
Michoacán. 


Gooch, G. P (1978), \textit{Historia e historiadores en el siglo XIX}, México,
FCE\@.


Hegel, George W. F. (1980), \textit{Lecciones sobre la Filosofía de la Historia
Universal}, Madrid, Alianza Editorial. 


Herrera Ibáñez, Alejandro (1972), \textit{Antología del Renacimiento a la
Ilustración}. Textos de Historia Universal, México, UNAM\@. 


Humboldt, Alejandro de (1991), \textit{Ensayo Político sobre el Reino de la
Nueva España}, México, Editorial Porrúa. 


Hume, David (1981), \textit{Investigación sobre el conocimiento humano,
}, Madrid, Alianza (El libro de bolsillo, 787). 


Jaspers, Karl (1980), \textit{Origen y meta de la historia}, Madrid, Alianza Editorial. 


Lamas, Marta (2003), «Usos, dificultades y posibilidades de la categoría de
género, en Lamas Marta, \textit{El género. La construcción cultural de la
diferencia sexual}, Programa Universitario de Estudios de Género, México,
Editorial Porrúa.


Leacock, Eleanor (1983), «La interpretación de los orígenes de la desigualdad
entre los géneros: problemas conceptuales e históricos», en Ramos C.
(comp), \textit{El género en perspectiva}, México, UAM Iztapalapa. 

Lefebvre, Henri (1983), \textit{La presencia y la ausencia. Contribución a la
teoría de las representaciones}, México, FCE. 


Marx, Carlos; Engels, Federico (1973), \textit{Manifiesto del Partido
Comunista}, Moscú, Editorial Progreso. 


Marx, Carlos (1985), \textit{El 18 brumario de Luis Bonaparte}, Moscú,
Progreso.

 
Marx, Carlos (1975), \textit{Materiales para la historia de América Latina},
Argentina, Córdoba. 


Mendiola, Alfonso (2000), «El giro lingüístico: la observación de las
observaciones de pasado» en {\itshape Historia y Grafía}, México, Universidad
Iberoamericana, núm. 15. 


Michelet, Jules (1991), \textit{El pueblo}, México, FCE\@. 


Mommsen, Teodoro (1982), \textit{El mundo de los Césares}, México, FCE\@. 


Montesquieu (1990), \textit{El espíritu de las leyes}, México, Editorial Porrúa
(Sepan cuantos, 191).


Ortega y Gasset, José (2007), \textit{Historia como sistema}, Madrid,
Biblioteca nueva. 


Ortega y Medina, Juan (1980), \textit{Teoría y crítica de la historiografía
científico idealista alemana}, México, UNAM, IIH\@.


Pérez Ruiz, Maya Lorena (2004), \textit{Tejiendo historias. Tierra, género y
poder en Chiapas}, México, Colección Científica, Instituto Nacional de
Antropología e Historia. 


Ranke, Leopold Von (1979), \textit{Pueblos y estados en la historia moderna},
México, FCE\@. 


Rousseau, Juan Jacobo (1972), \textit{El origen de la desigualdad entre los
hombres}, México, Grijalbo. 

Scott, Joan (2003), «El género: una categoría útil para el análisis histórico»
en Marta Lamas, \textit{El género. La construcción cultural de la diferencia
sexual}, México, Programa Universitario de Estudios de Género\slash{}Editorial
Porrúa.


Toynbee, Arnold J. (1970), \textit{Estudio de la historia}, Madrid, Alianza Editorial. 


Tuñón, Esperanza (1997), \textit{Mujeres en escena: de la tramoya al
protagonismo (1982--1994)}, México, Editorial Porrúa. 


Vázquez de Knauth, Josefina (1973), \textit{Historia de la historiografía},
México, SepSetentas.


Vico, Giambattista (1978), \textit{Principios de una ciencia nueva en torno a
la naturaleza común de las naciones}, México, FCE\@. 

Voltaire, Francois (1954), \textit{El siglo de Luis XIV}, México, FCE\@.

Wagner, Fritz (1980), \textit{La ciencia de la Historia}, México, UNAM\@.

Weber, Max (1989), \textit{La ética protestante y el espíritu del capitalismo},
México, Ed\@. Premia.
\newpage
\thispagestyle{empty}
\phantom{abc}
%%\end{document}

%\documentclass{article}
%\usepackage{amsmath,amssymb,amsfonts}
%\usepackage{fontspec}
%\usepackage{xunicode}
%\usepackage{xltxtra}
%\usepackage{polyglossia}
%\setdefaultlanguage{spanish}
%\usepackage{color}
%\usepackage{array}
%\usepackage{supertabular}
%\usepackage{hhline}
%\usepackage{hyperref}
%\hypersetup{colorlinks=true, linkcolor=blue, citecolor=blue, filecolor=blue, urlcolor=blue}
% Text styles
%\newcommand\textstylehps[1]{#1}
%\makeatletter
%\newcommand\arraybslash{\let\\\@arraycr}
%\makeatother
%\setlength\tabcolsep{1mm}
%\renewcommand\arraystretch{1.3}
%\newtheorem{theorem}{Theorem}

%%%\title{Comentarios en torno a la tarea de revisar los objetivos de la línea de investigación del nuevo Plan de Estudios de la Facultad de Historia y elaborar los nuevos Programas}
%\author{Centro}
%\date{2014-08-25}
%\begin{document}

%\clearpage\setcounter{page}{167}
\thispagestyle{empty}
\phantomsection{}
\addcontentsline{toc}{chapter}{Problemática y alternativas en el área\newline de investigación para la formación\newline del historiador\newline $\diamond$
\normalfont\textit{Arturo Carrillo Rojas y Luis Demetrio Meza López}}
{\centering {\scshape \large Problemática y alternativas en el área de investigación para 
la formación del historiador}\par}
\markboth{la formación del historiador}{problemática y alternativas}
\setcounter{footnote}{0}


\bigskip
\begin{center}
{\bfseries Arturo Carrillo Rojas \\
Doctorante Luis Demetrio Meza López}\\
{\itshape Universidad Autónoma de Sinaloa \par}
\end{center}

\bigskip
{\bfseries Resumen}

La experiencia en la impartición de varios cursos en el área de 
investigación de la Licenciatura en Historia de la Universidad Autónoma 
de Sinaloa nos ha permitido detectar una serie de deficiencias en 
varios aspectos de la práctica docente y en las formas de aprendizaje 
de los alumnos, que repercuten en la formación de los historiadores y 
pese a que se han modificado varias veces los programas las 
dificultades persisten. La finalidad de la presente ponencia es 
precisar los problemas más importantes y tratar de aportar algunas 
sugerencias o alternativas que nos permitan enfrentarlos. Sabemos que 
cada licenciatura en el país enfrenta situaciones particulares, dadas 
las especificidades regionales, pero consideramos que algunos de ellas 
tienen similitudes que nos permitirán buscar soluciones comunes y 
ayudar a la formación integral de nuestros egresados.\\ 
\textbf{Palabras clave:} investigación histórica, enseñanza, formación,\linebreak 
historia, práctica docente.


\bigskip
{\bfseries Abstract}
\enlargethispage{1\baselineskip}

The experience in the teaching of several courses in the field of 
investigation of degree in History of the Universidad Autónoma de 
Sinaloa has allowed us to detect a series of deficiencies in several 
aspects of the teaching practice and in the forms of learning of the 
pupils, who affect in the formation of the historians and although 
several times have modified the programs the difficulties they persist. 
The purpose of the present paper is to identify the most important 
problems and to try to make some suggestions or alternatives that allow 
us to face them. We know that every school in the country faces 
particular situations, considering the regional specificities, but we 
think that some of them have similarities that will allow us to look 
for common solutions and to help to the integral formation of our 
graduates.\\
\textbf{Keywords:} historical research, teaching, training, history, 
teaching practice.


\medskip
{\bfseries Introducción}
\enlargethispage{1\baselineskip}

Considerando las diversas opiniones de alumnos y egresados de la 
Licenciatura de Historia de la Universidad Autónoma de Sinaloa (UAS) 
sobre su formación en el área de investigación destaca la idea de que 
no están suficientemente preparados en este renglón. Los maestros 
coinciden con esto en lo general y se acercan a la verdad si se toma en 
cuenta el desempeño de los alumnos en los seminarios de investigación, 
sus calificaciones, el número de reprobados, la calidad de los 
proyectos y avances, el interés en la materia y las experiencias que se 
han teniendo con los egresados de licenciatura en la Maestría de 
Historia.

La información sobre esta situación existe diseminada en los distintos 
informes que se han realizado para evaluar la licenciatura en los 
últimos años, los resultados de las encuestas de opinión entre alumnos 
y maestros, el número de ponencias que los estudiantes presentan en los 
eventos académicos (estudiantiles o generales), cantidad de artículos 
publicados (solos o en compañía de algún maestro), el número de alumnos 
graduados bajo la modalidad de tesis y los cambios efectuados en las 
asignaturas del área de investigación del programa de estudio, todo 
esto nos muestra que pese a tener mejores resultados que otras 
licenciaturas de la universidad, nos enfrentamos a múltiples 
dificultades que nos esforzamos por superar.

A la opinión subjetiva de los maestros o estudiantes sobre esta 
problemática se suman elementos concretos que demuestran las 
características del tipo de investigador que estamos formando en esta 
área de estudio y los resultados reales que hemos obtenido. 
Lamentablemente no se ha realizado ninguna investigación exhaustiva 
sobre el tema, únicamente aproximaciones parciales a las cuales con 
este trabajo pretendemos brindar más elementos analíticos y algunas 
propuestas.

\medskip
{\bfseries Principales dificultades en el área de investigación}
\enlargethispage{1\baselineskip}

En general consideramos que no se ha logrado consolidar una propuesta 
definitiva de formación para la investigación de los alumnos de la 
licenciatura en historia y esto se debe a varios factores que incluyen 
a la administración de la facultad, a los maestros, al programa, a los 
mismos alumnos y al estado cambiante del conocimiento histórico, los 
cuales mencionaremos a continuación:

\begin{Obs}
\item[1.-] La Dirección de la Facultad, en sus diferentes administraciones, no 
ha podido generar una dinámica de discusión de éste y otros ejes del 
programa, pese a los esfuerzos realizados no se ha logrado que 
funcionen adecuadamente las áreas académicas, además, en algunos 
periodos, ha predominado la idea que se deben delegar las tareas de la 
práctica y el aprendizaje de la investigación estudiantil a los 
llamados Cuerpos Académicos, los cuales por su estructura y 
normatividad de ninguna forma les corresponde dicha función, aunque 
pueden coadyuvar en su implementación.

\item[2.-] En la planta de maestros también existen problemas serios 
relacionados con el tema: en primer lugar la exigencia institucional, 
sobre todo de CONACYT, de alto nivel de desempeño académico y de\linebreak 
investigación de los profesores, lo cual no permite que se le dedique 
mucho tiempo al factor docencia, mucho menos a la indagación 
sistemática sobre los procesos de enseñanza-aprendizaje, así que la 
mayoría de los maestros que pertenecen al Sistema Nacional de 
Investigadores e imparten las materias del área en lugar de dedicarle 
un mayor tiempo y esfuerzo a este tipo de cursos delegan esta tarea en 
los alumnos por considerar que como están en licenciatura ya deben 
saber investigar, sin realizar un proceso de formación centrado en el 
alumno. A esto se suma, en segundo lugar, la diversidad de concepciones 
que existen sobre la investigación y su proceso, la mayoría de la veces 
limitándose a su experiencia personal, sin considerar el gran cúmulo de 
conocimientos que existen sobre el tema y sobre todo la diversidad de 
niveles en el terreno investigativo, así se da que los maestros 
reproducen lo que aprendieron ya sea en maestría o en doctorado, sin 
tomar en cuenta que no basta con realizar la práctica investigativa 
(bajo la consigna de que se aprende investigando) sino que hay procesos 
cognitivos propios de esta área del conocimiento que hay que dominar y 
por lo tanto reconocer para poder desarrollar, impulsar y generar esta 
habilidad en los alumnos. En tercer lugar, en ocasiones los profesores 
no logran identificar los niveles específicos de cada taller o 
seminario, a tal grado que los alumnos se quejan que los maestros les 
imparten casi lo mismo en cada uno de ellos, demostrando la falta de 
especialización en el área, lo cual es difícil de lograr si a los 
maestros se les carga periódicamente con materias diferentes o si estos 
cursos se asignan a maestros de asignatura en proceso de formación, en 
lugar de seleccionar a los profesores con mayor experiencia en este 
terreno.

\enlargethispage{1\baselineskip}
\item[3.-] Relacionado con lo anterior está el problema de los programas de 
estudio que incluyen cursos, talleres y seminarios que, aunque existen 
formalmente en la tira de materias y sus objetivos están definidos en 
lo general no se ha logrado consensuar los contenidos específicos y en 
ocasiones ni tan siquiera el número óptimo de materias que permitirán 
la mejor formación del historiador en el área de investigación.

\item[4.-] A nivel de los alumnos el problema viene desde la preparatoria 
donde les inculcan la idea de que la investigación es un conjunto de 
técnicas y procedimientos que de seguirlas fielmente llegarán 
felizmente a su objetivo final, pero al momento de aplicarlas y 
relacionarlas no ven los resultados esperados y dudan de su eficacia. 
Por eso al llegar a la licenciatura no le dan la importancia a estas 
materias y las ven como asignaturas de relleno, privilegiando las 
asignaturas informativas.

\item[5.-] Es una realidad que el estado del conocimiento histórico esta en 
constante cambio y evolución, no sólo por las prioridades que se otorga 
a ciertos temas y periodos, dependiendo del momento que la sociedad 
este viviendo, sino que hay una constante renovación de los métodos y 
las fuentes que utilizan los historiadores (Bloch 1996; Camarena y 
Villafuerte 2001; Carr 2000; Eco 2005; González 1998). Esto 
repercute en las formas en que realizamos investigación y sobre todo en 
las concepciones y enfoques sobre esta tarea. El constante estado de 
reconstrucción del conocimiento histórico implica cambios frecuentes en 
la labor de investigación y su enseñanza.
\end{Obs}


\smallskip
{\bfseries La búsqueda de un modelo adecuado para el área de 
investigación}
\enlargethispage{1\baselineskip}

En la Licenciatura de Historia de la UAS, desde su fundación a la fecha, 
han existido cinco programas de estudio: 1988\footnote{Reformulación de la Oferta Educativa de la Licenciatura en Historia, Facultad de Historia, 1995.}, 
1995\footnote{Acuerdo Núm\@. 681, sesión ordinaria del 7 de julio de 1995 del H\@. 
Consejo Universitario de la UAS\@.}, 2000\footnote{Acuerdo Núm\@. 053, sesión ordinaria del 12 de junio de 2001 del H\@. Consejo Universitario de la UAS\@.}, 
2006\footnote{Acuerdo Núm\@. 353, sesión ordinaria del 4 de julio de 2006 del H\@. 
Consejo Universitario de la UAS\@.} y 2013. El primero de ellos se diseño a partir 
de la experiencia de la Maestría de Historia Regional, que había dado 
fruto a su primera generación (1984--1987), en ella se formaron una gran 
parte de los primeros maestros de la licenciatura y en gran medida se 
reprodujeron los programas impartidos en ella. La carrera se diseñó con 
una duración de 9 semestres y los seminarios de investigación iniciaban 
en el quinto semestre.
%\enlargethispage{1\baselineskip}

\bigskip
\begin{flushleft}
\tablefirsthead{}
\tablehead{}
\tabletail{}
\tablelasttail{}
\setlength{\extrarowheight}{1pt}
\begin{supertabular}{|m{0.2\textwidth}|m{0.75\textwidth}|}
\hline
\rowcolor{lsLightBlue}\multicolumn{2}{|m{0.969\textwidth}|}{\centering{\bfseries Cuadro 1\par Área de Investigación en
el Plan de Estudios\\ de 1988}}\\\hline
\rowcolor{lsLightGray}{\small\bfseries Semestre} &
{\small\bfseries Asignaturas de la línea de investigación}\\\hline
5to. &
 Seminario de Investigación I\\\hline
6to.  &
 Seminario de Investigación II\\\hline
7mo. &
 Seminario de Investigación III\\\hline
8vo. &
 Seminario de Investigación IV\\\hline
9o. &
Seminario de Tesis\\\hline\hline
\rowcolor{lsLightGray}\multicolumn{2}{|p{0.969\textwidth}|}{\scriptsize{\bfseries Fuente:} Actas de examen de la licenciatura
(1988--1993). Departamento de Control Escolar.\par Facultad de Historia de la
Universidad Autónoma de Sinaloa.}\\\hline%\hline
\end{supertabular}
\end{flushleft}

%\bigskip
Como en este tiempo en la Universidad no se contemplaba la tesis como 
requisito de egreso se optó por la alternativa de que en el programa de 
Licenciatura en Historia el último seminario fuera aprobado con la 
presentación y defensa de una tesis. Esta medida, en todo sentido 
correcto desde el punto de vista de la formación de nuevos 
historiadores profesionales, ocasionó que el índice de egresados fuera 
muy bajo durante los primeros años de la existencia de este programa, 
pues únicamente los alumnos más avezados que tuvieron la suerte de 
contar con una buena asesoría lograron terminar sus estudios.

%\smallskip
\begin{flushleft}
\tablefirsthead{}
\tablehead{}
\tabletail{}
\tablelasttail{}
\setlength{\extrarowheight}{1pt}
\begin{supertabular}{|m{0.2\textwidth}|m{0.75\textwidth}|}
\hline
\rowcolor{lsLightBlue}\multicolumn{2}{|m{0.969\textwidth}|}{\centering {\bfseries Cuadro 2\par Área de Investigación en el Plan de Estudios\\ de 1995}}\\\hline
\rowcolor{lsLightGray}{\small\bfseries Semestre} &
{\small\bfseries Asignaturas de la línea de investigación}\\\hline
1ro. &
Taller de lectura\\\hline
2do.  &
Taller de redacción\\\hline
3ro. &
Metodología y Técnicas de Investigación I\\\hline
4to. &
Metodología y Técnicas de Investigación II\\\hline
5to. &
 Seminario de Investigación I\\\hline
6to. &
 Seminario de Investigación II\\\hline
7mo. &
 Seminario de Investigación III\\\hline
8avo. &
 Seminario de Investigación IV\\\hline
9o. &
Seminario de Tesis\\\hline%\hline
\rowcolor{lsLightGray}\multicolumn{2}{|p{0.969\textwidth}|}{\scriptsize{\bfseries Fuente:} Actas de examen de la licenciatura (1995--2000). Departamento de Control Escolar.\par Facultad de Historia de la Universidad Autónoma de Sinaloa.}\\%\hline
\end{supertabular}
\end{flushleft}

Esto llevó a que en la reforma del Plan de Estudios de 1995 se 
fortaleciera esta área, añadiéndose cuatro materias más, como se 
muestra en el Cuadro 2 y a su vez se permitió que el último seminario 
se aprobara con la presentación de la tesis.

%\bigskip 
Aunque el número de egresados se incrementó el índice de 
titulados no fue tan alto como se esperaba, en esto influyó que en la 
UAS, en 1995, se formó el primer doctorado y un grupo de cinco 
profesores de la Facultad se inscribieron en él (alrededor del 50\,\% 
de los profesores de tiempo completo que laboraban en ese tiempo en la 
licenciatura), solicitando licencia en el trabajo para dedicarse a los 
estudios de doctorado como éste lo exigía. Esto impactó en el área de 
investigación pues las materias fueron cubiertas por maestros de 
asignatura con pocos años de experiencia. 

Al regreso de los nuevos doctores se planteó una nueva revisión del 
plan de estudios que contempló la disminución a cuatro años de la duración 
de la carrera y la modificación de algunas materias. Con respecto al área  de 
investigación, se mantuvo el mismo número de materias (Cuadro 3), pero con 
algunas modificaciones sustanciales.

Uno de los cambios fue que en el primer semestre se introdujo un curso 
de Introducción a la Investigación Histórica, donde el alumno recibía 
los primeros conocimientos sobre el tema y realizaba prácticas 
relacionadas con la investigación bibliográfica, archivística y de 
campo. En el segundo semestre se incorporaba un curso de Redacción y 
estilo para historiadores, después se incluían dos cursos sobre Métodos 
y Técnicas, el primero relacionado con el llamado método científico y 
el segundo con los métodos en la historia, para dar paso a los tres 
seminarios donde los alumnos debían elaborar el proyecto, avanzar en la 
redacción de los capítulos y concluir el último seminario con el 
borrador de la tesis. Si el principal interés del alumno era la 
investigación podía seleccionar en el último año dos materias optativas 
relacionadas con los métodos de investigación histórica que lo 
especializaba y le permitía avanzar y profundizar en la tesis. 

\bigskip
\begin{flushleft}
\tablefirsthead{}
\tablehead{}
\tabletail{}
\tablelasttail{}
\setlength{\extrarowheight}{1pt}
\begin{supertabular}{|m{0.2\textwidth}|m{0.75\textwidth}|}
\hline
\rowcolor{lsLightBlue}\multicolumn{2}{|m{0.969\textwidth}|}{\centering{\bfseries Cuadro 3\par Área de Investigación en
el Plan de Estudios\\ del 2000}}\\\hline
\rowcolor{lsLightGray}{\small \bfseries Semestre} &
{\small \bfseries Asignaturas de la línea de investigación}\\\hline
1ro. &
Introducción a la Investigación Histórica\\\hline
2do.  &
Redacción y estilo para historiadores\\\hline
4to. &
Métodos y Técnicas de Investigación I\\\hline
5to. &
Métodos y Técnicas de Investigación II\\\hline
6to. &
 Seminario de Investigación I\\\hline
7mo. &
 Seminario de Investigación II\\\hline
8vo. &
 Seminario de Investigación III\\\hline
Optativa &
Métodos de Investigación Histórica I\\\hline
Optativa &
Métodos de Investigación Histórica I\\\hline\hline
\rowcolor{lsLightGray}\multicolumn{2}{|p{0.969\textwidth}|}{\scriptsize{\bfseries Fuente:} Actas de examen de la licenciatura (2000--2005). Departamento de Control Escolar.\par Facultad de Historia de la Universidad Autónoma de Sinaloa.}\\\hline
\end{supertabular}
\end{flushleft}

\smallskip
\enlargethispage{1\baselineskip}
Cinco años después cuando empezaban a verse los resultados de este 
nuevo plan se volvió a plantear la modificación de éste, por la 
necesidad de cumplir con las exigencias de los organismos evaluadores 
de actualizar y modificar periódicamente los Planes de estudio. Para 
ser mejor evaluados la Dirección de la facultad impulsó este nuevo 
cambio que consideramos fue apresurado y dejó fuera la opinión de 
varios maestros y en particular los de historia económica.

En general el área de investigación del nuevo plan quedó conformada por 
6 asignaturas impartidas de forma discontinua desde el primer al octavo 
semestre, quedando el tercero y cuarto sin ninguna materia relacionada 
con el área de investigación (véase el Cuadro 4).


\medskip
\begin{flushleft}
\tablefirsthead{}
\tablehead{}
\tabletail{}
\tablelasttail{}
\begin{supertabular}{|m{0.2\textwidth}|m{0.75\textwidth}|}
\hline
\rowcolor{lsLightBlue}\multicolumn{2}{|m{0.969\textwidth}|}{\centering{\bfseries Cuadro 4\par Área de Investigación en el Plan de Estudios\\ del 2006}}\\\hline
\rowcolor{lsLightGray}{\small\bfseries Semestre} &
{\small\bfseries Asignaturas de la línea de investigación}\\\hline
1ro. &
 Técnicas de Investigación documental\\\hline
2do.  &
 Introducción a la Investigación\\\hline
5to. &
 Seminario de Investigación I\\\hline
6to. &
 Seminario de Investigación II\\\hline
7mo. &
 Taller de Investigación I\\\hline
8avo. &
 Taller de Investigación II\\\hline\hline
\rowcolor{lsLightGray}\multicolumn{2}{|m{0.969\textwidth}|}{\scriptsize{\bfseries Fuente:} Actas de examen de la licenciatura (2006--2010). Departamento de Control Escolar.\par Facultad de Historia de la Universidad Autónoma de Sinaloa.}\\\hline
\end{supertabular}
\end{flushleft}

\bigskip
Desde la puesta en marcha del Plan de Estudios 2006, en las asignaturas 
de la línea o área de investigación histórica, se detectaron una serie 
de problemas que no podían ser superados fácilmente pues eran de 
estructura y de contenido. La Comisión encargada del Plan de Estudios 
tuvo demasiadas tareas y faltó más cooperación de la planta académica 
en la formulación del Programa del 2006, esto aunado al apresuramiento 
para que entrara en funciones lo más pronto posible ocasionó serios 
problemas.
%\enlargethispage{1\baselineskip}

En un estudio realizado en 2008 (Carrillo 2008) para revisar los 
contenidos de las asignaturas de la línea correspondiente a 
Investigación y tratar de corregir en lo posible algunas deficiencias 
en su aplicación se detectaron diversos problemas en la estructura y 
objetivos de las asignaturas de esta línea:

\begin{Obs}
\item[1.] En primer lugar destacaba el orden de las cuatro últimas 
asignaturas, pues se ponían primero los dos seminarios de investigación 
y después los talleres. Esto no era correcto porque los seminarios se 
realizan cuando los alumnos ya tienen cierto avance en sus trabajos de 
investigación y pueden ser discutidos en ellos. 

\item[2.] Los cursos de seminarios y talleres no tenían una coherencia 
entre si y no definían una línea clara en la formación de los alumnos: 
El Seminario I era más bien un curso de metodología de la 
investigación, el Seminario II, aunque se planteaba que los 
conocimientos estaban encaminados a elaborar los respectivos proyectos 
de investigación, se concentraba en el análisis de discurso, historia 
oral, etnografía, investigación cualitativa, y la segunda parte se 
convertía en una  propuesta de técnicas (de recopilación de datos: 
observación, entrevistas, cuestionarios, etc.), siendo que esas 
técnicas debieran contemplarse en el primer curso de Técnicas de 
investigación.

\item[3.] El Taller I, era más bien un seminario, donde el alumno veía 
nuevamente <<los componentes de la investigación científica y el diseño 
de la investigación>> y se pretendía que realizaran <<el análisis, 
catalogación y clasificación de las principales fuentes documentales 
que se encuentren en los archivos locales>>, para que el alumno 
presentara el proyecto de investigación y el primer capítulo de la 
tesis elaborado a partir de <<un análisis historiográfico de la 
bibliografía trabajada durante el curso>> (De hecho, lograr todos estos 
objetivos era algo prácticamente imposible de realizar en un semestre).
\enlargethispage{-1\baselineskip}

\item[4.] El Taller II planteaba que se continuara con el fichaje de 
material bibliográfico y de fuentes archivísticas, e <<intercambio de 
experiencias con investigadores de otras instituciones del estado y del 
país\ldots >>, además <<publicándose los mejores>> resultados, aunque lo 
fundamental era que <<concluya su trabajo de tesis>>. Estas tareas de 
trabajo de archivo, intercambio de experiencias con investigadores 
foráneos, publicaciones y terminación de la tesis no se podían realizar 
en un semestre teniendo la carga de otras materias, además del servicio 
social, entre otras obligaciones. 
\end{Obs}

%\medskip
Como se observa existían buenas intenciones en todo lo que se 
planteaba, pero en algunos apartados los objetivos que se fijaban para 
un curso semestral eran muy ambiciosos y sobre todo no se percibía la 
secuencia lógica de las materias y sus contenidos.

Ante tales problemas se hizo una propuesta de adecuación de objetivos y 
contenidos que fue avalada por los órganos competentes, para esto se 
partió de las experiencias acumuladas y se trató de respetar en lo 
posible las propuestas originales, sin cambiar nombres ni secuencia en 
las materias Después de dos años observamos que esto no bastó para 
solucionar los problemas de fondo porque en la práctica se detectó que 
no había claridad de lo que se impartía en cada materia debido 
sobretodo al orden de las materias y sus nombres, que se confundían al 
no corresponder con los objetivos y contenidos.

%\smallskip
{\bfseries Hacia la formación de competencias en el área\\ de investigación}
\enlargethispage{2\baselineskip}

En el 2011 se tuvo la oportunidad de corregir estos problemas mediante 
una propuesta al Foro de Licenciatura para reformar el Plan de Estudios 
(Carrillo 2011). Esta propuesta incorporaba las competencias 
específicas que se estaban manejado para la formación del historiador, 
para ello se retomaron las planteadas en el proyecto Alfa Tuning para 
nuestro continente, el cual tiene entre sus objetivos impulsar en 
América Latina un mayor nivel de convergencia de la educación superior 
en doce aéreas temáticas, entre las cuales se encuentra la Historia, 
así como el de desarrollar perfiles profesionales en términos de 
competencias (Proyecto Tuning América 2011).

Retomando la experiencia planteada hasta aquí se propuso que las 
asignaturas de la línea de investigación comenzaran en el segundo 
semestre y se continuaran hasta el séptimo, quedando para el octavo, 
como materia opcional, un Seminario de Tesis para aquellos que optaran 
por esta forma de titulación. Cada curso estaría relacionado con las 
competencias específicas para el historiador, de tal manera que los 
objetivos de cada curso se definieran en gran medida por los propósitos 
de éstas. El siguiente cuadro nos muestra la relación entre competencia 
y asignatura que se proponía:
%\enlargethispage{3\baselineskip}

\begin{scriptsize}
\begin{flushleft}
\tablefirsthead{}
\tablehead{}
\tabletail{}
\tablelasttail{}
\begin{supertabular}{|p{0.57\textwidth}|p{0.3\textwidth}|p{0.07\textwidth}|}
\hline
\rowcolor{lsLightBlue}\multicolumn{3}{|p{0.977\textwidth}|}{\centering{\bfseries Cuadro 5\par Competencias y materias del área de
investigación}}\\\hline
\rowcolor{lsLightGray}\centering{\bfseries COMPETENCIAS} & \centering{\bfseries ASIGNATURAS} &
\centering\arraybslash{\bfseries SEM.}\\\hline
Conciencia de que el debate y la investigación histórica están en permanente
construcción & \centering Comprende toda la línea de investigación & ~\\\hline
Habilidad para usar los instrumentos de recopilación de información
(catálogos, inventarios, fuentes electrónicas etc.) &
\centering Técnicas de Investigación documental &
\centering\arraybslash 2do.\\\hline
Capacidad para participar en trabajos de investigación &
\centering Introducción a la Investigación Histórica &
\centering\arraybslash 3ro.\\\hline
Conocimiento de los métodos y problemas de las diferentes ramas de la
investigación histórica &
\centering Metodología de la Investigación Histórica &
\centering\arraybslash 4to.\\\hline
Habilidad para diseñar, organizar y desarrollar proyectos de investigación 
en historia &
\centering Taller de Investigación &
\centering\arraybslash 5to.\\\hline
Capacidad para definir temas de investigación que puedan contribuir al
conocimiento y debate historiográfico. &
\centering Seminario de Investigación I &
\centering\arraybslash 6to.\\\hline
Capacidad para identificar y utilizar fuentes de información para la
investigación histórica &
\centering Seminario de Investigación II &
\centering\arraybslash 7mo.\\\hline\hline
\rowcolor{lsLightGray}\multicolumn{3}{|m{0.977\textwidth}|}{{\bfseries Fuente:} Arturo Carrillo Rojas, \textit{Las
competencias y las asignaturas}. 2011.}\\\hline
\end{supertabular}
\end{flushleft}
\end{scriptsize}

\medskip
Por diversas razones internas en la Facultad de Historia, que 
incluyeron una dirección interina de un año, se detuvo el proceso de 
cambio de plan de estudios retomándose posteriormente. Para realizar el 
nuevo diseño curricular del programa de Licenciatura en Historia en el 
2013 se realizó una consulta con empleadores, docentes, egresados y 
estudiantes para determinar cuales eran las competencias genéricas más 
adecuadas para el perfil de Licenciado en Historia, en los primeros 
lugares aparecieron el actuar éticamente (90.6), aplicar los 
conocimientos a la práctica (82.6), el uso de las TIC (81.8) y otras, 
en cambio las relacionadas con la investigación tuvieron un lugar 
intermedio:


\bigskip
\begin{footnotesize}
\begin{center}
\tablefirsthead{}
\tablehead{}
\tabletail{}
\tablelasttail{}
\begin{supertabular}{|m{0.744\textwidth}|m{0.163\textwidth}|}
\hline
\rowcolor{lsLightBlue}\multicolumn{2}{|p{0.925\textwidth}|}{\centering{\bfseries Cuadro 6\par Competencias Genéricas\\ relacionadas con la investigación}}\\\hline
\rowcolor{lsLightGray}{\bfseries Competencias Genéricas} & {\centering{\bfseries Promedio}}\\\hline
Busca y analiza información procedente de fuentes diversas y la comunica en
forma oral y escrita. &
{\hfil 72.1}\\\hline
Formula y gestiona proyectos &
{\hfil 71.1}\\\hline
Identifica, plantea y resuelve problemas &
{\hfil 64.1}\\\hline\hline
\rowcolor{lsLightGray}\multicolumn{2}{|m{0.925\textwidth}|}{{\bfseries Fuente:} \textit{Diseño curricular del
Programa Educativo Licenciatura en Historia},\par Culiacán, Facultad de
Historia, UAS, 2013.}\\\hline
\end{supertabular}
\end{center}
\end{footnotesize}

\bigskip
Con respecto a la encuesta aplicada sobre el grado de importancia para 
el historiador de las competencias específicas los resultados arrojaron 
que las más valoradas fueron los conocimientos de historia nacional 
(90.4), conocimientos de la historia universal o mundial (88.3), 
capacidad para comunicarse y argumentar en forma oral y escrita (86.7), 
en cuarto lugar aparece ya la habilidad para diseñar proyectos de 
investigación (85.7) reflejando la importancia que se le da a este tipo 
de competencias para la profesión. En el siguiente cuadro se muestran 
las competencias que se relacionan con el tema analizado:
\newpage

\bigskip
\begin{center}
\begin{footnotesize}
%\tablefirsthead{}
%\tablehead{}
%\tabletail{}
%\tablelasttail{}
\begin{supertabular}{|m{0.79\textwidth}|m{0.172\textwidth}|}
\hline
\rowcolor{lsLightBlue}\multicolumn{2}{|m{0.98\textwidth}|}{\centering {\bfseries Cuadro 7\par Competencias Específicas relacionadas con la investigación}}\\\hline
\rowcolor{lsLightGray}{\bfseries Competencias Específicas} &
{\centering \bfseries Promedio}\\\hline
Habilidad para diseñar, organizar y desarrollar proyectos de investigación
histórica &
{\hfil 85.7}\\\hline
Capacidad para identificar y utilizar apropiadamente fuentes de información:
bibliográfica, documental, testimonios orales, etc\@. para la investigación
histórica &
{\hfil 85.4}\\\hline
Conciencia de que el debate y la investigación histórica están en permanente
construcción &
{\hfil 84.1}\\\hline
Habilidad para organizar información histórica completa de manera coherente
&
{\hfil 83.3}\\\hline
Habilidad para usar instrumentos de recopilación de la información, tales
como catálogos bibliográficos, inventarios de archivo y referencias
electrónicas &
{\hfil 81.0}\\\hline
Conocimiento de los métodos y problemas de las diferentes ramas de la
investigación histórica: economía, social, política, estudios de género,
etc. &
{\hfil 80.5}\\\hline
Capacidad para definir temas de investigación que puedan contribuir al
conocimiento y debate historiográficos &
{\hfil 78.4}\\\hline
Capacidad para participar en trabajos de investigación interdisciplinaria &
{\hfil 77.1}\\\hline
Conocimiento y habilidad para usar teorías, métodos y técnicas de otras
ciencias sociales y humanas &
{\hfil 75.3}\\\hline
\rowcolor{lsLightGray}\multicolumn{2}{|m{0.98\textwidth}|}{{\bfseries Fuente:} \textit{Diseño curricular del
Programa Educativo Licenciatura en Historia},\par Culiacán, Facultad de
Historia, UAS, 2013.}\\\hline
\end{supertabular}
\end{footnotesize}
\end{center}

\smallskip
\enlargethispage{1\baselineskip}
El currículo de la Licenciatura en Historia quedó estructurado en seis 
ejes: genérico, historiográfico, histórico, metodológico, acentuaciones 
y optativas, organizados por Unidades de Aprendizaje, las cuales se 
dividen en obligatorias y optativas, contemplando  el desarrollo de 
competencias de lo simple a lo complejo. El término de <<Unidades de 
aprendizaje>> se emplea en el sentido de abordar contenidos en torno a 
un objeto de conocimiento en el que se integran saberes teóricos, 
prácticos y actitudinales, adoptando una estrategia de trabajo\linebreak 
fundamentada en la investigación (Diseño curricular 2013). 


\bigskip
\begin{scriptsize}
\begin{flushleft}
\tablefirsthead{}
\tablehead{}
\tabletail{}
\tablelasttail{}
\begin{supertabular}{|m{0.062\textwidth}|m{0.32\textwidth}|m{0.56\textwidth}|}
\hline
\rowcolor{lsLightBlue}\multicolumn{3}{|m{0.98\textwidth}|}{\centering{\bfseries Cuadro 8\par Área de
Investigación del Plan de Estudios de 2013}}\\\hline
\rowcolor{lsLightGray}{\bfseries Sem.} &
{\bfseries Unidad de Aprendizaje} &
\centering\arraybslash{\bfseries Contenido General}\\\hline
1ro &
Introducción a los métodos de la historia &
Conoce los métodos de las diferentes corrientes historiográficas para
construir conocimiento histórico, en base a criterios de ética
científica.\\\hline
2do.  &
Introducción a la Investigación &
Recopila y utiliza apropiadamente información histórica, para realizar
investigación con rigor crítico y ética en el manejo de las
fuentes.\\\hline
5to. &
Software para investigación histórica &
Maneja software especializado para investigación, docencia y aprendizaje
atendiendo a normas y reglamentos en su utilización.\\\hline
6to. &
Seminario de Investigación &
Los estudiantes analizarán y serán capaces de explicar algunos de los
problemas relacionados con el conocimiento científico y la ciencia, las
diversas etapas del proceso de investigación científica, sus fundamentos
epistemológicos, la metodología de la investigación histórica, así como
adquirir  los elementos para la elaboración de un proyecto de tesis a nivel
licenciatura.\\\hline
7mo. &
Seminario de Investigación &
Se contempla realizar un
trabajo de localización, consulta y fichaje de material bibliográfico,
archivístico o producto del trabajo de campo, acompañado del procesamiento
y análisis de la información obtenida, esto le permitirá elaborar un buen
avance. Conocerá el proceso de análisis y descripción del fenómeno
histórico.\\\hline
8avo. &
Seminario de Investigación &
Familiariza al alumno con un formato de tesis (escritura y forma). Que el
alumno elabore  el borrador de tesis completo que contenga en términos
generales los elementos de lo que podría ser una tesis. Prepara al alumno
para defender su trabajo en público.\\\hline\hline
\rowcolor{lsLightGray}\multicolumn{3}{|m{0.98\textwidth}|}{{\bfseries Fuente:} Diseño curricular del Programa
Educativo Licenciatura en Historia, Culiacán,\par Facultad de Historia, UAS,
2013 y Programas de Estudio 2013--2017.}\\%\hline
\end{supertabular}
\end{flushleft}
\end{scriptsize}

\bigskip
Actualmente estos son los cursos del área de investigación que se están 
implementando en la licenciatura. En muchos de los casos se retoman los 
contenidos de las materias que ya se impartieron en programas 
anteriores y se van adecuando al sistema de competencias.

\enlargethispage{1\baselineskip}
Aunque apenas se esta implementando el nuevo plan de estudios, vale la 
pena dejar asentadas algunas de las dudas que surgieron en la\linebreak discusión 
de esta área, a futuro la práctica nos dirá hasta que punto eran 
acertadas o erróneas estas opiniones. 

\begin{Obs}
\item[$\bullet$] En el primer semestre la asignatura Introducción a los métodos de 
la historia se debería cambiar para más adelante e iniciar con algo de 
técnicas de investigación, porque al iniciar la carrera los alumnos 
carecen de los conocimientos necesarios de la disciplina que les 
permita valorar y dominar correctamente los distintos métodos que se 
utilizan en historia. 
\item[$\bullet$] La Introducción a la investigación 
histórica en el segundo semestre es correcta siempre y cuando se 
mantenga a un nivel introductorio y de continuidad al curso de 
técnicas. 
\item[$\bullet$] En el quinto semestre la materia de Software para 
investigación histórica aparece como algo aislado después de dos 
semestres sin ninguna materia relacionada con el área, en nuestra 
opinión debería de quedar subsumida en otros cursos donde se aborde lo 
de las TIC o en algún taller. En su lugar debería ir una materia sobre 
los métodos que el historiador utiliza para analizar la realidad. 
\item[$\bullet$] Los tres seminarios siguientes si están bien ubicados y cumplirán con 
su función si los cursos anteriores crearon las competencias necesarias 
para que el alumno pueda elaborar su proyecto y avanzar en su 
investigación. 
\end{Obs}

\medskip
{\bfseries Una propuesta en discusión}

Consideramos que la experiencia acumulada en este terreno nos permite 
plantear que combinando lo del área de investigación del Plan del 2000 
con la propuesta basada en competencias del 2011 y algunos elementos de 
la reforma del 2013 son la base para estructurar una propuesta de 
cursos con sus objetivos generales de lo que debe contener la línea o 
área de investigación en las licenciaturas de historia del país, mismas 
que se pueden adecuar a las condiciones concretas de cada estado. En 
general estas serían las materias (cursos, talleres y seminarios):

{\itshape\ Técnicas de Investigación  Documental\/}

El objetivo principal de este curso, consiste en que el alumno adquiera 
los conocimientos técnicos y habilidades necesarias para la búsqueda, 
recopilación y procesamiento de la información bibliográfica, 
hemerográfica y de archivo.

{\itshape\ Introducción a la Investigación histórica\/}

En este curso se pretende que el alumno a partir de sus conocimientos y 
habilidades en el manejo de la información bibliográfica y 
hemerográfica adquiera la capacidad de participar en trabajos de 
investigación mediante el uso de técnicas archivísticas y de 
investigación de campo (entrevistas, historia oral, etcétera).

\textit{Metodología de la Investigación Histórica} 

\enlargethispage{1\baselineskip}
El objetivo central de este curso consiste en que los alumnos adquieran 
las bases científico metodológicas necesarias para la realización de 
trabajos de investigación académica (Proceso del conocimiento y método 
científico), así como para la realización de trabajos de investigación 
histórica (método histórico y métodos específicos de la historia política, 
social, económica y cultural).


%\medskip
{\itshape\ Taller de  Investigación\/}

Uno de los objetivos de este taller consiste en preparar al alumno para 
que a través de un proceso permanente de orientación y de trabajo 
adquiera la habilidad para diseñar, organizar y desarrollar proyectos 
de investigación y avance en la elaboración de su propio protocolo.

{\itshape\ Seminario de Investigación I\/}

En este seminario los alumnos desarrollarán la capacidad de definir 
temas de investigaciones pertinentes y viables y con la aplicación de 
las técnicas y métodos aprendidos para explotar las fuentes conocidas 
elaborar un avance de la investigación.  

{\itshape\ Seminario de Investigación II\/}

El alumno con ayuda del material y conocimientos adquiridos en los 
cursos anteriores continuará con el rastreo y fichaje de material 
bibliográfico y de fuentes archivísticas, así como procesamiento y 
análisis de la información obtenida que le permita elaborar un borrador 
de la investigación.

Para concluir podemos afirmar que la definición y uso de las 
competencias en el proceso de enseñanza aprendizaje en la línea o área 
de investigación nos permitirá mejorar la formación integral del 
estudiante, siempre y cuando los maestros transformemos y adecuemos 
nuestra práctica docente y eso es motivo de otra discusión.

\bigskip
{\bfseries Referencias}

\medskip
Bloch, Marc (1996), \textit{Apología para la historia o el oficio del
historiador}, México, INAH, FCE\@.

\begin{sloppypar}
Camarena Ocampo, Mario y Villafuerte García, Lourdes (coords.) (2001),
\textit{Los andamios del historiador. Construcción y tratamiento de
fuentes}, México, AGN\slash{}INAH\@.
\end{sloppypar}

Carr, Edward H. (2000), \textit{¿Qué es la historia?}, México, Editorial Ariel.

Carrillo Rojas, Arturo (2008), \textit{Comentarios en torno a la 
revisión de los objetivos de la línea de investigación del Plan de 
Estudios de la Facultad de Historia (2006--2010) y propuesta  para los 
nuevos Programas}, Culiacán, Facultad de Historia. 

Carrillo Rojas, Arturo (2011), \textit{Las competencias y las 
asignaturas del área de investigación histórica en la Licenciatura de 
Historia de la Universidad Autónoma de Sinaloa}, 2do. Foro de 
restructuración del Plan de Estudios de la Licenciatura en Historia de 
la UAS\@.

\textit{Diseño curricular del Programa Educativo Licenciatura en Historia}
(2013), Culiacán, Facultad de Historia, UAS\@.

Eco, Humberto (2005), \textit{Como se hace una tesis. Técnicas y
procedimientos de estudio, investigación y escritura}, Barcelona, Gedisa.

González, Luis (1998), \textit{El oficio de historiar}, México, El Colegio
Nacional, Clío.

\textit{Informe de Autoevaluación Reacreditación de la Licenciatura en
Historia por la ACCECISO} (2013), Culiacán.

\begin{sloppypar}
Proyecto Tuning América, 21 de enero de 2011, disponible en \\ 
\url{http://tuning.unideusto.org/tuningal/index.php?option=content&task=view&id=173&Itemid=201}
\end{sloppypar}

\textit{Reformulación de la Oferta Educativa de la Licenciatura en Historia}
(1995), Culiacán, Facultad de Historia, UAS.

