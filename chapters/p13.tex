%\documentclass{article}
%\usepackage{amsmath,amssymb,amsfonts}
%\usepackage{fontspec}
%\usepackage{xunicode}
%\usepackage{xltxtra}
%\usepackage{polyglossia}
%\setdefaultlanguage{spanish}
%\usepackage{color}
%\usepackage{array}
%\usepackage{supertabular}
%\usepackage{hhline}
%\usepackage{hyperref}
%\hypersetup{colorlinks=true, linkcolor=blue, citecolor=blue, filecolor=blue, urlcolor=blue}
%\makeatletter
%\newcommand\arraybslash{\let\\\@arraycr}
%\makeatother
%\setlength\tabcolsep{1mm}
%\renewcommand\arraystretch{1.3}
%\newtheorem{theorem}{Theorem} 
%\title{}
%\author{Ofelia}
%\date{2014-06-15}
%\begin{document}
%%\clearpage\setcounter{page}{1}
%%\clearpage\setcounter{page}{235}
\markboth{}{}
\thispagestyle{empty}
\phantom{abc}
\phantomsection{}
\addcontentsline{toc}{chapter}{PARTE II.\ LAS ÁREAS TERMINALES DE LAS\newline LICENCIATURAS EN HISTORIA} 

\vspace{0.35\textheight}
{\centering \bfseries Parte II\par}
{\centering \bfseries LAS ÁREAS TERMINALES DE LAS LICENCIATURAS\newline EN HISTORIA \par}
\markboth{}{}
\thispagestyle{empty}	
\cleardoublepage{}

\thispagestyle{empty}%
\phantomsection{}
\addcontentsline{toc}{chapter}{Opciones de titulación y la  eficiencia terminal de la Licenciatura en Historia de la UAS\newline $\diamond$
\normalfont\textit{Ofelia Janeth Chávez Ojeda, Mayra Lizzete Vidales Quintero\newline y  Edna Elizabeth Alvarado Mascareño}}

{\centering{\scshape \large Opciones de titulación y la  eficiencia terminal\\ de  la Licenciatura en Historia de la UAS }\par}
\markboth{la formación del historiador}{opciones de titulación}
\setcounter{footnote}{0}


\bigskip
\begin{center}
{\bfseries Ofelia Janeth Chávez Ojeda\\
Mayra Lizzete Vidales Quintero\\
Edna Elizabeth Alvarado Mascareño}\\
{\itshape Universidad Autónoma de Sinaloa
\par}
\end{center}

\bigskip
\textbf{Resumen}
\enlargethispage{1\baselineskip}

En el año de 1999 se reglamenta la ley general de titulación de la
Universidad Autónoma de Sinaloa (en adelante UAS), aprobada por el máximo
órgano colegiado, el H.\ Consejo Universitario, en la cual se enlistan una
serie de opciones de titulación de las que los egresados pueden solicitar
la que mejor les convenga o para la que reúnen los requisitos para obtener
el título de licenciado, según sea el caso, de la carrera que han cursado. 

La Facultad de Historia, y más específicamente la Licenciatura en Historia, no
ha quedado ajena a este reglamento de titulación. Debido a la diversidad de
propuestas, los alumnos se pueden postular ante la Comisión Académica de
Titulación de la Facultad para aplicar en la opción que más les convenga. 
A pesar de que en un principio, y aún en la mayoría de los casos, la
titulación de los estudiantes de historia ha sido por medio de la
presentación y defensa de tesis, en los últimos años los estudiantes han
elegido las diferentes opciones que propone el reglamento General de
Titulación de la UAS, lo que ha apoyado la eficiencia terminal del programa
educativo en mención. 

En vísperas de cumplir 30 años formando historiadores, la Facultad de
Historia de la Universidad Autónoma de Sinaloa ha visto egresar de sus
aulas a 22 generaciones del programa de licenciatura, que\linebreak suman
aproximadamente más de trescientos  estudiantes. Manteniendo los propósitos
originales de su fundación,\footnote{La formación de historiadores que
contribuyan a la investigación del pasado en sus múltiples manifestaciones
sociales: económica, política, cultural, etc.} desde la primera generación
de la licenciatura, los egresados que optan por la profesión académica en
los niveles superiores se han inclinado por la obtención del grado a
través de la elaboración y defensa de tesis, además de continuar con los
temas de investigación para ampliarlos y desarrollarlos aún más en estudios
de Posgrado. 

En 1999 se presenta ante el H.\ Consejo Universitario la propuesta de ampliar
y diversificar las opciones de titulación del nivel superior. Para el año
2001 se aplica la reforma hecha al reglamento de titulación, y es incluida
en la Nueva Legislación Universitaria. Lo anterior abre un abanico de
oportunidades para los futuros egresados, quienes ahora podrían obtener 
el grado de licenciado a través de diversos medios, como son:
\enlargethispage{1\baselineskip}

\begin{Obs}
\item[1.-] Por promedio.
\item[2.-] Elaboración de memoria de Servicio Social con rigor metodológico o
tesina.
\item[3.-] Elaboración o defensa de tesis producto de participación en proyectos de
investigación.
\item[4.-] Examen de inglés Toefl o examen aplicado por el Centro de Estudios de
Idiomas de la Universidad. 
\item[5.-] Examen general de conocimientos interno o externo.
\item[6.-] Diplomado.
\item[7.-] Práctica profesional.
\end{Obs}


%\medskip
Para hablar de las opciones de titulación, debemos comenzar con las
disposiciones y plazos de titulación reglamentados en la Legislación
Universitaria  de la UAS, en la que se dispone que el plazo máximo para que
un alumno pueda titularse después de haber cubierto la totalidad de las
asignaturas de acuerdo al Artículo 27 del reglamento escolar, para el caso
de nivel Licenciatura, es de tres años (Universidad Autónoma de
Sinaloa 2009, p. 492).  Lo que, aunado al interés de algunos alumnos de 
cursar un posgrado, las
diferentes opciones brindan la oportunidad de titularse de una forma más
rápida y elevar los índices de titulación no solo de la Facultad de
Historia, sino también del resto de las Unidades Académicas de la propia
Universidad.

Para un mejor estudio de la eficiencia terminal de la Licenciatura en
Historia, dividimos este estudio en relación a los titulados a través de
las diferentes opciones de titulación, tomando como punto de partida la
primera generación de licenciados en historia hasta la generación vigésimo
primera, egresada en 2013.

\medskip
{\bfseries Titulación por Promedio}
\enlargethispage{1\baselineskip}
 
El promedio de excelencia es una de las primeras opciones que nos brinda la
Nueva Legislación Universitaria, contenida dentro de la Sección Primera, la
cual indica que

\begin{quotation}
podrán optar por la modalidad de titulación por promedio los pasantes que
hayan obtenido un promedio general de 9.0 o superior en el plan de estudios
correspondiente (\textit{ibid.}, p. 492).
\end{quotation}

%\medskip
En el caso de la Licenciatura en Historia, hasta el momento solo 28 alumnos
se han titulado a través de esta modalidad. Cabe señalar que en un
principio la Comisión de titulación de la propia Facultad se rehusaba a
aceptar otros medios de titulación que no fuera la defensa de tesis, esto
debido a que si bien uno de los objetivos de la institución es el formar
investigadores de alto nivel, el optar por otro medio no garantizaba el
desarrollo de aptitudes investigativas. Sin embargo, debido a la
normatividad universitaria, y después de 15 años de obtener resultados de
investigación convertidos en tesis, en 2003 se presenta el primer caso de
titulación por promedio (véase el Cuadro~I).

Si tomamos como punto de partida el 2003, que es cuando se inicia con esta
opción por parte de los alumnos de la Licenciatura en Historia, podemos
observar que para el 2006 hay un alza en la eficiencia terminal dado que un
mayor número de alumnos se inclinan por esta opción como forma de
titulación. Otro beneficio, además de aumentar la eficiencia terminal, es
el desempeño académico de los alumnos, pues algunos incluso se propusieron
como meta el conservar y aumentar un promedio de 9.0.

El titularse por promedio de excelencia, implica esfuerzo y dedicación e
incluso la habilidad de desarrollar proyectos de investigación para cumplir
con las asignaturas que así lo requieren. Además, podemos encontrar a
alumnos que cuentan con el promedio necesario para aplicar en esta
modalidad y que sin embargo decidieron defender su tesis y dar a conocer
sus trabajos de investigación.

%\smallskip 
\textit{Eficiencia Terminal por año a través de la modalidad de titulación
por\linebreak promedio}
\enlargethispage{1\baselineskip}

En el 2003 solo 3 alumnos obtuvieron la titulación por promedio de
excelencia, para el 2004 y 2005 no se recibieron solicitudes de titulación,
lo que no implica que el desarrollo académico de los alumnos hubiera
bajado, sino que buscaron la titulación por otro medio. Sin embargo, para
el 2006, se registra el mayor número de titulados por promedio alcanzando
la cifra de 9. En los años siguientes hasta llegar al 2014 la titulación
por esta modalidad fue disminuyendo, quedando para el 2007: 4, 2008:1,
2009: 2, 2010: 2, 2012:3, 2013: 1 y 2014: 3, presentándose de nuevo sin
referencia de solicitudes el 2011.

\smallskip
\textbf{Titulación por Elaboración de Memoria de Servicio Social 
con rigor metodológico o Tesina}
\enlargethispage{2\baselineskip}

En la sección segunda del instructivo de titulación de la Nueva Legislación
Universitaria, encontramos la opción de elaboración de memoria de Servicio
Social con rigor metodológico o tesina, en la cual se estipula que los
alumnos que opten por esta modalidad deben de cubrir los requisitos de:

\begin{quotation}
presentar ante la Comisión de Titulación una memoria o tesina cuyo rigor
metodológico será valorado de acuerdo con los siguientes criterios: a) el
marco teórico que sustente el proyecto o programa que el pasante desarrollo
durante su servicio social\ldots; b) El proyecto o programa que el prestador
ejecutó durante su servicio social, deberá acompañarse de un protocolo que
contenga al menos la introducción, justificación, objetivos, metodología
utilizada y metas alcanzadas; c) La extensión mínima de la memoria o tesina
será de 60 cuartillas escritas a doble espacio; d) La redacción deberá ser
aprobada previamente por el asesor correspondiente (\textit{ibid.}, p.
492).
\end{quotation}

%\smallskip
Además de cubrir con ese requisito, el estudiante deberá comprobar que ha
concluido su servicio social a través de la presentación de la carta
expedida por la Dirección General de Servicio Social Universitario, lo que
le permitirá continuar con el trámite de titulación. 

En la Unidad Académica de Historia, hasta el momento se han presentado solo
3 casos en esta modalidad: 1 en 2013; 1 en 2006 y 1 en 2010 (véase el Cuadro II). 
Sin embargo, es de nuestro conocimiento que hay alumnos de las diferentes generaciones de
egresados que se encuentran realizado esta práctica para hacer la solicitud
correspondiente de titulación a la Comisión de Titulación de la Facultad.

%\medskip
\textbf{Titulación por elaboración o defensa de tesis\linebreak producto de la
participación en proyectos de\linebreak investigación}

La modalidad de titulación por elaboración o defensa de tesis dentro de la
Licenciatura en Historia es una de las más recurrentes y la que más demanda
tiene. Se entiende por tesis el texto escrito resultado de un proceso de
investigación documental y de campo que, concluye con una postura en torno
a un problema especifico en el área de conocimiento de formación del
alumno.

Una característica importante que debe cubrir un proyecto de tesis es tener
asignado para su presentación un jurado, compuesto de un presidente,
secretario y vocal, que son nombrados por la Comisión de Titulación. En el
caso del vocal es elección del alumno ya que ese cargo lo desarrolla el
asesor o director de tesis, previamente registrado por el estudiante.

Otro de los aspectos contemplados en el reglamento de titulación vigente es
que los proyectos de tesis podrán ser desarrollados en las modalidades
individual o colectiva. En el caso de la modalidad colectiva, se aceptan
hasta tres alumnos de ser necesario, previamente aceptado por la Comisión
de Titulación, que a su juicio la profundidad y amplitud del tema requiere
del trabajo conjunto. Sin embargo, la réplica se hace de manera individual,
en examen por separado.

Los proyectos de tesis deben seguir los lineamientos de estilo y forma
que se piden en este tipo de trabajos, ya sean individuales o colectivos.
El principal es la extensión mínima que deben cubrir, que para el caso
del nivel licenciatura es de 80 cuartillas en las individuales y 150 en las
colectivas, escritas ambas a doble espacio (véase~el~Cuadro~III).

Como ya se ha mencionado, el grueso de la población estudiantil de la
Licenciatura en Historia, debido a su formación, se inclina más hacia la
defensa de un trabajo de investigación. Durante los primeros 15 años de
vida del programa educativo, la única opción de titulación fue la
presentación y defensa de la tesis, sin embargo, en los años siguientes se
sigue observando esta modalidad como una de las principales elecciones de
los estudiantes.

\smallskip 
\textit{Relación de titulación por la modalidad de elaboración y defensa de
tesis de 1993 a 2003}
\enlargethispage{1\baselineskip}

\smallskip
Iniciando las actividades de presentación de tesis en 1993, cuando egresa la
primera generación de licenciados en historia,  8 alumnos presentan sus
proyectos de investigación y aunque se registra una baja en los años
siguientes, se observa la persistencia de esta modalidad. Salvo en el año
de 1999 que no quedo asentado en el libro de actas de la Facultad la
presentación de algún trabajo de esta índole.


\smallskip 
\textit{Relación de titulación por la modalidad de elaboración y defensa de
tesis de 2004 a 2014}

\smallskip
En el 2005 se vuelve a dar un aumento en la presentación de Tesis. En 2009 y
2012, un número de alumnos considerable, para el área de humanidades, ve
más cercano el proceso de titulación al hacer la defensa de sus trabajos.
Cabe mencionar que en 2008 se presenta la primera tesis colectiva en la
Unidad Académica de Historia, lo que abrió paso a otras dos en la misma
tesitura en el año de 2009.

La eficiencia terminal de la carrera de Historia gira en torno a esta
modalidad de titulación, donde el 48\,\% de los egresados hasta la fecha, se
han decidido por la elaboración y defensa de tesis. Lo anterior demuestra
que, a pesar de que la docencia es una  de las principales áreas de
oportunidades para nuestros estudiares, se inclinan más por la
investigación de los procesos históricos y buscan ampliar sus horizontes al
incursionar a un Posgrado.


\medskip
\begin{sloppypar}
\textbf{Titulación por Examen de In\-glés Toefl o Exa\-men apli\-cado por el
Cen\-tro de Es\-tu\-dios de Idiomas de la Universidad}
\end{sloppypar}

Para los estudiantes que se decidan a optar por esta modalidad titulación
deben de cumplir con los siguientes requisitos:


I. Haber cubierto la totalidad de los créditos o asignaturas del plan de
estudios respectivo; II. Acreditar al menos 450 puntos en examen de inglés
Toefl; III. Demostrar a través de un examen de conocimientos el dominio de
las cuatro habilidades del idioma (expresión oral, escritura, comprensión
lectora y auditiva) (\textit{ibid.}, p. 495).


La titulación por examen de inglés, es una de las opciones poco solicitadas
en la Licenciatura en Historia, aun cuando se tiene la referencia que gran
parte de sus alumnos estudian otro idioma en el Centro de Idiomas de la
Universidad. Hasta la fecha son solo dos los alumnos que han tomado esta
opción y que mediante la documentación requerida han realizado el tramité
correspondiente para formar parte de los seguidores de Clío (véase el Cuadro~IV).
%\newpage

\textbf{Titulación por Examen General de Conocimientos Interno o Externo}
\enlargethispage{1\baselineskip}

\medskip
Según consta en el reglamento de titulación de la Universidad, para optar
por la modalidad de titulación por examen general de conocimientos, lo
alumnos deben haber cubierto la totalidad de los créditos o asignaturas del
plan de estudios respectivo, con el objetivo de demostrar el objeto teórico
práctico fundamental de su carrera.

Este examen consiste en un interrogatorio mediante el cual el aspirante,
demuestre fehacientemente  haber alcanzado los objetivos de la carrera
planteados en el plan de estudios respectivo. En el caso de ser examen
interno la Comisión de Titulación será la encargada de programar y evaluar
el desarrollo del mismo. En caso de ser externo, la misma Comisión
gestionará los trámites necesarios ante el organismo correspondiente.


Para el caso de la Licenciatura en Historia, esta representa una de las
opciones que no se ha implementado o bien no ha sido requerida por los
egresados de nuestra Facultad. Sin embargo, forma parte de una de las metas
planteadas por la Facultad en el Programa Institucional de Fortalecimiento
Integral, el promocionar la titulación a través de esta modalidad 
(véase el Cuadro~V).


\medskip
\textbf{Titulación por Diplomado}
\enlargethispage{1\baselineskip}

Un diplomado es un estudio fuera de los planes de estudio de las carreras,
que tiene como propósito el profundizar en un área del conocimiento y que
puede ser  impartido como una opción de titulación por la propia
Universidad. Los requisitos a cubrir por los alumnos que participen en
ellos son los que señalan la convocatoria que para tal efecto emita la
Comisión de Titulación de la Unidad Académica respectiva. En caso de optar
por un diplomado ofrecido por otra Unidad Académica, se solicitará
autorización a la Comisión de Titulación mediante solicitud y programa de
estudios del diplomado de su elección.


Cabe señalar, que con la reforma hecha a la Nueva Legislación
Universitaria, se fusiona a esta opción la de titulación por Seminario de
Titulación que en su momento se llevó a cabo, en el caso de la carrera de
Historia. Persiguiendo los mismos fines se ofertaban en un principio estos
seminarios donde los alumnos participantes continuaran desarrollando
funciones de investigación. En este sentido en 2002 se implementa esta
modalidad para que alumnos egresados de la Licenciatura en Historia
pudieran tomarlo como cierre de carrera y como una opción de 
titulación (véase el Cuadro~VI).


Por situaciones ajenas a la propia Facultad, a pesar de tener una gran
convocatoria, de los alumnos que llevaron el Seminario y que culminaron,
fue un número reducido quienes realizaron trámites de titulación. Cuando se
hace una nueva reforma a las opciones de titulación, se fusionan Seminarios
de Titulación con los Diplomados. Siendo esta última otras de las opciones
a las que más recurren los estudiantes egresados de la Facultad.

En esta modalidad La Facultad de Historia, ha ofertado los diplomados
siguientes: en Historia de Sinaloa, en dos ediciones; en Historia del Arte
I e Historia del Arte II, en dos ediciones; y en Humanidades: Filosofía,
Historia y Literatura, este último en colaboración con la escuela de
Filosofía y Letras de la Universidad, desarrollado en tres ediciones.

\textit{Eficiencia terminal  por titulación por Diplomado}
\enlargethispage{1\baselineskip}

En el Cuadro VI podemos observar que, en 2005, 11 alumnos de los que
participaron en estos programas educativos realizaron su trámite de
titulación, representando ese año el más fuerte con referencia a la
titulación a través de esta modalidad, seguida del 2011 con 9 alumnos.

Hasta nuestros días se sigue implementando la educación continua, enfocada
con fines de titulación, lo que apoya a varios programas educativos a
elevar sus índices de eficiencia terminal. Estos diplomados, avalados por
Secretaría Académica y Secretaria General Universitarias, son presentados
ante el H. Consejo Universitario para ser aprobados y ponerse en marcha.
Cada uno con requerimientos especiales a su área del conocimiento y basados
en las competencias profesionales necesarias.
%\newpage

\textbf{Titulación por Práctica Profesional}
\enlargethispage{1\baselineskip}

En la modalidad de práctica profesional pueden titularse aquellos
estudiantes que acrediten la realización de esta forma en el área de que se
trate, durante tres años a partir de la haber concluido sus estudios, con
la realización de actividades propias del campo profesional de la carrera
cursada y con el visto bueno de la Comisión de Titulación de la Unidad
Académica correspondiente.


En casos especiales los estudiantes pueden acreditar conocimientos
correspondientes a niveles educativos o grados escolares adquiridos en
forma autodidacta o a través de la experiencia laboral o con base en el
régimen de certificación referido a la formación para el trabajo, conforme
a lo estipulado en el Acuerdo Secretarial número 328 (Diario de la
Federación, 30-07-2003), que modifica al 286 del 30 de octubre del 2000
(Diario de la Federación), siempre y cuando cumplan con los requisitos para
hacerlo (\textit{ibid.}, pp. 497--498).

Para titularse por esta vía el interesado debe presentar a la Comisión de
Titulación de la Unidad Académica la solicitud correspondiente y la
documentación necesaria para continuar con el trámite, como: acta de
nacimiento, copia de la CURP, identificación oficial con fotografía,
Kardex, Currículum Vitae ampliado y con respaldo documental, constancia de
experiencia laboral y lo más importante, carta aval de honorabilidad y
correcto ejercicio del desempeño laboral expedida por un representante de
una persona moral legalmente constituida y con un objeto social vinculado
con los conocimientos que se desean acreditar.

En esta modalidad de titulación la Facultad de Historia no ha tenido mucho
eco, a pesar de que gran parte de la población de egresados de la
Licenciatura en Historia, se dedican a la docencia en centros educativos
públicos o particulares. Sin embargo, uno de los requisitos que estas
instituciones solicitan, para poder sostener en nomina  a estos egresados,
es que presenten su titulo de licenciatura. Lo anterior hace suponer que,
por esta razón los egresados se inclinan más por cualquiera de las otras
modalidades y en menor medida por esta otra.

Hasta el momento solo 5 alumnos, de los cuales ha quedado registro en el
departamento de control escolar, han realizado trámite de titulación por
este medio. Se observa además, que desde 2011 no se ha registrado ningún
otro caso (véase el Cuadro~VII). 

A esta modalidad de titulación también se le adhiere otra de las opciones
que se contemplaron en un primer momento, que es la opción de titulación
por elaboración de manuales o material didáctico. De esta solo se encontró
el registro de una persona titulado a través de este medio.
\enlargethispage{1\baselineskip}

A manera de conclusión, podemos decir que las diferentes opciones de
titulación vinieron a colaborar, no solo con la Licenciatura en Historia,
sino con las demás licenciaturas ofertadas en la UAS, a elevar los índices
de titulación y con ello a la eficiencia terminal de cada uno de los
programas educativos. Además, cada una de las modalidades cubre las
características necesarias para titular a sus egresados, así como a hacer
un análisis de los recursos humanos que se están formando, las competencias
profesionales que están desarrollando y los valores que están alcanzando.

\medskip
\textbf{Referencias}

Universidad Autónoma de Sinaloa (2009), \textit{Nueva Legislación Universitaria},
México, UAS.
\newpage

%\medskip
\textbf{\footnotesize Cuadro I. Titulación por promedio}\par

%\smallskip
\begin{center}
\begin{footnotesize} 
\setlength{\extrarowheight}{0.5pt}      
%\begin{table*}[h]
\tabulinesep=1.5mm
\begin{longtabu*} to \textwidth {X[7,l,p]X[50,c,p]X[22,c,p]X[21,c,p]} 
\toprule
\rowcolor{sLightGreen} {\bfseries No.} & {\bfseries Nombre} &  {\bfseries Año\par de\par titulación} &  
{\bfseries Generación}\\ 
    \midrule
  \endfirsthead%
\multicolumn{4}{r}{{Viene de la página anterior\ldots}} \\ \midrule 
\toprule
\rowcolor{sLightGreen} {\bfseries No.} & {\bfseries Nombre} &  {\bfseries Año\par de\par titulación} &  
{\bfseries Generación}\\ 
\midrule
\endhead%
\bottomrule
\multicolumn{4}{r}{{Continúa en la siguiente página\ldots}} %\\\ %hline
\endfoot%
\midrule\endlastfoot%
% Now the regular content:
1  &  Ofelia Janeth Chávez Ojeda & 2003 &  1996--2001\\\midrule
2  &  Reginaldo Santoyo García   & 2003 &  1996--2002 \\\midrule
3  &  Sergio Romo Santos         & 2003 &  1996--2001\\\midrule
4  &  Yanel Arreola Urrea       & 2006 &  1994--2005\\\midrule
5  &  Aida Rodríguez Campaña     & 2006 &  2000--2005\\\midrule
6  &  Andrea Olivia Ramírez Acosta & 2006 & 1999--2004\\\midrule
7  &  Cintia Dalladi Linares Cacique & 2006 & 2000--2004\\\midrule
8  &  Cristobal Gómez Vazquez    & 2006 & \phantom{2000}\\\midrule
9  &  Flor de los Ángeles Machorro Pérez & 2006 & 2000--2005\\\midrule
10 &  Janitzio Guadalupe Osorio Flores &   2006 & 1999--2004\\\midrule
11 &  Luis Martín Padilla Ordoñes & 2006 & 2001--2005\\\midrule
12 &  Mónica del Rosario Osorio Flores & 2006 & 1999--2004\\\midrule
13 &  Fernando Rodelo Mendoza & 2007 & 2002--2006\\\midrule
14 &  Leonidastenoch Carlos Farrera Ruíz & 2007 & 2003--2007\\\midrule
15 &  Miriam Faviola Soto Quintero &  2007 & 2003--2007\\\midrule
16 &  Natali Gaxiola Soto & 2007 & 2002--2006\\\midrule
17 &  Efraín López Sánchez & 2008 & 2002--2006\\\midrule
18 &  Cinthia Nashelli Audelo Martínez & 2009 & 2005--2009\\\midrule
19 &  Karely Lucia Elizalde Félix & 2009 & 2005--2009\\\midrule
20 &  David Enrique Alaniz Longoria & 2010 & 2004--2008\\\midrule
21 &  María Antonia Vega Covarrubias & 2010 & 2005--2009\\\midrule
22 &  Ramiro Estudillo Pérez & 2012 & 1998--2004\\\midrule
23 &  Karla Verónica López Ley & 2012 & 2008--2012\\\midrule
24 &  Roberto Meza Flores & 2012 & 2001--2005\\\midrule
25 &  Ramona de Jesús Bernal & 2013 & 2009--2013\\\midrule
26 &  Brenda Karina Duarte Oros & 2014 & 2007--2011\\\midrule
27 &  Lilia Margarita Moreno Cervantes & 2014 & 2006--2010\\\midrule
28 &  Esther Italia Rojo Arellanes & 2014 & 2008--2012\\%\midrule
\bottomrule
\end{longtabu*}
\end{footnotesize} 
\end{center} 
\newpage

\textbf{\footnotesize Cuadro II. Titulación por Memoria de Servicio Social con 
rigor metodológico o tesina}\par

%\smallskip
\begin{center}
\begin{footnotesize} 
\setlength{\extrarowheight}{0.5pt}      
%\begin{table*}[h]
\tabulinesep=1.5mm
\begin{longtabu*} to \textwidth {X[7,l,p]X[50,c,p]X[22,c,p]X[21,c,p]} 
\toprule
\rowcolor{sLightGreen} {\bfseries No.} & {\bfseries Nombre} &  {\bfseries Año\par de\par titulación} &  
{\bfseries Generación}\\ 
    \midrule
  \endfirsthead%
\multicolumn{4}{r}{{Viene de la página anterior\ldots}} \\ \midrule 
\toprule
\rowcolor{sLightGreen} {\bfseries No.} & {\bfseries Nombre} &  {\bfseries Año\par de\par titulación} &  
{\bfseries Generación}\\ 
\midrule
\endhead%
\bottomrule
\multicolumn{4}{r}{{Continúa en la siguiente página\ldots}} %\\\ %hline
\endfoot%
\midrule\endlastfoot%
% Now the regular content:
1  & Omar Tena Garrido  & 2003  & 1995--2001 \\\midrule
2  & Guadalupe de Jesús Meza Martínez  & 2006  & 2000--2005 \\\midrule
3  & María Alicia Larrañaga Cerda  & 2010  & 2005--2009 \\%\midrule
\bottomrule
\end{longtabu*}
\end{footnotesize} 
\end{center} 

\bigskip
\textbf{\footnotesize Cuadro III. Titulación por elaboración y defensa de tesis}\par

%\smallskip
\begin{center}
\begin{scriptsize} 
\setlength{\extrarowheight}{0.5pt}      
%\begin{table*}[h]
\tabulinesep=1.5mm
\begin{longtabu*} to \textwidth {X[6,l,p]X[35,l,p]X[40,l,p]X[19,c,p]} 
\toprule
\rowcolor{sLightGreen} {\bfseries No.} & {\bfseries Alumno} &  {\bfseries Título de tesis} &  
{\bfseries Fecha  de presentación}\\ 
    \midrule
  \endfirsthead%
\multicolumn{4}{r}{{Viene de la página anterior\ldots}} \\ \midrule 
\toprule
\rowcolor{sLightGreen} {\bfseries No.} & {\bfseries Alumno} &  {\bfseries Título de tesis} &  
{\bfseries Fecha  de presentación}\\ 
\midrule
\endhead%
\bottomrule
\multicolumn{4}{r}{{Continúa en la siguiente página\ldots}} %\\\ %hline
\endfoot%
\midrule\endlastfoot%
% Now the regular content:
1 & Alfonso Mercado Gómez  & Compostela, Nayarit en el siglo XVI:
rasgos de su historia  &  28/04/1993 \\\midrule
2 &
   Beatriz Rico Álvarez   &
   Los comerciantes de Culiacán durante el
porfiriato  &
   28/04/1993 \\\midrule
   3  &
   Eduardo Frías Sarmiento  &
   Origen de las compañías privadas de
alumbrado público en Culiacán: 1895--1915  &
   29/04/1993 \\\midrule
   4  &
   Félix Brito Rodríguez   &
    \-\-\-\-\-\-\- y tecnología en Rosario y
Concordia durante los años 1895--1910  &
   29/041993 \\\midrule
   5  &
   Rosa Amelia Félix Lara  &
   Los Redo: una familia empresarial, 1870--1920  &
   30/04/1993 \\\midrule
   6  &
   Mayra Lizzete Vidales Quintero  &
   Comerciantes en Culiacán. Un proceso de
transición: 1900--1920  &
   30/04/1993 \\\midrule
   7  &
   Samuel Octavio Ojeda Gastelum  &
   El mezcal: una fuente de riqueza en
Sinaloa durante el porfiriato  &
   11/04/1993 \\\midrule
   8  &
   José Manuel Frías Quintero  &
   Tacuichamona: origen y fundación  &
   25/11/1993 \\\midrule
   9  &
   Sergio Arturo Sánchez Parra  &
   El movimiento estudiantil universitario,
1966--1974  &
   31/05/1994 \\\midrule
   10  &
   Gilberto López Castillo  &
   Propiedad territorial en la provincia de
Culiacán, 1691--1810; la llanura costera  &
   17/06/1994 \\\midrule
   11  &
   María Elda Rivera Calvo  &
   Principales empresarios agrícolas en la
región de Ahome. Su evolución histórica. 1886--1930  &
   29/08/1995 \\\midrule
   12  &
   Blanca Mireya Lara Madrid  &
   La resistencia indígena frente a la
evangelización (Sinaloa en los siglos XVI y XVII)  &
   22/02/1996 \\\midrule
   13  &
   Manuel Hernández Martínez  &
   Breves noticias acerca de la educación en
Sinaloa, durante el porfiriato (1877--1902)  &
   03/07/1997 \\\midrule
   14  &
   Héctor Carlos Leal Camacho  &
   Sinaloa durante la Revolución. El papel de
los intelectuales en la transformación social: 1909--1922  &
   04/03/1998 \\\midrule
   15  &
   Carlos Enrique Rubio Juárez  &
   La hacienda de Pericos durante el
porfiriato  &
   08/07/1997 \\\midrule
   16  &
   Edi Omar Audelo Gastelum  &
   La industria textil en Sinaloa
(1877--1911)  &
   02/06/1998 \\\midrule
   17  &
   Rosendo Romero Guzmán  &
   Inmigración asiática a Sinaloa. El caso de
los chinos: 1880--1934  &
   09/10/1998 \\\midrule
   18  &
   María del Rosario Heras Torres  &
   Vida social en Culiacán durante el
cañedismo, 1895--1909  &
   23/08/2000 \\\midrule
   19  &
   Olivia Loza Vera  &
   La modernización de la agricultura
sinaloense y la contaminación por agroquímicos, 1950--1970  &
   01/09/2000 \\\midrule
   20  &
   Jesús Armando Monreal Ceyca  &
   El impacto de la política educativa
nacional en Sinaloa: del Colegio Civil Rosales a la Universidad Socialista
del Noroeste (1935--1940)  &
   26/06/2001 \\\midrule
   21  &
   Olga Martínez Sandoval  &
   Proceso histórico de la erradicación de la
poliomielitis en México. El caso de Sinaloa (1986--1994)  &
   09/07/2001 \\\midrule
   22  &
   Laura Elena Lira Morales  &
   Servicio público de agua en Culiacán
(Empresa de Agua de Sinaloa, S. A., 1887--1909)  &
   18/01/2002 \\\midrule
   23  &
   Rafael Santos Cenobio  &
   El movimiento estudiantil en la UAS
(1966--1972)  &
   25/09/2002 \\\midrule
   24  &
   Víctor Adrián González Pérez  &
   Las rebeliones de los indios cocoyomes en
el reino de la Nueva Vizcaya, 1691--1693  &
   03/10/2002 \\\midrule
   25  &
   Melina Carrillo Gutiérrez  &
   La instrucción femenina en Sinaloa.
Aspectos generales sobre su orientación y desarrollo. 1877--1910  &
   17/01/2003 \\\midrule
   26  &
   José Francisco Pérez Ríos  &
   CAADES: una institución de la agricultura
sinaloense (1980--2000)  &
   19/05/2003 \\\midrule
   27  &
   Mario Sánchez Aguirre  &
   Una mirada histórica al narcocorrido en
Sinaloa: apología, censura y tragedia social  &
   06/06/2003 \\\midrule
   28  &
   Wilfrido Llanes Espinoza  &
   Iglesia y autoridad secular: una disputa
de poder en las postrimerías de la colonia. La \-\-\- de la comunidad
eclesiástica de la casa cural de Mocorito  &
   21/08/2003 \\\midrule
   29  &
   Sara Nohemy Velarde Sarabia  &
   La encomienda en las provincias de
Chiametla, Culiacán y Sinaloa. Siglo XVI: un enfoque geohistórico  &
   12/09/2003 \\\midrule
   30  &
   Annabel García Carlos  &
   El Partido Acción Nacional en Sinaloa;
entre pasado y memoria. Las dos primeras décadas de su proceso
organizativo  &
   12/03/2004 \\\midrule
   31  &
   Liliana Plascencia Sánchez  &
   Entre la preocupación y la amenaza social.
El discurso de los jóvenes en Culiacán (1960--1968)  &
   31/03/2004 \\\midrule
   32  &
   Luis Felipe Días Cruz  &
   La ganadería en Sinaloa, 1946--1974; y las
implicaciones del ejido-ganadero  &
   30/11/2004 \\\midrule
   33  &
   Moisés Medina Armenta  &
   Formas, espacios y medios de diversión en
el Culiacán cañedista, 1895--1910  &
   15/03/2005 \\\midrule
   34  &
   Mario Cesar Islas Flores  &
   Tres aproximaciones a Clío: literatura,
vida cotidiana e imágenes. El problema de la cientificidad de la historia 
&
   16/03/2005 \\\midrule
   35  &
   Pedro Cazares Aboytes  &
   El movimiento obrero-campesino en la
United Sugar Companies: 1903--1939  &
   08/04/2005 \\\midrule
   36  &
   Ana Lilia Altamirano Prado  &
   El estudio de la nupcialidad en la
provincia de Culiacán: 1760--1778  &
   13/04/2005 \\\midrule
   37  &
   Javier Fuentes Posadas  &
   Una rebelión indígena al amparo de la
revolución: Felipe Bachomo y los mayos, 1913--1916  &
   29/08/2005 \\\midrule
   38  &
   Diana Sugey Burgos Aguilar  &
   La familia Clouthier; parte de la elite
culiacanense. Su trayectoria en los años 1920--1950  &
   11/10/2005 \\\midrule
   39  &
   Pedro Arturo Santos Díaz  &
   Origen, desarrollo y crisis en la Liga de
la Costa del Pacífico (1945--1958)  &
   07/06/2006 \\\midrule
   40  &
   María Anita Félix Osuna  &
   Las mujeres de Culiacán a través de la
prensa local, 1965--1968  &
   26/09/2006 \\\midrule
   41  &
   Nancy Sugey Moreno Matus  &
   Bocetos históricos sobre tres templos
católicos de Culiacán  &
   28/05/2007 \\\midrule
   42  &
   Víctor Hugo Sosa Ortíz  &
   Usos y manejos del agua en Sinaloa,
1877--1910: motor para el crecimiento económico  &
   06/07/2007 \\\midrule
   43  &
   Juan Carlos Díaz Arroyo  &
   El reparto agrario en el Valle de San
Lorenzo (1915--1940)  &
   29/08/2007 \\\midrule
   44  &
   Fabiola Guadalupe Gaxiola López  &
   Expresión plástica monumental en Culiacán,
1958--2007  &
   17/01/2008 \\\midrule
   45  &
   Araceli Santiago Ramírez  &
   Trabajadores en las haciendas azucareras
del norte de Sinaloa, 1900--1910  &
   25/02/2008 \\\midrule
   46  &
Ana Julieta Rueda Morales   &
 &
   28/02/2008 \\\midrule
   47  &
   Francisco Javier Osuna Félix  &
   La situación de la minería en el sur de
Sinaloa durante el porfiriato  &
   09/04/2008 \\\midrule
   48  &
   Nubia Gabriela Valenzuela Frías  &
   Escuela Normal de Sinaloa  &
   12/02/2009 \\\midrule
   49  &
Yaneth Guadalupe Gámez Rivera   &  \phantom{abc}  & \phantom{123}\\\midrule
   50  &
   Luis Salvador Morales Zepeda   &
   La escritura de la historia  &
   13/02/2009 \\\midrule
   51  &
   Gerardo Jiménez Maldonado  &
   Vida cotidiana; futbol en Culiacán  &
   13/02/2009 \\\midrule
   52  &
   Jesica Rosas Villa  &
   Solidaridad estudiantil en el Valle de
Culiacán  &
   13/02/2009 \\\midrule
   53  &
   Miguel Ángel Higuera Félix  &
   La otra cara del cañedismo: una sociedad
amenazada por calamidades y penurias  &
   23/06/2009 \\\midrule
   54  &
Milagros Millán Rocha   &  \phantom{abc}  & \phantom{123}\\\midrule
   55  &
   Rosa Yuneiry Ramírez Topete  &
   Voces del Culiacán ausente. Esparcimiento,
idilios y vida familiar, 1940--1960  &
   26/08/2009 \\\midrule
   56  &
   Elizabeth Álvarez Castro  &
   La música en Sinaloa; de la revolución
mexicana hasta 1960  &
   02/10/2009 \\\midrule
   57  &
   Elma Leticia Araujo Leyva  &
   Agrarismo en Sinaloa de Leyva, Sinaloa,
1915--1934  &
   20/01/2010 \\\midrule
   58  &
   Catarino Escobar Macías  &
   La enseñanza de la ingeniería en el
Colegio Rosales, 1874--1909  &
   09/07/2010 \\\midrule
   59  &
   Pedro Pablo Favela Astorga  &
   Un inciso irrelevante. Génesis de una
política cultural en Sinaloa, 1966--1975  &
   21/09/2010 \\\midrule
   60  &
   Ma. Benita Escarcega Ríos  &
   La moral transgredida. Bigamia y castigo
en Sinaloa y Sonora (siglo XVIII)  &
   29/09/2010 \\\midrule
   61  &
   Omar Hernández Millán  &
   Las bebidas embriagantes en Culiacán
1949--1954. Problemas y diversiones, una visión entre lo prohibido y lo
legal  &
   07/10/2010 \\\midrule
   62  &
   París Padilla Salazar  &
   Redes empresariales en Culiacán durante el
cañedismo  &
   01/04/2011 \\\midrule
   63  &
   Abel Astorga Morales  &
   Entre la satisfacción y el desencanto.
Experiencias de braceros sinaloenses (1942--1964)  &
   27/06/2011 \\\midrule
   64  &
   Diego Marcel Benítez Ramírez  &
   La esperanza deshecha. Ramón López Velarde
ante la revolución  &
   13/12/2011 \\\midrule
   65  &
   Antonio Santiago León  &
   El \-\-\- mixteco influenciado por la
política agraria de Adolfo López Mateos; de la región mixteca a la Sierra
Norte: la Mixtequita 1961--1981  &
   16/03/2012 \\\midrule
   66  &
   Oscar Antonio Aguilar Bastidas  &
   Familia Díaz Angulo: trayectoria
empresarial en la zona centro norte del estado de Sinaloa, 1930--1960  &
   22/03/2012 \\\midrule
   67  &
   Sandra Luz Gaxiola Valdovinos  &
   Del mercado a la mesa: alimentos y comida
en Culiacán porfirista  &
   29/03/2012 \\\midrule
   68  &
   Amanda Liliana Osuna Rendón  &
   Prestigio y poder a través de los retratos
de la elite sinaloense de finales del siglo XIX  &
   27/06/2012 \\\midrule
   69  &
   Eddy Yamir Ojeda Delgado  &
   Protagonistas y escenarios de la rebelión
escobarista en Sinaloa, 1929  &
   12/12/2012 \\\midrule
   70  &
   Roberto Carlos Verdugo Pompa  &
   El transporte urbano en Culiacán
(1940-1960)  &
   12/12/2012 \\\midrule
   71  &
   Juan Luis Ríos Treviño  &
   Sociabilidades políticas de finales del
porfiriato a inicios de la revolución en Sinaloa  &
   13/12/2012 \\\midrule
   72  &
   José Guadalupe Zamora Medina  &
   ¡Que me siga la tambora! Las bandas de
viento en Culiacán: 1940--1963  &
   16/01/2013 \\\midrule
   73  &
   Hugo Gabriel Cruz Martínez  &
   Yo fui bracero. La trayectoria laboral de
Gregorio Villanueva Vital, 1958--1962. Rumbo al norte, vivencias, penurias y
deslumbramiento de dólares  &
   24/04/2013 \\\midrule
   74  &
   Josué David Piña Valenzuela  &
   La literatura modernista en el cañedismo:
Enrique González Martínez, Julio G. Arce y Amado Nervo  &
   07/06/2013 \\\midrule
   75  &
   Josefina Raquel Favela Ahumada  &
   La participación de la mujer en la lucha
por el derecho al voto en Sinaloa (1940--1960)  &
   30/10/2013 \\\midrule
   76  &
   Bárbara Anahí Tolosa Arámburo  &
   Los ingenios azucareros en Culiacán,
1890-1940. Introducción al estudio de patrimonio industrial  &
   13/11/2013 \\\midrule
   77  &
   Héctor Castro Ahumada  &
   Irregularidades en la ocupación y
repartimiento de tierras en Ocoroni durante la primera mitad del siglo
XVIII  &
   20/05/2014 \\\midrule
   78  &
   Gustavo Telechea Saldaña  & \phantom{123} & 23/05/214\\\bottomrule
\end{longtabu*}
\end{scriptsize} 
\end{center} 

\textbf{\footnotesize Cuadro IV. Titulación por segundo idioma}

%\smallskip
\begin{center}
\begin{footnotesize} 
\setlength{\extrarowheight}{0.5pt}      
%\begin{table*}[h]
\tabulinesep=1.5mm
\begin{longtabu*} to \textwidth {X[7,l,p]X[50,c,p]X[22,c,p]X[21,c,p]} 
\toprule
\rowcolor{sLightGreen} {\bfseries No.} & {\bfseries Nombre} &  {\bfseries Año\par de\par titulación} &  
{\bfseries Generación}\\ 
    \midrule
  \endfirsthead%
\multicolumn{4}{r}{{Viene de la página anterior\ldots}} \\ \midrule 
\toprule
\rowcolor{sLightGreen} {\bfseries No.} & {\bfseries Nombre} &  {\bfseries Año\par de\par titulación} &  
{\bfseries Generación}\\ 
\midrule
\endhead%
\bottomrule
\multicolumn{4}{r}{{Continúa en la siguiente página\ldots}} %\\\ %hline
\endfoot%
\midrule\endlastfoot%
% Now the regular content:
1  & Alain Arturo Pulido Barrón  & 2007  & 2002--2006 \\\midrule
2  & Héctor Alberto Félix Derat  & 2010  & 2003--2009 \\%\midrule
\bottomrule
\end{longtabu*}
\end{footnotesize} 
\end{center} 

\bigskip
\textbf{\footnotesize Cuadro V. Titulación por Seminario de Titulación}\par

%\smallskip
\begin{center}
\begin{footnotesize} 
\setlength{\extrarowheight}{0.5pt}      
%\begin{table*}[h]
\tabulinesep=1.5mm
\begin{longtabu*} to \textwidth {X[7,l,p]X[50,c,p]X[22,c,p]X[21,c,p]} 
\toprule
\rowcolor{sLightGreen} {\bfseries No.} & {\bfseries Nombre} &  {\bfseries Año\par de\par titulación} &  
{\bfseries Generación}\\ 
    \midrule
  \endfirsthead%
\multicolumn{4}{r}{{Viene de la página anterior\ldots}} \\ \midrule 
\toprule
\rowcolor{sLightGreen} {\bfseries No.} & {\bfseries Nombre} &  {\bfseries Año\par de\par titulación} &  
{\bfseries Generación}\\ 
\midrule
\endhead%
\bottomrule
\multicolumn{4}{r}{{Continúa en la siguiente página\ldots}} %\\\ %hline
\endfoot%
\midrule\endlastfoot%
% Now the regular content:
1  & María Luisa Acuña Félix  & 2002  & 1994--2002\\\midrule
2  & Alicia Angélica Urrecha López  & 2003  & 1993--1998\\\midrule
3  & María Gabriela Fuentes García  & 2004  & 1994--2003\\\midrule
   4  &
   Rosa del Carmen Ibarra Aispuro  &
    2005  &
     1989--2004\\\midrule
   5  &
   Felipe Gastélum Navarrete  &
    2007  &
     1988--1993\\\midrule
   6  &
   Jaime Sánchez Carrizosa  &
    2007  &
     1988--1993\\\midrule
   7  &
   Cristian Contreras López  &
    2008  &
     2002--2006\\\midrule
   8  &
   Jesús Marbella García Medina  &
    2008  &
     2003--2007\\\midrule
   9  &
   María Antonia Gastélum Torres  &
    2008  &
     2004--2008\\\midrule
   10  &
   María Isabel Herrera Vega  &
    2009  &
     2002--2006\\\midrule
   11  &
   José Alberto López Montoya  &
    2009  & \phantom{123}\\\midrule
   12  &
   Francisca Olga Daniela Ovalles Camargo  &
    2009  &
     2004--2008\\\midrule
   13  &
   Ilán Antonio Aguilera Benítez  &
    2010  &
     2003--2008\\\midrule
   14  &
   Jorge Alberto Bernal Castellanos  &
    2010  &
     2003--2007\\\bottomrule
\end{longtabu*}
\end{footnotesize} 
\end{center} 

\bigskip 
\textbf{\footnotesize  Cuadro VI. Titulación por Diplomado}\par

%\smallskip
\begin{center}
\begin{footnotesize} 
\setlength{\extrarowheight}{0.5pt}      
%\begin{table*}[h]
\tabulinesep=1.5mm
\begin{longtabu*} to \textwidth {X[7,l,p]X[50,c,p]X[22,c,p]X[21,c,p]} 
\toprule
\rowcolor{sLightGreen} {\bfseries No.} & {\bfseries Nombre} &  {\bfseries Año\par de\par titulación} &  
{\bfseries Generación}\\ 
    \midrule
  \endfirsthead%
\multicolumn{4}{r}{{Viene de la página anterior\ldots}} \\ \midrule 
\toprule
\rowcolor{sLightGreen} {\bfseries No.} & {\bfseries Nombre} &  {\bfseries Año\par de\par titulación} &  
{\bfseries Generación}\\ 
\midrule
\endhead%
\bottomrule
\multicolumn{4}{r}{{Continúa en la siguiente página\ldots}} %\\\ %hline
\endfoot%
\midrule\endlastfoot%
% Now the regular content:
1 & 
  Adriana Valdez Ruíz &
  2005 &
  2000-2004\\\midrule
  2 &
  Diana Zulema Zazueta Salas &
  2005 &
  1998-2004\\\midrule
  3 &
  Jesús Clemente Sánchez Miranda &
  2005 &
  1998-2004\\\midrule
  4 &
  María de los ángeles Sitlalit García
Murillo &
  2005 &
  1996-2001\\\midrule
  5 &
  María Elena Verdugo Ovalles &
  2005 &
  2000-2004\\\midrule
  6 &
  María de los Ángeles Aguilar Rivera &
  2005 &
  2000-2004\\\midrule
  7 &
  Martha Alicia Camacho Loaiza &
  2005 &
  1997-2002\\\midrule
  8 &
  Marco Antonio Ortíz &
  2005 &
  1998-2004\\\midrule
  9 &
  Perla Anaiz Padilla Hernández &
  2005 &
  2000-2004\\\midrule
  10 &
  Santa Isabel Heras Camacho &
  2005 &
 1998-2003\\\midrule
  11 &
  Sergio Bustamante Escamilla &
  2005 &
  1998-2005\\\midrule
  12 &
  Jairo Bernardo Narvaez Llamas &
  2006 &
  1998-2004\\\midrule
  13 &
  Angélica Mabel Gómez Medina &
  2007 &
  1999-2004\\\midrule
  14 &
  Elizabeth Osuna Salazar &
  2007 &
  2000-2005\\\midrule
  15 &
  Iris Martínez Sánchez &
  2008 &
  1998-2006\\\midrule
  16 &
  Angélica Zaray Ramírez Zavala &
  2009 &
  1997-2003\\\midrule
  17 &
  María Rosalina Medina Armenta &
  2010 &
  1998-2003\\\midrule
  18 &
  Felipe Ramiro Aguilar Trujillo &
  2011 &
  1997-2006\\\midrule
  19 &
  Alfonso Fernando Castro Guerrero &
  2011 &
  2005-2009\\\midrule
  20 &
  Luis Ángel Chaidez Félix &
  2011 &
  2005-2009\\\midrule
  21 &
  María Josefina Heras Ceballos &
  2011 &
  2004-2010\\\midrule
  22 &
  Norma Yaremi Leyva Guzmán &
  2011 &
  2000-2008\\\midrule
  23 &
  Martha López Hernández &
 2011 &
  2006-2010\\\midrule
  24 &
  María Fernanda López Ibarra &
  2011 &
  2007-2011\\\midrule
  25 &
  Lorena Ramírez Ávila &
  2011 &
  2006-2010\\\midrule
  26 &
  Betzabe Reyes Trujillo &
  2011 &
  2002-2009\\\midrule
  27 &
  Teresita de Jesús Germán Gastélum &
  2012 &
  2006-2010\\\midrule
  28 &
  Juan Manuel Lizarraga Sánchez &
  2012 &
  2007-2011\\\midrule
  29 &
  Paúl Antonio Félix Meza &
  2013 &
  2006-2010\\\midrule
  30 &
  Patricia María Hernández Salazar &
  2014 &
  2003-2009\\\bottomrule
\end{longtabu*}
\end{footnotesize} 
\end{center} 
\newpage

\textbf{\footnotesize  Cuadro VII. Titulación por Práctica Profesional}\par

%\smallskip
\begin{center}
\begin{footnotesize} 
\setlength{\extrarowheight}{0.5pt}      
%\begin{table*}[h]
\tabulinesep=1.5mm
\begin{longtabu*} to \textwidth {X[7,l,p]X[50,c,p]X[22,c,p]X[21,c,p]} 
\toprule
\rowcolor{sLightGreen} {\bfseries No.} & {\bfseries Nombre} &  {\bfseries Año\par de\par titulación} &  
{\bfseries Generación}\\ 
    \midrule
  \endfirsthead%
\multicolumn{4}{r}{{Viene de la página anterior\ldots}} \\ \midrule 
\toprule
\rowcolor{sLightGreen} {\bfseries No.} & {\bfseries Nombre} &  {\bfseries Año\par de\par titulación} &  
{\bfseries Generación}\\ 
\midrule
\endhead%
\bottomrule
\multicolumn{4}{r}{{Continúa en la siguiente página\ldots}} %\\\ %hline
\endfoot%
\midrule\endlastfoot%
% Now the regular content:
1  &
   Lamberto Vizcarra Cárdenas  &
   2002  &
   1988--1992 \\\midrule
   2  &
   Guadalupe Acosta Aragón  &
   2003  &
   1991--2003 \\\midrule
   3  &
   Manuel Heras Petris  &
   2003  &
   1994--2002 \\\midrule
   4  &
   Miram Lizeth Palomares Villarreal  &
   2010  &
   1995--2000 \\\midrule
   5  &
   Arely Félix Villaseñor  &
   2011  &
   1996--2001 \\\bottomrule
\end{longtabu*}
\end{footnotesize} 
\end{center} 
%\newpage
%\thispagestyle{empty}
%\phantom{abc}

%\documentclass{article}
%\usepackage{amsmath,amssymb,amsfonts}
%\usepackage{fontspec}
%\usepackage{xunicode}
%\usepackage{xltxtra}
%\usepackage{polyglossia}
%\setdefaultlanguage{spanish}
%\usepackage{color}
%\usepackage{array}
%\usepackage{hhline}
%\usepackage{hyperref}
%\hypersetup{colorlinks=true, linkcolor=blue, citecolor=blue, filecolor=blue, urlcolor=blue}
%\usepackage{graphicx}
%% Text styles
%\newcommand\textstyleInternetLink[1]{\textcolor{blue}{#1}}
%\newtheorem{theorem}{Theorem}
%\title{}
%\author{Pat}
%\date{2014-06-14}
%\begin{document}
%%\clearpage\setcounter{page}{261}

\thispagestyle{empty}
\phantomsection{}
\addcontentsline{toc}{chapter}{<<HISTORIA ACA>>: una experiencia en el proceso terminal de los alumnos de la FES Acatlań\newline $\diamond$
\normalfont\textit{Patricia Montoya Rivero y María Cristina Montoya Rivero}}
{\centering {\scshape \large <<HISTORIA ACA>>: una experiencia en el proceso terminal de los 
alumnos de la FES Acatlań}\par}
\markboth{la formación del historiador}{historia aca}
\setcounter{footnote}{0}

\bigskip
\begin{center}
{\bfseries Patricia Montoya Rivero\\
María Cristina Montoya Rivero}\\
{\itshape Facultad de Estudios Superiores Acatlán, UNAM}
\end{center}

\medskip
\epigraph{La técnica y el arte tienen  como mira final servir al hombre del pueblo.\par
La Universidad dejará de producir profesionistas decimonónicos: individualistas, egocentristas 
y simuladores, en cambio, se forjará al profesional capaz y solidario.}{\itshape — José Vasconcelos.}

\bigskip
\textbf{Resumen} 
\enlargethispage{1\baselineskip}

En el año  2001 se inició en la FES Acatlán el proyecto <<Historia ACA:
Catalogación y balance historiográfico de revistas mexicanas especializadas
en historia>>, en el cual se estableció la opción para que los alumnos
integrantes realizaran su Servicio Social catalogando alguna revista y, al
mismo tiempo, quienes estaban en la posibilidad de iniciar sus trabajos de
titulación lo hicieran mediante las modalidades de:  tesina, tesis o el
Servicio Social Profesional,  para ello, además del catálogo tendrían que
hacer un balance y análisis historiográfico de la revista  seleccionada. 

Con esta ponencia queremos compartir nuestra experiencia de cómo se ha
llevado al cabo el servicio social y la titulación de los alumnos que han
participado en nuestro seminario desde que éste se fundó.  Para ello
presentaremos algunas de las características y problemas de los procesos
mencionados.\\
\textbf{Palabras clave:} Servicio social, tesis, tesina, catalogación de\\
revistas, balance historiográfico, titulación.


\medskip
\textbf{Abstract}

In the 2001 year, the "Historia ACA: Cataloging and historiographical 
balance of specialized magazines in Mexican history" began in the FES 
Acatlán, and was established for students that members do, the option 
their Social Service cataloging some magazine and at the same time, 
those on the possibility of obtaining the degree they did in the 
categories of: thesis, dissertation or Professional Social Service, the 
selected magazine.

With this exposition we want to share our experience of what has
been within the social service and certification of students who have
participated in our seminar since it was founded. This will present some of
the characteristics and problems of these processes.\\
\textbf{Keywords:} social service, thesis, dissertation, cataloging
magazines, historiographical balance, titling.


\bigskip
\textbf{Presentación}

En el año  2010 en la FES Acatlán se aprobó el proyecto <<Historia ACA:
Catalogación y balance historiográfico de revistas mexicanas especializadas
en historia, siglo XX>>, para que formara parte del Programa de Apoyo a
Proyectos de Innovación y Mejoramiento de la Enseñanza de la UNAM (con
clave PAPIME PE400310).  En dicho proyecto se estableció la opción para que
los alumnos integrantes realizaran su servicio social catalogando alguna
revista y, al mismo tiempo, quienes estaban en la posibilidad de iniciar
sus trabajos de titulación lo hicieran mediante las modalidades de: tesina,
tesis o Servicio Social Profesional, para ello, además del catálogo
tendrían que hacer un estudio introductorio al catálogo que consistiría en
un balance y análisis historiográfico  de la revista seleccionada, estos
trabajos se realizarían al interior de un Seminario académico organizado
para tal fin. 

Con este trabajo queremos compartir nuestra experiencia de cómo se ha
llevado al cabo el servicio social y la titulación de los alumnos que han
participado en nuestro seminario desde que éste se fundó; por lo cual
presentaremos algunas de las características y problemas de los procesos
mencionados. Para lograr el objetivo propuesto, hemos dividido  el texto en
los siguientes apartados: El Servicio Social en la UNAM, donde se hace un
recuento histórico del mismo, se exponen las características que debe
tener y se presenta su reglamentación. Enseguida, se aborda el punto
titulado \textit{El Proyecto Historia ACA y el Servicio Social}, en este caso se
mencionan los objetivos y características de dicho proyecto, así como su
relación con el servicio social. Finalmente, expondremos la experiencia del
proyecto relacionada con el proceso terminal de los alumnos inscritos en
este PAPIME PE400310.


\bigskip
\textbf{El Servicio Social  en la UNAM}
\enlargethispage{1\baselineskip}

A raíz de la revolución mexicana y para atender aspectos de justicia social,
en los años 30 se estableció en México el servicio social; se trataba de
una respuesta de los egresados de la Universidad para favorecer a las
comunidades marginadas. Ya desde la década anterior, José Vasconcelos había
declarado que: <<La técnica y el arte tiene  como mira final servir al
hombre del pueblo. La Universidad dejará de producir profesionistas
decimonónicos: individualistas, egocentristas y simuladores, en cambio, se
forjará al profesional capaz y solidario>> (Universia 2014).


En marzo de 1934 el rectos de la Universidad, Enrique Gómez Morín presentó
una propuesta de Servicio social en el congreso Nacional de Profesionistas
y el 2 de diciembre del siguiente año el director de la Escuela Nacional de
Medicina, Gustavo Baz Prada propuso ante el rector  el servicio social de
los pasantes de la carrera de medicina en zonas rurales carentes de
servicios de salud, propuesta que se llevó al presidente Lázaro Cárdenas. 
Así fue como a través de un convenio de la UNAM con el departamento de
Salubridad Pública quedó establecido que a partir de entonces el servicio
social de los pasantes de medicina sería obligatorio para quienes
pretendían obtener el título de médico cirujano. Hacia 1940 en la UNAM se
crearon las brigadas multidisciplinarias, llamadas <<misiones
universitarias>>,  de esta manera, egresados de distintas carreras
trabajaban en las vacaciones en proyectos para el desarrollo de las
comunidades del campo.

Con los antecedentes señalados, sería en 1945 cuando se estableció la
obligatoriedad del servicio social en la UNAM en la Constitución Política
de los Estados Unidos Mexicanos. Al paso del tiempo, el servicio social se
ha generalizado y hoy en día todos los pasantes de la UNAM y escuelas
incorporadas tienen la obligación de realizarlo para retribuir en parte a
la sociedad la educación recibida en sus aulas.

En los ochenta años que lleva establecido el servicio social ha habido
logros muy importantes; algunos estudiantes sobresalieron por ello, ya sea
en trabajos individuales o bien colectivos. Para reconocer la excelencia de
los estudiantes en la prestación del servicio social, se fundó el Premio
Gustavo Baz Prada, que consiste en una medalla de plata y un diploma, que
se otorgan uno por carrera de cada una de las\linebreak facultades y/o escuelas,
asimismo el profesor asesor del servicio social recibe también un
reconocimiento. El premio se otorga para gratificar a <<quienes se hayan
distinguido por su participación en programas de servicio social con
impacto social, dirigidos a la población menos favorecida, que coadyuven a
mejorar sus condiciones de vida, contribuyendo así al desarrollo económico,
social y educativo del país>> (Dirección General de Servicios Educativos
2014) 


Fue con motivo de la conmemoración de los 50 años de haberse establecido el
servicio social en la UNAM, que en 1986 se inició el otorgamiento del
premio anual del servicio social <<Gustavo Baz Prada.>>


Hay que recordar la UNAM  cuenta con un reglamento en donde se establecen
las bases y fijan los lineamientos para la prestación del servicio social
de los pasantes, éstos son regulados por: el  Reglamento General de
Estudios Técnicos y Profesionales, el Reglamento General de Exámenes y  los
reglamentos internos que en cada facultad o escuela establezcan  los
consejos técnicos correspondientes.


De acuerdo con el artículo 3 del Reglamento General del Servicio Social de
la Universidad Nacional Autónoma de México, que plantea la Normatividad
Administrativa de la UNAM, se entiende por Servicio Social Universitario: 

%\enlargethispage{1\baselineskip} 
La realización obligatoria de actividades temporales que ejecutan los
estudiantes de carreras técnicas y profesionales, tendientes [sic] a la
aplicación de los conocimientos que hayan obtenido y que impliquen el
ejercicio de la práctica profesional en beneficio o en interés de la
sociedad (Dirección General de Estudios de Legislación Universitaria
2014). 
\newpage

En el mismo reglamento se establecen cuales son los objetivos que persigue
el servicio social:

\begin{Obs}
\item[I.] Extender los beneficios de la ciencia, la técnica y la cultura a la
sociedad; 
\item[II.] Consolidar la formación académica y capacitación profesional del
prestador del servicio social; 
\item[III.] Fomentar en el prestador una conciencia de solidaridad con la comunidad
a la que pertenece. (Dirección General de Estudios de Legislación
Universitaria, 2014). 
\end{Obs}


El servicio social es un requisito que se establece para todas las carrera
de la UNAM y las escuelas incorporadas, indispensable para obtener el grado
académico de licenciatura.

Por otra parte, a partir de 1991 se establecieron en la UNAM nuevas opciones
para obtener el título de licenciatura además de las existentes:
presentación de una tesis o tesina y examen profesional. De acuerdo a los
lineamientos de los consejos técnicos o internos de las diferentes
facultades o escuelas y a las características propias de las carreras, los
egresados pudieron optar para titularse entre diversas modalidades. En la
FES Acatlán quedaron establecidas las siguientes: tesis y examen
profesional, tesina y examen profesional, titulación por ampliación y
profundización de conocimientos, titulación por haber cubierto la totalidad
de créditos con alto nivel académico, titulación mediante estudios de
posgrado, titulación por trabajo profesional y examen profesional,
titulación por servicio social profesional, titulación por actividad de
apoyo a la docencia, titulación por seminario curricular, titulación por
seminario-taller extracurricular, titulación por actividad de investigación
y titulación mediante examen general de conocimientos (Facultad de Estudios
Superiores Acatlán 2014).

\enlargethispage{1\baselineskip}
Hay que destacar que no todas las opciones de titulación están disponibles
para todas las carreras, pues  cada Comité de Programa por licenciatura,
adoptó las más idóneas. (Véase la Nota 1)

\begin{sloppypar}
El Departamento de Servicio Social y Bolsa de Trabajo de la FES Acatlán,
definió que la titulación por Servicio Social deberá prestarse con base en
el Reglamento General de Servicio Social, durante un tiempo no menor de
seis meses ni mayor a un año. El egresado que haya optado por tal modalidad
deberá entregar al Departamento seis informes avalados por su asesor y el
número de horas que requiera deberá ser determinado por las características
del proyecto y la licenciatura a la que se encuentre adscrito.
\end{sloppypar}

El servicio social para titulación podrá realizarse en proyectos de
investigación, proyectos de innovación y\slash{}o proyectos de divulgación que
respondan a las siguientes características: Realizar análisis reales, con
referencias reales, mostrar cambios, beneficios e impacto, mostrar
habilidades sobresalientes en investigación, deducción y\slash{}o experimentación
y saber muy bien en donde se ubica el profesional dentro de la institución
donde prestará el servicio social (Facultad de Estudios Superiores Acatlán
2014).

Para aprobar que un determinado servicio social sea considerado como opción
de titulación, se han establecido los siguientes criterios: que  fomente la
investigación, que desarrolle nuevos materiales o alternativas, que haya
innovación tecnológica, que se promueva el desarrollo de \textit{software},
\textit{interfaces}, objetos de aprendizaje y redes sociales, el análisis
de soluciones recicladas,  y actualizadas, mejorar el clima humano,
organizacional y financiero en diversos ambientes y/o promover el
desarrollo sustentable y el manejo ambiental. Es claro que el proyecto de
Servicio Social puede cubrir tan solo alguno o algunos de estos criterios y
no necesariamente todos (Facultad de Estudios Superiores Acatlán 2014).

El prestador de un servicio social para titularse deberá contar con un plan
de trabajo sobre las actividades a desarrollar, avalado por un asesor
académico y aprobado por el Comité de Programa respectivo. Para ello
existen diferentes proyectos de servicio social aprobados por el
Departamento, ya sea al interior de la Facultad o bien en instituciones
públicas o que tienen convenio con Acatlán.

Una vez que el egresado se ha inscrito a un servicio social bajo la
singularidad de titulación, podrá obtener su grado de licenciatura cuando
haya cubierto los siguientes requisitos: aprobar la totalidad de las
asignaturas de su carrera y cubierto el 100\,\% de créditos que establece el
plan de estudios correspondiente, de igual manera deberá comprobar que ha
cumplido satisfactoriamente con otros requisitos para titulación, como el
haber acreditado los idiomas que marca su plan de estudios, y desde luego,
haber concluido su servicio social,  pero sin liberarlo en la modalidad
tradicional, sino de acuerdo a los lineamientos del reglamento (Véase el Anexo
1). 

\medskip 
\textbf{El proyecto <<Historia ACA>> y el servicio social}

Si bien en otras ocasiones hemos presentado en algún Encuentro de la Red
Nacional de Licenciaturas en Historia y Cuerpos Académicos (RENALHICA) y de
la Red de Especialistas en Docencia, Difusión e Investigación en Enseñanza
de la Historia (REDDIEH) algunos aspectos de este proyecto, cabe aquí
sintetizar algunos elementos relacionados con el mismo. (Véase la Nota 2) 
\newpage

Debemos recordar que los objetivos de este proyecto son: 

\begin{Obs}
\item[$\bullet$] Aportar a la comunidad de historiadores, investigadores y público en general
interesado en el pasado una herramienta de consulta que acortará el tiempo
de investigación que se emplea en la revisión de volúmenes de diferentes
tipos de revistas, en búsqueda de la información sobre el estado de la
cuestión a tratar.
\item[$\bullet$] Tener acercamientos particulares a la historia de las revistas mexicanas
especializadas, sobre alguna época o tema en particular tratado en una
publicación periódica.
\item[$\bullet$] Apoyar a los programas de titulación, de tal manera que los estudiantes que
cuenten con los requisitos estipulados por el Departamento de Servicio
Social  y que decidan optar como modalidad para su titulación por el
Servicio Social Profesional,  o por la tesis o tesina, en caso de que ya
tengan cubierto el requisito del servicio social, puedan incorporarse a
este proyecto colectivo, en el que podrán trabajar en la elaboración de un
catálogo y demostrar las habilidades para ejercer el oficio del
historiador. (Véase la Nota 3)
\end{Obs}

\medskip
De acuerdo a lo señalado, observamos que el servicio social en el proyecto
<<Historia ACA>> está acorde con lo que se establece en  el Reglamento
General del Servicio Social de la UNAM al que ya hemos hecho referencia en
párrafos anteriores. 

A través del proyecto <<Historia ACA>>, se ha venido trabajando en el índice y
el análisis de las publicaciones periódicas del siglo XX, que tratan temas
históricos. Si bien hoy en día existen catálogos electrónicos que es
posible consultar en diferentes medios (CDROM  e Internet), no
hay ningún catálogo completo de esta naturaleza que contenga un índice de
los artículos de las revistas mexicanas especializadas en\linebreak historia, y que
incluya información necesaria que facilite a cualquier investigador o
estudioso de los temas históricos la localización del tema de su interés.
Es por ello que el proyecto se justifica por sí mismo; en la actualidad el
profesional de la historia debe revisar gran cantidad de números de
revistas y\slash{}o diversos índices para localizar artículos y ensayos que den
apoyo a sus investigaciones. Sobra decir que el catálogo de <<Historia ACA>>
tal como lo establece el Reglamento General del Servicio Social en su
primer objetivo, constituye una aportación significativa en beneficio y en
interés de la sociedad.

Con relación al segundo objetivo del proyecto, los estudiantes adscritos a
éste realizan una serie de actividades en las que demuestran  su capacidad
de síntesis, análisis e investigación histórica, así como su capacitad para
realizar crítica histórica. ya que el proyecto no solo radica en la
construcción de una gran base de datos, sino que implica también un
acercamiento significativo a la historia de la comunidad de historiadores
mexicanos y extranjeros que han publicado en revistas especializadas de
temas históricos artículos en torno a sus investigaciones. En este sentido,
la profundización en la historia cultural a través de los textos publicados
y el análisis historiográfico mediante los autores, editores y lectores de
las publicaciones periódicas, vienen a ser parte fundamental de este
proyecto. Por ello, podemos afirmar que se satisface plenamente el segundo
objetivo del Reglamento General del Servicio Social que dice que el
servicio social prestado debe de consolidar la formación académica y la
capacitación profesional del prestador. 

Para cumplir con el último objetivo del Proyecto Historia ACA, de apoyar a
los programas de titulación en la FES Acatlán, se trabaja a través de un
Seminario con sesiones mensuales en el que se presentan los avances de los
prestadores en la elaboración de tablas y gráficas de autores, temáticas y
temporalidades de las revistas que se trabajan (ver Anexo 2); se buscan
soluciones a los problemas que se les presentan a los alumnos en el trabajo
cotidiano de la elaboración de fichas catalográficas, tabulaciones,
gráficas y análisis historiográficos. Asimismo en estas reuniones se
analizan y discuten lecturas de carácter teórico que servirán de fundamento
a sus trabajos de titulación.  Cada alumno cuenta con un asesor, al que
eligen de entre los profesores la planta docente de la carrera de historia.
De esta manera se conforman equipos de trabajo de acuerdo el tipo de
revistas estudiadas al interior del seminario. Por lo anterior el trabajo
en equipo que se lleva a cabo en el Proyecto cumple con la solidaridad con
la comunidad universitaria y al mismo tiempo el trabajo final, producto del
servicio social, será un aporte para la comunidad de historiadores e
interesados en la historia, de ahí que también el trabajo realizado
represente un apoyo, tal y como lo establece el tercer punto del Reglamento
General del Servicio Social Universitario. 

\bigskip
\textbf{Una experiencia}

Cuando se inició el Proyecto Historia ACA, se pensó que en éste todos los
alumnos realizarían servicio social y se titularían, sin embargo, en los
años que llevamos coordinando el proyecto, no todos los estudiantes lo han
hecho así.

A continuación se presentan unas gráficas relacionadas con el proceso
terminal de los alumnos inscritos en el Proyecto Historia ACA, en el cual
han sido aceptados un número de treinta estudiantes, los que componen
nuestro universo. En algunos casos ha habido estudiantes qué únicamente han
cumplido con el requisito del servicio social, para ello se les ha asignado
una revistas o algunos números determinados de alguna publicación para que
elaboren el catálogo correspondiente, sin embargo, la gran mayoría,
participa en el proyecto con el fin de obtener su título profesional.

\bigskip
\begin{center} 
\includegraphics[scale=0.31]{p14-img1.jpg} 
\end{center}

\enlargethispage{1\baselineskip}
En la Gráfica 1 podemos observar el porcentaje de titulación de los
pasantes integrantes del Proyecto: A la fechas se han titulado a través del
trabajo realizado en el Seminario un 43\,\% de los participantes; hay alumnos
que están por terminar su trabajo próximamente, en el curso de los meses
que faltan de este 2014, éstos son el 16.66\,\%; finalmente hay un 40\,\% de
alumnos que aún están en el proceso de elaboración de sus trabajos de
titulación, lo que se espera que logren en su mayor parte en 2015. Cabe
aclarar que este es un proyecto a largo plazo, por lo que al concluir los
trabajos de los pasantes que hoy están inscritos, nuevamente se abrirá para
otra promoción. 

\bigskip
\includegraphics[scale=0.31]{p14-img2.jpg} 

Como ya se apuntó anteriormente, existen varias modalidades de titulación en
la Universidad Nacional, y por consecuencia en la FES Aca\-tlán. En el
proyecto de catalogación de revistas de historia de México siglo XX, según
se observa en la Gráfica 2, la  mayor parte de los alumnos  que se
han titulado, lo ha hecho con un trabajo de tesina, siendo éstos el
36.66\,\%.  En tanto que 10\,\% han optado por titularse con una tesis. La
diferencia entre estos dos tipos de trabajo radica en: que la tesina
implica la realización de un trabajo de análisis historiográficos de los
artículos catalogados, mientras que la tesis supone que, a través del
estudio historiográfico de las colaboraciones en la revista tratada, existe
el planteamiento de una hipótesis que deberá ser comprobada en el trabajo.
Desde luego, un trabajo de tesis implica una mayor complejidad y
dedicación, tanto por parte del alumno como del asesor; en tanto que la
tesina representa un estudio introductorio al catálogo.

En esta misma gráfica se observa que sólo un 33.33\,\% corresponde a la
modalidad de titulación por Servicio Social Profesional, porcentaje
representado solamente por un alumno que obtuvo su título en el año de
2013. Este caso particular se sale de la generalidad porque el pasante era
de la licenciatura en Ciencias Políticas y Administración Pública; dada su
formación a él le interesó el proyecto, ya que la revista Cuadernos
Americanos, publicación periódica que se caracteriza por divulgar textos
relacionados con la actualidad política de América Latina, maneja muchos
temas que tienen que ver con su carrera (Otero~2013). Una vez concluida
la etapa de catalogación y haber cumplido con los requisitos para obtener
el título bajo esta modalidad, el estudiante fue nominado para el premio
Dr.\ Gustavo Baz Prada de Servicio Social; si bien no quedó vencedor, quedó
como finalista, lo que trajo una gran satisfacción al proyecto al
corroborar  que éste ofrece un beneficio social a la sociedad en el ámbito
educativo.

Finalmente, la modalidad de titulación por actividad de investigación fue
elegida por el 10\,\%. Esta forma consiste en realizar un artículo de
investigación, relacionado con las revistas estudiadas, para publicarse en
un libro colectivo producto del proyecto. Para culminar esta actividad
también se requiere de un docente asesor que guíe al estudiante por el
camino de oficio del historiador.

Presentamos una tercera gráfica, relativa
al servicio social al interior del Seminario de revistas, en la que podemos
apreciar que un 36.66\,\% de alumnos se han titulado y realizado el servicio
social en este Proyecto; podemos ver que únicamente el 16.66 cumplieron con
el requisito del servicio social tradicional, y se retiraron sin titularse.
Como ya se mencionó con anterioridad, solamente un 3.33\,\% optó por el
Servicio Social Profesional para titularse.
 

\includegraphics[scale=0.28]{p14-img3.jpg} 

\medskip
En la actualidad tenemos un 40\,\% en proceso de realización del servicio
social tradicional y un 3.33\,\% ha ingresado habiendo cumplido el requisito
de servicio social en una instancia ajena a este proyecto.
%\newpage

\textbf{Consideraciones finales}

De esta manera, podemos señalar que el Proyecto Historia ACA, ha significado
entre los egresados de la carrera de Historia, y en algún caso de la
carrera de Ciencias Políticas, una nueva opción para realizar el Servicio
Social, el cual puede hacerse de tres maneras; la primera, de la manera
tradicional, o sea que el estudiante solamente cubre el requisito
establecido, y, ya posteriormente elige una de las opciones de titulación,
lo que puede realizar fuera del proyecto; una segunda opción es la de hacer
el servicio, con la catalogación de una revista o parte de ella,  y en
seguida, al interior del mismo proyecto la tesina o tesis profesional, que
realiza mediante un estudio y balance historiográficos, siendo esta la
manera por la que optan la mayoría de nuestros alumnos, lo que consideramos
resulta para ello más acorde con la formación académica que debe tener el
futuro historiador. Así mismo, los egresados que quieren seguir una carrera
académica realizando un posgrado, pueden mostrar este tipo de trabajos de
titulación como una mejor carta de presentación para se aceptados en las
instituciones que imparten maestrías y doctorados;  tercera forma, implica
cumplir con el Servicio Social Profesional, donde a la vez se realiza el
Servicio Social y se lleva a cabo la titulación, pensamos que esta
modalidad no se ha generalizado entre los miembros del Seminario, ya que
existen muchos trámites para lograrlo (véase el Anexo 1).

Para los profesores que coordinamos el Proyecto de catalogación y balance
historiográfico de revistas mexicanas, siglo XX, así como también para los
que son asesores en el mismo, la participación en este grupo académico ha
sido muy enriquecedora, ya que se está llevando al cabo un trabajo
novedoso, útil y de servicio a la comunidad.
%\newpage

\bigskip
\textbf{Notas:}

\begin{Obs}
\item[1.-] En la estructura de la FES Acatlán, existen los Comités de
Programa, como órganos auxiliares del Consejo Técnico, ya que la facultad
cuenta con dieciséis licenciaturas de diferentes áreas de
conocimiento.
\item[2.-]  <<Una opción de titulación>> análisis historiográfico de
publicaciones periódicas especializadas en historia, ponencia presentada en
el\linebreak \textit{II Encuentro Nacional de Licenciaturas de Historia de
Instituciones Públicas de Educación Superior}. Benemérita Universidad
Autónoma de Puebla, 29, 30 y 31 de Octubre, 2007.
\newpage

<<El giro historiográfico en la enseñanza de la historia: la red y las
revistas especializadas>>, ponencia presentada en \textit{VII
Encuentro de la RENALIHCA y I Encuentro Iberoamericano de Licenciaturas en
Historia,} Universidad Autónoma de Chiapas. San Cristóbal de las
Casas, Chis., 29--31 Octubre, 2012.


<<Historiando historias. Publicaciones periódicas: nueva vertiente
en la enseñanza de la historia>>, ponencia presentada y publicada en
\textit{Memorias: }\textit{III Encuentro Nacional de Docencia, Difusión y
Enseñanza de la Historia, I Encuentro internacional de Enseñanza de la
Historia}, México, Universidad Pedagógica Nacional, 2012, ISBN
978-607-413-152-9.
\item[3.-]  El informe de Servicio Social Profesional fue aprobado  por el
Consejo Técnico de la Facultad el doce de marzo de 1991, el que permite que
al tiempo que se realiza el Servicio se tiene la posibilidad de titularse.
Para inscribirse en esta opción es requisito no haber realizado el servicio
social con anterioridad, tener cubierto el 100\,\% de créditos y realizar 960
horas de servicio social,  en un periodo mínimo de seis meses y máximo de
dos años.
\end{Obs}
\newpage

\textbf{Referencias} 


\medskip
\begin{sloppypar}
Dirección General de Estudios de Legislación Universitaria (2014),
<<Reglamento general del servicio social de Universidad Nacional Autónoma de
México>>. Recuperado de \url{http://www.dgelu.unam.mx/nad3-3.htm}.  
\end{sloppypar}

\begin{sloppypar}
Dirección General de Servicios Educativos (2014), <<Premios y
reconocimientos>> Premio al Servicio Social Dr.\ Gustavo Baz Prada,
Universidad Nacional Autónoma de México, 1985. Recuperado de
\href{http://www.dgose.unam.mx/dgose/pregbp.html}{\textstyleInternetLink{www.dgose.unam.mx/dgose/pregbp.html}}.
\end{sloppypar}

Facultad de Estudios Superiores Acatlán (2014), <<Opciones de titulación en
la FES Acatlán. Recuperado de
\url{http://www.acatlan.unam.mx/licenciaturas/212/}.  


Otero Dávila, David Alejandro (2011), \textit{Catalogación y balance
historiográfico de la revista en temas sociopolíticos
}. Cuadernos Americanos periodo 1993-1995, Trabajo
de Servicio Social, UNAM, FES Acatlán.

\begin{sloppypar}
Sozzan Carlos (2007) <<Antecedentes del servicio social universitario>> en
\textit{Debates Universitarios}. Recuperado el 7 de junio de
\url{http://debatesuniversitarios.blogspot.mx/2007/04/antecedentes-del-servicio-social.html},
\end{sloppypar}

\begin{sloppypar}
Universia (2014), <<Antecedentes del Servicio Social>>. Recuperado de
\url{http://noticias.universia.net.mx/tiempo-libre/noticia/2004/12/27/1133113/antecedentes-servicio-social.html}
\end{sloppypar}
\newpage

\begin{footnotesize}
\textbf{Anexo 1}

\textbf{Procedimiento a seguir para la modalidad de Servicio Social
Titulación en la FES Acatlán}

\begin{Obs}
\item[1.] El alumno solicita en el cubículo 2 del COESI (edificio A8, planta baja),
con el responsable de Servicio Social, orientación sobre los procedimientos
y trámites para la modalidad. 

\item[2.] El responsable envía vía correo electrónico al alumno: procedimientos a
seguir, guía de actividades y guía de trabajo final, así como información
sobre el tiempo y requisitos previos para optar por la modalidad. 

\item[3.] El alumno debe revisar los documentos que se le enviaron y comparar la
modalidad con las otras opciones de titulación, si decide por servicio
social resolverá sus dudas sobre el procedimiento con el responsable de
servicio social. 

\item[4.] El alumno revisará la página www.serviciosocial.unam.mx, para elegir la
institución en donde realizará su servicio social. 

\item[5.] El alumno elaborará la Guía de plan de actividades, la cual presentará
con el jefe inmediato de la institución en donde realizará el servicio
social, así como con su asesor académico para revisión y firmas de visto
bueno. 

\item[6.] Elaborar una carta de solicitud, dirigida al Comité de Programa de su
licenciatura, en donde pida que se revise y se apruebe su proyecto (plan de
actividades) y en la cual tendrá que incluir los siguientes datos: 

Nombre del proyecto.\\
Nombre de la Institución donde realizará el servicio.\\
Periodo sugerido para la realización del mismo.\\ 
Nombre completo del alumno.\\
Carrera.\\ 
Número de cuenta.\\
Teléfono.\\ 
Correo Electrónico.

\item[7.] El alumno entregará al responsable los siguientes documentos en fólder: 

Carta de solicitud firmada por el alumno (original).\\ 
Guía de actividades acompañada del Vo.\,Bo, con las firmas del jefe
inmediato, el asesor académico y la del alumno.\\ 
Historial original con el 100\,\% de créditos.(original)\\ 
2 fotografías tamaño infantil.\\ 
Si el plan de estudios lo solicita, copia de la o las constancias de
idiomas. 

\smallskip
NOTA: Estos documentos se entregarán al comité engrapados y solicitamos que
por favor no alteren el orden del expediente.
\item[8.] El responsable de Servi\-cio So\-cial re\-gis\-tra en for\-ma\-to de excel: nom\-bre
del alum\-no, carrera, número de cuenta, nombre del proyecto, institución
donde realizará el servicio y asigna la homoclave para la modalidad de
titulación. 

\item[9.] En el proyecto el responsable de Servicio Social, anotará en la parte
final del mismo la homoclave, firmará y sellará el documento como visto
bueno. Sólo con estos últimos datos el Comité de Programa puede estar
seguro que el trabajo ya pasó por el procedimiento con servicio social. 
\item[10.] Por parte del responsable, se realizará el envío a la Jefatura de
Programa de la licenciatura correspondiente, de los documentos del alumno
vía oficio para su revisión y aprobación. 
\item[11.] Una vez que el Comité de Programa ha revisado y aprobado los documentos
y el proyecto, la Jefatura de programa deberá enviar por oficio dirigido al
responsable de servicio social, los documentos del alumno, junto con una
copia del acta de comité donde aprueban el trabajo en un plazo no mayor a
30 días hábiles. El secretario técnico del programa hace del conocimiento
de la minuta de aprobación del comité de programa, junto con la solicitud
de la modalidad de titulación a la unidad de administración escolar, en
tanto el responsable de servicio social, notifica al alumno de la
resolución tomada. 
\item[12.] El alumno registra en su Programa de Carrera la modalidad a titulación y
obtendrá: 

Formatos oficiales de registro en 4 tantos (alumno, asesor, escolares y
jefatura del programa). 
\item[13.] El alumno entregará los documentos en el cubículo 2 del COESI, con el
responsable de Servicio Social, para su registro y control. 
\item[14.] El alumno tendrá que acudir a realizar su registro oficial. Llevando
copia de alguno de los formatos oficiales de registro de la modalidad y
recibirá los siguientes documentos: 

Formato de inscripción.\\ 
Hoja de registro de firmas para informe de avances.\\  
Acuse de carta de aceptación.\\ 
Guía de trabajo final. 
\item[15.] El alumno deberá construir durante su servicio social, la guía de
trabajo final y entregarla al término junto con sus seis informes
parciales, además de la hoja de firmas, avalada por su asesor y su jefe
inmediato, así como su formato de inscripción. El proyecto concluido con
base en la guía, será el documento con el cual el alumno se titulará. 
\item[16.] El Área de Servicio Social entregará en un plazo no mayor a 45 días
hábiles la carta original de liberación a Servicios Escolares y una copia
al alumno. 
\item[17.] El alumno deberá realizar en su Jefatura de Programa los trámites
correspondientes para la notificación de trabajo concluido a la Unidad de
Servicios Escolares.

Véase: Opciones de titulación en la FES Acatlán en:\\
\url{http://www.acatlan.unam.mx/licenciaturas/212/} 
\end{Obs}

\medskip
\textbf{Anexo 2}

Revistas que se catalogan y\slash{}o estudian en el Proyecto Historia ACA

\textit{Arqueología Mexicana}\\
\textit{Artes de México}\\
\textit{Cuadernos americanos}\\
\textit{Cuicuilco}\\
\textit{Estudios de cultura maya}\\
\textit{Estudios de cultura náhuatl}\\
\textit{Estudios de historia moderna y contemporánea}\\
\textit{Estudios de historia novohispana}\\
\textit{Frenia}\\
\textit{Fuentes humanísticas}\\
\textit{Letras libres}\\
\textit{Luna córnea}\\
\textit{Historia mexicana}\\
\textit{Historias}\\
\textit{Historia y grafía}\\
\textit{I Cuaderni Storici}\\
\textit{Istor}\\
\textit{Quipu}\\
\textit{Revista de ciencias sociales y humanidades}\\
\textit{Saber ver}\\
\textit{Secuencia}\\
\textit{Vuelta}
\end{footnotesize}