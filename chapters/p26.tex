%\documentclass{article}
%\usepackage{amsmath,amssymb,amsfonts}
%\usepackage{fontspec}
%\usepackage{xunicode}
%\usepackage{xltxtra}
%\usepackage{polyglossia}
%\setdefaultlanguage{spanish}
%\usepackage{color}
%\usepackage{array}
%\usepackage{hhline}
%\usepackage{hyperref}
%\hypersetup{colorlinks=true, linkcolor=blue, citecolor=blue, filecolor=blue, urlcolor=blue}
%% Text styles
%\newcommand\textstyleEndnoteAnchor[1]{\textsuperscript{#1}}
%\newtheorem{theorem}{Theorem}
%\clearpage\setcounter{page}{1}

\thispagestyle{empty}
\phantomsection{}
\addcontentsline{toc}{chapter}{Avances y problemáticas detectados en el\newline nuevo Plan 
de estudios (2011) de la\newline Licenciatura en Historia de la Universidad\newline Autónoma de
Zacatecas\newline $\diamond$
\normalfont\textit{Lidia Medina Lozano, José Luis Raigoza Quiñónez\newline y 
Luis Román Gutiérrez}}
{\centering {\scshape \large Avances y problemáticas detectados en el nuevo Plan 
de estudios (2011) de la Licenciatura en Historia de la Universidad Autónoma de
Zacatecas}\par}
\markboth{la formación del historiador}{avances y problemáticas}
\setcounter{footnote}{0}

\bigskip
\begin{center}
{\bfseries Lidia Medina Lozano\newline
José Luis Raigoza Quiñónez\newline
Luis Román Gutiérrez}\newline
{\itshape Universidad Autónoma de Zacatecas}
\end{center}

\bigskip
\textbf{Resumen}

La licenciatura en Historia de la Universidad Autónoma de Zacatecas, 
tuvo su origen en 1987 como Escuela de Humanidades con tres áreas 
terminales: Historia, Letras y Filosofía. Comenzó con un plan rígido 
modular que en pocos años se separó por licenciaturas tomando cada una 
su propio rumbo disciplinar. En todos estos años, el plan de estudios 
de la Licenciatura en Historia se ha reformado en varias ocasiones.


Primeramente, en 1998 el programa quedó evaluado en el nivel 2 de los 
CIEES. Posteriormente se acreditó por los mismos CIEES en 2005, después 
de una reforma curricular al plan de estudios de 1999 que se implementó 
en 2004 bajo las características de una educación “abierta, flexible, 
integral y polivalente”. Se re-acreditó en 2009 por el mismo organismo 
y dos años más tarde, nuevamente se reformó el plan de estudios 
aumentando las materias optativas y pasando de tres a seis las áreas 
terminales sin perder las características del anterior. Con el mismo 
plan de estudios, en enero pasado abrió la Licenciatura a distancia con 
el objetivo inicial de capacitar cronistas a través de la Asociación 
Nacional de Cronistas. En estos momentos está solicitada la evaluación 
por el Consejo para la Acreditación de Programas Educativos en 
Humanidades (COAPEHUM) que esperamos para este mismo año. En este 
trabajo pretendemos presentar los cambios curriculares del plan 2004 al 
de 2011 así como sus ventajas y problemáticas que se han detectado a 
tres años de su implementación.\\ 
\textbf{Palabras clave:} Currícula, titulación, acreditación, flexibilidad,
créditos.


\bigskip
\textbf{Abstract}

The bachelor degree in history of the Autonomous University of 
Zacatecas had its origin in 1987 like School of Humanities with three 
terminal areas: History, Letters, and Philosophy. It began with a 
modular rigid plan that in few years each one took its own course 
separated by bachelors degrees. In all these years, the curriculum of 
the Bachelor Degree in History has reformed in several occasions.

Firstly, in 1998 the bachelor was evaluated in level 2 of the CIEES. 
Later it was acredited by the same CIEES in 2005, after a curricular 
reform that was implemented in 2004 under the characteristics of an 
“opened, flexible, integral and multipurpose education”.  The bachelor 
was reacredited in 2009 by the same organism, and two years later the 
curriculum reformed again increasing the optative matters and going 
from three to six the terminal areas without losing the characteristics 
of the previous one. With the same curriculum, in past January it 
opened the remote bachelor degree with the initial objective to enable 
chroniclers through the National Association of Chroniclers. At the 
present, was asked the evaluation by the Council for the Accreditation 
of Educative Programs in Humanidades (COAPEHUM) that we hoped for this 
same year. In this work we try to present the changes of curriculum 
2004 to the one of 2011 as well as their problems and advantages that 
have been detected to three years of their implementation.\\
\textbf{Keywords:} Curriculum, university degree,
accreditation, flexibility, credits.

\bigskip 
La licenciatura en Historia de la Universidad Autónoma de 
Zacatecas, tuvo su origen en 1987 como Escuela de Humanidades con tres 
áreas terminales: Historia, Letras y Filosofía. Comenzó con un plan 
rígido modular que en pocos años se separó por licenciaturas tomando 
cada una su propio rumbo disciplinar. En todos estos años, el plan de 
estudios de la Licenciatura en Historia se ha reformado en varias 
ocasiones.

Primeramente, en 1998 el programa quedó evaluado en el nivel 2 de los 
CIEES. Posteriormente se acreditó por los mismos CIEES en 2005, después 
de una reforma curricular al plan de estudios de 1999 que se implementó 
en 2004 bajo las características de una educación “abierta, flexible, 
integral y polivalente”. Se re-acreditó en 2009 por el mismo organismo 
y dos años más tarde, nuevamente se reformó el plan de estudios 
aumentando las materias optativas y pasando de tres a seis las áreas 
terminales sin perder las características del anterior. Con el mismo 
plan de estudios, en enero pasado abrió la Licenciatura a distancia con 
el objetivo inicial de capacitar cronistas a través de la Asociación 
Nacional de Cronistas, el resultado ha sido una fuerte demanda a nivel 
nacional y el sur de Estados Unidos.  En estos momentos está solicitada 
la evaluación por el Consejo para la Acreditación de Programas 
Educativos en Humanidades (COAPEHUM) que esperamos para este mismo año. 
En este trabajo pretendemos presentar los cambios curriculares del plan 
2004 al de 2011 así como sus ventajas y problemáticas que se han 
detectado a tres años de su implementación. 


Como resultado del Congreso General de Reforma, los planes de estudio 
de la Licenciatura en historia han sido modificados en dos ocasiones 
para responder a las nuevas estrategias didácticas y para implementar 
un plan de estudios con las características de abierto, flexible, 
integral polivalente y pertinente. El primer plan de estudios que fue 
diseñado fue el del año 1999 el cual contenía 54 asignaturas divididas 
en: área informativa con las siguientes materias: dos Introducciones; 
cinco Historias de la Historia; seis Historias universales; seis 
Historias de México; cinco Historias generales; tres Historias del 
arte.

En el área formativa-productiva; cuatro materias sobre Teoría y 
Filosofía de la Historia; tres metodologías; Nueve Talleres; dos 
seminarios; Siete instrumentales y sólo dos materias optativas. De la 
currícula anterior se llevaban de todas, todas las materias sin opción 
y con una sola formación sin ejes de especialización. 


Al innovar nuestros planes de estudio acordes con una currícula 
flexible y polivalente, que diera respuesta a las exigencias 
gubernamentales y al mercado laboral donde nos exigían profesionales 
más aptos para el desempeño de la labor histórica con habilidades 
extras, por un lado y por el otro que diera satisfacción a las 
inquietudes de formación del propio alumnado que quería formarse con 
habilidades personales para el desempeño de su quehacer de manera 
óptima.

Así después de celebrar sendos congresos de egresados y empleadores, y 
en concordancia a las políticas educativas, nuevamente los docentes nos 
reunimos en comisiones para analizar las nuevas propuestas y ofertar 
una currícula variada y con tres ejes de especialización. Así de esta 
manera se conformó el plan de estudios 2004 con las siguientes 
características y contenidos: abierto, flexible, integral, polivalente 
y pertinente. Que ofertara un mayor número de materias optativas las 
cuales debieron ser optativas específicas de cada eje de 
especialización: Docencia, Extensión y Difusión. Se trabajó por 14 
meses tras los cuales con discusiones y puestas de acuerdo se terminó 
el Plan 2004 que contenía 54 asignaturas o llamadas UDI,s ( unidades 
didácticas integradoras): se comenzó con un Tronco Común de las 
Humanidades donde se pretendía que los programas académicos de Letras, 
Filosofía y Antropología, participaran con el concurso de profesores y 
alumnos donde éstos convivieran con sus pares de dichos programas a fin 
de enriquecer su conocimiento de esas disciplinas y terminaran por 
definir su vocación.


No hubo eco en nuestras intenciones de que nuestros alumnos tomaran las 
introducciones a Letras, Filosofía y Antropología, pues esos programas 
no estaban en la misma sintonía, continuaban teniendo sus programas 
rígidos y no permitieron que nuestros alumnos tomaran esas UDI´s y 
solamente se logró que los profesores de dichos programas impartieran 
las introducciones referidas en nuestras propias instalaciones.

El plan entonces además del tronco común incluía cinco UDI´s y a partir 
del segundo semestre nuestra oferta educativa incluía treinta y tres 
materias optativas; quince disciplinares; nueve básicas y diez del 
programa académico común. Sin tomar en cuenta Estancia profesional y 
Servicio Social; Seminario de Elaboración de Proyectos; Teorías de la 
Comunicación y Liderazgo y desarrollo organizacional que se sugerían 
para los tres ejes, el número de asignaturas optativas disciplinares 
por eje eran cinco, muy ligadas a la especialización.  

Además se incluyeron ejes transversales que incluyeron: trece materias 
en el eje de derechos humanos, género y democracia; sobre Identidad y 
valores se contemplaron veintiún UDI´s; Ecología, globalización, y 
desarrollo sustentable se incluyeron trece materias, y finalmente un 
eje integrador compuesto por dos materias sobre la práctica 
profesional. Es oportuno señalar que varias materias compartieron dos o 
más ejes tanto verticales como transversales. 

Así esta oferta educativa estuvo conformada por: veintiuna materias 
obligatorias y treinta y tres optativas siendo un 57\,\% de materias 
optativas lo cual ya era un avance sustantivo para la oferta educativa. 
Esta situación hacía necesaria la tutoría como acompañamiento y 
asesoría académica al alumno para la elección tanto de su eje de 
especialización como por las materias optativas de su preferencia, 
haciendo atractivo el plan de estudios con miras al abatimiento de la 
deserción escolar y la eficiencia terminal pero sobre todo una mejor 
preparación del alumnado para enfrentarse al mercado laboral. El 
trabajo del tutor es de suma importancia en el acompañamiento de los 
estudiantes. Desde que el estudiante ingresa a nuestro programa se le 
da un acercamiento general del Plan de estudios, una vez que es 
aceptado se le nombra su Tutor el cual le ofrece toda la información 
del Plan de estudios, incluyendo los respectivos Reglamentos internos 
(el Reglamento General del Programa, el Reglamento de  Titulación y el 
Reglamento de Servicio Social)


No obstante lo anterior y que con estas acciones se logró obtener el 
nivel 1 de los CIIES, continuaron los congresos de egresados y 
empleadores con el objetivo de medir la capacidad de nuestros egresados 
y su inserción al mercado laboral. Esto dio como resultado que nuestros 
egresados en su mayoría se estaban empleando como docentes y 
extensionistas y en menor número como investigadores.

 
Con miras a continuar dentro de un programa académico de excelencia, 
nuevamente se hizo necesario un nuevo ejercicio de modificar la 
currícula a fin de dar satisfacción tanto a egresados como a 
empleadores. Así desde el 2010 nuevamente y en comisiones trabajamos 
por once meses y elaboramos un nuevo plan de estudios que comenzó a 
operar desde el 2011.

El Programa de Licenciatura de la Unidad Académica de Historia 
consideró desde el 2004 actualizar y dar pertinencia a sus planes y 
programas de estudios. Desde mayo del 2003 la planta docente del PE de 
la Unidad Académica de Historia, bajo la coordinación de la Responsable 
del Departamento de Docencia procedió a realizar el nuevo Plan de 
estudios, bajo una serie de lineamientos: Se hizo un análisis 
comparativo de veintiún planes de estudios de Licenciatura en Historia  
de universidades nacionales, con el fin de ubicar la situación del plan 
vigente y proponer por lo tanto los puntos nuevos a la Reforma del Plan 
de Estudios 2004. Se convocó a un encuentro con egresados y empleadores 
con el objetivo de establecer lazos y comunicación permanente entre los 
egresados y la Universidad, conocer las demandas actuales propias del 
ejercicio profesional en que se desenvuelven los historiadores; 
observar las tendencias para el perfil profesional de los historiadores 
en materia de conocimientos, valores y competencias en el desempeño 
laboral de calidad; obtener información acerca las aspiraciones y 
expectativas sobre titulación, educación continua y acciones de 
crecimiento profesional y personal de los egresados; recuperar la 
información respecto a la formación recibida en esta Institución; las 
áreas de conocimiento que es necesario incluir o fortalecer y las 
recomendaciones dirigidas a la modificación y requerimiento del 
proyecto curricular. Este mismo ejercicio se volvió hacer en el año del 
2010 para realizar la siguiente Reforma al Plan de Estudios.


El reto ahora fue una reforma curricular en el año 2011 que estuviera a la
altura de las nuevas exigencias del mundo actual, de la sociedad y para los
retos del futuro, es decir, una educación abierta, flexible, integral y
polivalente, con la finalidad “de entregar a la sociedad nuevos
profesionales en el campo de las humanidades y de la educación”.\footnote{
Antonio González Barroso, María del Refugio Magallanes Delgado y Ángel
Román Gutiérrez (Cuerpo Académico Enseñanza y Difusión de la Historia).
\textit{La reforma curricular de la licenciatura en historia de la UAZ: un
esfuerzo compartido, en el VI Encuentro de la Red Nacional de
Licenciaturas en Historia y sus Cuerpos Académicos.}}

La estructura curricular consistente en cuatro áreas (básica, 
disciplinar, programa académico común y optativa); ejes terminales de 
especialización (docencia, investigación, difusión, organización y 
administración de acervos, historia del arte e historiografía), ejes 
transversales (género, democracia, derechos humanos, ecología, 
globalización y desarrollo sustentable) y el eje integrador donde el 
servicio social está integrado a la currícula; se promueve el liderazgo 
y se presta especial atención a la comunicación, la identidad y los 
valores. Lo anterior permite que cada estudiante, junto con su tutor, 
diseñe la currícula que más satisfaga sus intereses, potencialidades y 
expectativas.\footnote{El área de Básicas, según el cuadernillo once, 
forma al universitario en aspectos teóricos, epistemológicos y 
metodológicos propios del área de conocimiento e implica una formación 
tanto en los conceptos teóricos de las ciencias básicas de cada 
profesión, como en la génesis, la lógica de construcción y el uso 
social de ese conocimiento; asimismo implica el análisis de la lógica 
del poder que hace posible tal conocimiento, es decir de cómo se piensa 
y se estructuran los conocimientos (teóricos) en función de 
determinadas exigencias del poder.  Programa académico común, consiste 
en formar a los estudiantes sobre conocimientos, saberes y habilidades 
necesarios para su formación superior y su futuro desempeño 
profesional; representan el bagaje cultural e instrumental demandado 
por el mundo contemporáneo, es el punto de entrada a la educación 
abierta, integral, flexible, polivalente y pertinente. Es una formación 
obligatoria para todos los estudiantes inscritos en la Universidad, y 
puede cursarse en cualquier Unidad Académica de la institución, aunque 
preferentemente en el Área Académica correspondiente. Optativas, 
el alumno junto con el tutor elegirá del menú, aquellas UDIs para 
profundizar en el conocimiento y así tener la opción de egresar como 
licenciado en historia o bien en algunas de las salidas colaterales.\\   
Disciplinar, esta fase comprende un conjunto integrado de disciplinas 
ligadas estrechamente a la realización de la profesión y a los saberes, 
técnicas o artes que le corresponden; este conjunto de UDIs 
relacionadas con un campo profesional específico es propiamente el 
programa académico de licenciatura que se imparte en una Unidad 
Académica. Esta fase prepara al universitario para su ejercicio 
profesional-laboral al insertarse y/o vincularse con la sociedad, 
incluye tanto una formación teórica especializada y a profundidad, como 
en habilidades, actitudes y valores inherentes a la profesión; esta 
formación fortalece la conceptualización y el acercamiento a la 
práctica profesional e implica la incorporación ágil, oportuna y 
significativa de los avances de la ciencia y la tecnología en el 
currículo del programa académico correspondiente.}  

En el nuevo Plan de estudios  se pone en práctica un modelo de servicio
social, de estancias y prácticas universitarias acorde al modelo académico
“UAZ-Siglo XXI”,  incorporando el Servicio Social a la Currícula, generando
instrumentos de seguimiento puntual, bajo proyectos claros de las unidades
receptoras que incidan con los intereses académicos de los alumnos y de la
sociedad de acuerdo a los ejes en que fue formado el estudiante.  El
servicio Social tiene el carácter atribuido el Departamento de Servicio
Social y de Vinculación de la Universidad Autónoma de Zacatecas, ya que la
UAZ debe contribuir a socializar los beneficios derivados de la ciencia, la
cultura, el arte y la tecnología, con el fin de impulsar el desarrollo
económico y social de nuestro entorno local, regional y nacional con un
contexto de globalización. El objetivo es promover el acercamiento real del
estudiante de historia con la sociedad, consolidar la formación académica,
desarrollar sus valores y actitudes y favorecer su inserción al mercado de
trabajo. La realización del servicio social por parte de los alumnos de
nuestro PE pretende que sea debidamente aprovechado y recupere su
importancia académica a través de labores llevadas a cabo por los alumnos
en contacto con una realidad social,  bajo proyectos claros de las unidades
receptoras que incidan con los intereses académicos de los alumnos y de la
sociedad de acuerdo a los ejes en que fue formado el estudiante.  Las metas
principales que se han logrado realizar en el servicio social desde el
2004, ha sido realizar el reglamento de incorporación del Servicio Social a
la Currícula bajo el modelo educativo basado en “competencias”,
organización de Foros anuales sobre experiencia de los estudiantes en el
Servicio Social, llevar a cabo un registro sobre la inserción, seguimiento
y evaluación de servicio social del PE. Conformar y actualizar
permanentemente convenios con instancias receptoras y proponer anualmente
un plan de prácticas profesionales con los sectores sociales y productivos.

El modelo curricular comprende una estructura abocada al impulso de la
educación abierta, flexible, integral, polivalente y pertinente de acuerdo
a la exigencia del mundo actual, de la sociedad y para los retos del
futuro. 


a) Se propone el carácter abierto con el fin de romper con el modo cerrado
mediante la libertad de cátedra, con un fuerte compromiso de aprendizaje
con el educando para la integración del conocimiento y la transformación de
los docentes de consumidores pasivos a líderes constructivos del currículo.
 


b) El carácter integral rebasa la visión tradicional centrada solamente en
lo temático o conceptual, con el fin de impulsar perfiles en todas sus
facetas: intelectual, afectiva, relacional, psicomotora y axiológica. El
fin es de generar en los jóvenes una actitud abierta al cambio permanente. 


 
c) El carácter flexible impulsa la conexión entre áreas del conocimiento
elevando la versatilidad e interdisciplinariedad, fomentando el
autoaprendizaje y la movilidad del estudiante en planos intra e
interinstitucionales.

 
d) El currículo adquiere un carácter polivalente capaz de responder a la
sociedad cambiante y a la imprevisibilidad del futuro, integrando
coherentemente los contenidos de las distintas áreas del conocimiento
científico y humanístico; así como la virtud para formar individuos con
capacidad para adaptarse crítica y creativamente a cualquier contexto y
circunstancia, demostrando habilidades para la toma de decisiones oportuna
y decididamente y para la solución de problemas en los ámbitos
sociocultural y laboral. 

 
e) El carácter de pertinencia corresponde a los mecanismos de vinculación
con los sectores sociales, productivos y de servicios, para generar nuevos
ambientes de aprendizaje para la formación profesional del alumnado. 

 
La Reforma al Plan de Estudios considera que se debe de formar a los
estudiantes autónomos a partir de una educación centrada en el aprendizaje,
por lo que el tiempo en el aula se reduce para favorecer las actividades en
bibliotecas, archivos, museos, recorridos de campo, etcétera.  Al concebir
que la formación de los estudiantes debe ser integral, se promueve la
asistencia a eventos tales como: conferencias, exposiciones, cine, debates,
presentación de libros, obras de teatro, etcétera. Asimismo, se le
introduce al ámbito de su profesión y se promueve a proponer proyectos en
las diferentes esferas de su competencia. Lamentablemente no existe un
programa específico que regule una actividad permanente de cursos,
talleres, diplomados, etc. Dirigidos a los egresados y a los sectores de la
sociedad, de manera sistemática. 

Entre los beneficios que propone el nuevo plan de estudios es la educación
continua, por medio de los cursos optativos; sin embargo a tres años de
haberse implementado la oferta dirigida a los egresados no ha tenido ningún
impacto positivo, puesto que los alumnos una vez que egresan no vuelven al
programa para su actualización profesional,  a pesar de la difusión que al
menos por un tiempo se logró establecer por medio de la Comisión de
seguimiento de egresados. Una de las oportunidades que el plan de estudio
ofrece a los estudiantes que no cuentan con las condiciones  para cubrir
los créditos de la licenciatura, pueden optar por la salida lateral de
Técnico Superior Universitario (flexibilidad en tiempo y contenido). Pero
igualmente el estudiante que no logra terminar sus estudios, tampoco tiene
interés por recuperar el documento del TSU. En este sentido, ha sido
necesario tener información permanente del Departamento Escolar del
Programa que dispone de una encuesta-cuestionario para aquellos alumnos que
se dan de baja en la carrera, la cuál nos ayuda a discutir en la academia
de profesores los problemas detectados del estudiantado, siendo una
constante la falta de vocación o prioridad por otra carrera, puntos
detectados desde los primeros filtros para la selección de los alumnos.


Desde el 2009 a la fecha  se cuenta con el Programa de Movilidad estudiantil
con Universidades nacionales y extranjeras, sin embargo la falta de
información  oportuna por parte de la Institución y los pocos convenios que
existen con las Universidades extranjeras, y las pocas becas que se
ofertan, han disminuido la oportunidad de que los estudiantes se inscriban
en el proceso de movilidad o intercambio académico en universidades tanto
nacionales como extranjeras.  En enero del 2010 inició el proceso de
Reforma integral a su Plan de Estudios, siguiendo un análisis sistémico. 


Las buenas intenciones del Plan curricular era ofrecer diplomados, 
talleres, seminarios conferencias, presentaciones de libros; y que sus 
profesores asistieran a congresos, coloquios y otro tipo de eventos 
tanto a nivel nacional como internacional. Lo anterior se traduciría en 
la mejor calidad de la enseñanza y en una constante producción 
investigativa. Sin embargo la falta de Cuerpos Académicos en el 
Programa y proyectos en común de los profesores a derivado en una 
diversificación de los temas de interés que no se han visto reflejados 
en el enriquecimiento curricular del programa. La falta de 
consolidación está acompañada de algunos problemas: existencia de un 
solo cuerpo académico registrado en el programa: “Enseñanza y difusión 
de la historia”.\footnote{Antonio González Barroso, María del Refugio 
Magallanes Delgado y Ángel Román Gutiérrez \textit{(Cuerpo Académico 
Enseñanza y Difusión de la Historia). La Reforma Curricular de la 
Licenciatura en Historia de la UAZ: Un esfuerzo compartido, en  el VI 
Encuentro de la Red Nacional de Licenciaturas en Historia y sus Cuerpos 
Académicos.}} 

Para elaborar este plan, que contempla las siguientes asignaturas: 
ciento catorce UDI´s de las cuales sólo veinte son obligatorias y 
noventa y cuatro son optativas arrojando un porcentaje de la siguiente 
manera: obligatorias: el 18\,\% y optativas el 82\,\%. Ello obedece a 
la diversificación de los ejes terminales de especialización de la 
carrera de Licenciatura en historia, distribuidas de la siguiente 
manera: Área básica con cinco UDI´s; la disciplinar con diez materias; 
el programa académico común con tres asignaturas; el eje integrador 
está compuesto de dos materias; y los ejes de especialización se 
diversificaron de la siguiente manera: Eje de Difusión; Docencia; 
Investigación; Organización y Administración de Acervos; Historia del 
Arte y Historiografía. Cada uno de estos ejes de especialización debe 
contener seis materias optativas disciplinares de cada eje para 
complementar su preparación.\footnote{\textbf{DOCENTE}\textit{\ \ 
Conocimiento }\ \ \ \ Manejará los paradigmas contemporáneos de la 
educación y su derivación en la práctica docente. Será capaz de 
diseñar, implementar y evaluar programas didácticos de la historia, 
para los distintos niveles educativos, ámbitos culturales e 
institucionales. \textit{\ \ Competencias }\par \ \ Contará con la 
habilidad para elaborar y aplicar material didáctico tipográfico y 
electrónico propicio para la enseñanza y difusión de la historia. 
Habilidad para construir modelos de enseñanza-aprendizaje, incluida su 
instrumentación y evaluación. Desarrollar programas de formación y 
capacitación docente para la enseñanza de la historia.  \par \textbf{\ 
\ INVESTIGADOR }\par \textit{\ \ Conocimiento }\par \ \ Demostrará 
dominio del método científico y de la integración  de herramientas 
teórico-metodológicas e instrumentales para la realización de tareas de 
investigación que le permita esclarecer, razonar y explicar los 
diversos procesos históricos. A fin de  descubrir,  reinterpretar, 
reconstruir y corregir la representación de los acontecimientos 
históricos. Será capaz de aportar interpretaciones conceptuales 
respecto a la construcción histórica y su vinculación con otras 
disciplinas en torno a la identidad y de proponer elementos 
teórico-metodológicos para el rescate, preservación y fortalecimiento 
de los procesos identitarios en los diversos planos: individual, 
colectivo, cultural, etc. Será capaz de proponer, diseñar, implementar 
y evaluar proyectos de investigación, abocados a enriquecer el 
conocimiento histórico y  solucionar problemas desde la perspectiva 
disciplinaria.  \par \textit{\ \ Competencias }\par \ \ Habilidades 
para buscar, seleccionar, procesar, explicar y representar –a través de 
la escritura– información sobre el pasado.  Comprobar la veracidad de 
la información, así como cotejar las distintas versiones sobre un mismo 
hecho. Identificar los diferentes lenguajes del discurso histórico. 
Diestro para el manejo de los diferentes tipos de acervos. Capacidad 
para la lectura y comprensión de manuscritos e impresos, imágenes, 
fotos y en movimiento, arquitectonografía, etcétera.  Capacidad para 
formular, aplicar e interpretar cuestionarios, encuestas, entrevistas y 
bitácoras para el trabajo de campo. Dominio de nuevas tecnologías para 
la búsqueda, procesamiento y difusión de la información.  \par 
\textbf{\ \ DIVULGADOR} \par \textit{\ \ Conocimiento }\par \ \ Estará 
al corriente de las teorías de la comunicación y de su instrumentación. 
Sabrá discernir el impacto social de los diferentes medios de 
información. Será capaz de diseñar e implementar recursos idóneos para 
la difusión de la historia. \par \textit{\ \ Competencias} \par \ \ 
Contará con la habilidad para manejar todo tipo de medios y recursos 
informáticos. Habilidad para gestionar espacios de información. 
Habilidad para despertar el interés en todo tipo de auditorio. 
Habilidad para utilizar innovadoramente viejas tecnologías y recursos 
limitados en la composición y transmisión del conocimiento histórico.  
\par \textbf{\ \ ORGANIZACIÓN Y ADMINISTRACIÓN DE ACERVOS }\par 
\textit{\ \ Conocimiento} \par \ \ Conocer y analizar los diferentes 
tipos de documentos. Capacitar en el diseño, organización y 
administración de acervos de carácter histórico, bibliográfico y de 
otro tipo de repositorios o fondos documentales. Conocer y dominar las 
técnicas y teorías archivísticas, biblio- hemerográficas, así como sus 
principales principios. Capacitar en el manejo de las tecnologías del 
conocimiento aplicadas en el resguardo documental.  \par \textit{\ \ 
Competencias }\par \ \ Desarrollará la competencia para organizar y 
administrar repositorios documentales, ya sean instituciones públicas, 
privadas o particulares. Será capaz de distinguir los diferentes tipos 
de documentos, sus características internas y externas, su tipología de 
clasificación, etcétera. Será capaz de organizar, clasificar, conservar 
diferentes tipos de expedientes y obras, considerando el soporte 
material.  Desarrollar la competencia de transcribir, analizar e 
interpretar la documentación histórica.  \par \textbf{\ \ HISTORIADOR 
DEL ARTE} \par \textit{\ \ Conocimiento }\par \textit{\ \ A}dquiere una 
visión general de los diferentes procesos histórico-artísticos, en los 
ámbitos teórico-metodológicos e historiográficos. \par \textit{\ \ 
Competencias}  \par \ \ Desarrollo crítico y analítico de la producción 
artística en sus distintos lenguajes (arquitectura y urbanismo, 
escultura, pintura, fotografía, cine, música, artes decorativas y 
suntuarias) representativos de cada época de la historia.  \par 
\textbf{\ \ HISTORIOGRAFO} \par \textit{\ \ Conocimiento }\par \ \ El 
alumno conocerá las diversas teorías y metodologías de hacer y pensar 
la historia, tanto nacional como internacional a través del tiempo, 
además aprenderá a relacionar los textos historiográficos con sus 
contextos ideológicos y circunstanciales.  \par \textit{\ \ 
Competencias }\par \ \ El alumno analizará y criticará diferentes 
formas de escritura de la historia. Además aprenderá a utilizar la 
metodología de otros historiadores para sus trabajos de investigación, 
en los que aplicará tanto teoría como elementos narrativos y 
discursivos para pensar y escribir la historia.\par \par \par \par }

Además se cuenta con un listado de optativas generales que complementen 
su formación de ellas se describen las siguientes: Historias generales 
con siete asignaturas; Historia especializadas, veintitrés; Historias 
del Arte, trece; turismo con siete; complementos con seis UDI´s. El 
plan de estudios es por créditos por lo que la duración dependerá de la 
planeación individual de cada uno de los estudiantes, es decir, de un 
total de 469 créditos repartidos en 113 UDIs, se necesitan para 
titularse de 230 créditos, de los que el 39\,\% (20 UDIs) corresponden a 
las UDIs obligatorias y el 61\,\% (35 UDIs) a las optativas, 
especificando que de estas últimas, cuatro deben ser historiografías. 
Además de las actividades en el aula, el alumno deberá administrar su 
tiempo para realizar lecturas, investigar, elaborar trabajos, tareas y 
asistir a diversos eventos cultural-académicos (conferencias, 
presentación de libros, coloquios, cine, conciertos, etcétera).  

La variada y rica oferta de UDI´s en el Plan de estudios 
lamentablemente no es aprovechada por los estudiantes ni por los 
profesores. En el caso de los seis ejes y su justificada pertinencia, 
la realidad es distinta. Los ejes que siguen manteniendo vigencia y 
predominio es el de docencia e investigación; seguida por la de 
difusión e  historia del arte con un mínimo de estudiantes, mientras 
que los ejes de  organización y administración de acervos e 
historiografía son pocos solicitados. En éste sentido las razones 
pueden ser varias, desde una realidad laboral que prevalece en el 
estado y en el país, donde la docencia es lo que prima en el ejercicio 
laboral del historiador hasta el poco apoyo que existe en proyectos 
culturales y de difusión aún y a pesar de que Zacatecas tenga la 
declaratoria de Patrimonio Mundial de la Humanidad. Sin embargo 
consideramos que la clave para fortalecer los ejes terminales se 
encuentra en la orientación de los tutores con los estudiantes para 
poder orientarlos correctamente y de acuerdo a sus aptitudes, 
conocimientos y destrezas del alumnado. 

Respecto a los mismos ejes transversales (género, democracia, derechos 
humanos, ecología, globalización y desarrollo sustentable) se observa 
que en los programas no se integran como valores de discusión o de toma 
de conciencia. Existe el planteamiento en el plan de estudios pero en 
la práctica, pocas veces permea el espacio áulico y en la comunidad 
educativa.


La gran variedad de materias optativas presenta todo un desafío para la 
planta docente, pues además de que se han realizado nuevas 
contrataciones de profesores al programa que logren cubrir las 
necesidades de materia laboral disponible. La realidad es otra. El 
perfil laboral y experiencia profesional de la nueva planta docente no 
cubre con las necesidades planteadas del programa, es decir, no tienen 
el perfil de  profesores, tienen escasa experiencia en el proceso de 
enseñanza aprendizaje, y su perfil y especialización no cubre la 
experiencia necesaria para las nuevas materias que ofrece el plan de 
estudios. A tres años de implementarse el nuevo plan de estudio, varias 
UDI´s no se han podido ofertar ya sea por falta de profesores,  o por 
que los estudiantes no están interesados en ellas.

A partir del 2004 nuestro Plan de estudios eliminó la rigidez de la 
currícula  por uno de carácter flexible. Ahora el alumno que repruebe 
la asignatura en cualquier oportunidad (ordinario, extraordinario y a 
título de suficiencia), podrá volver a cursarla o darla de baja y/o en 
su caso sustituirla por otra, siempre y cuando reúna el número de 
créditos necesarios y corresponda al eje seleccionado previamente por 
el alumnado.

En esta ocasión se trató de diversificar aún más las materias optativas 
en vías de una preparación más integral del estudiante, y con el 
objetivo principal de ser evaluados por instancias externas además de 
los CIIES y en esta ocasión para el 2014 nos estamos preparando para 
ser supervisados en aras de la certificación por el COAPEHUM y mantener 
los estándares de calidad que exigen nuestras autoridades tanto 
universitarias como gubernamentales.

El nuevo plan de estudios del programa de licenciatura en historia está 
diseñado para que el alumno se inserte de una manera inmediata al 
mercado una vez concluidos sus estudios, ya que el contenido de las 
UDIs proporciona las herramientas necesarias para que el egresado tenga 
una amplia oportunidad y competencia para el empleo, sin embargo la 
realidad laboral es otra, algunos estudiantes continúan estudiando en 
algún posgrado o buscan trabajo como profesores. Pocos son los 
egresados que se desempeñan en el ámbito de la investigación, a pesar 
de que egresan con la capacidad de consultar en bibliotecas 
especializadas, archivos y realizar textos de carácter histórico que 
incluyan las nuevas visiones de la historia. Pocos estudiantes han 
podido incorporarse laboralmente en los  archivos históricos a pesar de 
que sus competencias los prepararon para elaborar proyectos, clasificar 
y organizar documentos así como saber crear instrumentos de consulta 
como catálogos, guías, versiones paleográficas y otros. En algunos 
casos y temporalmente los egresados se han podido emplear como 
auxiliares de investigador y/o por su cuenta.  

El ámbito laboral común de los egresados es como docentes en los 
niveles medio, medio superior y superior, y entre las competencias con 
las que egresan son primordialmente la elaboración de materiales 
didácticos, aplicación de dinámicas e impulso por  el aprendizaje por 
competencias. Pocos son los casos de egresados que se ocupen en 
distintos ámbitos interinstitucionales, a través de proyectos 
culturales; o en áreas laborales de gestoría y comunicación.  

Para que el alumno pueda titularse requiere haber cubierto 230 créditos, de
los que el 47\,\% (109 UDIs) corresponden a las UDIs obligatorias y el 53\,\%
(121 UDIs) a las optativas, especificando que de estas últimas, cuatro
deben ser historiografías.  Cubrir alguna de las modalidades de titulación
a saber: 1) Tesis individual o colectiva (máximo dos personas), 2)
evaluación de la memoria y del producto de la estancia profesional, 3)
elaboración de catálogos o ediciones paleográficas, 4) promedio (mínimo de
9.0) sin interrupción de los estudios y con la obtención de las
calificaciones en períodos ordinarios, 5) curso-taller de titulación y 6)
examen general de conocimientos.  El estudiante deberá acreditar 6
trimestres de un idioma extranjero cursado en el centro de idiomas de la
UAZ.  Los Títulos que se otorgan son: 1) Técnico Superior Universitario. 
Para el TSU se requieren 140 créditos, de los cuales 80 deberán
corresponder a UDIs obligatorias, además de la Estancia y el servicio
profesional, mientras que los 24 créditos restantes corresponden a seis
UDIs optativas del eje  de especialización que se haya elegido, tomando en
cuenta que son seis ejes: Docencia, Investigación, Difusión, Organización y
Administración de Acervos, Historia del Arte e Historiografía.  2)
Licenciado en Historia.\footnote{Antonio
González Barroso, María del Refugio Magallanes Delgado y Ángel Román
Gutiérrez (Cuerpo Académico Enseñanza y Difusión de la Historia).
\textit{La reforma curricular de la licenciatura en historia de la UAZ: un
esfuerzo compartido, en el }\textit{VI Encuentro de la Red Nacional de
Licenciaturas en Historia y sus Cuerpos Académicos.}}  

La opción de titulación más aceptada por los estudiantes es la tesis. 
Esto se debe a que tradicionalmente la licenciatura ofrecía como única 
opción de titulación la tesis individual. Y la innovación de nuevas 
opciones se verificó en el 2004, apareciendo hasta fechas recientes la 
elección de las nuevas formas de titulación, prefiriéndose el promedio  
y los cursos talleres de titulación con sus productos. De éstos últimos 
el de mayor demanda es el curso de titulación, puesto que ésta 
modalidad se dirigió en un primer momento a todos aquellos alumnos 
rezagados que no habían podido adquirir el grado permitiéndoles la 
titulación en seis meses.

A pesar de la variedad de oferta de titulación el estudiante considera 
que la tesis es la mejor opción para el desempeño profesional del 
egresado, sin embargo la realidad se contradice con las opiniones 
vertidas por los mismos ya que muchos de ellos al terminar sus créditos 
correspondientes y con un buen desempeño en su carrera optan por la 
titulación por promedio, la otra gran mayoría que no fue tan 
beneficiada en sus calificaciones se inclina por los cursos que se 
ofertan anualmente en el programa. A pesar de las ventajas en la oferta 
de titulación  la eficiencia terminal sigue por debajo de lo requerido. 

El objetivo general que pretende la Unidad Académica de Historia es
consolidarse como una alternativa de excelencia académica en la región.
 {El programa de licenciatura en historia de la UAZ,  al
multiplicar sus orientaciones terminales adopta el modelo “eficientista”
dominado por el mercado, pero también se mantiene el compromiso de generar,
avanzar y transmitir el conocimiento a través de los cuerpos académicos, no
sólo de conocimiento útil sino de conocimiento que hace trascender  a la
humanidad.}\footnote{Antonio
González Barroso, María del Refugio Magallanes Delgado y Ángel Román
Gutiérrez (Cuerpo Académico Enseñanza y Difusión de la Historia).
\textit{La reforma curricular de la licenciatura en historia de la UAZ: un
esfuerzo compartido, en el }\textit{VI Encuentro de la Red Nacional de
Licenciaturas en Historia y sus Cuerpos Académicos.}}

Actualmente, la Universidad Autónoma de Zacatecas se rige por un modelo 
educativo implementado en 2005, denominado \textit{UAZ Siglo }XXI, en 
el cual se determinó a “la actividad académica como el centro de su 
competencia con base en sus funciones sustantivas.”\footnote{Comisión 
general operativa, \textit{Proyecto curricular} (cuadernillo No. 11), 
Zacatecas, UAZ, 2000, p. 4.} Al efecto se reivindicó la tradición 
humanística, democrática y de autonomía que le dio origen y que ha 
estado presente a lo largo de su historia reciente. De acuerdo con este 
nuevo modelo académico, que se basa en el constructivismo crítico, el 
estudiante es el centro del proceso académico mientras que el profesor 
se convierte en un facilitador del aprendizaje; además, se propone la 
flexibilización de los planes y programas académicos. Como resultado 
del contexto local, nacional e internacional caracterizado por la 
desigualdad, la competitividad producto del neoliberalismo, la 
globalización y la revolución digital, uno de los principales objetivos 
del modelo \textit{UAZ Siglo XXI }consiste en ofrecer a los y las 
zacatecanas “…una educación sustentada en la investigación; prioridad a 
los procesos de aprendizaje del estudiante; los currículos abiertos y 
flexibles para todos y para toda la vida, y la incorporación y 
generalización del uso de nuevas tecnologías…”\footnote{UAZ 
(2005)\textit{ Modelo académico UAZ}, Siglo XXI, Zacatecas, UAZ, p. 9.}


%\end{document}