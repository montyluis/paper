%\documentclass{article}
%\usepackage{amsmath,amssymb,amsfonts}
%\usepackage{fontspec}
%\usepackage{xunicode}
%\usepackage{xltxtra}
%\usepackage{polyglossia}
%\setdefaultlanguage{spanish}
%\usepackage{color}
%\usepackage{array}
%\usepackage{hhline}
%\usepackage{hyperref}
%\hypersetup{colorlinks=true, linkcolor=blue, citecolor=blue, filecolor=blue, urlcolor=blue}
%\newtheorem{theorem}{Theorem}
\pagenumbering{arabic}
\clearpage\setcounter{page}{1}
\thispagestyle{empty}
\phantomsection{}
\addcontentsline{toc}{chapter}{Introducción\newline $\diamond$
\normalfont\textit{Elva Rivera Gómez}}
{\centering \large {\scshape\ Introducción}\newline {\itshape\ La enseñanza de la Historia y las opciones terminales\newline en las Licenciaturas en Historia en México\/}\par}
\markboth{la formación del historiador}{introducción}
\setcounter{footnote}{0}

\bigskip
\begin{center}
{\bfseries Elva Rivera Gómez\par}
\end{center}

\bigskip
\epigraph{¿Cómo se forman los historiadores? ¿Para qué se estudia la licenciatura en 
Historia en la Universidad? [\ldots] Tanto la formación como la información
del futuro historiador se orienta mucho más hacia la investigación 
que hacia la docencia, pese a que, por el contrario, 
el terreno de la práctica profesional es mucho 
más amplio en la enseñanza escolar.}{\itshape\ \textemdash{} Andrea Sánchez Quintanar\/}
\enlargethispage{2\baselineskip}

\bigskip
Por los pasillos de las licenciaturas en Historia, un fantasma atraviesa la enseñanza de la disciplina: 
 el fantasma de las competencias diseñadas por el mercado neoliberal. 

En las universidades mexicanas la enseñanza de la Historia como disciplina estuvo fuertemente orientada a la investigación, al menos durante la segunda mitad del siglo XX.\ La mayoría de las licenciaturas fundadas durante este periodo tomaron como
modelo el plan de estudios de la UNAM.\ Por ejemplo, las carreras de Historia
que se fundaron en 1945 en el Instituto de Historia de la Universidad Nacional Autónoma de México y en la casa de España, hoy Colegio de México, y la que se estableció en 1957 en la Universidad Iberoamericana. Mientras que en 1965 se fundó el Colegio de Historia en la Escuela de Filosofía y Letras de la Universidad Autónoma de Puebla. La UNAM inauguró la modalidad de enseñanza abierta en 1972, y  no fue sino hasta 1977 que la licenciatura en Historia se empezó a ofrecer por esta vía. Más tarde, en la década de los $80$,
algunas licenciaturas se fueron decantando por un modelo híbrido entre el plan de estudios de la UNAM y el de la UAM\@.

Si las y los historiadores\slash{}as son quienes conocen el saber-hacer de la disciplina, ¿qué ha pasado en la enseñanza, la investigación y la difusión, en el
diseño de los planes y programas de estudios a lo largo de los últimos treinta años? ¿Cómo se forma a los futuras generaciones de profesionales de la Historia, qué áreas formativas del perfil profesional son reconocidas en el mercado laboral de las y los egresados? Estas son algunas de las interrogantes que necesitamos responder hoy por hoy en la academia. 

\enlargethispage{1\baselineskip}
Algún barrunto de respuesta a estas preguntas seguramente lo encontraremos en los objetivos y el perfil de egreso de los planes y programas de estudio de nuestras carreras. Empero, si deseamos preservar los rasgos característicos de un perfil disciplinar consistente, es importante
enseñar a construir el conocimiento, a elaborarlo, a descubrirlo, a
resolver problemas, como alguna vez lo sugirió  Andrea Sánchez Quintanar (1996), quien también reconoció que los problemas a que se enfrentan tanto los\slash{}as historiadores\slash{}as que enseñan, como los\slash{}as que aprenden, abarcan un abanico tal que 

\begin{quote}
(\ldots) desde el abordaje filosófico de la disciplina \textemdash{}el por qué y para qué de la historia\textemdash{}; los diversos aspectos epistemológicos{} \textemdash{}las formas de
construcción del conocimiento histórico; las diversas teorías de la
historia y los métodos que las teorías implican; su aplicación a las áreas
específicas del conocimiento histórico y, por último, su elaboración y
formulación en un estructura formal orgánicamente estructurada, y con una
forma de expresión al menos clara o, si es posible, estéticamente valiosa
(\textit{ibid.}).
\end{quote}
\enlargethispage{1\baselineskip}
Sánchez Quintanar sustentó estas propuestas al revisar el
Plan de Estudios de Historia de la UNAM, que en ese entonces sólo contemplaba
dos cursos de didáctica de la historia. Desde su punto de vista, en estas
materias se debían estudiar los problemas que plantea la enseñanza de la
historia, tales como:

\begin{quotation}
(\ldots) problemas de organización educativa, elementos de sociología de la
educación, los problemas específicos de la transmisión-difusión del
conocimiento histórico (temporalidad, espacialidad y varios más) y, desde
luego, los propiamente didácticos: formulación de objetivos, técnicas
didácticas, proyectos de trabajo docente, programas y planes de estudio,
etcétera (\textit{ibid.}).
\end{quotation}

Estas reflexiones de Sánchez Quintanar se hicieron en el contexto de la UNAM,
justo cuando a nivel internacional se discutían los problemas referentes a
la enseñanza de la historia en Europa e Iberoamérica, y cuando en la
mayoría de las universidades estatales se iniciaban las reformas de los planes y programas 
 de estudio sugeridas por las políticas educativas neoliberales.

En el contexto internacional, tanto en Europa como  en Latinoamérica se
abordó la discusión sobre la formación del profesorado, y se propusieron un conjunto de
competencias para la enseñanza de la historia. A este respecto, Joan Pagés nos relata de que modo el Consejo de Europa se interesó por la enseñanza de la Historia, pues a finales de la década de los 90 se organizaron tres seminarios destinados a la reflexión sobre la formación inicial del profesorado de historia: en Viena (1998), Praga (1999) y Atenas
(2000). El objetivo central de tales eventos fue <<Aprender y enseñar la historia de Europa del siglo XX>>. 
De estas reuniones surgió una \textit{recomendación} relativa a la enseñanza de la historia en Europa en el siglo XXI (Pagés 2004, p. 163).

Para el caso iberoamericano, la Organización de Estados Iberoamericanos, a partir del Nodo Coordinador de la Cátedra de Historia de Iberoamérica, propuso 

\begin{quotation}
(\ldots) la necesidad de combinar la formación del profesorado de
historia con la formación para poder enseñar Historia de Iberoamérica,
haciendo hincapié en los valores y las competencias académicas y pedagógicas,
cuyas bases se apoyen en el sentido humanista, las experiencias comunes de los procesos históricos y el aprendizaje autónomo:

Unos valores y unas actitudes proclives a esta enseñanza basada en la
comprensión del otro, en la solidaridad y en la responsabilidad hacia un
pasado común y hacia un futuro en el que sean posibles los procesos de la
integración regional.

Unas competencias académicas referidas a la Historia en general y a la
Historia de Iberoamérica en particular, a sus rasgos específicos y a las
experiencias comunes de los procesos históricos que en ella se han
producido.

Unas competencias pedagógicas-didácticas facilitadoras de proceso de
aprendizaje autónomos, que permitan al profesor conocer, seleccionar,
utilizar, evaluar y desarrollar estrategias de intervención didáctica
(Pagés 2004, pp. 164\textendash{}165).
\end{quotation}
\enlargethispage{1\baselineskip}
Además, es importante señalar que se diseñaron las competencias académicas y
pedagógicas relacionadas con la naturaleza y el procedimiento del conocimiento histórico, consistentes en:

\begin{quotation}
(\ldots) las principales tendencias historiográficas y su evolución; el tipo de conceptos, datos y hechos que utiliza la Historia y el carácter relativista
de su interpretación; la comprensión y orientación en los sistemas temporal
y espacial; la comprensión del sentido procesal de la Historia, etcétera.
Por su parte las competencias pedagógicas las clasifican en tres grandes
grupos: la adquisición de las competencias sociales y de comunicación; la
concepción, planificación y programación de la actividad docente y la
organización, dirección y evaluación del proceso de enseñanza-aprendizaje
(\textit{ibid.}).
\end{quotation}

En este contexto internacional e iberoamericano de la década de los 
$90$, fue
que se empiezan a introducir en México nuevas reformas a los planes y programas de
estudio, y, paulatinamente, a transformarse el curriculum en las Licenciaturas en Historia 
de las universidades estatales. Entre otras cosas, se trataba de que ahora el curriculum se hiciera
flexible, que contará con un sistema de créditos, y que incluyera las opciones terminales.

Finalmente, en la primera década del siglo XXI, el Proyecto Tuning del año
2001 constituyó un espacio de reflexión sobre la educación superior, y
fue el marco para que en la IV Reunión de Seguimiento del Espacio Común de
Enseñanza Superior de la Unión Europea, América Latina y el Caribe (UEALC),
en octubre de 2002, se propusiera un proyecto para América Latina, el cual fue
presentado ante la Comisión Europea en el 2003. Un año más tarde surgió el
Proyecto ALFATuning-América Latina (2004\textendash{}2008).

\enlargethispage{1\baselineskip}
Los objetivos de este proyecto fueron, entre otros:

\begin{Obs}
\begin{sloppypar}
\item[$\bullet$] Contribuir al desarrollo de titulaciones fácilmente comparables y
comprensibles en una forma articulada en toda América Latina.
\end{sloppypar}

\item[$\bullet$] Impulsar a escala latinoamericana, un importante nivel de convergencia de la
educación superior en doce áreas temáticas, \textemdash{}entre éstas Historia\textemdash{}
mediante las definiciones aceptadas en común de resultados profesionales y
de aprendizajes.
\item[$\bullet$] Desarrollar perfiles profesionales en términos de competencias genéricas y
relativas a cada área de estudios incluyendo destrezas,\linebreak conocimientos y
contenido en las cuatro áreas temáticas del proyecto.
\item[$\bullet$] Desarrollar e intercambiar información relativa al desarrollo de los
currículos en las áreas seleccionadas y crear una estructura curricular
modelo expresada por puntos de referencia para cada área, promoviendo el
reconocimiento y la integración latinoamericana de titulaciones (Tuning
2004\textendash{}2008).
\end{Obs}

El Proyecto Tuning para América Latina (2004\textendash{}2008) señalaba cuatro líneas de
trabajo: (1) Competencias (genéricas y específicas), (2) Enfoques de
enseñanza, aprendizaje y evaluación, (3) Créditos académicos, y (4) Calidad
de los programas. La tercera, especialmente, propuso el trabajo del alumnado,
el sistema de créditos académicos, es decir, una medida que incidía directamente en la duración de los estudios del alumnado en el nivel superior.

Este antecedente internacional fue el marco referencial para que en las
diversas licenciaturas en Historia de las universidades mexicanas se
llevaran a cabo revisiones y reformas a los planes y programas de estudio
de esta disciplina. Para ese proceso ha sido central la evaluación y
acreditación de las carreras \textemdash{}ahora llamados Programas Educativos (PE)\textemdash{}, 
lo que ha obligado a confirmar y reconfirmar la calidad y pertinencia educativa de las licenciaturas. Asimismo, los organismos evaluadores (CIEES) y acreditadores
(COPAES) reconocidos por la Secretaría de Educación Pública, en sus
recomendaciones de evaluación, sugirieron que para llevar a cabo reformas a
los planes y programas se debían considerar los estudios y\slash{}o diagnósticos
del mercado laboral de la disciplina de historia. Para ellos,  la
opinión de los empleadores y de los egresadas\slash{}os, por una parte, era decisiva. Por la otra, la Secretaría de Educación Pública diseñó directrices sustentadas en los acuerdos de la Conferencia Mundial de Educación Superior (1989), el plan 
Bolonia y el proyecto Tuning, a fin de introducir el modelo 
de las competencias genéricas, disciplinares y profesionales en el diseño de los planes y programas de estudio.

Este nuevo escenario condujo a una transformación, en algunos casos parcial,
de los planes y programas de estudio de las licenciaturas en Historia. A las llamadas asignaturas y\slash{}o materias se les denominó a partir de entonces
Unidades de Aprendizaje, y se puso especial atención a las opciones y\slash{}o
áreas terminales de la carrera.

\enlargethispage{1\baselineskip}
En un trabajo reciente, Sebastián Plá señala que en la enseñanza de la historia 
como objeto de investigación en México,  a pesar de la falta de
conceptualización,  es notorio el reconocimiento como parte constitutiva de la identidad del
historiador  (Plá 2012, p. 166).  Este autor reconoce que, aunque los planes y programas de estudios sigan orientados esencialmente a la formación en la investigación,  la inserción laboral real de los\slash{}as egresados\slash{}as en Historia da fe de que estos planes y programas también ofrecen opciones en otras vertientes del conocimiento relacionadas con el quehacer del historiador\slash{}a, como la docencia y el patrimonio histórico. En ese mismo tenor, también son reveladoras las cifras del Observatorio Laboral de la Secretaría del Trabajo y Previsión Social (STPS) relativas a la inserción laboral de los\slash{}as egresados\slash{}as de Historia, que el mismo Plá nos trae a colación: 

\begin{quotation}
(\ldots) el 20.7\,\% de los egresados de las licenciaturas en Historia son maestros de secundaria, 4.9\,\% trabaja dando clases en bachillerato y 9.9\,\% como docentes universitarios, es decir, un poco más de la tercera parte de los
historiadores\slash{}as no se dedican a la investigación sino a la docencia, y una
cuarta parte lo hace lejos de los ambientes universitarios (\textit{ibid.}, p.
166).
\end{quotation}

Al reflexionar en torno a la dicotomía investigación\slash{}docencia, Plá reconoce que ésta se encuentra disociada o tiene poca relevancia en la formación del\slash{}a historiador\slash{}a en los planes y programas de estudio de las respectivas licenciaturas de las universidades mexicanas. Además, en un análisis sucinto expone cual es el lugar que ocupa el \textit{área de docencia} en algunas licenciaturas, encontrando, por ejemplo, algunas donde apenas se le asignan dos o tres asignaturas, como didáctica, enseñanza o docencia de la historia. Plá subraya, más adelante, que son varios los casos en los que estas asignaturas se engarzan con opciones terminales que abarcan otras temáticas, 
 como aprendizajes de la historia, instrumentos de evaluación y habilidades docentes (\textit{ibid.},~p.~167).

Así, pues, en la primera década del siglo XXI se han institucionalizado, a
partir de las reformas y de los estudios del mercado laboral \textemdash{}de
empleadores y egresados\textemdash{}, las áreas terminales y\slash{}o de especialización.
Éstas, según los perfiles de egreso, se orientan a la inserción en el
mercado laboral, cubriendo principalmente tres rubros: investigación, docencia y difusión. A continuación citaré sólo unos cuantos casos significativos, ya que este tema amerita un estudio más detallado que comprenda a todos los planes y programas de las carreras de Historia en el país, tanto de las universidades públicas como de las privadas.

Un primer ejemplo lo tenemos en las cuatro áreas de especialización que contempla la Licenciatura en Historia de la Universidad de Sonora:\linebreak (1)
\textit{Investigación histórica}, que incluye las subáreas de: (a) Historia
económica, (b) Historiografía,  (c) Historia cultural, (d) Historia política,
(e) Historia social, (f) Historia colonial, (g) Historia de las comunidades
étnicas; (2) \textit{Divulgación histórica}, que incorpora la enseñanza de
cursos referentes a la Producción y realización fotográfica I y II,
Producción y realización radiofónica I y II, Producción y realización
audiovisual I y II, Producción y realización de medios impresos I y II, y
Producción multimedia I y II;\ (3) \textit{Formación docente}, la cual abarca las
asignaturas de Políticas educativas, Administración educativa, Desarrollo
curricular, Sociología de la educación, Historia de la educación y
Multimedia educativa; y (4) \textit{Organización y administración de
archivos}, que integra los cursos de Taller de organización y
administración de archivos, Archivística y Producción multimedia I y II\@.

En tanto que la carrera de Historia de la Universidad de Guadalajara ofrece
las áreas de formación especializante en \textit{Historia y Comunicación},
\textit{Historia del Arte}, \textit{Prehistoria y Estudios Mesoamericanos},
\textit{Estudios Coloniales}, \textit{Historia Moderna de México}, y
\textit{Docencia en Historia}.

\enlargethispage{1\baselineskip}
Mientras que la Universidad de Autónoma de Nuevo León, en la Licenciatura en
Historia y Estudios en Humanidades, incluye el área de
\textit{Investigación}: Metodología de la investigación, Seminario de
Investigación (10º semestre); el área de \textit{Patrimonio y gestión
cultural}: Metodología de la difusión histórica (6º semestre), Patrimonio y
gestión cultural (7º semestre), Producción editorial (10º semestre); y 
el área de \textit{Docencia}, en la cual se insertan los contenidos de Teorías
del aprendizaje (6º semestre), Metodología de la enseñanza de la historia
(7º semestre) y Programación didáctica (10º semestre). En el octavo semestre se 
cursa Protocolo de investigación, y se realiza el servicio social (16~créditos); 
mientras que en el noveno semestre el alumnado tiene libre elección con 22 créditos.

Por su parte, la Licenciatura en Historia de la BUAP, en el plan de estudios
2009 (MUM), incluye las áreas terminales de \textit{Investigación} (Seminario
metodológico, Seminario de Investigación Histórica I y II y Seminario de
Tesis con un total de 16 créditos); \textit{Docencia} (Didáctica de la
Historia, Metodología de la Enseñanza de la Historia y Práctica Docente, con 
valor de 12 créditos); y \textit{Gestión del Patrimonio histórico cultural} (Gestión
del Patrimonio Cultural y Organización y administración de fondos
históricos, con 8 créditos). El Servicio Social tiene un valor de 10 créditos
(500 horas), y la práctica profesional 100\slash{}150 horas cubre ahora 10 créditos.

La Licenciatura en Historia de la Universidad Juárez Autónoma de Tabasco
presenta cuatro opciones denominadas Áreas de Formación Integral
Profesional, donde se insertan diversos cursos. En \textit{Docencia}:
Introducción a la Didáctica, Corrientes actuales de la Didáctica, Taller de
material didáctico de la Historia y práctica docente, con un total de\linebreak 24
créditos. En \textit{Investigación}: Taller de investigación I y II,
Práctica de campo, Taller de historia oral, Seminario de tesis e Historia y
filosofía de la ciencia con un total de 37 créditos. La opción
\textit{Investigación histórica y Cultural}: Administración y organización
de archivos, Biblioteconomía, Museografía, Taller de paleografía y
Diplomática, Arqueología y Diplomática. La opción \textit{Arte y
Literatura}: Historia de la literatura clásica y renacentista, Historia del
arte y literatura moderna, Historia del arte y literatura latinoamericana,
Historia del arte y literatura mexicana, Historia y literatura y arte de
Tabasco, Producción y elaboración de guiones y Diseño y producción de
material audiovisual con un total de 46 créditos. En el área de formación
transversal se ubican el Servicio Social (con un valor de 10 créditos) y la
Práctica docente (con\linebreak 6 créditos).

\enlargethispage{1\baselineskip}
La Licenciatura en Historia del Instituto Mora ofrece tres áreas terminales:
\textit{Didáctica de la historia, Divulgación de la Historia y Gestión del
patrimonio cultural}. En tanto que la Licenciatura en Historia de la
Universidad Autónoma Metropolitana, a las opciones terminales les denomina áreas
de integración. Y ofrece cuatro líneas: \textit{Investigación histórica}
(I, II y III), \textit{Didáctica de la Historia} (I,~II~y~III),
\textit{Difusión de la Historia}~(I,~II~y~III) y \textit{Trabajo de fuentes
históricas}~(I,~II~y~III).

El Plan de Estudios de Licenciatura de Historia (2006) de la Universidad
Autónoma de Yucatán contempla los ejes Teórico, Formativo,\linebreak Metodológico y
Optativos. Por ejemplo, el eje metodológico comprende las materias de
Didáctica de la Historia, Paleografía Indiana, Patrimonio Documental,
Métodos y Técnicas de Investigación Histórica, Taller de investigación I,
II, III, IV, V y VI.\ Como curso optativo oferta Enseñanza de la Historia y
Métodos de la Historia Regional. El Servicio Social tiene un valor de 12
créditos.

El programa de Historia de la Escuela Nacional de Antropología e Historia,
sólo incluye 3 áreas con una asignatura en área de especialización, y son:
\textit{Patrimonio cultural} (con 4 créditos y se cursa en el VI semestre),
\textit{Difusión de la Historia} (con 4 créditos y se cursa en el VII
semestre) y \textit{Docencia} (8 créditos, que se cursa en el 8º semestre).

En otras universidades, en la carrera de Historia se le dedican más asignaturas
al área de investigación, y tiene menor relevancia el área de docencia. Tal
es el caso de la Universidad Autónoma del Estado de Hidalgo, donde sólo
existe un curso de didáctica de la Historia en el séptimo semestre, con un valor 
de 7.5 créditos.

Finalmente, el Plan de estudios 2014 de la Licenciatura en Historia de la Universidad de
Aguascalientes abarca cuatro áreas: Básica, Disciplinar,
Metodológica y Terminal. En el área metodológica incluye las asignaturas de
Manejo de fuentes, Taller de archivonomía y paleografía, Taller de
implementación de la Historia, Estadística descriptiva, Metodología
cualitativa y Metodología de la investigación. Mientras que en la rama Terminal
incorpora la Enseñanza de la Historia, Ética profesional, Taller de integración
I y II.\  Además, presenta una correlación formativa de la curricula con el
perfil de egreso en relación con las Historias regional, nacional y mundial;
fundamentos del Patrimonio histórico y cultural; Sistema de catalogación y
registro de colecciones históricas, método paleográfico, fuentes y
archivos; Enfoques teórico y\linebreak metodológicos de la investigación histórica
(énfasis en nuevas tendencias); Didáctica general y nuevas teorías y
métodos para el proceso de enseñanza/aprendizaje de la historia (Didáctica,
Diseño, selección y uso de recursos didácticos, Enseñanza de la Historia,
Taller de Integración I y II).

Con estos antecedentes,  podemos señalar que en cada universidad las áreas
terminales que se han ofertado están diseñadas en función del campo y
mercado laboral disciplinar, y están siendo  reguladas por las reformas curriculares diseñadas desde los organismos financiadores de la educación superior, tanto a nivel internacional como por la SEP y los organismos evaluadores y acreditadores de la educación superior mexicana.

Por todo lo anterior, la Red Nacional de Licenciaturas en Historia y sus
Cuerpos Académicos (RENALIHCA) ha considerado pertinente efectuar un balance de los primeros efectos de la transformación curricular de la carrera de Historia,
que se ha venido dando en el contexto de las directrices propuestas por las políticas educativas internacionales mediante las recientes reformas y reestructuraciones curriculares de la enseñanza de esta disciplina en las universidades estatales de México.

El libro \textit{La formación del historiador. Áreas terminales, 
prácticas profesionales, servicio social y tutorías en las
Licenciaturas de Historia en México} reúne las experiencias
educativas de quienes enseñan a investigar, a enseñar y a divulgar el
conocimiento histórico en las universidades mexicanas. 

%\enlargethispage{1\baselineskip}
La obra empieza con una reflexión de Saúl Jerónimo
Romero en torno a la historiografía como eje
articulador de la enseñanza de la historia. En esta ponencia el autor revisa puntualmente las dos vertientes de la enseñanza de esta asignatura en las carreras de Historia de las universidades de México.

A continuación, en el apartado intitulado \textit{Las áreas terminales de las
Licenciaturas en Historia}, se presentan las experiencias del
\textit{Seminario de catalogación de documentos de archivo} en la
Universidades Autónoma del Estado de México, de la autoría de la maestra
emérita María Elena Bibriesca Sumano y Guadalupe Zárate Barrio y \textit{El
catálogo documental como vía de titulación en la Facultad de Historia: un
instrumento necesario para la investigación histórica} de la Universidad
Michoacana de San Nicolás Hidalgo, de Catalina Sáenz
Gallegos, María Guadalupe Carapia Medina y Ruben Dario Nuñez Altamirano. En
ambos trabajos se sustenta ampliamente la importancia de la preservación
documental, el manejo de los aspectos teórico-metodológicos y las técnicas
que debe adquirir y dominar el\slash{}la profesional de la Historia para
preservar la memoria colectiva, a través de la elaboración de catálogos
documentales como trabajo recepcional para obtener el grado de
licenciatura.

\enlargethispage{1\baselineskip}
En este mismo apartado, un número importante de trabajos
abordan el área de la enseñanza de la historia y la investigación,
así como las experiencias en áreas no escolarizadas. En este rubro se
inscriben los trabajos intitulados \textit{Programa de
Estudios con Línea Terminal en Enseñanza de la Historia en la
Universidad Autónoma de Querétaro} de
Paulina Latapí Escalante, \textit{La
enseñanza de la historia y su pertinencia en los planes de
estudio} de Ma. Gabriela Guerrero Hernández y
María del Rocío Rodríguez Román,
\textit{La didáctica en la
formación docente: retos y perspectivas} de
Hugo Torres Salazar, \textit{La
enseñanza de la Historia y la crisis del medio
ambiente} de Gil Arturo Ferrer Vicario,
\textit{Dimensión didáctica de la Historia y su valor
formativo} de Jaime Salazar Adame y
Smirna Romero Garibay, \textit{La
enseñanza de la Historia: Una modalidad no
convencional} de Wilfrido Llanes
Espinoza y Eduardo Frías Sarmiento, \textit{Estrategias
de enseñanza desde el aula} de Vanessa Magaly Moreno
Coello, Patricia Gutiérrez Casillas y Mario Heriberto Arce
Moguel, \textit{Problemática y
alternativas en el área de investigación para la formación del
historiador} de Arturo Carrillo Rojas y Luis Demetrio Meza López,
\textit{Trayecto de una línea académica en peligro de extinción.
El caso de la Difusión de la Historia en la
Universidad Autónoma de Zacatecas} de
Antonio F. de Jesús González Barroso, María R.
Magallanes Delgado y Ángel Román
Gutiérrez, y \textit{El periodo virreinal de México y su alcance en
nuestros días: la difusión del conocimiento histórico}
de Marco Antonio Peralta Peralta y Marcela Janette Arellano González.

En la segunda parte, \textit{La
eficiencia terminal y la estancia y práctica profesional en las
Licenciaturas en Historia}, se presentan algunos trabajos que versan sobre
las experiencias en torno a las opciones de titulación y la
eficiencia terminal, como es el caso de la disertación de Ofelia Janeth
Chávez, Mayra Lizzete Vidales Quintero y Edna Elizabeth Alvarado Mascareño,
de la Universidad Autónoma de Sinaloa. Otra experiencia en el trabajo con
el alumnado que realiza el Servicio Social es la que nos comparte el trabajo intitulado \textit{Historia ACA:\ Una experiencia en el proceso terminal de los alumnos de la FES Acatlán} de la autoría de Patricia Montoya Rivero y
María Cristina Montoya Rivero. Mientras que al
estudio del \textit{Clima social en estudiantes universitarios:
análisis obligado para el mejoramiento de la eficiencia terminal}, nos invitan
Ivett Reyes-Guillén y Carlos Arcos Vázquez de
la Universidad Autónoma de Chiapas. De la misma manera, en \textit{La
estancia profesional en el plan de estudios de la
Licenciatura en Historia de la Facultad de Humanidades Universidad Autónoma
del Estado de México}, de Georgina Flores
García y Marcela J. Arellano González, podemos encontrar una interesante reflexión
en torno al significado de este ejercicio práctico
disciplinar que le permite al alumnado vincularse al mercado laboral.
Finalmente, esta parte del libro concluye con una experiencia en el área de investigación:  el trabajo \textit{Historiografía de la guerra de castas en Campeche: una historia fragmentaria} de  Miriam  Edith León Méndez.

El tercer apartado del libro se dedica a los
\textit{Modelos y programas de tutorías en las
Licenciaturas en Historia}. En él se abordan
principalmente los distintos modelos que han puesto en marcha las
universidades. Ese es el caso, por ejemplo, de \textit{El Modelo
Institucional de Tutorías en la Universidad Autónoma de Aguascalientes y
sus implicaciones en la Licenciatura en Historia} de
Laura Elena Dávila Díaz de León; así como en 
\textit{El Programa de Tutorías en la Licenciatura en Historia
de la Universidad Autónoma de Ciudad Juárez} de María
Socorro Aguayo Ceballos y Ana Karent Muñoz Chávez, 
\textit{La tutoría académica: una experiencia de vida Facultad
de Humanidades Universidad Autónoma del Estado de México} de Georgina
Flores García y  Belén Benhumea Bahena,
y en \textit{La tutoría un compromiso de apoyo a lo largo de la
formación de los estudiantes de la Licenciatura en Historia de la UATx} de 
Juliana Angélica Rodríguez Maldonado y
Teodolinda Ramírez Cano. En todos estos trabajos se reseñan las
experiencias y retos que han significado para la labor docente, la asesoría
académica dirigida al alumnado que cursa estos programas educativos. Tema que es de capital importancia a la hora de intentar mejorar los indicadores relativos a la eficiencia terminal  y la deserción.

\enlargethispage{1\baselineskip}
Por último, el libro termina con el apartado
\textit{La acreditación, certificación e innovación en
los Programas Educativos de Historia}, en el cual hemos reunido un
conjunto de experiencias en torno a los procesos de evaluación, tanto de los
programas educativos como del profesorado.
Procesos en los que se involucran tanto las licenciaturas como los y las
docentes, y de los cuales dependen las acreditaciones y evaluaciones
individuales, así como los indicadores de el PROMEP y el SNI, que son importantes
para la competitividad y la calidad de los programas educativos. En esta vertiente se inscriben los trabajos intitulados \textit{Evaluación de PE:\
estrategia de gestión institucional para mejorar la calidad de la
educación} de Alfonso Mercado Gómez y
María de los Ángeles Sitlalit García Murillo, \textit{La
Licenciatura en Historia de la UAZ en la antesala del COAPEHUM.\ Los
maestros frente a lo real, lo posible y lo deseable} de
María del Refugio Magallanes Delgado, Norma Gutiérrez
Hernández y  Ángel Román Gutiérrez, \textit{Entre PROMEPsas y
PROEDsas} de Gloria Pedrero Nieto y Graciela Isabel Badía Muñoz,
\textit{Programas de Estudio en Modalidad no Convencional} de
María de los Ángeles Sitlalit García Murillo,
Beatriz Rico Álvarez y Sonia Bouchez
Caballero, y \textit{Avances y problemáticas detectados en el
nuevo plan de estudios (2011) de la Licenciatura en Historia de la
Universidad Autónoma de Zacatecas} de la mano de Lidia Medina Lozano,
José Luis Raigoza Quiñónez y  Luis Román Gutiérrez.

Esta obra colectiva es el resultado del interés de cada uno\slash{}a de los
integrantes de la RENALIHCA y del profesorado que se dedica a la
investigación, docencia y difusión del conocimiento histórico en el nivel
superior; y forma parte de la memoria colectiva de los procesos y avatares 
que se viven en el interior de las universidades públicas en México.

\enlargethispage{1\baselineskip}
Para concluir, esperamos que esta recopilación consiga sumar algunos votos 
a la propuesta de Andrea Sánchez Quintanar de que <<(\ldots) los problemas de la enseñanza de la historia deben ser propuestos y resueltos por los historiadores, sin soslayar, el valioso apoyo de los especialistas 
relacionados con el estudio de la educación.>> 

\bigskip
{\raggedleft\ La Mora, Puebla, Agosto de 2014\par}
\newpage

{\bfseries Referencias}

\medskip
Pagès i Blanch, Joan (2004), <<Enseñar a enseñar historia: la formación
didáctica de los futuros profesores de historia>>, en Nicolás, Encarna y
Gómez Hernández, José A. (coords.), \textit{Miradas a la Historia.
Reflexiones historiográficas en recuerdo de Miguel Rodríguez  Llopis},
Murcia, Universidad de Murcia. Consultado el 18 de junio de 2014 en 
\url{https://www.um.es/campusdigital/Libros/textoCompleto/historia/12pages.pdf}

\begin{sloppypar}
Plá, Sebastián (2012), <<La enseñanza de la historia como objeto de
investigación>>, en \textit{Secuencia}, Revista de Historia y Ciencias
Sociales, núm.\ 84, septiembre-diciembre. pp.\ 163\textendash{}184. En
\url{http://secuencia.mora.edu.mx/index.php/Secuencia/article/view/5967}
Consultado el 22 de junio de 2014.
\end{sloppypar}

\textit{Plan de Estudios de la Licenciatura en Historia} (MUM 2009) de la
Benemérita Universidad Autónoma de Puebla. México, BUAP.

\textit{Plan de estudios de la Licenciatura en Historia (2014)} de la
Universidad Autónoma de Aguascalientes, México, UAA\@.

\begin{sloppypar}
\textit{Plan de estudios de la carrera de Historia} de la Universidad de
Guadalajara, en
\url{http://guiadecarreras.udg.mx/licenciatura-en-historia/#03}, consultado
el 26 de junio de 2014.
\end{sloppypar}

\textit{Plan de Estudios de Historia} de la Escuela Nacional de Antropología
e Historia. En \url{http://www.enah.edu.mx/index.php/plan-l-his}, consultado el
27 de junio de 2014.

\textit{Plan de Estudios de la Licenciatura} en Historia de México de la
Universidad Autónoma de Hidalgo. Disponible en
\url{http://www.uaeh.edu.mx/campus/icshu/investigacion/aaha/oferta.html}.
Consultado el 26 de junio de 2014.

\textit{Plan de Estudios de la Licenciatura en Historia} de la Universidad
Juárez Autónoma de Tabasco (2010), México, UJAT\@.
%\newpage

\textit{Plan de Estudios de la Licenciatura en Historia} de la Universidad
Autónoma Metropolitana. En \url{http://www.uam.mx/licenciaturas/index.html}.
Consultado el 26 de junio de 2014.

\textit{Plan de Estudios de la Licenciatura en Historia y Estudios en
Humanidades} de la Universidad Autónoma de Nuevo León. Disponible en
\url{http://www.filosofia.uanl.mx:8080/web/?page_id=690}. Consultado el 27
de junio de 2014.


\begin{sloppypar}
\textit{Plan de Estudios de la Licenciatura en Historia} de la Universidad
de Sonora. En \url{http://www.uson.mx/oferta_educativa/pe/lichistoria.htm#esp}
Consultado el 26 de junio de 2014.
\end{sloppypar}

\begin{sloppypar}
\textit{Plan de Estudios de la Licenciatura en Historia} del Instituto Mora. En
\url{http://www.mora.edu.mx/Docencia/LicenciaturaHistoria/SitePages/Inicio.aspx}
Consultado el 26 de junio de 2014.
\end{sloppypar}

\textit{Plan de Estudios de la Licenciatura en Historia} (2006) de la
Universidad Autónoma de Yucatán. Disponible en
\url{http://www.antropologia.uady.mx/programas/historia/ejes_plan.php}.
Consultado el 27 de junio de 2014.

\enlargethispage{2\baselineskip}
\begin{sloppypar}
Sánchez Quintanar, Andrea (1995), <<Enseñar historia en la universidad y
fuera de ella>>, en \textit{Perfiles Educativos}, núm. 68, abril-junio,
México, Instituto de Investigaciones sobre la Universidad y la Educación.
En \url{http://www.redalyc.org/articulo.oa?id=13206809}. Consultado el 20 de junio
de 2014.
\end{sloppypar}

\begin{sloppypar}
Proyecto Tuning 2004\textendash{}2008. En \url{http://tuning.unideusto.org/tuningal/}
Consultado 15 de junio de 2014.
\end{sloppypar}%%%

%\clearpage\setcounter{page}{1}
\thispagestyle{empty}
\phantomsection{}
\addcontentsline{toc}{chapter}{La historiografía como eje articulador\\ de la 
enseñanza de la historia\newline $\diamond$
\normalfont\textit{Dr.\ Saúl Jerónimo Romero
}}
{\centering \large {\scshape La historiografía como eje articulador de la 
enseñanza de la historia }}
\par
\markboth{la formación del historiador}{la historiografía}
\setcounter{footnote}{0}

\bigskip
\begin{center}
{\bfseries Dr.\ Saúl Jerónimo Romero}\\
{\itshape Departamento de Humanidades}
\end{center}
%\enlargethispage{1\baselineskip}

\bigskip
Hace tres años en la VI reunión de RENALIHCA realizada en la UABC 
presenté un trabajo relativo a la formación del historiador en el siglo 
XXI. En dicha conferencia sostenía que a pesar de los avances en la 
teoría de la historia, el giro lingüístico, el giro espacial y ahora el 
giro cultural, en México se seguía enseñando la disciplina histórica de 
manera muy similar a como lo harían los historiadores del siglo XIX o, en 
última instancia, como a principios del siglo XX. También mencioné que 
había un fuerte divorcio entre la práctica histórica y su teorización, 
lo que limitaba la reflexión historiográfica y daba como resultado la 
realización de trabajos de investigación descriptivos, de recorta y 
pega. En ese diagnóstico suponía que el problema radicaba en una 
enseñanza memorística y repetitiva; basada en técnicas y no en 
reflexiones teóricas sobre el saber histórico. Este tipo de enseñanza, 
decía en aquel momento, provoca serios problemas, entre otros, que 
nuestros estudiantes sean poco competitivos comparados con otras 
disciplinas y con historiadores de otras latitudes, que están 
preparados con un enfoque multi y transdisciplinario, lo que les 
permite abordar problemas complejos, tanto del pasado como del 
presente; los prepara con mayor éxito para el ingreso a los posgrados y 
los dota de herramientas para ser investigadores y no solo 
compiladores de información, lo que en principio los orienta hacia uno 
de los campos de trabajo más codiciados, la investigación, y por otro 
lado, los prepara para ser mejores docentes en la medida en que no 
están repitiendo libros de texto, sino que constantemente están 
investigando sobre las materias que imparten, etcétera.

Al tratar de ubicar en dónde reside el problema, he planteado la 
siguiente hipótesis de trabajo: A pesar de que todos los programas de 
estudio de las licenciaturas en historia se han modificado en los años 
recientes, la mayoría de estas modificaciones se han realizado 
obedeciendo criterios exógenos a la disciplina y a las discusiones 
teóricas en torno a las ciencias sociales, y, salvo algunas excepciones, el eje 
teórico que debería articular el mapa curricular. Esta desarticulación 
provoca que no haya correspondencia entre objetivos, perfil de egreso y 
el mapa curricular. 

En las siguientes líneas se abunda en esta discusión. A partir del 
análisis de treinta y dos planes y programas, se exponen algunas de las 
transformaciones relevantes, se señalan algunas limitaciones y, 
finalmente, se hace una propuesta sintética de algunos contenidos 
mínimos. 

\medskip
\textbf{Las evaluaciones}
\enlargethispage{1\baselineskip}

Los criterios de evaluación que imponen las agencias evaluadoras y 
acreditadoras son mecanismos de medición estándar, aplicable para 
cualquier carrera que se imparte en el nivel superior, por lo que no se 
ocupan de plantear mecanismo de evaluación de calidad. Por ejemplo, las 
estrategias de CONACyT es impulsar la vocación de investigación entre 
los jóvenes del bachillerato y fortalecer los doctorados, por lo tanto 
las licenciaturas y las maestrías no son prioritarias, por lo que los 
recursos escasean para las licenciaturas y maestrías. Además priorizan 
la eficiencia terminal sobre cualquier otro indicador de calidad. Estas 
instancias están más preocupadas por resultados numéricos como: índices 
de reprobación, uso intensivo o no de las tecnologías de la 
comunicación,  eficiencia de egreso, eficiencia terminal, 
diversificación de las formas de titulación, mecanismos de evaluación 
para profesores y alumnos; número de profesores de tiempo completo, 
cuántos de ellos pertenecen al SNI, cuántos al PROMEP, etcétera; si las 
aulas tienen o no suficiente equipo, cuántos libros hay en la 
biblioteca, etcétera. Las instituciones han procurado cumplir con estos 
criterios porque son políticas nacionales vinculadas a financiamiento. 
Así que esos indicadores numéricos hay que cumplirlos.

Celebro que estas políticas no se hayan ocupado de la orientación 
académica de los planes y programas, pues creo que sería un enorme 
desastre que desde alguna oficina gubernamental se decidiera la vida 
académica de las universidades. También me parece bien que estos 
mecanismos de evaluación hayan servido para que en las universidades 
tengan más y mejor acceso a Internet, para que haya más computadoras y 
acceso a bibliotecas físicas y virtuales; para que se impartan cursos 
de aprender a aprender, de escritura de géneros universitarios, de 
diversos tipos de \textit{software}, etcétera. No me alegra tanto que en la 
búsqueda de mejor eficiencia terminal se titule a los alumnos por 
promedio, pues a nuestros mejores alumnos les evitamos realizar una 
pequeña investigación, que es una de la labores relevantes del quehacer 
del historiador. En algunos casos se ha procurado la aprobación de 
seminarios que más o menos cumplen esa función.  No obstante estos 
apoyos, considero que en muchos casos no se ha logrado proponer un 
currículo que permita tener egresados con criterio, que entiendan las 
dimensiones históricas de tiempo y espacio, que estén ciertos de 
necesidad de interpretar y del papel que juega el lenguaje en las 
construcciones historiográficas. 

Incluso intentos como el proyecto Tuning Latinoamérica, que ha tratado 
de ofrecer una guía de lo que debían contener planes y programas en 
materia de capacidades, habilidades y conocimientos básicos para cada 
una de las carreras, no se ocuparon del diseño del mapa curricular. Sus 
recomendaciones son tan genéricas que se prestan a confusión o llegan 
a ser poco prácticas. En un trabajo anterior hice una crítica a esa 
propuesta, en particular lo relativo al espacio, donde quedaba claro 
que la noción de espacio propuesta no se enfocaba de manera histórica, 
si no como algo ya dado, ahistórico, y cargado de ideología 
nacionalista, regionalista o estatal.

\medskip
\textbf{Objetivos declarados}
\enlargethispage{1\baselineskip}

La mayor parte de los planes y programas de historia tratan de formar
profesionales dedicados a la investigación y a la problematización de asuntos
históricos; en algunos casos con una visión crítica; en otros, desde una
perspectiva racional; otros más hacen hincapié en recuperar la función que
juega la historia en vincular presente, pasado y futuro; y en otros se
enfatiza en las funciones docentes, de difusión y preservación del
patrimonio histórico cultural.\footnote{Véase los programas de BUAP, UNAM,
Aguascalientes, etcétera. El de la Universidad Juárez de Tabasco dice lo
siguiente: Formar profesionales que sean capaces de interpretar
científicamente los procesos históricos realizando estudio de las diversas
dimensiones del hombre y la sociedad a través del tiempo, para revalorar el
presente y participar en la transformación del futuro. En \textit{Licenciatura en
historia}, Universidad Juárez de Tabasco: 
\url{http://www.ujat.mx/interioradentro.aspx?ID=157&NODO=78}} 

En todos ellos se puede apreciar la intención de formar profesionales 
que puedan manejar información con fines investigativos, y el manejo de 
resultados varía de acuerdo a las circunstancias sociales en las que 
opera cada programa. En todos ellos se hace mención de la necesidad de 
formar a los futuros historiadores con los fundamentos teóricos y 
metodológicos necesarios para llevar a cabo estas tareas. Estas 
pretensiones me parecen fundamentales para el quehacer histórico; 
sin embargo, considero que todavía estamos distantes de lograr la 
consecución de estos objetivos con eficiencia. Vale la pena destacar 
que en cualquiera de las diversas ocupaciones que tienen los egresados, 
en todas se requiere saber allegarse conocimientos y tener claridad 
sobre la responsabilidad sobre lo que se está publicando, difundiendo o 
enseñando. Así entonces, dos elementos son básicos en la formación del 
historiador: saber investigar y tener claridad sobre su responsabilidad 
sobre su producción.\footnote{En el Plan de estudios de la UAEM se dice 
al respecto: Como características colaterales ser capaz de investigar, 
insertarse en el paradigma del aprendizaje con el manejo de recursos 
didácticos y técnicos innovadores, así opios procedimientos. Esta 
capacidad le permite no sólo apropiarse permanentemente de nuevos 
conocimientos sino también de ser capaz de transmitir en su actividad 
como docente, esta misma capacidad (Plan de Estudios de la UAEM 
2004, p.\ 2).} 

A propósito, recupero un artículo de Hayden White publicado en el 
número de Nexos de mayo de 1982, nótese que es un texto de hace 32 
años, en el que provocativamente decía de la profesión histórica no era 
ni ciencia ni arte, pero que sin embargo los historiadores desde el 
siglo XIX habían reclamado los privilegios del artista y el científico, 
pero al mismo tiempo se negaban a someterse a los rigores críticos y 
creativos del arte y la ciencia (White 1982). Explica que ello se debe a dos causas: 

Una es la naturaleza de la profesión histórica misma, quizá la 
disciplina conservadora  {\itshape par excellence}. Desde mediados del siglo XIX 
los historiadores acostumbran fingir una especie de ingenuidad 
metodológica voluntaria, que originalmente servía a un buen propósito: 
protegerlos de la tendencia a abrazar los sistemas explicativos 
monistas, el idealismo militante en la filosofía o el positivismo 
militante en la ciencia (\textit{ibid.}).

White después retoma la crítica que se le hizo a la disciplina 
histórica a principios del siglo XX; la cual, le acusaba de ser 
acomodaticia y políticamente servil. Me parece que esa crítica ya no es 
aplicable a todos los tipos y formas de hacer historia que se hacen 
desde el último tercio del siglo XX y lo que va de este siglo. Sobre 
todo, porque los emisores de discurso histórico se han multiplicado y 
las versiones oficiales de la historia, son unas más entre muchas 
otras. Basta considerar desde dónde y con qué fines se investiga, por 
mencionar algunos ejemplos: universidades públicas y privadas, 
estatales y nacionales; centros de investigación privados y públicos; 
organizaciones no gubernamentales y asociaciones diversas; centros 
culturales e incluso religiosos. Además, los fines de estas historias 
varían en función de los intereses sociales desde los que se 
construyen, quisiera poner como ejemplo las investigaciones que tienen 
que ver con justicia transicional, que evidentemente no reproducen la 
historia desde el punto de vista gubernamental. 

\enlargethispage{1\baselineskip}
En lo que si estoy de acuerdo con White es la idea de que:

La historia tiene hoy la oportunidad de allegarse las nuevas 
perspectivas del mundo que ofrecen una ciencia y un arte igualmente 
dinámicos. La ciencia y el arte han trascendido los conceptos 
tradicionales que les exigían ser copias literales de una realidad 
presumiblemente estática; y ambas han descubierto el carácter 
esencialmente provisional de las construcciones metafóricas destinadas 
a comprender un universo dinámico. Así confirman implícitamente la 
verdad a la que llegó Camus cuando escribió <<Antes fue cuestión de 
saber si la vida tenía o no un significado para vivirla. Ahora está 
claro, al contrario, de que será vivida mejor si no tiene sentido>>. 
Podemos corregir la afirmación para leer: será vivida mejor si tiene 
muchos sentidos en lugar de un solo sentido (\textit{ibid.}).
\newpage

%\medskip 
Si este es el tipo de historiador necesario en el siglo XXI, 
vale la pena reflexionar sobre cuáles serían los contenidos necesarios 
para que los historiadores tuvieran esas posibilidades de innovación y 
de responsabilidad sobre lo que investigan, publican y enseñan. 

\medskip
\textbf{Los Planes de Estudio}

Ante este panorama, creo que es esencial que el futuro historiador 
curse un eje teórico sólido, con materias teóricas, tales como: 
nociones básicas del conocimiento historiográfico, teoría de la 
historia, epistemología histórica, historiografía general, etcétera. De 
tal suerte que pueda reconocer: cuál es su función como productor de un 
tipo de conocimiento; cuál es la naturaleza del conocimiento que 
produce el historiador a lo largo de su carrera profesional; el papel 
que juega el lenguaje en la construcción del conocimiento humano y en 
particular del histórico; las diferencias entre filosofía de la 
historia y teoría de la historia; las posibilidades de conocer el 
pasado; la historicidad de la práctica histórica y de su teoría. 
Asimismo, el reconocimiento de los conceptos históricos claves como 
tiempo, espacio, discurso. Así como el conjunto elecciones sobre las 
que se estructura el discurso histórico y la responsabilidad ética y 
profesional que conlleva su práctica.

¿Todo esto cuándo? Desde el primer día en la universidad y a lo largo 
de toda la carrera. Este tipo de materias le servirá para entender y 
analizar la críticamente la formación que están recibiendo; así podrán 
valorar adecuadamente historiografía que utilizaran durante su proceso 
formativo y tendrán consciencia de que el objetivo central de su 
formación es generar conocimiento histórico y divulgarlo y de ninguna 
manera obtener un conocimiento enciclopédico. Es decir que no están 
aprendiendo todo lo que ha ocurrido a la humanidad desde los griegos 
hasta nuestros días y que en su momento a ellos les tocará hacer 
selecciones similares. Proponer problemas de investigación de su 
localidad, de México o del mundo global y estar preparados para 
seleccionar las fuentes y métodos para resolverlos.  

Al momento de diseñar el plan de estudios los profesores encargados 
siempre encuentran con el dilema, de qué incluir y qué dejar fuera. Es 
decir, que también el currículum es una selección que hacen un grupo de 
profesionistas y expertos. Por ejemplo, si necesitamos que el alumnos 
tengan una ubicación, en términos histórico espaciales, de la dimensión 
local hasta los grandes asuntos mundiales, que le den un bagaje 
histórico-cultural suficiente para entender sus objetos de 
investigación. Nos vamos a encontrar con la imposibilidad de incluir 
todo ese conocimiento en el currículo de los planes de estudio, ya que 
en cuanto los alumnos tienen que familiarizarse con la historia mundial 
o universal, que cabe mencionar siempre es un problema delimitar lo que 
es historia mundial o universal, sin caer en eurocentrismo o en 
algún otro ismo.\footnote{Por ejemplo el perfil de egreso de la 
Universidad de Ciencias y Artes de Chiapas es el siguiente: 
\begin{Obs}
\item[-] Comprenderá los procesos sociales, políticos, culturales, geográficos y 
económicos que han caracterizado a la historia mundial, americana y 
nacional, desde la prehistoria hasta nuestros días. 
\item[-] Conocerá la historia de Chiapas y sus regiones, desde la época prehispánica 
hasta el pasado más reciente, y en el contexto de la historia nacional 
y mundial. 
\item[-] Adquirirá conocimientos acerca de las 
características y métodos de las principales tradiciones 
historiográficas a nivel mundial, y en particular de la historiografía 
de México y de Chiapas. 
\item[-] Contará con un amplio conocimiento 
de los planteamientos teóricos y metodológicos de las ciencias 
sociales. 
\item[-] Obtendrá conocimientos teóricos-prácticos para 
realizar actividades de investigación, docencia y difusión del 
conocimiento histórico.
\end{Obs}

En \textit{Folleto informativo}. Consultado en 
\url{http://www.unicach.mx/_/descargar/licenciatura/historia.pdf} } 


Siempre resulta complicado determinar qué tanto se incluye de los 
procesos que fueron transformando el espacio en: la época de los 
griegos y romanos, en la edad media, de los ocurridos en México y en 
los Estados de la república y en las localidades. Si se intentara, el 
plan de estudios quedaría como un rompecabezas con piezas de diferente 
escala que no encajan entre sí. Creo que sería más fácil una materia en 
dónde se explique la noción de espacio histórico, que permita al alumno 
comprender las diversas formas en que se divide el espacio, de acuerdo 
a los sistemas políticos, interacciones sociales, relaciones 
comerciales, interacciones culturales, etcétera y que para 
comprenderlas y delimitarlas es necesarios tomar en cuenta su 
historicidad. Es decir, un planteamiento más teórico que descriptivo, y, 
por supuesto,  en la enseñanza se puede hacer uso de todos los ejemplos 
que se quiera. 

Esto aplica para todas las áreas de conocimiento que se quieran 
abordar: por ejemplo para las historiografías, las relativas a la 
historia de la historiografía, la manera tradicional de enseñarlas es 
cronológicamente desde los griegos hasta lo más cercano a nuestros 
días, muy pocas son las facultades que cuando menos han cambiado a 
enseñarla a través de corrientes historiográficas o través de las 
metodologías de cada tipo de historiografía.\footnote{Mediante 
corrientes lo hacen: la ENAH y la Universidad de Sinaloa y a través de 
metodologías enseñan, la Universidad de Aguascalientes y la del Estado 
de México. Todas las demás siguen el método tradicional.}  La mayoría 
sigue siendo en estricto orden temporal y por autores. Ninguna ha 
cambiado a enseñarla por problemas, tales como sujeto, discurso, 
géneros discursivos, memoria, procesos de significación, identidades 
históricas, usos de la historia, etcétera. Por ejemplo, la Benemérita 
Universidad Autónoma de Puebla tiene como primer párrafo de su perfil 
de egreso: <<El historiador egresado estará comprometido con la 
preservación de la memoria histórica y recuperará la historia de su 
espacio regional, nacional y universal, manteniendo una visión crítica 
del presente vinculándolo con el pasado.>>\footnote{BUAP, 
Licenciatura en Historia, en 
\url{http://cmas.siu.buap.mx/portal_pprd/wb/EDUCATIVA/licenciatura_en_historia_1} 
} Y no hay en su mapa curricular ninguna materia en la que directamente 
se trate la problemática de la memoria histórica. Quizá en las 
relativas en Área de Sistematización de Documentos y Patrimonio 
Históricos, pero la comprensión de los fenómenos asociados a Memoria 
histórica no son exclusivos de los referidos a patrimonio. 

Un enfoque por grandes problemáticas historiográficas permitiría exponer
estos problemas de manera comparativa, alentaría la curiosidad por
descubrir cómo se han tratado estos problemas en diferentes momentos,
culturas o situaciones. Es decir que estas materias dejarían de ser
informativas para ser un elemento más de formación, de tal suerte que los
alumnos de historia tendrían un contacto permanente con el pensamiento
histórico y su problematización. 

De otra forma, la información que reciben los alumnos es sumamente 
parcial y parece no tener otra justificación esa selección de hechos y 
acontecimientos que se les enseñan, más que la costumbre. Donde parece 
que lo central, por ejemplo, es que lo alumnos recuerden que Tucídides 
escribió Guerra del Peloponeso y no la discusión, sobre evidencia, 
testigo, crítica etcétera que abrió su texto, preocupaciones que serán 
importantes entre los historiadores de los siglos XIX y XX. Si no se 
quiere cambiar el modelo de enseñanza por problemas, estas materias 
deberían incluir discusiones sobre horizonte, cultural, enunciación, 
recepción, análisis de textos, etcétera. La historiografía llamada 
clásica, griega y romana se presta adecuadamente para estos debates y 
muestra con claridad que no hay una sucesión evolutiva de los modos, 
formas y métodos de hacer historia. Hay momentos que se retoman 
problemáticas que se discuten con perspectivas distintas, propias del 
momento de enunciación o recepción del contexto del momento. Un enfoque 
de esta naturaleza integraría los elementos teóricos con las diversas 
prácticas historiográficas. 

Al revisar los planes y programas de estudio, encuentro que las uni\-ver\-si\-da\-des
de Coahuila, Morelos, San Nicolás de Hidalgo, Baja
California, Baja California Sur, FES Acatlán, Facultad de Filosofía y
Letras, UNAM y la ENAH han incluido cursos tales como: introducción a la
historia, introducción a la teoría de la historia, al pensamiento
histórico, a la investigación histórica, la interdisciplina, lo cual me
parece un avance significativo, aunque en algunos casos habría que ver el
contenido exacto de las materias de introducción, pues algunos de estos
cursos se ocupan de temas tales como introducción al uso de archivos y
fuentes documentales, lo cual tiene su importancias, pero no está enfocado
en el mismo sentido que esta propuesta. 

No obstante, en la mayoría de los planes no hay continuidad de esta 
primera introducción al ámbito teórico del quehacer historiográfico; en 
casi todos los planes incluyen dos cursos relativos a Teoría de la 
historia, salvo en los de las universidades Iberoamericana y de San 
Luis Potosí que contienen tres cursos. La mayoría de las instituciones 
los imparten entre el segundo y quinto semestre. En otras instituciones 
no hay una propuesta sistemática de estos contenidos. Pongo algunos 
ejemplos: en Universidad de Tamaulipas cursan esas materias en 
el cuarto año; la Universidad de Nuevo León solo ofrece un curso de 
teoría de la historia en quinto semestre y en sexto semestre se ofrece 
como optativa la materia de Filosofía de la historia; la Universidad de 
Sinaloa solo brinda en el primer semestre Introducción a la teoría de 
la historia y después ya no hay materias teóricas, sino únicamente 
historiografías  específicas. La Universidad Juárez Autónoma de Tabasco 
es obligatorio cursar en tercer semestre Filosofía y Teoría de la 
historia y después no hay más materias teóricas. En la UAM Iztapalapa 
se imparten en el eje teórico tres materias denominadas, Teoría y 
Problemas Sociopolíticos Contemporáneos, nada que ver con la propuesta 
teórica de la historiografía. En la universidad de Guadalajara el curso 
de Teoría de la Historia es optativo. En fin, no se trata de agotar 
todos los ejemplos; pero sí mostrar que en el currículo de la mayoría 
de las licenciaturas éste no es un eje central, o tiene poca 
integración con el resto el de las materias que cursan. 

Otro asunto que me parece relevante comentar es que algunas 
instituciones han optado por enseñar una forma específica de hacer 
historia, ya sea por corriente historiográfica, autor o pensamiento 
político. En el plan de estudios de la Universidad Juárez de Tabasco se 
reconoce:

Aunque en el caso de las Ciencias Sociales es difícil mantener una 
posición de neutralidad, ya que los que se ocupan de estas ciencias se 
inclinan por alguna corriente en particular ya sea por su formación y 
posición de clase, resulta peligroso establecer en una estructura 
curricular una determinada orientación hacia alguna de las diversas 
corrientes de la historia, ya que no todos coinciden en esa determinada 
corriente o en muchos casos pueden diferir completamente, por lo cual 
es adecuado plantear una posición abierta e 
interdisciplinaria.\footnote{Universidad Juárez Autónoma de Tabasco 
División Académica de Ciencias Sociales y Humanidades 
\textit{Reestructuración del Plan de Estudios de la Licenciatura en 
Historia}. En 
\url{http://www.archivos.ujat.mx/2014/dacsyh/plan_estudios/LIC.\%20EN\%20\%20HISTORIA.pdf} 
}
\newpage

A pesar de este reconocimiento de la importancia de ofrecer diversas 
opciones, solo se ofrecen al final un Taller de historia oral y una 
materia específica de historia de género. Seguramente es un problema 
relacionado con la planta docente. En otros momento se priorizó el 
materialismo histórico. La BUAP, por ejemplo, menciona que desde la 
creación de la licenciatura ha habido siete planes de estudio con 
orientaciones distintas: una tradicional, otra plegada al materialismo 
histórico y la actual, que sigue a la Escuela de los \textit{Annales}.\footnote{ 
BUAP, Facultad de Filosofía y Letras Programa de de Licenciatura en 
Historia. Informe de autoevaluación para la acreditación del Programa 
Educativo de la Licenciatura en Historia de la Facultad de Filosofía y 
Letras de la  Benemérita Universidad Autónoma de Puebla  En 
\url{http://historia.dosmildiez.net/revisionplan/wp-content/uploads/2007/10/acreditacion_colegio-de-historia-buap.pdf}, pp.~28--29} Evidentemente, desde la aparición de los \textit{Annales} en 1929 a 
la fecha ha corrido mucha tinta y muy diversas formas de hacer 
historia, pero a pesar de ello no es la única corriente. (¿Y si un 
alumno quiere hacer historia intelectual?). Quizás con los elementos 
teóricos adecuados los alumnos podrían escoger problemas de 
investigación y luego elegir la conceptualización teórica que les 
permita comprender ese fenómeno, en donde no solo las corrientes de 
pensamiento histórico le brinden las herramientas para comprender esos 
fenómenos, sino los de las ciencias sociales en su conjunto. Esto le 
permitiría elegir el método adecuado a su proyecto de investigación y 
buscar allegarse los conocimientos necesarios para desempeñar su 
actividad. 

Estoy consciente que las plantas académicas a veces están formadas en la
misma escuela histórica, son seguidoras de uno o varias figuras señeras,
todo eso está bien, pero sí creo que debe haber la apertura y la
disposición para apoyar a los alumnos en sus indagaciones y apoyarlos
metodológicamente, en estructura investigativa y no condicionarlos a seguir
determinada corriente histórica. 

No es mi intención agobiarlos con esta descripción de planes y programas, y
tampoco que pensemos que le historiador se tiene que dedicar a la teoría o
filosofía de la historia, pero retomo aquí las palabras de Luis Villoro, quien
en 1960 escribió un artículo en \textit{Historia Mexicana} y llegaba a la siguiente
conclusión: 

\begin{quotation}
Creemos que los historiadores americanos necesitan plantearse con mayor 
gravedad el problema del  objeto y métodos de su ciencia. Con ello no 
pedimos que hagan filosofía. Quien tal pensara sólo demostraría tener 
una pobre idea del historiador, al reducirlo al papel de simple técnico 
o ingenuo narrador. Al historiador compete reflexionar sobre los 
fundamentos y fines humanos de su ciencia. Sólo él puede formular 
nuevas hipótesis de trabajo y aplicarlas en procedimientos completos; 
mientras no haga esto, todas teorías filosóficas acerca de la historia 
serán vacías especulaciones. Por eso, las grandes reformas de la 
historiografía un nunca fueron resultado de los filósofos de la 
historia en cuanto tales, sino de los mismos historiadores: Solo si el 
historiador cobra cabal conciencia de la especificidad de su objeto y 
redescubre en él la vida creadora del hombre en toda su riqueza, sólo 
si se percata de la dignidad de su función humana, podrá recuperar el 
papel de director en la sociedad que antaño le correspondiera (Villoro 1960). 
\end{quotation}

Este enfoque parece ir ganando adeptos en las licenciaturas de San Luis
Potosí, la UNAM y la ENAH y en algunas otras instituciones, ya tenemos
cuando menos cinco materias que apuntan a comprender los ejes teóricos, pero
considero que todavía no es suficiente. Las propuestas contemporáneas de
trabajo historiográfico requieren: una cultura histórica, que se puede
formar únicamente si hay una cabal comprensión de la naturaleza del aporte
que hace la historiografía al conocimiento historiográfico. Este enfoque se
reconoce por ejemplo en la presentación del plan de estudios de la Escuela
Nacional de Antropología: 

Las grandes transformaciones experimentadas por las sociedades en el siglo
XX han tenido como consecuencia la ampliación del horizonte intelectual de
la historia: la disciplina ha incorporado nuevos enfoques, temáticas y
metodologías. Este cambio se ha plasmado en el terreno de la teoría, pues
los cambios sociales y culturales del último siglo dieron paso a la
comprensión de los fenómenos sociales con una perspectiva más crítica, a la
vez que se sostiene la necesidad del conocimiento riguroso del acontecer
histórico.

\medskip
\textbf{\bfseries Propuesta:}

\medskip
Analizar la pertinencia de integrar un eje teórico a lo largo de toda la
licenciatura

Este eje debe comprender desde cursos introductorios, epistemología
histórica, filosofía de la historia, teoría de la historia y ética 

Las materias de historiografía deberían estar integradas por problemas tanto
las generales, como las nacionales y locales. 

El alumno tiene derecho a elegir el tipo de problemas que desea investigar,
los métodos que necesita utilizar y es responsabilidad de las plantas
académicas y de la institución proporcionarle las condiciones adecuadas
para el aprendizaje de su profesión.

Debe haber cursos de actualización para profesores para que puedan impartir
estos cursos. 
\newpage

\textbf{Referencias}

\bigskip
\textit{Plan de Estudios de la UAEM} (2004), Toluca, UAEM.

White, Hayden V. (1982).  «El peso de la 
historia», en \textit{Nexos}, mayo de 1982. Consultado en 
\url{http://www.nexos.com.mx/?p=4057}.

Villoro, Luis (1960),  «La tarea del historiador desde la perspectiva 
mexicana», en \textit{Historia Mexicana}, vol. IX, núm. 3, enero-marzo 
de 1960, pp. 329--339, El Colegio de México. Compilado en Evelia Trejo. 
\textit{La historiografía del siglo XX en México. Recuento, 
perspectivas teóricas y reflexiones}, México, UNAM, p.\  290.
