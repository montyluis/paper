%\documentclass{article}
%\usepackage{amsmath,amssymb,amsfonts}
%\usepackage{fontspec}
%\usepackage{xunicode}
%\usepackage{xltxtra}
%\usepackage{polyglossia}
%\setdefaultlanguage{spanish}
%\usepackage{color}
%\usepackage{array}
%\usepackage{hhline}
%\usepackage{hyperref}
%\hypersetup{colorlinks=true, linkcolor=blue, citecolor=blue, filecolor=blue, urlcolor=blue}
%% Text styles
%\newcommand\textstyleEndnoteAnchor[1]{\textsuperscript{#1}}
%\newtheorem{theorem}{Theorem}
%\title{}
%\author{Usuario}
%\date{2014-06-16}
%\begin{document}
%%\clearpage\setcounter{page}{521}
\thispagestyle{empty}
\phantomsection{}
\addcontentsline{toc}{chapter}{Avances y problemáticas detectados en el\newline nuevo plan de estudios (2011) de la\newline Licenciatura en Historia de la Universidad\newline Autónoma de Zacatecas\newline $\diamond$
\normalfont\textit{Lidia Medina Lozano,  José Luis Raigoza Quiñónez\newline y
Luis Román Gutiérrez}}
{\centering {\scshape \large Avances y problemáticas detectados en el nuevo plan\newline 
de estudios (2011) de la Licenciatura en Historia\newline de la Universidad Autónoma de Zacatecas}\par}
\markboth{la formación del historiador}{avances y problemáticas}
\setcounter{footnote}{0}

\bigskip
\begin{center}
{\bfseries Lidia Medina Lozano\\
José Luis Raigoza Quiñónez\\
Luis Román Gutiérrez}\\
{\itshape Universidad Autónoma de Zacatecas}
\end{center}

\bigskip
\textbf{Resumen}
\enlargethispage{1\baselineskip}

La licenciatura en Historia de la Universidad Autónoma de Zacatecas 
tuvo su origen en 1987 como Escuela de Humanidades y contaba con tres áreas 
terminales: Historia, Letras y Filosofía. Comenzó con un plan rígido y 
modular, que en pocos años se separó en licenciaturas, tomando cada una 
de ellas su propio rumbo disciplinar. Durante estos años, el plan de estudios 
de la Licenciatura en Historia se ha reformado en varias ocasiones.


Primeramente, en 1998 el programa quedó evaluado en el nivel 2 de los 
CIEES. Posteriormente se acreditó por los mismos CIEES en 2005, después 
de una reforma curricular al plan de estudios de 1999 que se implementó 
en 2004 bajo las características de una educación «abierta, flexible, 
integral y polivalente». Se volvió a acreditar en 2009 por el mismo organismo, 
y, dos años más tarde, nuevamente se reformó el plan de estudios, 
aumentando las materias optativas, y pasando de tres a seis las áreas 
terminales, sin perder las características del anterior. Con el mismo 
plan de estudios, en enero pasado abrió la Licenciatura a distancia con 
el objetivo inicial de capacitar cronistas con ayuda de la Asociación 
Nacional de Cronistas. En estos momentos ya se ha solicitado la evaluación 
por el Consejo para la Acreditación de Programas Educativos en 
Humanidades (COAPEHUM), la cual esperamos para este mismo año. En este 
trabajo intentaremos presentar los cambios curriculares del plan 2004 al 
de 2011, así como las ventajas y problemáticas que se han detectado a 
tres años de su implementación.\\ 
\textbf{Palabras clave:} Currícula, titulación, acreditación, flexibilidad,\\
créditos.


\bigskip
\textbf{Abstract}

%\enlargethispage{1\baselineskip}
The bachelor degree in history of the Autonomous University of 
Zacatecas had its origin in 1987 like School of Humanities with three 
terminal areas: History, Letters, and Philosophy. It began with a 
modular rigid plan that in few years each one took its own course 
separated by bachelors degrees. In all these years, the curriculum of 
the Bachelor Degree in History has reformed in several occasions.

\begin{sloppypar}
Firstly, in 1998 the bachelor was evaluated in level 2 of the CIEES. 
Later it was acredited by the same CIEES in 2005, after a curricular 
reform that was implemented in 2004 under the characteristics of an 
«opened, flexible, integral and multipurpose education».  The bachelor 
was reacredited in 2009 by the same organism, and two years later the 
curriculum reformed again increasing the optative matters and going 
from three to six the terminal areas without losing the characteristics 
of the previous one. With the same curriculum, in past January it 
opened the remote bachelor degree with the initial objective to enable 
chroniclers through the National Association of Chroniclers. At the 
present, was asked the evaluation by the Council for the Accreditation 
of Educative Programs in Humanidades (COAPEHUM) that we hoped for this 
same year. In this work we try to present the changes of curriculum 
2004 to the one of 2011 as well as their problems and advantages that 
have been detected to three years of their implementation.\\
\textbf{Keywords:} Curriculum, university degree,
accreditation, flexibility, credits.
\end{sloppypar}

\bigskip 
La licenciatura en Historia de la Universidad Autónoma de 
Zacatecas tuvo su origen en 1987 como Escuela de Humanidades y contaba con tres 
áreas terminales: Historia, Letras y Filosofía. Comenzó con un plan 
rígido y modular, que en pocos años se separó en licenciaturas, tomando 
cada una de ellas su propio rumbo disciplinar. En todos estos años, el plan de 
estudios de la Licenciatura en Historia se ha reformado en varias 
ocasiones.
\enlargethispage{1\baselineskip}

Primeramente, en 1998 el programa quedó evaluado en el nivel 2 de los 
CIEES. Posteriormente se acreditó por los mismos CIEES en 2005, después 
de una reforma curricular al plan de estudios de 1999 que se implementó 
en 2004 bajo las características de una educación «abierta, flexible, 
integral y polivalente». Se volvió a acreditar en 2009 por el mismo organismo, 
y, dos años más tarde, nuevamente se reformó el plan de estudios, 
aumentando las materias optativas, y pasando de tres a seis las áreas 
terminales, sin perder las características del anterior. Con el mismo 
plan de estudios, en enero pasado abrió la Licenciatura a distancia con 
el objetivo inicial de capacitar cronistas con ayuda de la Asociación 
Nacional de Cronistas. En estos momentos ya se ha solicitado la evaluación 
por el Consejo para la Acreditación de Programas Educativos en 
Humanidades (COAPEHUM), la cual esperamos para este mismo año. En este 
trabajo intentaremos presentar los cambios curriculares del plan 2004 al 
de 2011, así como las ventajas y problemáticas que se han detectado a 
tres años de su implementación. 


Como resultado del Congreso General de Reforma, los Planes de estudios 
de la Licenciatura en Historia han sido modificados en dos\linebreak ocasiones 
consecutivas con el objetivo de responder a las nuevas estrategias didácticas
e implementar un plan de estudios que tenga las características de ser abierto, flexible, 
integral polivalente y pertinente. El primer plan de estudios que fue 
rediseñado fue el del año 1999, el cual contenía 54 asignaturas divididas 
en un área informativa y un área formativa-productiva. El área informativa 
cubría las siguientes materias: dos Introducciones, 
cinco Historias de la Historia, seis Historias universales, seis 
Historias de México, cinco Historias generales y tres Historias del 
arte. Mientras que el área formativa-productiva abarcaba cuatro materias 
sobre Teoría y Filosofía de la Historia, tres metodologías, nueve Talleres, dos 
seminarios, siete instrumentales y sólo dos materias optativas. De la 
currícula anterior se llevaban de todas, y todas las materias no tenían opción 
y se centraban en una sola formación que no permitía ejes de especialización. 

\enlargethispage{1\baselineskip}
Al innovar nuestros planes de estudio para hacerlos compatibles con una currícula 
flexible y polivalente que respondiera a las exigencias 
gubernamentales y al mercado laboral, que nos exigían profesionales 
más aptos para el desempeño de la labor histórica y con habilidades 
extras, por un lado; y, por el otro, que diera satisfacción a las 
inquietudes de formación del propio alumnado, que deseaba formarse con 
habilidades personales para el desempeño de su quehacer de manera 
óptima.

Así, después de celebrar un congreso de egresados y otro de empleadores, y 
en concordancia con las políticas educativas, nuevamente los docentes nos 
reunimos en comisiones para analizar las nuevas propuestas y ofrecer 
una currícula variada y con tres ejes de especialización. Así, de esta 
manera fue que se conformó el Plan de estudios 2004 con las siguientes 
características: abierto, flexible, integral, polivalente 
y pertinente. En cuanto a su contenido, se buscaba que ofertara un mayor 
número de materias optativas, las cuales debían ser específicas para cada uno de los ejes de 
especialización: Docencia, Extensión y Difusión. Se trabajó a lo largo de 14 
meses, tras los cuales, y después de varias discusiones y negociaciones, se elaboró 
el Plan 2004 con 54 asignaturas o UDIs (unidades 
didácticas integradoras). Se comenzó con un Tronco Común de las 
Humanidades, con el que se pretendía que los programas académicos de Letras, 
Filosofía y Antropología participaran con el concurso de profesores y 
alumnos, de tal suerte que los alumnos convivieran con sus pares de dichos programas 
para que pudieran enriquecer su conocimiento de esas disciplinas y terminaran por 
definir su vocación.

No tuvimos eco en nuestras intenciones de que nuestros alumnos tomaran las 
introducciones a Letras, Filosofía y Antropología, pues esos programas 
no estaban en la misma sintonía, continuaban teniendo sus programas 
rígidos que impidieron que nuestros alumnos tomaran esas UDIs, de tal forma que 
solamente se logró que los profesores de dichos programas impartieran 
las introducciones referidas en nuestras propias instalaciones.

%\enlargethispage{1\baselineskip}
El plan, entonces, además del tronco común incluía cinco UDIs, y a partir 
del segundo semestre nuestra oferta educativa abarcaba treinta y tres 
materias optativas: quince disciplinares, nueve básicas y diez del 
programa académico común. Sin tomar en cuenta la Estancia profesional y el 
Servicio Social, el Seminario de Elaboración de Proyectos, las Teorías de la 
Comunicación y la de Liderazgo y desarrollo organizacional, que se sugerían 
para los tres ejes, el número de asignaturas optativas disciplinares 
por eje eran cinco, las cuales estaban muy ligadas a la especialización.  

%\enlargethispage{1\baselineskip}
Además, se incorporaron 4 ejes transversales. El de {\itshape Derechos humanos, 
género y democracia\/} que incluía trece materias;  el de {\itshape Identidad y 
valores\/} contemplaba veintiún UDIs; el de {\itshape Ecología, globalización y 
desarrollo sustentable\/} cubría trece materias; y, finalmente, un 
{\itshape eje integrador\/} compuesto por dos materias relacionadas con la práctica 
profesional. Es oportuno señalar que varias materias compartían dos o 
más ejes, tanto verticales como transversales. 

Así, esta oferta educativa estuvo conformada por: veintiuna materias 
obligatorias y treinta y tres optativas, siendo un 57\,\% de materias 
optativas, lo cual ya era un avance sustantivo para la oferta educativa. 
Esta situación hacía necesaria la tutoría como acompañamiento y 
asesoría académica del alumno a  la hora de la elección, tanto de su eje de 
especialización como de las materias optativas de su preferencia, 
volviendo atractivo el plan de estudios con miras al abatimiento de la 
deserción escolar y el incremento de la eficiencia terminal; pero, sobre todo, 
con el objeto de brindar una mejor preparación al alumnado para que
encare de mejor manera el mercado laboral. El trabajo del tutor, en este
sentido, es de suma importancia en el acompañamiento de los 
estudiantes. Desde que el estudiante ingresa a nuestro programa se le 
da un acercamiento general del Plan de estudios; una vez que es 
aceptado, se le nombra su Tutor, el cual le ofrece toda la información 
sobre el Plan de estudios, incluyendo los respectivos Reglamentos internos 
(el Reglamento General del Programa, el Reglamento de  Titulación y el 
Reglamento de Servicio Social).

\enlargethispage{1\baselineskip}
No obstante lo anterior y de que con estas acciones se pudo 
alcanzar el nivel 1 de los CIIES, se siguieron realizando congresos de egresados y 
empleadores con el objetivo de medir la capacidad de nuestros egresados 
y evaluar su inserción en el mercado laboral. Obtuvimos como resultado 
que nuestros egresados en su mayoría se estaban empleando como docentes y 
extensionistas y, en menor número, como investigadores.

 
Con el objeto de permanecer dentro de un programa académico de excelencia, 
otra vez se hizo necesario realizar una nueva modificación de la 
currícula a fin de dar satisfacción tanto a egresados como a 
empleadores. Así, desde el 2010 volvimos nuevamente a trabajar en comisiones  
durante once meses, con lo cual conseguimos elaborar el nuevo plan de estudios 
que comenzó a operar desde el 2011.

\enlargethispage{1\baselineskip}
Ya desde el 2004 el Programa de Licenciatura de la Unidad Académica de Historia 
consideraba ineludible actualizar y dar pertinencia a sus planes y 
programas de estudios. Desde mayo del 2003, la planta docente del PE de 
la Unidad Académica de Historia, bajo la coordinación de la Responsable 
del Departamento de Docencia, procedió a realizar el nuevo Plan de 
estudios, siguiendo una serie de lineamientos: Se hizo un análisis 
comparativo de veintiún planes de estudios de Licenciatura en Historia  
de universidades nacionales, con el fin de ubicar la situación del plan 
vigente y proponer por lo tanto los puntos nuevos a la Reforma del Plan 
de Estudios 2004. Se convocó a un encuentro de egresados y empleadores 
con el objetivo de establecer lazos y comunicación permanente entre los 
egresados y la Universidad; conocer las demandas actuales propias del 
ejercicio profesional en que se desenvuelven los historiadores; 
observar las tendencias para el perfil profesional de los historiadores 
en materia de conocimientos, valores y competencias en el desempeño 
laboral de calidad; obtener información acerca de las aspiraciones y 
expectativas sobre titulación, educación continua y acciones de 
crecimiento profesional y personal de los egresados; recuperar la 
información respecto a la formación recibida en esta Institución; las 
áreas de conocimiento que es necesario incluir o fortalecer; y las 
recomendaciones dirigidas a la modificación y requerimiento del 
proyecto curricular. Un ejercicio similar se volvió a realizar en el año  
2010, con el fin de acometer la siguiente reforma del Plan de estudios.

El reto era, ahora, una reforma curricular en el año 2011 que estuviera a la
altura de las nuevas exigencias del mundo actual, de la sociedad y de los
retos del futuro; es decir, una educación abierta, flexible, integral y
polivalente, con la finalidad «de entregar a la sociedad nuevos
profesionales en el campo de las humanidades y de la educación» 
(González Barroso, Magallanes Delgado y Román Gutiérrez 2011). 

\enlargethispage{1\baselineskip}
La estructura curricular consta ahora de cuatro áreas (básica, 
disciplinar, programa académico común y optativa); contiene los ejes terminales de 
especialización (docencia, investigación, difusión, organización y 
administración de acervos, historia del arte e historiografía) y los ejes 
transversales (género, democracia, derechos humanos, ecología, 
globalización y desarrollo sustentable), así como un eje integrador, donde el 
servicio social está vinculado a la currícula, y con el cual se promueve el liderazgo 
y se presta especial atención a la comunicación, la identidad y los 
valores. Lo anterior permite que cada estudiante, junto con su tutor, 
diseñe la currícula que más satisfaga sus intereses, potencialidades y 
expectativas.\footnote{El área de \textit{Básicas}, según el cuadernillo once, 
forma al universitario en aspectos teóricos, epistemológicos y 
metodológicos propios del área de conocimiento, e implica una formación 
tanto en los conceptos teóricos de las ciencias básicas de cada 
profesión, como en la génesis, la lógica de construcción y el uso 
social de ese conocimiento; asimismo implica el análisis de la lógica 
del poder que hace posible tal conocimiento, es decir, de cómo se piensa 
y se estructuran los conocimientos (teóricos) en función de 
determinadas exigencias del poder. El \textit{Programa académico común} consiste 
en formar a los estudiantes sobre conocimientos, saberes y habilidades 
necesarios para su formación superior y su futuro desempeño 
profesional; representan el bagaje cultural e instrumental demandado 
por el mundo contemporáneo, es el punto de entrada a la educación 
abierta, integral, flexible, polivalente y pertinente. Es una formación 
obligatoria para todos los estudiantes inscritos en la Universidad, y 
puede cursarse en cualquier Unidad Académica de la institución, aunque 
preferentemente en el Área Académica correspondiente. En \textit{Optativas}, 
el alumno junto con el tutor elegirá del menú, aquellas UDIs para 
profundizar en el conocimiento y así tener la opción de egresar como 
licenciado en historia o bien en algunas de las salidas colaterales.  
La fase \textit{Disciplinar} comprende un conjunto integrado de disciplinas 
ligadas estrechamente a la realización de la profesión y a los saberes, 
técnicas o artes que le corresponden; este conjunto de UDIs 
relacionadas con un campo profesional específico es propiamente el 
programa académico de licenciatura que se imparte en una Unidad 
Académica. Esta fase prepara al universitario para su ejercicio 
profesional-laboral al insertarse y\slash{}o vincularse con la sociedad, 
incluye tanto una formación teórica especializada y a profundidad, como 
en habilidades, actitudes y valores inherentes a la profesión; esta 
formación fortalece la conceptualización y el acercamiento a la 
práctica profesional e implica la incorporación ágil, oportuna y 
significativa de los avances de la ciencia y la tecnología en el 
currículo del programa académico correspondiente.}  
%\enlargethispage{1\baselineskip}
%\newpage

En el nuevo Plan de estudios  se pone en práctica un modelo de servicio
social, de estancias y prácticas universitarias acorde con el modelo académico
«UAZ--Siglo XXI»,  incorporando el Servicio Social a la Currícula, generando
instrumentos de seguimiento puntual, bajo proyectos claros de las unidades
receptoras que incidan con los intereses académicos de los alumnos y de la
sociedad de acuerdo con los ejes en que fue formado el estudiante.  El
servicio Social tiene el carácter atribuido por el Departamento de Servicio
Social y de Vinculación de la Universidad Autónoma de Zacatecas, ya que la
UAZ debe contribuir a socializar los beneficios derivados de la ciencia, la
cultura, el arte y la tecnología, con el fin de impulsar el desarrollo
económico y social de nuestro entorno local, regional y nacional, en  un
contexto de globalización. El objetivo es promover el acercamiento real del
estudiante de historia a la sociedad; consolidar la formación académica;
desarrollar sus valores y actitudes; y favorecer su inserción al mercado de
trabajo. La realización del servicio social por parte de los alumnos de
nuestro PE pretende que sea debidamente aprovechado y recupere su
importancia académica a través de las labores llevadas a cabo por los alumnos
en contacto con una realidad social,  bajo proyectos claros de las unidades
receptoras que incidan con los intereses académicos de los alumnos y de la
sociedad de acuerdo con los ejes en que fue formado el estudiante.  Las 
principales metas que se han logrado cristalizar con el servicio social desde el
2004, han sido: realizar el reglamento de incorporación del Servicio Social a
la Currícula bajo el modelo educativo basado en «competencias», la
organización de Foros anuales sobre experiencia de los estudiantes en el
Servicio Social, y el llevar a cabo un registro sobre la inserción, seguimiento
y evaluación del servicio social del PE. Conformar y actualizar
permanentemente convenios con instancias receptoras, y proponer anualmente
un plan de prácticas profesionales con los sectores sociales y productivos.

El modelo curricular comprende una estructura abocada al impulso de la
educación abierta, flexible, integral, polivalente y pertinente, que esté en sintonía
con las demandas del mundo actual y de la sociedad, y que sirva para enfrentar los retos del
futuro. 

\enlargethispage{1\baselineskip}
\begin{Obs}
\item[a)] Se propone el carácter abierto con el fin de romper con el modo cerrado
mediante la libertad de cátedra, con un fuerte compromiso de aprendizaje
con el educando para la integración del conocimiento y la transformación de
los docentes de consumidores pasivos a líderes constructivos del currículo.
\item[b)] El carácter integral rebasa la visión tradicional centrada solamente en
lo temático o conceptual, con el fin de impulsar perfiles en todas sus
facetas: intelectual, afectiva, relacional, psicomotora y axiológica. Tiene el
propósito de generar en los jóvenes una actitud abierta al cambio permanente. 
\item[c)] El carácter flexible impulsa la conexión entre áreas del conocimiento
elevando la versatilidad e interdisciplinariedad, fomentando el
autoaprendizaje y la movilidad del estudiante en planos intra e
interinstitucionales. 
\item[d)] El currículo adquiere un carácter polivalente, capaz de responder a una
sociedad cambiante y a la imprevisibilidad del futuro, integrando
coherentemente los contenidos de las distintas áreas del conocimiento
científico y humanístico; así como la virtud de formar individuos con
capacidad para adaptarse crítica y creativamente a cualquier contexto y
circunstancia, demostrando habilidades para la toma  oportuna
y decidida de decisiones, y para la solución de problemas en los ámbitos
sociocultural y laboral. 
\item[e)] El carácter de pertinente corresponde a los mecanismos de vinculación
con los sectores sociales, productivos y de servicios, para generar nuevos
ambientes de aprendizaje durante la formación profesional del alumnado. 
\end{Obs}

%\enlargethispage{1\baselineskip}
La Reforma del Plan de estudios considera que se debe formar a 
estudiantes autónomos mediante una educación centrada en el aprendizaje,
por lo que el tiempo en el aula se reduce para favorecer las actividades en
bibliotecas, archivos, museos, recorridos de campo, etcétera.  Al concebir
que la formación de los estudiantes debe ser integral, se promueve la
asistencia a eventos, tales como conferencias, exposiciones, cine, debates,
presentación de libros, obras de teatro, etcétera. Asimismo, se le
introduce al ámbito de su profesión y se le incita a proponer proyectos en
las diferentes esferas de su competencia. Lamentablemente, no existe un
programa específico que regule de manera sistemática las actividades permanentes
 de los cursos, talleres, diplomados, etc.\ que se destinan a los 
 egresados y a los sectores de la sociedad. 

Entre los beneficios que propone el nuevo plan de estudios está la educación
continua, por medio de los cursos optativos; sin embargo, a tres años de
haberse implementado la oferta dirigida a los egresados no ha tenido ningún
resultado positivo, puesto que los alumnos, una vez que egresan, no regresan al
programa para su actualización profesional,  a pesar de la difusión que, al
menos durante un tiempo, se logró establecer por medio de la Comisión de
seguimiento de egresados. Una de las oportunidades que el plan de estudio
ofrece a los estudiantes que no cuentan con las condiciones  para cubrir
los créditos de la licenciatura, es que puedan optar por la salida lateral de
Técnico Superior Universitario (flexibilidad en tiempo y contenido). Pero,
igualmente, el estudiante que no logra terminar sus estudios tampoco tiene
interés por recuperar el documento del TSU. En este sentido, ha sido
necesario disponer de información permanente sobre el
Programa por parte que del Departamento Escolar, quien dispone de una 
encuesta-cuestionario para aquellos alumnos que se dan de baja en la carrera, 
lo cuál nos ayuda a discutir en la academia
de profesores los problemas detectados entre el estudiantado, siendo una
constante la falta de vocación o prioridad por otra carrera, aspectos
detectados desde los primeros filtros para la selección de los alumnos.


Desde el 2009 a la fecha se cuenta con el Programa de Movilidad estudiantil
con Universidades nacionales y extranjeras; sin embargo, la falta de
información  oportuna por parte de la Institución y los pocos convenios que
existen con las Universidades extranjeras, y las pocas becas que se
ofertan, han mermado la oportunidad de que los estudiantes se inscriban
en el proceso de movilidad o intercambio académico hacia  otras universidades, 
tanto nacionales como extranjeras.  

En enero del 2010 inició el proceso de Reforma integral a su Plan de Estudios, 
siguiendo un proceso de análisis sistémico. 

%\enlargethispage{1\baselineskip}
Las buenas intenciones del Plan curricular eran ofrecer diplomados, 
talleres, seminarios conferencias y presentaciones de libros; y que sus 
profesores asistieran a congresos, coloquios y otro tipo de eventos, 
tanto a nivel nacional como internacional. Lo anterior se traduciría en 
la mejor calidad de la enseñanza y en una constante producción 
investigativa. Sin embargo, tanto la falta de Cuerpos Académicos en el 
Programa como de proyectos en común por parte de los profesores, ha derivado en una 
diversificación de los temas de interés, los cuales no se han visto reflejados 
en el enriquecimiento curricular del programa. La falta de 
consolidación está acompañada de algunos problemas: existencia de un 
solo cuerpo académico registrado en el programa, el de «Enseñanza y difusión 
de la historia» (González Barroso, Magallanes Delgado y Román Gutiérrez 2011). 

Este nuevo plan contempla ahora las siguientes asignaturas: 
ciento catorce UDIs, de las cuales sólo veinte son obligatorias (18\,\%) y 
noventa y cuatro son optativas (82\,\%). Ello obedece a 
la diversificación de los ejes terminales de especialización de la 
carrera de Licenciatura en historia, que distribuye de la siguiente 
manera las asignaturas: Área básica con cinco UDIs; la disciplinar con diez materias; 
el programa académico común con tres asignaturas; el eje integrador 
está compuesto de dos materias; mientras que los ejes de especialización se 
diversificaron de la siguiente manera: Eje de Difusión; Docencia; 
Investigación; Organización y Administración de Acervos; Historia del 
Arte y Historiografía. Cada uno de estos ejes de especialización debe 
contener seis materias optativas disciplinares de cada eje para 
complementar su preparación:
\newpage

%\smallskip
%\phantom{abc}
%\bigskip
\begin{scriptsize}
\begin{framed}
%\begin{minipage}[c]{.9\linewidth}
\textsc{Docente}

\textit{Conocimiento}\quad Manejará los paradigmas contemporáneos de la 
educación y su derivación en la práctica docente. Será capaz de 
diseñar, implementar y evaluar programas didácticos de la historia, 
para los distintos niveles educativos, ámbitos culturales e 
institucionales. 

\textit{Competencias}\quad Contará con la 
habilidad para elaborar y aplicar material didáctico tipográfico y 
electrónico propicio para la enseñanza y difusión de la historia. 
Habilidad para construir modelos de enseñanza-aprendizaje, incluida su 
instrumentación y evaluación. Desarrollar programas de formación y 
capacitación docente para la enseñanza de la historia.  

\textsc{Investigador} 

\textit{Conocimiento}\quad Demostrará 
dominio del método científico y de la integración  de herramientas 
teórico-metodológicas e instrumentales para la realización de tareas de 
investigación que le permita esclarecer, razonar y explicar los 
diversos procesos históricos. A fin de  descubrir,  reinterpretar, 
reconstruir y corregir la representación de los acontecimientos 
históricos. Será capaz de aportar interpretaciones conceptuales 
respecto a la construcción histórica y su vinculación con otras 
disciplinas en torno a la identidad y de proponer elementos 
teórico-metodológicos para el rescate, preservación y fortalecimiento 
de los procesos identitarios en los diversos planos: individual, 
colectivo, cultural, etc. Será capaz de proponer, diseñar, implementar 
y evaluar proyectos de investigación, abocados a enriquecer el 
conocimiento histórico y  solucionar problemas desde la perspectiva 
disciplinaria.  

\textit{Competencias}\quad Habilidades 
para buscar, seleccionar, procesar, explicar y representar ---a través de 
la escritura--- información sobre el pasado.  Comprobar la veracidad de 
la información, así como cotejar las distintas versiones sobre un mismo 
hecho. Identificar los diferentes lenguajes del discurso histórico. 
Diestro para el manejo de los diferentes tipos de acervos. Capacidad 
para la lectura y comprensión de manuscritos e impresos, imágenes, 
fotos y en movimiento, arquitectonografía, etcétera.  Capacidad para 
formular, aplicar e interpretar cuestionarios, encuestas, entrevistas y 
bitácoras para el trabajo de campo. Dominio de nuevas tecnologías para 
la búsqueda, procesamiento y difusión de la información.  

\textsc{Divulgador} 

\textit{Conocimiento}\quad Estará 
al corriente de las teorías de la comunicación y de su instrumentación. 
Sabrá discernir el impacto social de los diferentes medios de 
información. Será capaz de diseñar e implementar recursos idóneos para 
la difusión de la historia. 

\textit{Competencias}\quad  Contará con la habilidad para manejar todo 
tipo de medios y recursos 
informáticos. Habilidad para gestionar espacios de información. 
Habilidad para despertar el interés en todo tipo de auditorio. 
Habilidad para utilizar innovadoramente viejas tecnologías y recursos 
limitados en la composición y transmisión del conocimiento histórico.  
\newpage

\textsc{Organización y administración de acervos}

%\enlargethispage{1\baselineskip} 
\textit{Conocimiento}\quad Conocer y analizar los diferentes 
tipos de documentos. Capacitar en el diseño, organización y 
administración de acervos de carácter histórico, bibliográfico y de 
otro tipo de repositorios o fondos documentales. Conocer y dominar las 
técnicas y teorías archivísticas, biblio-hemerográficas, así como sus 
principales principios. Capacitar en el manejo de las tecnologías del 
conocimiento aplicadas en el resguardo documental.  

\textit{Competencias }\quad Desarrollará la competencia para organizar y 
administrar repositorios documentales, ya sean instituciones públicas, 
privadas o particulares. Será capaz de distinguir los diferentes tipos 
de documentos, sus características internas y externas, su tipología de 
clasificación, etcétera. Será capaz de organizar, clasificar, conservar 
diferentes tipos de expedientes y obras, considerando el soporte 
material.  Desarrollar la competencia de transcribir, analizar e 
interpretar la documentación histórica.  

\textsc{Historiador del arte} 

\textit{Conocimiento}\quad Adquiere una 
visión general de los diferentes procesos histórico-artísticos, en los 
ámbitos teórico-metodológicos e historiográficos. 

\textit{Competencias}\quad Desarrollo crítico y analítico de la producción 
artística en sus distintos lenguajes (arquitectura y urbanismo, 
escultura, pintura, fotografía, cine, música, artes decorativas y 
suntuarias) representativos de cada época de la historia.  
\newpage

\textsc{Historiógrafo} 

\textit{Conocimiento}\quad El 
alumno conocerá las diversas teorías y metodologías de hacer y pensar 
la historia, tanto nacional como internacional a través del tiempo, 
además aprenderá a relacionar los textos historiográficos con sus 
contextos ideológicos y circunstanciales.  

\textit{Competencias }\quad El alumno analizará y criticará diferentes 
formas de escritura de la historia. Además aprenderá a utilizar la 
metodología de otros historiadores para sus trabajos de investigación, 
en los que aplicará tanto teoría como elementos narrativos y 
discursivos para pensar y escribir la historia.
%\end{minipage}
\end{framed}
\end{scriptsize}

\smallskip
Además, se cuenta con un listado de materias optativas generales que 
complementan su formación. Entre estas materias sobresalen las siguientes: 
Historias generales (7 asignaturas); Historia especializadas (23); Historias 
del Arte (13); turismo (7); y complementos (6 UDIs). El 
Plan de estudios es por créditos, de tal manera que la duración depende de la 
planeación individual de cada uno de los estudiantes; es decir, de un 
total de 469 créditos, repartidos en 113 UDIs, se necesitan para 
titularse 230 créditos, de los cuales el 39\,\% (20 UDIs) corresponden a 
las del tipo obligatorio y  61\,\% (35 UDIs) a las del tipo optativo, 
especificando que cuatro de estas optativas deben ser historiografías. 
Además de las actividades en el aula, el alumno deberá administrar su 
tiempo para realizar lecturas, investigar, elaborar trabajos, tareas y 
asistir a diversos eventos culturales y académicos (conferencias, 
presentación de libros, coloquios, cine, conciertos, etcétera).  

\enlargethispage{1\baselineskip}
La variada y rica oferta de UDIs en el Plan de estudios 
lamentablemente no ha sido aprovechada por los estudiantes ni por los 
profesores. En el caso de los seis ejes y su justificada pertinencia, 
la realidad es distinta. Los ejes que siguen manteniendo vigencia y 
predominio son el de docencia e investigación; seguidos por el de 
difusión e  historia del arte con un menor número de estudiantes; mientras 
que los ejes de  organización y administración de acervos e 
historiografía son pocos demandados. En este sentido, las razones 
pueden ser varias, desde una realidad laboral que prevalece en el 
estado y en el país, donde la docencia es lo que prima en el ejercicio 
laboral del historiador, hasta el poco apoyo que se da a los proyectos 
culturales y de difusión, a pesar de que Zacatecas posee la 
declaratoria de Patrimonio Mundial de la Humanidad. Sin embargo, 
consideramos que la clave para fortalecer los ejes terminales se 
encuentra en la actuación de los tutores con el alumnado, a la hora de 
orientar a cada  estudiante para que explote sus aptitudes, 
conocimientos y destrezas de mejor manera. 

Respecto a los mismos ejes transversales (género, democracia, derechos 
humanos, ecología, globalización y desarrollo sustentable), se observa 
que en los programas no se integran como valores de discusión o de toma 
de conciencia. Existe el planteamiento en el plan de estudios, pero en 
la práctica pocas veces permea el espacio del aula\linebreak y en la comunidad 
educativa.

La gran variedad de materias optativas representa todo un desafío para la 
planta docente, pues, además de que se han realizado nuevas 
contrataciones de profesores al programa que logren cubrir las 
necesidades de materia laboral disponible, la realidad es otra. El 
perfil laboral y la experiencia profesional de la nueva planta docente no 
cubre con las necesidades planteadas por el programa; es decir, no cuentan con 
el perfil de  profesores, poseen escasa experiencia en el proceso de 
enseñanza-aprendizaje, o su perfil y especialización no alcanza la 
experiencia necesaria para las nuevas materias que ofrece el plan de 
estudios. A tres años de implementarse el nuevo plan de estudios, varias 
UDIs no se han podido ofertar, ya sea por falta de profesores  o 
porque los estudiantes no están interesados en ellas.

%\enlargethispage{1\baselineskip}
A partir del 2004, nuestro Plan de estudios eliminó la rigidez de la 
currícula, convirtiéndose en un Plan de carácter flexible. Ahora, el alumno que reprueba 
una asignatura en cualquier oportunidad (ordinario, extraordinario y a 
título de suficiencia) puede volver a cursarla o darla de baja, y\slash{}o, en 
su caso, sustituirla por otra, siempre y cuando reúna el número de 
créditos necesarios y corresponda al eje seleccionado previamente por 
el alumno.

En esta ocasión se trató de diversificar aún más las materias optativas, 
procurando una preparación más integral del estudiante, y con el 
objetivo principal de ser evaluados por instancias externas, además de 
los CIIES. En este sentido, en el 2014 nos estamos preparando para 
ser supervisados y certificados por el COAPEHUM, y poder así mantener 
los estándares de calidad que exigen nuestras autoridades, tanto 
universitarias como gubernamentales.

%\enlargethispage{1\baselineskip}
El nuevo plan de estudios del programa de licenciatura en historia está 
diseñado para que el alumno se posicione de una manera inmediata en el 
mercado una vez concluidos sus estudios, ya que el contenido de las 
UDIs proporciona las herramientas necesarias para que el egresado tenga 
una amplia oportunidad y competencia para el empleo; sin embargo, la 
realidad laboral es otra, algunos estudiantes continúan estudiando en 
algún posgrado o buscan trabajo como profesores. Pocos son los 
egresados que se desempeñan en el ámbito de la investigación, a pesar 
de que egresan con la capacidad de consultar en bibliotecas 
especializadas y en archivos, y para realizar textos de carácter histórico que 
incluyan las nuevas visiones de la historia. Pocos estudiantes han 
podido incorporarse laboralmente en los  archivos históricos, a pesar de 
que sus competencias los prepararon para elaborar proyectos, clasificar 
y organizar documentos, así como saber crear instrumentos de consulta 
como catálogos, guías, versiones paleográficas y otros. En algunos 
casos, y temporalmente, los egresados se han podido emplear como 
auxiliares de investigador y\slash{}o por su cuenta.  

El ámbito laboral común de los egresados es la docencia en los 
niveles medio, medio superior y superior; y entre las competencias 
relacionadas con las que egresan son primordialmente la elaboración de materiales 
didácticos, aplicación de dinámicas e impulso del  aprendizaje por 
competencias. Pocos son los casos de egresados que se ocupan en 
distintos ámbitos interinstitucionales, a través de proyectos 
culturales; o en áreas laborales de gestoría y comunicación.  

\enlargethispage{1\baselineskip}
Para que el alumno pueda titularse requiere cubrir 230 créditos, de
los cuales el 47\,\% (109 UDIs) corresponden a las UDIs obligatorias y el 53\,\%
(121 UDIs) a las optativas, especificando que de estas últimas, cuatro
deben ser historiografías.  Abarcar alguna de las modalidades de titulación,
a saber: 1)~Tesis individual o colectiva (máximo dos personas); 2)~evaluación 
de la memoria y del producto de la estancia profesional; 3)~elaboración de 
catálogos o ediciones paleográficas; 4)~promedio mínimo de
9.0, sin interrupción de los estudios y con la obtención de todas las
calificaciones en períodos ordinarios; 5)~curso-taller de titulación, y 
6)~examen general de conocimientos.  El estudiante deberá acreditar 6
trimestres de un idioma extranjero cursado en el centro de idiomas de la
UAZ.  Los títulos que se otorgan en la carrera son: 1)~{\itshape Técnico Superior Universitario}. 
Para el TSU se requieren 140 créditos, de los cuales 80 deberán
corresponder a UDIs obligatorias, además de la Estancia y el servicio
profesional, mientras que los 24 créditos restantes corresponden a seis
UDIs optativas del eje  de especialización que se haya elegido, tomando en
cuenta que son seis ejes: Docencia, Investigación, Difusión, Organización y
Administración de Acervos, Historia del Arte e Historiografía; 2)~{\itshape Licenciado 
en Historia} (González Barroso, Magallanes Delgado y Román
Gutiérrez~2011).
 
La opción de titulación más aceptada por los estudiantes es la tesis. 
Esto obedece a que tradicionalmente la licenciatura ofrecía como única 
opción de titulación la tesis individual. Mientras que las nuevas 
opciones se abrieron a partir del 2004, apareciendo hasta fechas recientes la 
elección de las nuevas formas de titulación, prefiriéndose el promedio  
y los cursos talleres de titulación con sus productos. De éstos últimos, 
el de mayor demanda es el curso de titulación, puesto que esta 
modalidad se dirigió en un primer momento a todos aquellos alumnos 
rezagados que no habían podido adquirir el grado, permitiéndoles la 
titulación en seis meses.

%\enlargethispage{1\baselineskip}
A pesar de la variedad de ofertas de titulación, el estudiante considera 
que la tesis es la mejor opción para el desempeño profesional del 
egresado; sin embargo, la realidad se contradice con las opiniones\linebreak 
vertidas por los mismos, ya que muchos de ellos al terminar sus créditos 
correspondientes y con un buen desempeño en su carrera optan por la 
titulación por promedio, la otra gran mayoría que no fue tan 
beneficiada en sus calificaciones se inclina por los cursos que se 
ofertan anualmente en el programa. A pesar de las ventajas en la oferta 
de titulación,  la eficiencia terminal sigue por debajo de lo requerido. 

El objetivo general que pretende la Unidad Académica de Historia es
consolidarse como una alternativa de excelencia académica en la región.

El programa de licenciatura en historia de la UAZ,  al
multiplicar sus orientaciones terminales adopta el modelo «eficientista»
dominado por el mercado, pero también se mantiene el compromiso de generar,
avanzar y transmitir el conocimiento a través de los cuerpos académicos, no
sólo de conocimiento útil, sino de conocimiento que hace trascender  a la
humanidad (González Barroso, Magallanes Delgado y Román
Gutiérrez~2011).
%\newpage

%\begin{sloppypar}
Actualmente, la Universidad Autónoma de Zacatecas se rige por un modelo educativo que fue implementado en 2005, el de\-no\-mi\-na\-do {\textit{UAZ--Siglo XXI}}, en el cual se definió a «la actividad académica como el centro de su 
competencia con base en sus funciones sustantivas» (Comisión general 
operativa~2000, p.~4).  Al efecto, se reivindicó la tradición 
humanística, democrática y de autonomía que le dio origen y que ha 
estado presente a lo largo de su historia reciente. De acuerdo con este 
nuevo modelo académico, que se basa en el constructivismo crítico, el 
estudiante es el centro del proceso académico, mientras que el profesor 
se convierte en un facilitador del aprendizaje; además, se propone la 
flexibilización de los planes y programas académicos. Como resultado 
del contexto local, nacional e internacional, caracterizado por la 
desigualdad, la competitividad producto del neoliberalismo, la 
globalización y la revolución digital, uno de los principales objetivos 
del modelo \textit{UAZ--Siglo~XXI} consiste en ofrecer a los y las 
zacatecanas «(\ldots) una educación sustentada en la investigación; prioridad a 
los procesos de aprendizaje del estudiante; los currículos abiertos y 
flexibles para todos y para toda la vida, y la incorporación y 
generalización del uso de nuevas tecnologías\ldots» (UAZ~2005, p.~9).
%\end{sloppypar}
%\newpage
%\enlargethispage{1\baselineskip}

\bigskip
\textbf{Referencias}

Antonio González Barroso, María del Refugio Magallanes Delgado y Ángel
Román Gutiérrez (2011), \textit{La reforma curricular de la licenciatura 
en historia de la UAZ: un esfuerzo compartido}, VI Encuentro de la Red Nacional de
Licenciaturas en Historia y sus Cuerpos Académicos, Tijuana B.\ C\@.

Comisión general operativa (2000), \textit{Proyecto curricular} (cuadernillo\linebreak No.~11), 
Zacatecas, UAZ.

UAZ (2005), \textit{Modelo académico UAZ, Siglo XXI}, Zacatecas, UAZ.
\newpage
\thispagestyle{empty}
\phantom{abc}