%\documentclass{article}
%\usepackage{amsmath,amssymb,amsfonts}
%\usepackage{fontspec}
%\usepackage{xunicode}
%\usepackage{xltxtra}
%\usepackage{polyglossia}
%\setdefaultlanguage{spanish}
%\usepackage{color}
%\usepackage{array}
%\usepackage{hhline}
%\usepackage{hyperref}
%\hypersetup{colorlinks=true, linkcolor=blue, citecolor=blue, filecolor=blue, urlcolor=blue}
%% Text styles
%\newcommand\textstyletr[1]{#1}
%\newcommand\textstyleInternetLink[1]{\textcolor{blue}{#1}}
%\newtheorem{theorem}{Theorem}
%\title{}
%\author{MICROHISTORIA}
%\date{2014-06-15}
%\begin{document}
%%\clearpage\setcounter{page}{483}


\thispagestyle{empty}
\phantomsection{}
\addcontentsline{toc}{chapter}{La Licenciatura en Historia de la UAZ en la\newline antesala 
del COAPEHUM. Los maestros frente a lo real, lo posible y lo deseable\newline $\diamond$
\normalfont\textit{María del Refugio Magallanes Delgado,\newline Norma Gutiérrez Hernández  y
Ángel Román Gutiérrez}}
\normalfont\normalsize
{\centering {\scshape \large La Licenciatura en Historia de la UAZ en la antesala\newline
del COAPEHUM. Los maestros frente a lo real,\newline lo posible y lo deseable}\par}
\markboth{la formación del historiador}{licenciatura de la UAZ}
\setcounter{footnote}{0}


\bigskip
\begin{center}
{\bfseries María del Refugio Magallanes Delgado\\
Norma Gutiérrez Hernández\\
Ángel Román Gutiérrez}\\
{\itshape Universidad Autónoma de Zacatecas}
\end{center}

\bigskip
\textbf{Resumen}

El anuncio del proceso de acreditación por COAPEHUM en junio de 2013 provocó
dos reacciones entre el colectivo de profesores del programa de
licenciatura en historia de la UAZ: pesadumbre por la gran cantidad de
trabajo extraordinario que se avecinaba y preocupación por la búsqueda de
información y la escritura correcta de cada uno de los criterios y
lineamientos que serán evaluados en agosto de 2014. Pero esa preocupación
no era en vano. La desigualdad en la distribución de puntos en cada
criterio, muestra niveles diferenciados en la valoración que se hace en los
factores ponderables en la educación superior en México. La <<carpeta>> de
personal académico es el segundo rubro en importancia cuantitativa en la
evaluación. En los nueve criterios subyacen tres dimensiones: lo real, lo
posible y lo deseable de los docentes, el programa y la institución. La
relación entre estas tres esferas pocas veces es armoniosa. La innovación,
la calidad y el desarrollo sustentable de los profesores se esconden en las
relaciones jerarquizadas excluyentes que controlan las acciones
científicas, pedagógicas, de vinculación y gestión de los docentes.\\ 
{\bfseries Palabras clave:} evaluación, innovación, personal académico.


\bigskip
\textbf{Abstract}

\begin{sloppypar}
The History Degree at the University
Autonomous of Zacatecas (UAZ) is at the prelude of the COAPEHUM. Teachers
are facing up the reality, the possibilities and the desirable scenario.
\end{sloppypar}

{Announcement of certification processes conducted by COAPEHUM on the
academic programs at the History degree at the UAZ on June 2013 caused a
mixture of reactions among professors. They were overwhelmed by the amount
of additional work to do, and they were as well concerned for data mining
and proper writing of criteria and guidelines, documentation to be
evaluated in August 2014. This concern proved not to be vain, since there
are profound differences in scoring each criterion, a clear evidence of the
prevalence of variability of what can be evaluated in the Mexican
undergraduate education. The second most important issue for the evaluation
is the <<files>> of the academic staff. The reality, the possibilities and
the ideal scenarios for programs, professors, and institutional policies,
are inherent to the nine evaluable criteria. Rarely, there are harmonic
relationships among these three characteristics. Hidden behind hierarchical
relationships, which control social connections, scientific and pedagogic
actions, innovation, quality, and sustainable development of professors are
resting.\\
{\bfseries Keywords:} evaluation, innovation, academic staff.

\bigskip
Los profesores y directivos universitarios parten de dos supuestos: el 
siglo XXI se caracteriza por ser la era de la sociedad del conocimiento y 
la internacionalización de las instituciones de educación superior  es un
hecho ineludible. El telón de trasfondo de este fenómeno social es el valor
estratégico del conocimiento y la información, el rol que desempeñan las
instituciones de educación superior, la incorporación  masiva de los
jóvenes a este nivel escolar y la contribución al crecimiento económico de
este sector de la población. En consecuencia, los mecanismos de
aseguramiento de la calidad son necesarios para que los actores
involucrados en la educación superior tengan confianza en que el suministro
del servicio cumple con las expectativas socioeconómicas y culturales y se
alcanzan ciertos estándares.


En este sentido, la evaluación y la acreditación son procesos reconocidos en
todo el mundo, que fungen como medios para mejorar los sistemas de
educación superior. Estas acciones se asociación con el mejoramiento de la
calidad, la generación de información para la toma de decisiones, la
garantía pública de la calidad de las instituciones y de los programas que
ofrecen. Adicionalmente a esto, han servido también para garantizar la
equivalencia y reconocimiento de títulos y grados en instituciones de un
país o varios. 

Así pues, el gobierno federal reconoce al Consejo para la Acreditación de la
Educación Superior (COPAES) como instancia que confiere reconocimiento
formal a organizaciones cuyo fin es acreditar programas académicos de
educación superior ofrecidos por instituciones públicas y particulares.

 
La política de evaluación para la acreditación de los programas académicos
de educación superior (PAES), se sustenta en el hecho de que con ellos se
forman los futuros profesionistas y de que en un  PAES de alta calidad se
gesta el cambio social y tecnológico, por lo que se presume que son
instancias en transformación constante para mejorar la enseñanza, la
investigación y la vinculación (Arroyo 2010, 64--65). 
\newpage

%\enlargethispage{1\baselineskip}
Considerando lo anterior, el anuncio del proceso de acreditación por
COAPEHUM en junio de 2013 provocó dos reacciones entre el colectivo de
profesores del programa de licenciatura en historia de la UAZ: pesadumbre
por la gran cantidad de trabajo extraordinario que se avecinaba y
preocupación por la búsqueda de información y la escritura correcta de cada
uno de los criterios y lineamientos que serán evaluados en agosto de 2014.
Pero esa preocupación no era en vano. La desigualdad en la distribución de
puntos en cada criterio, muestra niveles diferenciados en la valoración que
se hace en los factores ponderables en la educación superior en México. 

La <<carpeta>> de personal académico es el segundo rubro en importancia
cuantitativa en la evaluación y en los nueve criterios subyacen tres
dimensiones: lo real, lo posible y lo deseable de los docentes, el programa
y la institución. La relación entre estas tres esferas pocas veces es
armoniosa. La innovación, la calidad y el desarrollo sustentable de los
profesores se esconden en las relaciones jerarquizadas excluyentes que
controlan las acciones científicas, pedagógicas, de vinculación y gestión
de los docentes. 

Esta ponencia ofrece los resultados que se obtuvieron del ejercicio de
autoevaluación que se realizó de la carpeta de personal docente de agosto a
diciembre de 2013. Únicamente se toman para esta investigación cuatro
criterios: el reglamento de evaluación docente para ser profesor del
programa,  las estrategias para evitar la endogamia,  revisar la existencia
de programas que incentiven a los docentes para formar redes académicas
nacionales e internacionales, y el reglamento para el otorgamiento de
estímulos al desempeño docente e investigación vía ESDEPED, PROMEP y SNI. 
\newpage

\textbf{Marco contextual}

La Licenciatura en Historia de la Universidad Autónoma de Zacatecas (UAZ)
nació el 7 de septiembre de 1987 con la fundación de la Escuela de
Humanidades. Ésta tenía como sus objetivos centrales: <<(\ldots) formar
humanistas que sepan atender aquellas disciplinas que cultiven el
pensamiento, el desarrollo de las diversas formas de expresión oral y
escrita y la reconstrucción del pasado>> (González 2004, pp. 13--14).  Este
proyecto contempló tres áreas de especialización: Letras, Filosofía e
Historia. Inicialmente había un tronco común de 4 semestres, mismo que en
1991 se redujo a un año. En 1992 la Escuela de Humanidades transitó a
Facultad de Humanidades, incorporando en 1996 la especialización en
Arqueología (\textit{ibid.}, p. 14). Posteriormente, en el 2000, el Consejo
Universitario avaló el establecimiento de cuatro Unidades, mismas que a la
fecha subsisten e integran las Licenciaturas de Historia, Letras, Filosofía
y Arqueología (\textit{ibid.}).

A la fecha, el Programa de Licenciatura en Historia de la UAZ ha tenido seis
planes de estudios durante los años de 1987, 1991, 1994, 1999, 2004 y 2011
(\textit{ibid.}, pp. 15, 18, 21, 23 y 29). En cada uno de estos años, las
reformas curriculares que se han realizado han obedecido al examen que ha
hecho la academia de profesores y profesoras en torno al impacto y
pertinencia del currículum en la carrera, a la par que por el resultado de
observaciones que ha externado la comunidad egresada del programa, los
encuentros con empleadores y empleadoras que se han llevado a cabo, las
políticas educativas institucionales que rigen la propia Universidad y, por
supuesto, el escenario contextual nacional y local en el que se ha inscrito
la Licenciatura.

Es importante señalar que de todas estas revisiones curriculares se marca un
parte aguas a partir del penúltimo Plan de Estudios (2004),  en virtud de
que a partir de esta nueva reestructuración curricular, la cual fue
diseñada por un colectivo de ocho docentes y contó con la asesoría de la
Coordinación de Docencia de la propia Universidad, se incorporaron por
primera vez cinco rasgos distintivos que modificaron sustancialmente la 
Licenciatura en Historia, al considerar un currículum flexible,
polivalente, abierto, integral y centrado en el aprendizaje. 

 
Por primera vez, el Plan de Estudios de la carrera planteó una duración de 8
semestres, opciones de titulación, créditos, estructura en áreas (común,
básica, disciplinar y optativa), ejes terminales (docencia, investigación y
extensión), ejes transversales, particularmente los de género, democracia,
derechos humanos, ecología, globalización, desarrollo sustentable,
identidad y valores; un eje integrador a partir de la práctica profesional;
y, un curso propedéutico denominado <<Estrategias de aprendizaje.>>

Las bondades de este Plan fueron significativas, considerando que el 58\,\% de
los cursos fueron de carácter optativo, además de que el servicio social se
anexó al currículum, se implementó el sistema de tutorías, se estableció el
programa de educación continua para la actualización permanente con las y
los egresados; además, se incluyó la salida lateral de Técnico/a Superior
Universitario/a (TSU) para quienes no pudieran concluir con la
Licenciatura. 


En términos generales, en contraposición a los planes de estudio anteriores,
en el del 2004 se hizo énfasis en la formación de estudiantes  a partir de
una educación centrada en el aprendizaje, de tal forma que, el tiempo
en el aula sería compartido con actividades académicas en otras instancias como
bibliotecas, archivos, museos, recorridos de campo, viajes de prácticas,
etc., al mismo tiempo que, se impulsaría la asistencia a conferencias,
exposiciones, cine, debates, presentación de libros y obras de teatro,
entre otras. En suma, el Plan de Estudios 2004 puso un énfasis especial en
la comunicación, la identidad y los valores,  juntamente con los procesos
de edificación de liderazgo entre los y las estudiantes (González~2014,
pp.~1--3).

 
En el 2011 la Licenciatura en Historia modificó su Plan de Estudios, el cual
sigue vigente en la actualidad. Se conservaron varios elementos del
anterior como la formación por créditos, opciones de titulación, sistema de
tutorías y educación continua; curso propedéutico; salida terminal de TSU;
intensificación del aprendizaje fuera del aula; una formación integral y,
los mismos ejes transversales de género, democracia, derechos humanos,
ecología, globalización y desarrollo sustentable, los cuales se
justificaron por el escenario contextual imperante:


\medskip
Los problemas actuales que padece la humanidad han llegado a todos los
lugares y los distintos niveles de la sociedad.  Consideramos de vital
importancia que las nuevas generaciones independientemente de la disciplina
o área que estudien hagan conciencia desde su formación profesional en los
siguientes temas: respeto a los derechos humanos, un buen ejercicio de la
democracia, mantener su identidad y fomentar los distintos valores éticos,
cívicos,  culturales, etc\@. De igual manera se insistirá en crear conciencia
para la protección y vigilancia de los recursos naturales que le rodean,
sin olvidar el fenómeno de la globalización (González~2011, p.~9).


\medskip
Las principales innovaciones en el año 2011 fueron las siguientes: se incrementó
el número de ejes terminales de especialización, aparte de los tres previos
se anexaron tres más: organización y administración de acervos, historia
del arte e historiografía; se cambió sustancialmente el número de materias
de 57 a 114; el eje integrador quedó\linebreak comprendido a partir de la estancia
profesional y servicio social, así como el seminario de elaboración de
proyectos; y, se implementó el modelo por competencias (González~2011,
pp.~1--15).

Es importante comentar que la Licenciatura en Historia de la Universidad
Autónoma de Zacatecas ha incrementado notablemente el número de su planta
docente: en el año 2004 éramos 8 docentes, mientras que desde el 2011 ya
sumábamos más del doble, y actualmente somos\linebreak 25 maestros y maestras: 10 con
el grado de doctorado, 14 con el de maestría (de éstos, 5 están haciendo
estudios doctorales y 5 más están en proceso de elaboración de tesis
doctoral o en espera de dictamen para obtención de este grado) y uno con
licenciatura. Es relevante precisar que varios de estos docentes han tenido
una formación de posgrado fuera de la Universidad Autónoma de Zacatecas, lo
que ha impactado positivamente en el programa, en tanto que no han
reproducido una endogamia académica y han sido estudiantes en posgrados de
calidad en el país.


\bigskip 
\textbf{La reglamentación de la autoevaluación\\ y evaluación en la
licenciatura}

Desde el semestre agosto-diciembre del 2013, las autoridades y la planta
docente de la Licenciatura en Historia decidieron iniciar un trabajo
colegiado para sumar a la acreditación de los CIEES del nivel 1 (que
ostenta desde el 2005),  la acreditación del Consejo para la Acreditación
de Programas Educativos en Humanidades (COAPEHUM). Esta decisión fue muy
importante porque se pusieron en una balanza los impactos positivos que
traería esta nueva acreditación para el Programa, particularmente en torno
a los y las estudiantes, quienes al egresar de su carrera estarían en
posibilidades de encontrar\linebreak mejores oportunidades laborales, o bien, cursar
un posgrado de calidad, en tanto que su formación profesional estaría
amparada por una certificación específica de su campo de estudio. Sin
embargo, también hay que precisar que, hubo incertidumbre sobre la cantidad
de trabajo que se incrementaría (sumada a la ya existente) en torno al
costo de recuperar la información, analizarla y redactar los textos para
cumplir con cada uno de los lineamientos. Hubo un fuerte debate sobre ello,
con opiniones a favor y en contra. Finalmente, la balanza se inclinó hacia
los beneficios que se podían obtener si se obtenía la certificación,
reconociendo y asumiendo el compromiso que se avecinaba. 

En este sentido, una vez que se conocieron los instrumentos de
autoevaluación y evaluación que precisa el COAPEHUM, se procedió a dividir
el personal docente en ocho comisiones, a saber: de personal académico o
docentes; de estudiantes; del plan de estudios o programa académico; de
investigación; de extensión; de recursos financieros e infraestructura; de
apoyo administrativo; y de normatividad y gestión. Vale la pena comentar
que las ponderaciones en cada una de estas carpetas la primera y la segunda
son las mayores; mientras que, la de apoyo administrativo es la que tiene
menos puntos (Instrumento de autoevaluación y evaluación~2011, p.~1).

De esta forma, ocupando parte de las asambleas semanales de la academia de
la Licenciatura en Historia, o bien, reuniéndose por comisión en distintos
tiempos, paulatinamente se ha iniciado y desarrollado el trabajo de esta
acreditación. A continuación se analizarán algunos lineamientos centrales
que competen al rubro: planta docente. 

El Programa de Licenciatura en Historia no tiene un Reglamento que integre
mecanismos para la evaluación de los y las profesoras, sin embargo, sí se
realiza tal acción con regularidad. La normatividad en este rubro se ubica
como parte de las funciones que tiene a su cargo la persona Responsable de
dicho programa académico. Específicamente, en el artículo 45, fracción IV,
en la cual se lee que es propio de su competencia lo siguiente: <<Aplicar
dos evaluaciones, por escrito, del desempeño del personal docente a todos
los alumnos y en todos los semestres. La primera tendrá lugar a mediados
del semestre y la segunda una vez pasados los exámenes ordinarios.>> 


Es importante señalar que la aplicación de este precepto tiene ya carta de
legitimidad en la Licenciatura, puesto que cada semestre se realizan las
dos evaluaciones que se mencionan, con la precisión de que la última de
ellas no se lleva a cabo al término de las evaluaciones ordinarias, sino
unos días antes de concluir el periodo de clases. Lo anterior, porque cada
unidad didáctica integral tiene diferentes mecanismos de evaluación en la
etapa ordinaria y, en ocasiones, no está todo el grupo junto, por lo que se
ha optado por aplicar la segunda evaluación para los y las docentes en la
última semana de clases. En este tenor, también es relevante enfatizar que
el Programa de la Licenciatura en Historia en su oferta de unidades
didácticas de verano y durante los meses de diciembre y enero, así como en
el curso propedéutico, también somete a evaluación a su profesorado.

\enlargethispage{1\baselineskip} 
Generalmente, las examinaciones de los profesores y profesoras se hacen en
los horarios de las materias obligatorias ---ya que se quiere que esté todo
el grupo---, solicitando unos momentos al maestro o titular, quien sale un
momento del aula. De un par de semestres a la fecha, esta
actividad la realiza la secretaria del Programa, ante la multitud de
actividades que definen las tareas administrativas de quien dirige la
Licenciatura.


Una vez evaluada la planta docente ---en dos momentos--- en cada uno de los
cursos que tienen a cargo durante el semestre, se hace un vaciado y
análisis de la información. Esta acción cuenta con el apoyo del personal
administrativo del programa, quien concentra en un solo formato la
evaluación del grupo en una asignatura determinada. Posteriormente, en
reunión del colectivo docente se presentan los resultados de evaluación a
cada profesor o profesora. Es importante resaltar que el carácter de las
evaluaciones se vincula con el enriquecimiento de la práctica docente, ya
que no ha ameritado alguna sanción como despido o no otorgamiento de carga
de trabajo. 


Actualmente, está en proceso la elaboración de un reglamento para la
evaluación docente en este programa académico. También es importante
comentar que, recientemente, apenas en el pasado mes de noviembre, la
administración central de la Universidad Autónoma de Zacatecas implementó
la evaluación en línea para que los y las estudiantes evalúen a sus
profesores y profesoras en cada uno de sus cursos. Aunque esto es una
iniciativa necesaria, en tanto que existen varios programas académicos en
la Universidad que no evalúan semestralmente a su planta docente, es
significativo mencionar que dicha evaluación\linebreak ---al menos en lo que se conoce---
ha tenido sólo como resultado cubrir la evaluación que plantea la
convocatoria de Estímulos en su presente emisión.


\bigskip 
\textbf{Los docentes y la  endogamia: frente a los cambios y la
continuidad  formativa}

La planta docente del Programa de Licenciatura en Historia de la Unidad
Académica de Historia de la Universidad Autónoma de Zacatecas (UAZ) en
agosto de 2013 estaba integrada por 24 profesores que tienen una formación
sólida en el área de las Humanidades y la Educación desde su formación
inicial y de posgrado. A partir del criterio de\linebreak <<institución otorgante>> se
observó que  20 docentes (83\,\%) obtuvieron el grado de licenciado de la UAZ
y 4 de ellos (17\,\%) de la Universidad Autónoma de Puebla, Universidad
Autónoma de Campeche y el Instituto Tecnológico de Zacatecas. 

Si bien la UAZ fue al mismo tiempo formadora y receptora de gran parte de
estos profesionales de las humanidades y la educación (un caso de medicina
humana), existe una diferencia significativa en el proceso formativo:
egresaron del Plan de Estudios de 1994, que contenía la primera reforma de
la Escuela de Humanidades, específicamente de la Facultad de Historia que
se fundó en 1987; en el Plan de 1994 se buscó retroalimentar a la Historia
tras las modificaciones que se hicieron a las Humanidades en 1992 y 1996
cuya composición era con cuatro especialidades: Filosofía, Letras, Historia
y Arqueología, mismas que en el 2000, tras una reforma curricular
universitaria se constituyeron en Unidades Académicas autónomas, pero con
acciones académicas interdisciplinares. 


La profesionalización de la planta docente egresada de la UAZ en estudios de
posgrado tomó tres líneas: continuidad en el campo disciplinar de la
Historia, incursión en la filosofía e historia de las ideas, las ciencias
sociales, la docencia y la especialidad disciplinar ---estudios
novohispanos---, dentro y fuera de la UAZ. Es importante subrayar que en el
2000 se funda la Maestría en Estudios Novohispanos, con lo cual se
contratan nuevos maestros, algunos de ellos egresados de El Colegio de
Michoacán (González 2004, p.19), quienes se incorporan también a la
Licenciatura en Historia.}

De los 23 profesores y profesoras, 13 de ellos (57\,\%) optaron por posgrados
de la UAZ y 10 (43\,\%) por posgrados registrados en el Programa Nacional de
Posgrados de Calidad (PNPC). De los 13 docentes,\linebreak 3 egresaron del Programa
de Maestría en Humanidades (Área de Historia), que en el 2005 se transformó en el
Programa de Maes\-tría-Doc\-to\-ra\-do en Historia que se encuentra acreditado en
el PNPC, de la cual egresaron una maestra y un maestro; tres docentes
egresaron  de la  Maestría en Filosofía e Historia de las Ideas,  uno
profesor de la  Maestría en Estudios Novohispanos, una docente egresada
Maestría en Ciencias Sociales, un maestrante de la Maestría en Humanidades
y Procesos Educativos de reciente creación que desarrolla tres
orientaciones: aprendizaje de la historia, cultura, currículum y procesos
institucionales, y educación en tecnologías. 


De los 10 profesores que continuaron su formación de posgrado fuera de la
UAZ, consiguieron su grado en programas inscritos en el PNPC, en
universidades del extranjero y universidades privadas. De programas PNPC,
dos egresaron de El Colegio de Michoacán,  cuatro de El Colegio de San
Luis, uno del  Instituto de Investigaciones históricas José María Luis Mora
y una de la Universidad de Guadalajara; se tiene un egresado de  la
University of Illinois at Chicago y un egresado de la Universidad
Interamericana para el Desarrollo. 

Trayectoria profesional que se traduce en las siguientes cifras, pero sobre
todo en una formación que se desplegó en la reforma curricular del Planes
de Estudios de la licenciatura en Historia 2004 y 2011 que se caracterizan
por estar articulados en orientaciones terminales  en torno a la
investigación, la docencia y la difusión (2004), que en 2011 se enriqueció
con  tres ejes más además con los temas de género y multiculturalidad en
sentido transversal. 


Si bien, de los 23 docentes, dos no cuentan con el grado de maestros ---uno en
proceso de titulación y otro en proceso de estudios--- apenas representan el
8.7\,\% en relación del total que ostenta el grado para ejercer la docencia
en la licenciatura. De este modo, de los 21 maestros en Historia, filosofía
e historia de las ideas, estudios novohispanos, historia moderna y
contemporánea de México, ciencias sociales, estudios latinoamericanos e
historia, 19 cuentan con estudios doctorales en diferentes etapas: en
curso, en proceso de titulación y titulados. 

 
Hay 10 profesores con grado de doctor (53\,\%), 5 en proceso de titulación
(26\,\%) y 4 en estudios de doctorado (21\,\%). Estas cifran dan cuenta de una
continuidad en la formación profesional de la planta docente. Si bien se
consolida el campo disciplinar,  también se traza la presencia del
acompañamiento interdisciplinar desde el área de la tecnología educativa
aplicada a la historia. En este nivel de desarrollo profesional hay que
destacar la composición institucional mixta, es decir, egresados de
programas de posgrados de la UAZ  y otras instituciones. 


Si bien predominan de manera real y potencial, 12 doctores con
formación en la UAZ (63\,\%) frente a los 7 doctores y doctorantes  de otras
instituciones (37\,\%),  se debe enfatizar que ocho ya tienen el grado 
(cuatro del programas de Doctorado en Humanidades y Artes,  dos del Programa
de Doctorado en Historia que posteriormente se convirtió en el Programa de
Maestría-Doctorado inscrito en PNPC, y dos  docentes del Programa de
Maestría-Doctorado PNPC; dos doctorantes que egresaron del Programa de
Doctorado en Historia están en proceso de titulación y uno del Programa de
Humanidades y Artes; un doctorante del programa PNPC  lo hará en diciembre
2014. 


Cabe mencionar que de los otros siete docentes, dos están tituladas, una
por parte de El Colegio de Michoacán y otra de la UNAM; los pasantes de
doctor son dos, egresadas de la Universidad Michoacana de San Nicolás de
Hidalgo (PNPC) y de la Universidad Autónoma Metropolitana-Iztapalapa
(PNPC); y tres de ellos se encuentran en proceso de formación en programas
de la Universidad Autónoma Metropolitana-Azcapotzalco (PNPC), la
Universidad de Guadalajara (PNPC) y la Escuela Normal Superior de Ciudad
Madero Tamaulipas.


Por lo expuesto, se aprecia que la planta docente mantiene un desarrollo
profesional continuo en los programas de posgrado en la UAZ y otras
instituciones acreditadas en el PNPC, de tal forma que de la formación
inicial dominante UAZ se dio el paso a programas externos; el resultado fue
el retorno de  nuevos profesionales capaces de incidir en la innovación del
Plan de Estudios de la Licenciatura en Historia 1994, 1999, 2004 y
recientemente 2011.  En cada uno de estos años, las reformas curriculares
que se han realizado han obedecido al examen que ha hecho la academia de
profesores y profesoras en torno al impacto y pertinencia del currículum en
la carrera, a la par que por el resultado de observaciones que ha externado
la comunidad egresada del programa, los encuentros con empleadores y
empleadoras que se han llevado a cabo, las políticas educativas
institucionales que rigen la propia Universidad y, por supuesto, el
escenario contextual nacional y local en el que se ha inscrito la
Licenciatura.

Se suma a ello, el influjo de la relación contractual que aconteció en junio
de 2013. En este proceso se otorgaron medios tiempos de base, tiempos
completos de base y diez horas base frente a grupo que retiene a ese
capital humano en el programa de Licenciatura e incide en la concesión de
dos descargas para el inicio de doctorados en dos programas PNPC externas;
y la acreditación del Programa de Maestría-Doctorado en Historia de la UAZ
en el PNPC, del que egresaron dos docentes y uno aún se mantiene
concluyendo sus estudios.
\newpage

\textbf{El docente-investigador: el reto de formar CA\\ y después redes}

El Programa de Licenciatura en Historia promueve desde la normatividad,
entre su colectivo de profesores la participación académica mediante la
investigación que producen las Líneas de Generación y Aplicación de
Conocimiento (LGAC) que detenta cada Cuerpo Académico (CA) a la que están
articulados como integrantes o colaboradores en aras de responder al
compromiso que ratificó en 2005 la Universidad Autónoma de Zacatecas en su
\textit{Modelo Académico UAZ Siglo XXI}: tender puentes para hacer real la
pertinencia social entre el núcleo universitario y la sociedad. 

\enlargethispage{1\baselineskip}
En este tenor, los cuatro CA en consolidación del Área de Humanidades y
Educación de la UAZ [<<Enseñanza  y Difusión de la Historia>> (UAZ--CA--184),
<<Teoría, Historia e Interpretación del Arte>> (UAZ--CA--172) e <<Historia,
Cultura y Sociedad en Hispanoamérica>> (UAZ--CA--130) y <<Cultura, Currículum y
Procesos Institucionales>>\linebreak (UAZ--CA--150)] sirven de escenario para que ocho
profesores participen como integrantes, y dos profesores como colaboradores
de dichos cuerpos. 
 
Los datos revelan un esfuerzo importante de parte de este sector del
colectivo de profesores, pues han  logrado de manera consecutiva participar
como ponentes en congresos nacionales e internacionales, ser gestores de
eventos de iguales magnitudes e incursionar en la formación de redes
temáticas financiadas por PROMEP, redes y asociaciones  disciplinares,
inter y multidisciplinares. Estos  resultados se suman a la participación
del resto del colectivo en actividades académicas relevantes que favorecen
la reflexión sobre temas y subtemas, marcos teóricos y perspectivas
historiográficas vinculados a sus proyectos de investigación doctoral. 


Con  todo, aún están excluidos quince profesores de esta actividad
académica. Sin embargo, estos docentes dinámicos, emprendedores y
comprometidos con la investigación histórica que se inscribe en la historia
social, historia de la filosofía, historia  del agua, historia ambiental,
historia económica, historia social de la educación, historia de la ciudad,
historia del patrimonio cultural, historia de la religiosidad y la historia
biográfica, e incursiona en la investigación educativa con temas educación
tecnológica de los que se desprenden temáticas inter y multidisciplinarias.
\enlargethispage{1\baselineskip}
 
La diversidad de enfoques en y de las temáticas ha propiciado un
acercamiento de facto y de manera formal a redes y asociaciones académicas
que no exigen a los participantes, la identidad colegiada del CA. Así pues,
los 24 docentes anual o bianualmente se reúnen con los especialistas en
investigación histórica, patrimonial y educativa, pues atienden a la
convocatoria que hace  la <<Red Temática de Estudios Estéticos y de las
Artes>>, la <<Red Arte, Música y Cultura>>, la <<Red Nacional de Licenciaturas
en Historia y sus Cuerpos Académicos>>, la <<Red Nacional de Especialistas en
Docencia, Difusión e Investigación en la Enseñanza de la Historia>>,
Sociedad Latinoamericana y Caribeña de Historia Ambiental, la <<Red de
Investigadores Sociales Sobre Agua>>, Sociedad Mexicana de Historia de la
Educación, el Comité Mexicano de Investigación Educativa, Asociación <<Polo
Académico>> de San Luis Potosí, Red de Investigación  Educativa de San Luis
Potosí, Red de Investigación Educativa de Campeche, Red de Investigación
Educativa de Durango,  la Asociación Civil <<Elías Amador>>, la Asociación
Identidad, Memoria y Cultura A.\ C., el Grupo de Investigación <<Cartografía
Hidráulica de México>> y la Red de Investigadores del Pensamiento
Novohispano.

 
A partir de 2004 y hasta diciembre de 2013, se sostiene la participación académica
del colectivo de profesores como ponentes con investigaciones que
desarrollan desde los CA, los proyectos de investigación registrados en la
Coordinación de Investigación y Posgrado de la UAZ, o se desprenden de la
investigación de tesis que desarrollan en posgrado. Se distinguen dos
periodos: el que va de 2004 a 2008 y el de 2009 a 2013, mismos que podemos
denominar de inicio-despegue y desarrollo-crecimiento, respectivamente. 

 
El financiamiento de estas actividades académicas ha sido a través de
recursos federales derivados de PROMEP a través de las convocatorias para
el fortalecimiento de los CA y los Programas de Integral de Fortalecimiento
Institucional (PIFI) 2010 y 2012. A pesar de que no existe un programa
institucionalizado interno en la UAZ para otorgar recursos a los
académicos, la Rectoría  y el Sindicato de Personal Académico de la
Universidad Autónoma de Zacatecas (SPAUAZ) se muestran sensibles a la
investigación de corte histórico que realizan sus docentes-investigadores y
les ofrecen recursos para la asistencia a congresos, encuentros, foros y
coloquios nacionales e internacionales.

 
Asimismo, se implementó una estrategia interna para incrementar la
participación de los profesores de la licenciatura a través de la
<<Formación de Grupos de Trabajo>> que tienen posibilidades de formalizar la
investigación y la participan que realizan de facto en la próxima
convocatoria PROMEP para que sean evaluados y tentativamente sean
registrados <<en formación>>.


\bigskip
\textbf{Dos mundos, dos realidades frente a los estímulos, el perfil
deseable y el sistema nacional}

 
No todos los  profesores y profesoras del Programa de Licenciatura en
Historia tienen la posibilidad de acceder al \textit{Programa de Estímulos
al Desempeño del Personal docente (ESDEPED) }que ofrece la Universidad
Autónoma de Zacatecas anualmente a este sector laboral de todas las
unidades académicas que integran las áreas de la institución. Si bien es
cierto que El Programa de ESDEPED se publica con anticipación en la 
\href{http://www.uaz.edu.mx/}{\textstyleInternetLink{página
web de la Universidad}} y con breves avisos en los recibos de nómina 
quincenales del personal académico, siempre será el profesor de tiempo 
completo de base el que acceda a dicho programa. 

 
El Programa \textit{ESDEPED 2012--2013} de la UAZ, efectivamente posee un
Reglamento que está compuesto de 9 secciones con miras a reconocer la
calidad, la dedicación y la permanencia, pero esos lineamientos trazan la
diferencia entre el profesor de tiempo completo de base y profesor de
tiempo determinado o suplente, que aunque estén cuantitativamente sean de
tiempo completo, su categoría es inferior a la de tiempo completo nivel A,
B o C. 

 
En este sentido, de los 24 profesores del programa de Licenciatura, 22
cuentan con el reconocimiento de docente-investigador, es decir, 96\,\% están
potencialmente en condiciones de acceder en alguna de las emisiones de la
convocatoria anual del \textit{ESDEPED} de la UAZ a corto y mediano plazo. 


 
Estas cifras reflejan un área de oportunidades significativa pues, el 88\,\%
cuenta con el grado de maestría, porcentaje que satisface el criterio de
calidad ---aspecto formación profesional del docente--- considerado en el
\textit{ESDEPED}. 

 
Este espectro de oportunidades tiende a ser real por el hecho de que un
<<profesor de carrera>> (término que alude al docente de tiempo completo y
medio tiempo de base),  permanece 15 horas semana\slash{}mes y 10 horas 
semana\slash{}mes, respectivamente frente a grupo, es decir cada uno de ellos ejerce la
práctica docente en tres grupos o dos, según sea el caso. Se aprecia que de
los 24 profesores, el 41.6\,\% de ellos (10) es profesor de Tiempo Completo
Base, el 45.8\,\% (11) Medio Tiempo Base y solamente el 8.3\,\% (2) son
profesores de asignatura (tiempo determinado). 

 
En este mismo tenor, el 100\,\% de la planta docente participa en el Programa
de Tutorías institucionalizado en la UAZ en 2005 en el Modelo Académico UAZ
siglo XXI, pero también por la aplicación y vigencia de la \textit{Comisión
de Tutorías} que se estableció por acuerdo en el colectivo de maestros de
la licenciatura en 2004 junto con las Comisiones de Titulación, Servicio
Social y Seguimiento de Egresados y Empleadores, y Movilidad. La Comisión
de Tutorías se compone de tres cargos: presidente, secretario y vocal;
distingue por estar vigente y ser renovable cada dos años. 
\enlargethispage{1\baselineskip}
 
El pasado 25 de septiembre de 2013, en asamblea plenaria del colectivo de
profesores, presidida por la responsable del Programa y el director de la
Unidad Académica de Historia se procedió a la elección de los nuevos
titulares de las mencionadas comisiones. Este mecanismo se satisface el
criterio de calidad docente en el área de tutorías a la vez que incluye a
toda la planta docente. 

 
Atendiendo a la explicación en torno a la relación laboral y el estatus
académico ya expuesto, se deriva que el 87\,\% del colectivo de profesores
está posibilidades de satisfacer los otros dos criterios de calidad docente
estipulados en las convocatorias \textit{ESDEPED}: la dedicación y la
permanencia. Aspectos que ponderan, como ya se expuso en los párrafos
anteriores, práctica docente frente a grupo y antigüedad institucional.

 
No obstante, existen dos obstáculo que tendrán que superar los docentes de
medio tiempo base (45.8\,\%), ser profesores de tiempo\linebreak completo de base y
que dicha base tenga un año de antigüedad, tiempo que es el que se somete a
evaluación. En este sentido, el SPAUAZ y la Comisión Mixta son las
instancias que determinan el número de bases de tiempo completo que se
pueden ofertar anualmente. 

 
En este sentido, el Programa \textit{ESDEPED} tiene dos componentes en su
base: la laboral y la académica, ambas están articuladas de tal forma que
funcionan como palanca para que el docente, en nuestro caso, del programa
de Licenciatura de Historia, vaya constituyendo y articulando su quehacer
académico con miras a su evaluación y acreditación en el Programa de
Mejoramiento Profesional (PROMEP) y el Sistema Nacional de Investigadores
(SNI) que en evalúan la docencia de manera integral: horas frente a grupo,
tutorías, asesorías de tesis e impacto de la generación del conocimiento.
Estas dos instancias no exigen que el profesor aspirante sea tiempo
completo de base, sino que tenga una relación laboral de tiempo completo y
que su adscripción institucional esté avalada por el Departamento de
Recursos Humanos. 

 
Bajo esta dinámica, de los 24 profesores que integran el colectivo del
Programa de Licenciatura, 12 pertenecen al PROMEP, es decir, el 48\,\%. El
indicador del 48\,\% se incrementará sustancialmente en la próxima
convocatoria (2014) debido a que dos docentes habrán concluido su proceso
de titulación del doctorado y los dos que ya se doctoraron tendrán
oportunidad de alcanzar el estatus de profesor de tiempo completo de forma
inmediata con el simple hecho de sumen a su medio tiempo de base que ya
desempeñan, veinte horas de tiempo determinado que el Programa de
Licenciatura puede asumir, o incluso el Programa de Licenciatura en Turismo
 por pertenecer a la Unidad Académica de Historia es receptor de docentes
con perfil profesional en Historia. 

 
De este modo se proyecta que para 2014, habrá un mayor número de profesores con
Perfil Deseable. En otras palabras, del 48\,\% en 2013 se pasará a un 64\,\%
en 2014, esto es, se tendrá un incremento de Perfiles Deseables del 16\,\% en
menos de un año.

 
Si bien es cierto, que en cuanto al PERFIL SNI, en la Convocatoria 2013,
tres docentes figuran con el estatus de Investigadoras Nacionales, dos en
calidad de candidatas y una de ellas por segunda vez en Nivel I, lo cual se
traduce en un 12\,\% del total de profesores; los dos doctores recién
regresados en agosto de 2013 y los dos que egresaran en junio 2014,
participarán en la convocatoria 2014 con grandes posibilidades pues
mantienen una trayectoria inicial, pero sólida en la investigación, son
jóvenes y egresaron de programas académicos inscritos en el PNPC. Por lo
tanto, se espera que para 2014, se cuente con 7 Investigadores Nacionales,
esto es un 28\,\%.

 
Si bien, el indicador del 28\,\% es importante, alcanzar esa meta en 2014 no
cierra las expectativas de que un futuro inmediato (2015--2016) continúe
incrementándose el número de SNI, pues en el programa existen dos doctoras,
cuatro candidatos a doctor y dos doctorantes de recién ingreso a Programas
PNPC. 

 
Es importante enfatizar que, en la medida en que el Programa de Licenciatura
en Historia pertenece a la Unidad Académica de Historia que cuenta con el
Programa de Licenciatura en Turismo y el Programa de Maestría-Doctorado en
Historia registrado en el PNPC que posee al menos con 10 Investigadores
Nacionales Nivel I y II, el profesorado es un académico con inscripción
laboral a la Unidad, por lo tanto, se genera la movilidad de su planta
docente en sentido horizontal y vertical. Por otro lado, esta movilidad se
potencializa en cuanto la Unidad es parte de un Área Académica, la de
Humanidades y Educación, que tiene a su interior un importante número de
SNI. Todos ellos en condiciones laborales de colaborar en cualquier
Programa del Área, sí y solo sí, se dejan a tras intereses individuales y
políticos. 


\bigskip
\textbf{Consideraciones finales}

\enlargethispage{1\baselineskip} 
Transitar de la evaluación de los CIIES a la acreditación COAPEHUM  no va
ser fácil para el programa de licenciatura en historia en 2014. El
diagnóstico de la carpeta de personal docente de agosto a diciembre de 2013
era prometedor en ese momento. Ahora está en crisis: cuatro docentes de
tiempo completo con  perfil PROMEP y dos SNI, formalizaron su cambio de
adscripción a la unidad académica de docencia superior en septiembre de
2013. Si bien, se desempeñan todavía como profesores de diez horas en el
programa de licenciatura, esas plazas no fueron recuperadas por el
director. Las gestiones para llenar ese agujero no se han dado por las
graves dificultades financieras y políticas que atraviesa la universidad. 

 
Las sesenta horas clase que dejaron esas vacantes se cubrieron con dos
profesores que tienen doctorado, pero son egresados de la UAZ; con el
incremento de horas clases a docentes de tiempo determinado y medio tiempo
de base; esta solución inmediata, abre la puerta a la endogamia. La
situación se agrava más por el cambio de adscripción del CA: <<Enseñanza y
difusión de la historia>>, reconocido por PROMEP como el único colegiado que
tenía ese programa. Se suma a esta difícil situación, el hecho de que los
profesores que realizaron su cambio, están perdiendo influencia en las
asambleas de pares y los alumnos.

 
Las instancias evaluadoras como COAPEHUM, valoran la calidad de los
programas en relación con los recursos humanos y materiales; con  bienes
disponibles y recursos; obtención de productos colegiados nacional e
internacionalmente, tasas de promoción docente y estudiantil. Lo real, lo
posible y lo deseable no guardan la misma proporción en los criterios de
evaluación. Lo deseable se superpone a lo posible y lo real. En la
Universidad Autónoma de Zacatecas, hay voces que hablan sobre la necesidad
de una reingeniería para la UAZ, entendida ésta como una revisión y
modificación estructural, capaz de sacar a la institución del estancamiento
académico, económico y político en que se encuentra. 


\bigskip
\textbf{Referencias}

 
Arroyp Alejandre, Jesús (2010), <<La acreditación en México en el contexto de
las políticas públicas de educación superior>>, en \textit{Consejo para la
Acreditación de la Educación Superior}, \textit{A.C.}, Universidad del Mar,
Noviembre, pp. 59-72.

 
González Barroso, Antonio (2004), \textit{Reforma curricular. Propuesta de
Plan de Estudios 2004--2008}, Zacatecas, Universidad Autónoma de
Zacatecas-Licenciatura en Historia.

 
\_\_\_\_\_\_ (2011)  \textit{Reforma curricular. Propuesta de
Plan de Estudios 2011}, Zacatecas, Universidad Autónoma de
Zacatecas-Licenciatura en Historia.


COAPEHUM (2011), \textit{Instrumentos de autoevaluación y evaluación}, 
México, COAPEHUM.
 
\textit{Programas académicos de la Licenciatura en Historia de la
Universidad Autónoma de Zacatecas 2004--2008}. Zacatecas, Unidad Académica
de Historia-Programa de Licenciatura.

\medskip
\textbf{Referencias documentales}

Archivo de datos curriculares de la planta docente.